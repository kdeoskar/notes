%
% This is the LaTeX template file for lecture notes for CS294-8,
% Computational Biology for Computer Scientists.  When preparing 
% LaTeX notes for this class, please use this template.
%
% To familiarize yourself with this template, the body contains
% some examples of its use.  Look them over.  Then you can
% run LaTeX on this file.  After you have LaTeXed this file then
% you can look over the result either by printing it out with
% dvips or using xdvi.
%
% This template is based on the template for Prof. Sinclair's CS 270.

\documentclass[twoside]{article}
\usepackage{graphics}
\usepackage{mathtools}
\usepackage[]{mdframed}
\usepackage{amsmath}
\usepackage{amsfonts}
\usepackage[shortlabels]{enumitem}
\usepackage{bbm}

%\usepackage{asmfonts}
\setlength{\oddsidemargin}{0.25 in}
\setlength{\evensidemargin}{-0.25 in}
\setlength{\topmargin}{-0.6 in}
\setlength{\textwidth}{6.5 in}
\setlength{\textheight}{8.5 in}
\setlength{\headsep}{0.75 in}
\setlength{\parindent}{0 in}
\setlength{\parskip}{0.1 in}

%
% The following commands set up the lecnum (lecture number)
% counter and make various numbering schemes work relative
% to the lecture number.
%
\newcounter{lecnum}
\renewcommand{\thepage}{\thelecnum-\arabic{page}}
\renewcommand{\thesection}{\thelecnum.\arabic{section}}
\renewcommand{\theequation}{\thelecnum.\arabic{equation}}
\renewcommand{\thefigure}{\thelecnum.\arabic{figure}}
\renewcommand{\thetable}{\thelecnum.\arabic{table}}

%
% The following macro is used to generate the header.
%
\newcommand{\lecture}[4]{
   \pagestyle{myheadings}
   \thispagestyle{plain}
   \newpage
   \setcounter{lecnum}{#1}
   \setcounter{page}{1}
   \noindent
   \begin{center}
   \framebox{
      \vbox{\vspace{2mm}
    \hbox to 6.28in { {\hfill Fall 2023} }
       \vspace{4mm}
       \hbox to 6.28in { {\Large \hfill Some exercises from Shankar \hfill} }
       \vspace{2mm}
       \hbox to 6.28in { {\it Lecturer: #3 \hfill #4} }
      \vspace{2mm}}
   }
   \end{center}
   \markboth{Lecture #1: #2}{Lecture #1: #2}
   {\bf Disclaimer}: {\it LaTeX template courtesy of the UC Berkeley EECS Department.}
   \vspace*{4mm}
}

%
% Convention for citations is authors' initials followed by the year.
% For example, to cite a paper by Leighton and Maggs you would type
% \cite{LM89}, and to cite a paper by Strassen you would type \cite{S69}.
% (To avoid bibliography problems, for now we redefine the \cite command.)
% Also commands that create a suitable format for the reference list.
\renewcommand{\cite}[1]{[#1]}
\def\beginrefs{\begin{list}%
        {[\arabic{equation}]}{\usecounter{equation}
         \setlength{\leftmargin}{2.0truecm}\setlength{\labelsep}{0.4truecm}%
         \setlength{\labelwidth}{1.6truecm}}}
\def\endrefs{\end{list}}
\def\bibentry#1{\item[\hbox{[#1]}]}

%Use this command for a figure; it puts a figure in wherever you want it.
%usage: \fig{NUMBER}{SPACE-IN-INCHES}{CAPTION}
\newcommand{\fig}[3]{
			\vspace{#2}
			\begin{center}
			Figure \thelecnum.#1:~#3
			\end{center}
	}
% Use these for theorems, lemmas, proofs, etc.
\newtheorem{theorem}{Theorem}[lecnum]
\newtheorem{lemma}[theorem]{Lemma}
\newtheorem{proposition}[theorem]{Proposition}
\newtheorem{claim}[theorem]{Claim}
\newtheorem{corollary}[theorem]{Corollary}
\newtheorem{definition}[theorem]{Definition}
\newenvironment{proof}{{\bf Proof:}}{\hfill\rule{2mm}{2mm}}

% **** IF YOU WANT TO DEFINE ADDITIONAL MACROS FOR YOURSELF, PUT THEM HERE:

\begin{document}
%FILL IN THE RIGHT INFO.
%\lecture{**LECTURE-NUMBER**}{**DATE**}{**LECTURER**}{**SCRIBE**}
%\footnotetext{These notes are partially based on those of Nigel Mansell.}


%%%%%%%%%%%%%%%%%%%%%%%%%%%%%%%%%%%%%%%%%%
%Additional commands

\newcommand{\ket}[1]{\mid#1\rangle}
\newcommand{\bra}[1]{\langle#1\mid}
\newcommand{\R}{\mathbb{R}}
\newcommand{\Prob}[1]{\mathbb{P}(#1)}
\newcommand{\mean}[1]{\left\langle #1 \right\rangle}
\newcommand{\inner}[2]{\left\langle #1 \bigg\rvert #2 \right\rangle}
\newcommand{\ham}{\hat{H}}
\newcommand{\mom}{\hat{P}}



%% Install amsfonts or amssymb package so that below command can be defined 
%\newcommand{\R}{\mathbb{R}

%%%%%%%%%%%%%%%%%%%%%%%%%%%%%%%%%%%%%%%%%%

% **** YOUR NOTES GO HERE:

%%%%%%%%%%%%%%%%%%%%%%%%%%%%%%%%%%%%%%%%%%
%%%%%         HOMEWORK 1
%%%%%%%%%%%%%%%%%%%%%%%%%%%%%%%%%%%%%%%%%%
\lecture{10}{November 21}{Chien-I Chiang}{Keshav Deoskar}


% Some general latex examples and examples making use of the
% macros follow.  
%**** IN GENERAL, BE BRIEF. LONG SCRIBE NOTES, NO MATTER HOW WELL WRITTEN,
%**** ARE NEVER READ BY ANYBODY.


%%%%%%%%%%%%%%%%%%%%%%%%%%%%%%%%%%%%%%%%%
%%%%%%%%%%%%% Question 1
%%%%%%%%%%%%%%%%%%%%%%%%%%%%%%%%%%%%%%%%%
\underline{\textbf{Exercise 10.1.2:}}

We have basis vectors $\ket{+}, \ket{-}$ and operators
\begin{align*}
   \sigma_1^{(1)} = \begin{bmatrix}
      a & b \\
      c & d
   \end{bmatrix}
\end{align*} 

and 

\begin{align*}
   \sigma_2^{(2)} = \begin{bmatrix}
      e & f \\
      g & h
   \end{bmatrix}
\end{align*} 

\begin{enumerate}
   \item We know that
   \begin{align*}
      \sigma^{(1) \otimes (2)} = \sigma^{(1)} \otimes \mathbbm{1}^{(2)}
   \end{align*}
   can be written in terms of how it acts on the basis vectors $\ket{+} \otimes \ket{-}$, $\ket{+} \otimes \ket{+}$, $\ket{-} \otimes \ket{-}$, and $\ket{-} \otimes \ket{-}$ in the sense that the element $\left(\sigma^{(1)} \otimes \mathbbm{1}^{(2)}\right)_{11}$ is given by 

   \begin{align*}
      \left(\sigma^{(1)} \otimes \mathbbm{1}^{(2)}\right)_{11} &= \left(\bra{+}\otimes\bra{+}\right)\left(\sigma_1^{(1)} \otimes \mathbbm{1}^{(2)}\right)\left(\ket{+}\otimes\ket{+}\right) \\
      &= \left(\bra{+}\otimes\bra{+}\right) \left( \ket{\sigma_1^{(1)} | 1} \otimes \ket{\mathbbm{1}|+}\right) \\
      &= \inner{+}{\sigma_1^{(1)}|+} \inner{+}{+} \\
      &= a
   \end{align*}
   and so on.

   Thus, carring out all such calculations, the matrix has $4 \times 4$ elements and is given by 

   \[ \sigma_1^{(1) \otimes (2)} = \sigma^{(1)} \otimes \mathbbm{1}^{(2)} = \begin{bmatrix}
         a & 0 & b & 0 \\
         0 & a & 0 & b \\
         c & 0 & d & 0 \\
         0 & c & 0 & d 
      \end{bmatrix} \]


   \item Similar procedure for $\sigma_2^{(1) \otimes (2)}$. We find that 
   \[ \sigma_2^{(1) \otimes (2)} = \mathbbm{1}^{(1)} \otimes \sigma^{(2)} = \begin{bmatrix}
      e & f & 0 & 0 \\
      g & h & 0 & 0 \\
      0 & 0 & e & f \\
      0 & 0 & g & h 
   \end{bmatrix} \]
   
   \item There is a better way to show parts (a) and (b). In this method, we prove a more general statement that if 
   \[ A_1^{(1)} = \begin{bmatrix}
      a_{11} & a_{12} \\
      a_{21} & a_{22} \\
   \end{bmatrix} \]

   and 
   \[ B_2^{(2)} = \begin{bmatrix}
      b_{11} & b_{12} \\
      b_{21} & b_{22} \\
   \end{bmatrix} \]

   Then, their direct product is given by 
   \[ M = A_1^{(1)} \otimes B_2^{(2)} = \begin{bmatrix}
      a_{11} \begin{pmatrix}
         b_{11} & b_{12} \\
      b_{21} & b_{22} \\
      \end{pmatrix} &       a_{12} \begin{pmatrix}
         b_{11} & b_{12} \\
      b_{21} & b_{22} \\
      \end{pmatrix} \\
      a_{21} \begin{pmatrix}
         b_{11} & b_{12} \\
      b_{21} & b_{22} \\
      \end{pmatrix} &       a_{22} \begin{pmatrix}
         b_{11} & b_{12} \\
      b_{21} & b_{22} \\
      \end{pmatrix}
   \end{bmatrix} \]

   \textbf{Proof:}
   The matrix element $M_{(i_1 \cdot i_2), (j_1 \cdot j_2)}$ is related to entries in $A$ and $B$ as 
   \begin{align*}
      M_{(i_1 \cdot i_2), (j_1 \cdot j_2)} &= \left( \bra{i_1} \otimes \bra{i_2} \right) \left( A_1^{(1)} \otimes B_2^{(2)} \right) \left( \ket{j_1} \otimes \ket{j_2} \right) \\
      &= \left( \bra{i_1} \otimes \bra{i_2} \right) \left( A_1^{(1)} \ket{j_1} \otimes B_2^{(2)} \ket{j_2} \right) \\
      &= \inner{i_1}{ A_1^{(1)} \biggr\rvert j_1}\inner{i_2}{ B_2^{(2)} \biggr\rvert j_2} \\
      &= (A_1^{(1)})_{i_1, j_1} \cdot (B_2^{(2)})_{i_2, j_2}
   \end{align*}
   and this is equivalent to the matrix form written above.

   Thus, we can directly apply this result in finding the tensor product between $\sigma_1^{(1)}$ and $\sigma_2^{(2)}$:
   
   \begin{align*}
      (\sigma_1 \sigma_2)^{(1)\otimes(2)} &= \sigma_1^{(1)} \otimes \sigma_2^{(2)} \\
      &= \begin{bmatrix}
         a \begin{pmatrix}
            e & f \\
            g & h
         \end{pmatrix} &          b \begin{pmatrix}
            e & f \\
            g & h
         \end{pmatrix} \\
         c \begin{pmatrix}
            e & f \\
            g & h
         \end{pmatrix} &          d \begin{pmatrix}
            e & f \\
            g & h
         \end{pmatrix} 
      \end{bmatrix} \\
   &= \begin{bmatrix}
      ae & af & be & bf \\
      ag & ah & bg & bh \\
      ce & cf & de & df \\
      cg & ch & dg & dh \\
   \end{bmatrix}
   \end{align*}

   DO SECOND METHOD OF PART 3 LATER
\end{enumerate}

\vskip 0.5cm
\hrule
\vskip 0.5cm




%%%%%%%%%%%%%%%%%%%%%%%%%%%%%%%%%%%%%%%%%
%%%%%%%%%%%%% Question 2
%%%%%%%%%%%%%%%%%%%%%%%%%%%%%%%%%%%%%%%%%
\underline{\textbf{Problem}}

\vskip 0.5cm
\hrule
\vskip 0.5cm






\end{document}





