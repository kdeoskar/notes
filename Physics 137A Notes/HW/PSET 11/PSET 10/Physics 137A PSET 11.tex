%
% This is the LaTeX template file for lecture notes for CS294-8,
% Computational Biology for Computer Scientists.  When preparing 
% LaTeX notes for this class, please use this template.
%
% To familiarize yourself with this template, the body contains
% some examples of its use.  Look them over.  Then you can
% run LaTeX on this file.  After you have LaTeXed this file then
% you can look over the result either by printing it out with
% dvips or using xdvi.
%
% This template is based on the template for Prof. Sinclair's CS 270.

\documentclass[twoside]{article}
\usepackage{graphics}
\usepackage{mathtools}
\usepackage[]{mdframed}
\usepackage{amsmath}
\usepackage{amsfonts}
\usepackage[shortlabels]{enumitem}
\usepackage{bbm}

%\usepackage{asmfonts}
\setlength{\oddsidemargin}{0.25 in}
\setlength{\evensidemargin}{-0.25 in}
\setlength{\topmargin}{-0.6 in}
\setlength{\textwidth}{6.5 in}
\setlength{\textheight}{8.5 in}
\setlength{\headsep}{0.75 in}
\setlength{\parindent}{0 in}
\setlength{\parskip}{0.1 in}

%
% The following commands set up the lecnum (lecture number)
% counter and make various numbering schemes work relative
% to the lecture number.
%
\newcounter{lecnum}
\renewcommand{\thepage}{\thelecnum-\arabic{page}}
\renewcommand{\thesection}{\thelecnum.\arabic{section}}
\renewcommand{\theequation}{\thelecnum.\arabic{equation}}
\renewcommand{\thefigure}{\thelecnum.\arabic{figure}}
\renewcommand{\thetable}{\thelecnum.\arabic{table}}

%
% The following macro is used to generate the header.
%
\newcommand{\lecture}[4]{
   \pagestyle{myheadings}
   \thispagestyle{plain}
   \newpage
   \setcounter{lecnum}{#1}
   \setcounter{page}{1}
   \noindent
   \begin{center}
   \framebox{
      \vbox{\vspace{2mm}
    \hbox to 6.28in { {\bf Physics 137A: Quantum Mechanics
                        \hfill Fall 2023} }
       \vspace{4mm}
       \hbox to 6.28in { {\Large \hfill PSET #1, Due #2  \hfill} }
       \vspace{2mm}
       \hbox to 6.28in { {\it Lecturer: #3 \hfill #4} }
      \vspace{2mm}}
   }
   \end{center}
   \markboth{Lecture #1: #2}{Lecture #1: #2}
   {\bf Disclaimer}: {\it LaTeX template courtesy of the UC Berkeley EECS Department.}
   \vspace*{4mm}
}

%
% Convention for citations is authors' initials followed by the year.
% For example, to cite a paper by Leighton and Maggs you would type
% \cite{LM89}, and to cite a paper by Strassen you would type \cite{S69}.
% (To avoid bibliography problems, for now we redefine the \cite command.)
% Also commands that create a suitable format for the reference list.
\renewcommand{\cite}[1]{[#1]}
\def\beginrefs{\begin{list}%
        {[\arabic{equation}]}{\usecounter{equation}
         \setlength{\leftmargin}{2.0truecm}\setlength{\labelsep}{0.4truecm}%
         \setlength{\labelwidth}{1.6truecm}}}
\def\endrefs{\end{list}}
\def\bibentry#1{\item[\hbox{[#1]}]}

%Use this command for a figure; it puts a figure in wherever you want it.
%usage: \fig{NUMBER}{SPACE-IN-INCHES}{CAPTION}
\newcommand{\fig}[3]{
			\vspace{#2}
			\begin{center}
			Figure \thelecnum.#1:~#3
			\end{center}
	}
% Use these for theorems, lemmas, proofs, etc.
\newtheorem{theorem}{Theorem}[lecnum]
\newtheorem{lemma}[theorem]{Lemma}
\newtheorem{proposition}[theorem]{Proposition}
\newtheorem{claim}[theorem]{Claim}
\newtheorem{corollary}[theorem]{Corollary}
\newtheorem{definition}[theorem]{Definition}
\newenvironment{proof}{{\bf Proof:}}{\hfill\rule{2mm}{2mm}}

% **** IF YOU WANT TO DEFINE ADDITIONAL MACROS FOR YOURSELF, PUT THEM HERE:

\begin{document}
%FILL IN THE RIGHT INFO.
%\lecture{**LECTURE-NUMBER**}{**DATE**}{**LECTURER**}{**SCRIBE**}
%\footnotetext{These notes are partially based on those of Nigel Mansell.}


%%%%%%%%%%%%%%%%%%%%%%%%%%%%%%%%%%%%%%%%%%
%Additional commands

\newcommand{\ket}[1]{\rvert#1\rangle}
\newcommand{\bra}[1]{\langle#1\rvert}
\newcommand{\R}{\mathbb{R}}
\newcommand{\Prob}[1]{\mathbb{P}(#1)}
\newcommand{\mean}[1]{\left\langle #1 \right\rangle}
\newcommand{\inner}[2]{\left\langle #1 \bigg\rvert #2 \right\rangle}
\newcommand{\ham}{\hat{H}}
\newcommand{\mom}{\hat{P}}



%% Install amsfonts or amssymb package so that below command can be defined 
%\newcommand{\R}{\mathbb{R}

%%%%%%%%%%%%%%%%%%%%%%%%%%%%%%%%%%%%%%%%%%

% **** YOUR NOTES GO HERE:

%%%%%%%%%%%%%%%%%%%%%%%%%%%%%%%%%%%%%%%%%%
%%%%%         HOMEWORK 1
%%%%%%%%%%%%%%%%%%%%%%%%%%%%%%%%%%%%%%%%%%
\lecture{11}{December 04}{Chien-I Chiang}{Keshav Deoskar}


% Some general latex examples and examples making use of the
% macros follow.  
%**** IN GENERAL, BE BRIEF. LONG SCRIBE NOTES, NO MATTER HOW WELL WRITTEN,
%**** ARE NEVER READ BY ANYBODY.


%%%%%%%%%%%%%%%%%%%%%%%%%%%%%%%%%%%%%%%%%
%%%%%%%%%%%%% Question 1
%%%%%%%%%%%%%%%%%%%%%%%%%%%%%%%%%%%%%%%%%
\underline{\textbf{Problem 1:}}

In this question, we consider the 3D Particle in a box potential.
\[ V(x,y,z) = \begin{cases}
   0, \;\;\;x,y,z\text{ all between zero and 'a'} \\
   \infty,\;\;\text{ otherwise}
\end{cases} \]

\begin{enumerate}[label=\alph*]
   \item To find the wavefunctions of the time-independent energy eigenstates we will solve the Time Independent Schroedinger Equations, which presents itself in 3D as 
   \[ -\frac{\hbar^2}{2m} \nabla^2 \psi + V\psi = E\psi\]

   Outside the region $\{ (x,y,z) : 0 \leq x,y,z \leq a \}$, the potential is infinite i.e. $V(x,y,z) = \infty$. So, the PDE is 
   \[ -\frac{\hbar^2}{2m} \nabla^2 \psi + \infty\psi = E\psi \]

   But the energy $E$ must be some finite value, so the only valid wavefunction for this region is
   \[ \boxed{\psi(x,y,z) = 0} \] 
   \vskip 0.5cm

   Inside of the box, we solve this PDE by \textbf{Separation of Variables} in cartesian coordinates.

   Let's assume that the solutions to the PDE have the form 
   \[ \psi(x,y,z) = X(x)Y(y)Z(z) \]

   Then, the TISE reads as 
   \begin{align*}
      &-\frac{\hbar^2}{2m} \left[ \partial_x^2 + \partial_y^2 + \partial_z^2 \right](XYZ) + V(x,y,z) XYZ = E \cdot XYZ \\
      \implies&\frac{-\hbar2}{2m} \left[ \frac{\partial^2 X}{\partial x^2} YZ +  \frac{\partial^2 Y}{\partial y^2} XZ +  \frac{\partial^2 Z}{\partial z^2} XY \right] + V(x,y,z) (XYZ) = E(XYZ)
   \end{align*}

   Dividing through by $X(x)Y(y)Z(z)$, we have 
   \begin{align*}
      \frac{-\hbar2}{2m} \left[ \frac{1}{X}\frac{\partial^2 X}{\partial x^2} +   \frac{1}{Y}\frac{\partial^2 Y}{\partial y^2}  +  \frac{1}{Z}\frac{\partial^2 Z}{\partial z^2}  \right] + V(x,y,z) = E
   \end{align*}

   And also, recall that inside the box, we have $V(x,y,z) = 0$. This, TISE in this region is equivalent to the equation 
   \begin{align*}
      \frac{-\hbar^2}{2m} \left[ \frac{1}{X}\frac{\partial^2 X}{\partial x^2} +   \frac{1}{Y}\frac{\partial^2 Y}{\partial y^2}  +  \frac{1}{Z}\frac{\partial^2 Z}{\partial z^2}  \right] = E
   \end{align*}
   
   Notice that the LHS consists of three different functions, each dependent SOLELY on one of $x$, $y$, or $z$ being summed together to some constant.

   The only way for this to holds for all possible $(x,y,z)$ inside the box is for each of the functions to be constant itself. i.e. we have 

   \[ \frac{-\hbar^2}{2m}\frac{1}{X}\frac{\partial^2 X}{\partial x^2} = E_x \]
   \[ \frac{-\hbar^2}{2m}\frac{1}{Y}\frac{\partial^2 Y}{\partial y^2} = E_y \]
   \[ \frac{-\hbar^2}{2m}\frac{1}{Z}\frac{\partial^2 Z}{\partial z^2} = E_z \]

   where $E_x + E_y + E_z = E$. 
   
   
   But each of these equations is just the TISE for a free particle in the region $[0, a]$ for each of the directions $x, y, z$! This is a problem we've already solved, but Let's go through the solution for one of the dimensions and then extrapolate.
   \vskip 0.5cm

   Considering the $x$-direction, we have 
   \begin{align*}
      &\frac{-\hbar^2}{2m}\frac{1}{X}\frac{\partial^2 X}{\partial x^2} = E_x \\
      \implies& X^{''}(x) = -\frac{2mE_x}{\hbar^2} X(x)
   \end{align*}

   Or, setting $k = \sqrt{\frac{2mE_x}{\hbar^2}}$, we have
   \[ X^{''}(x) = -k^2 X(x) \]

   and this differential equation has the general solution
   \[ \boxed{ X(x) = Ae^{ikx} + Be^{-ikx}} \]

   Further, the wavefunction $X(x)$ must be continuous. Outside the region $0 \leq x \leq a$, the wavefunction must be zero, so invoking continuity gives us the boundary conditions $X(0) = 0$ and $X(a) = 0$.

   The first condition gives us 
   \begin{align*}
      &X(0) = A + B = 0 \\
      \implies& X(x) = A(e^{ikx} - e^{-ikx}) = 2A\sin(kx)
   \end{align*}

   And so, now, the second condition tells us that 
   \begin{align*}
      &X(a) = 2A \sin(ka) = 0\\
      \implies& ka = n\pi,\;\; n \in \mathbb{Z} \\
      \implies& k = \frac{n\pi}{a}
   \end{align*}
   Our $k$ is quantized to only have certain values, which means the energy of the particle in the x-direction is also quantized.


   Collecting everything we've found so far, we know
   \[ X(x) = 2A\sin\left(\frac{n\pi}{a}x\right) \]

   Finally, we use the normalization condition to find $A$ as
   \begin{align*}
      &\int_0^a |X(x)|^2 dx = 1 \\
      \implies& 4A^2 \int_0^a \sin^2 \left( \frac{n\pi x}{a} \right) dx  = 1\\
      \implies& 4A^2 \int_0^a \frac{1 - \cos(\frac{2n\pi x}{a})}{2} dx  = 1 \\
      \implies& 4A^2 \left[ \frac{x}{2} - \frac{a}{2n\pi}\frac{\sin(\frac{2n\pi x}{a})}{2} \right]_{0}^{a}  = 1 \\
      \implies& 4A^2 \left[ \frac{a}{2} - 0 \right] = 1 \\
      \implies& 2aA^2  = 1 \\
      \implies& A  = \frac{1}{\sqrt{2a}}
   \end{align*}
   Thus, the eigenfunction to the differential equation in the x-direction is 
   \[ \boxed{X(x) = \sqrt{\frac{2}{a}} \sin\left(\frac{n\pi}{a}x\right) } \]
   with eigenvalue
   \[ \boxed{E_x = \frac{\hbar^2 k^2}{2m} = \frac{\hbar^2 n^2 \pi^2}{2ma^2}} \]
   where $n = 1,2,3...$ 


   Solving the differential equations in the y and z directions, we would get the exact same result. Thus, we can conclude that the energy eigenstates of the 3D particle in a box are 
   \begin{align*}
      \psi(x,y,z) &= X(x)Y(y)Z(z) \\
      &= \left(  \sqrt{\frac{2}{a}} \sin\left(\frac{n\pi}{a}x\right) \right) \cdot \left(  \sqrt{\frac{2}{a}} \sin\left(\frac{n\pi}{a}y\right) \right) \cdot \left(  \sqrt{\frac{2}{a}} \sin\left(\frac{n\pi}{a}z\right) \right)  \\
      &= \sqrt{\frac{8}{a^3}}\sin\left(\frac{n\pi}{a}x\right)\sin\left(\frac{n\pi}{a}y\right)\sin\left(\frac{n\pi}{a}z\right)
   \end{align*}
   with energy eigenvalues
   \begin{align*}
      E &= E_x + E_y + E_Z \\
     &= \frac{\hbar^2 n_x^2 \pi^2}{2ma^2} + \frac{\hbar^2 n_y^2 \pi^2}{2ma^2} + \frac{\hbar^2 n_z^2 \pi^2}{2ma^2} \\
     &= \frac{\hbar^2 \pi^2}{2ma^2} (n_x^2 + n_y^2 + n_z^2)
   \end{align*}
   \vskip 1cm

   To conclude,
   \[ \boxed{ \psi_E(x,y,z) = \sqrt{\frac{8}{a^3}}\sin\left(\frac{n_x\pi}{a}x\right)\sin\left(\frac{n_y\pi}{a}y\right)\sin\left(\frac{n_z\pi}{a}z\right) } \]

   \[ \boxed{ E = \frac{\hbar^2 \pi^2}{2ma^2} (n_x^2 + n_y^2 + n_z^2) } \]  
   where $n_x, n_y, n_z = 1,2,3...$ 

   \item The Distinct Energies $E_1, E_2, E_3, E_4, E_5, E_6$ are found by finding calculating $E$ for different values of $n_x, n_y, n_z$.
   
   If we denote the number of degeneracies corresponding to each energy level as $d$, then
   \begin{align*}
      &E_1 = \frac{\hbar^2 \pi^2}{2ma^2} (3) \;\;:\;\; d = 1; \\
      &E_2 = \frac{\hbar^2 \pi^2}{2ma^2} (6) \;\;:\;\; d = 3; \\
      &E_3 = \frac{\hbar^2 \pi^2}{2ma^2} (9) \;\;:\;\; d = 3; \\
      &E_4 = \frac{\hbar^2 \pi^2}{2ma^2} (11) \;\;:\; d = 3; \\
      &E_5 = \frac{\hbar^2 \pi^2}{2ma^2} (12) \;\;:\; d = 1; \\
      &E_6 = \frac{\hbar^2 \pi^2}{2ma^2} (14) \;\;:\; d = 6; \\
   \end{align*}


   \item Supposing we have 5 non-interacting particles doomed to live together in a box, the lowest attainable energy depends on what kind of particles are present.
   
   \begin{enumerate}
      \item If the particles are \textbf{classical, distinguishable particles} then the ground state energy is zero. This is because, classically, energy is not quantized and can attain any value in a continuous range which includes zero.
      \vskip 1cm
      
      \item If the particles are Spin-1 particles, the Spin-Statistics Theorem tells us that they are \textbf{Bosons}, which means they do not need to satisfy the Pauli Exclusion principle. 
      
      Thus, all five particles will have the lowest possible energies i.e. each one will have energy $E = \frac{3\hbar^2\pi^2}{2ma^2}$. So, the ground state energy of the 5 particle system will be 

      \[ \boxed{E_{ground} = \frac{15\hbar^2\pi^2}{2ma^2}} \]
      \vskip 1cm

      \item If the particles are Spin-1/2, the Spin-Statistics Theorem tells us they are \textbf{fermions}. Thus, by the Pauli Exclusion Principle, each energy level can only accomodate two particles (of opposite spin).
      
      Thus, two particles will have energy $E_1 = \frac{\hbar^2 \pi^2}{2ma^2} (3)$, two will have energy $E_2 = \frac{\hbar^2 \pi^2}{2ma^2} (6)$, and one will have energy $E_3 = \frac{\hbar^2 \pi^2}{2ma^2} (9)$.

      Thus, the ground state energy of the 5-particle system will be 
      \[ \boxed{E_{ground} = 18 \frac{\hbar^2 \pi^2}{2ma^2}} \]

      \item When it comes to degeneracy, 
      \begin{enumerate}
         \item In the case of 5 classical distinguishable particles, all five particles lie in the same energy state so we have $d = 5$.
         
         \item For the bosons, once again, all five particles have the same energy so $d = 5$.
         
         \item In the case of the 5 fermions, due to the Pauli Exlcusion Principle, the ground state has two particles, the first excited state has two particles, and the second excited state has one particle.
      \end{enumerate}
   \end{enumerate}
\end{enumerate}

\vskip 0.5cm
\hrule
\vskip 0.5cm


% %%%%%%%%%%%%%%%%%%%%%%%%%%%%%%%%%%%%%%%%%
% %%%%%%%%%%%%% Question 2
% %%%%%%%%%%%%%%%%%%%%%%%%%%%%%%%%%%%%%%%%%
% \underline{\textbf{Problem 2:}}


% \vskip 0.5cm
% \hrule
% \vskip 0.5cm



% %%%%%%%%%%%%%%%%%%%%%%%%%%%%%%%%%%%%%%%%%
% %%%%%%%%%%%%% Question 3
% %%%%%%%%%%%%%%%%%%%%%%%%%%%%%%%%%%%%%%%%%
% \underline{\textbf{Problem 3:}}



% \vskip 0.5cm
% \hrule
% \vskip 0.5cm


\end{document}





