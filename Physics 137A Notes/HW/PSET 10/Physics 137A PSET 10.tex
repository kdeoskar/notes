%
% This is the LaTeX template file for lecture notes for CS294-8,
% Computational Biology for Computer Scientists.  When preparing 
% LaTeX notes for this class, please use this template.
%
% To familiarize yourself with this template, the body contains
% some examples of its use.  Look them over.  Then you can
% run LaTeX on this file.  After you have LaTeXed this file then
% you can look over the result either by printing it out with
% dvips or using xdvi.
%
% This template is based on the template for Prof. Sinclair's CS 270.

\documentclass[twoside]{article}
\usepackage{graphics}
\usepackage{mathtools}
\usepackage[]{mdframed}
\usepackage{amsmath}
\usepackage{amsfonts}
\usepackage[shortlabels]{enumitem}
\usepackage{bbm}

%\usepackage{asmfonts}
\setlength{\oddsidemargin}{0.25 in}
\setlength{\evensidemargin}{-0.25 in}
\setlength{\topmargin}{-0.6 in}
\setlength{\textwidth}{6.5 in}
\setlength{\textheight}{8.5 in}
\setlength{\headsep}{0.75 in}
\setlength{\parindent}{0 in}
\setlength{\parskip}{0.1 in}

%
% The following commands set up the lecnum (lecture number)
% counter and make various numbering schemes work relative
% to the lecture number.
%
\newcounter{lecnum}
\renewcommand{\thepage}{\thelecnum-\arabic{page}}
\renewcommand{\thesection}{\thelecnum.\arabic{section}}
\renewcommand{\theequation}{\thelecnum.\arabic{equation}}
\renewcommand{\thefigure}{\thelecnum.\arabic{figure}}
\renewcommand{\thetable}{\thelecnum.\arabic{table}}

%
% The following macro is used to generate the header.
%
\newcommand{\lecture}[4]{
   \pagestyle{myheadings}
   \thispagestyle{plain}
   \newpage
   \setcounter{lecnum}{#1}
   \setcounter{page}{1}
   \noindent
   \begin{center}
   \framebox{
      \vbox{\vspace{2mm}
    \hbox to 6.28in { {\bf Physics 137A: Quantum Mechanics
                        \hfill Fall 2023} }
       \vspace{4mm}
       \hbox to 6.28in { {\Large \hfill PSET #1, Due #2  \hfill} }
       \vspace{2mm}
       \hbox to 6.28in { {\it Lecturer: #3 \hfill #4} }
      \vspace{2mm}}
   }
   \end{center}
   \markboth{Lecture #1: #2}{Lecture #1: #2}
   {\bf Disclaimer}: {\it LaTeX template courtesy of the UC Berkeley EECS Department.}
   \vspace*{4mm}
}

%
% Convention for citations is authors' initials followed by the year.
% For example, to cite a paper by Leighton and Maggs you would type
% \cite{LM89}, and to cite a paper by Strassen you would type \cite{S69}.
% (To avoid bibliography problems, for now we redefine the \cite command.)
% Also commands that create a suitable format for the reference list.
\renewcommand{\cite}[1]{[#1]}
\def\beginrefs{\begin{list}%
        {[\arabic{equation}]}{\usecounter{equation}
         \setlength{\leftmargin}{2.0truecm}\setlength{\labelsep}{0.4truecm}%
         \setlength{\labelwidth}{1.6truecm}}}
\def\endrefs{\end{list}}
\def\bibentry#1{\item[\hbox{[#1]}]}

%Use this command for a figure; it puts a figure in wherever you want it.
%usage: \fig{NUMBER}{SPACE-IN-INCHES}{CAPTION}
\newcommand{\fig}[3]{
			\vspace{#2}
			\begin{center}
			Figure \thelecnum.#1:~#3
			\end{center}
	}
% Use these for theorems, lemmas, proofs, etc.
\newtheorem{theorem}{Theorem}[lecnum]
\newtheorem{lemma}[theorem]{Lemma}
\newtheorem{proposition}[theorem]{Proposition}
\newtheorem{claim}[theorem]{Claim}
\newtheorem{corollary}[theorem]{Corollary}
\newtheorem{definition}[theorem]{Definition}
\newenvironment{proof}{{\bf Proof:}}{\hfill\rule{2mm}{2mm}}

% **** IF YOU WANT TO DEFINE ADDITIONAL MACROS FOR YOURSELF, PUT THEM HERE:

\begin{document}
%FILL IN THE RIGHT INFO.
%\lecture{**LECTURE-NUMBER**}{**DATE**}{**LECTURER**}{**SCRIBE**}
%\footnotetext{These notes are partially based on those of Nigel Mansell.}


%%%%%%%%%%%%%%%%%%%%%%%%%%%%%%%%%%%%%%%%%%
%Additional commands

\newcommand{\ket}[1]{\rvert#1\rangle}
\newcommand{\bra}[1]{\langle#1\rvert}
\newcommand{\R}{\mathbb{R}}
\newcommand{\Prob}[1]{\mathbb{P}(#1)}
\newcommand{\mean}[1]{\left\langle #1 \right\rangle}
\newcommand{\inner}[2]{\left\langle #1 \bigg\rvert #2 \right\rangle}
\newcommand{\ham}{\hat{H}}
\newcommand{\mom}{\hat{P}}



%% Install amsfonts or amssymb package so that below command can be defined 
%\newcommand{\R}{\mathbb{R}

%%%%%%%%%%%%%%%%%%%%%%%%%%%%%%%%%%%%%%%%%%

% **** YOUR NOTES GO HERE:

%%%%%%%%%%%%%%%%%%%%%%%%%%%%%%%%%%%%%%%%%%
%%%%%         HOMEWORK 1
%%%%%%%%%%%%%%%%%%%%%%%%%%%%%%%%%%%%%%%%%%
\lecture{10}{November 21}{Chien-I Chiang}{Keshav Deoskar}


% Some general latex examples and examples making use of the
% macros follow.  
%**** IN GENERAL, BE BRIEF. LONG SCRIBE NOTES, NO MATTER HOW WELL WRITTEN,
%**** ARE NEVER READ BY ANYBODY.


%%%%%%%%%%%%%%%%%%%%%%%%%%%%%%%%%%%%%%%%%
%%%%%%%%%%%%% Question 1
%%%%%%%%%%%%%%%%%%%%%%%%%%%%%%%%%%%%%%%%%
\underline{\textbf{Problem 1:}}

We consider two non-interacting particles in a harmonic oscillator potential with one particle being in state $\ket{\psi} = \ket{\psi_0}$ and the other is in the state $\ket{\phi} = \frac{1}{\sqrt{2}} \left( \ket{\psi_0} + \ket{\psi_1} \right)$ where $\ket{\psi_i}$ represents the $i^{th}$ energy eigenstate of the Harmonic Oscillator.

\begin{enumerate}
   \item First, we want to construct the symmetrized and anti-symmetrized wavefunctions of the two-particle system, corresponding to Bosons and Fermions respectively.
   
   The (un-normalized) symmetrized wavefunction would be 
   \begin{align*}
      \ket{\Psi, S} &= \ket{\psi} + \ket{\phi} \\
      &= \ket{\psi_0} + \frac{1}{\sqrt{2}}\left( \ket{\psi_0} + \ket{\psi_1} \right) \\
      &= \left(\frac{\sqrt{2} + 1}{\sqrt{2}}\right) \ket{\psi_0} + \frac{1}{\sqrt{2}}\ket{\psi_1}
   \end{align*}

   Whereas, the (unnormalized) anti-symmetrized wavefunction is 
   \begin{align*}
      \ket{\Psi, A} &= \ket{\psi} - \ket{\phi} \\
      &= \left( \frac{1-\sqrt{2}}{\sqrt{2}} \right) \ket{\psi_0} - \frac{1}{\sqrt{2}} \ket{\psi_1}
   \end{align*}
   The states can then be appropriately normalized.

   \vskip 1cm
   \item Perhaps we could do an experiment like the Stern-Gerlach, which separates particles into different classes of spin (bosons and fermions have different spins.)
\end{enumerate}

\vskip 0.5cm
\hrule
\vskip 0.5cm


%%%%%%%%%%%%%%%%%%%%%%%%%%%%%%%%%%%%%%%%%
%%%%%%%%%%%%% Question 2
%%%%%%%%%%%%%%%%%%%%%%%%%%%%%%%%%%%%%%%%%
\underline{\textbf{Problem 2:}}

\begin{enumerate}
   \item \textbf{Write down the totally symmetric three-particle state with proper normalization:}
   
   A totally symmetric three-particle state is one where switching any two indices does not change the state. i.e.
   \[ \ket{\omega_1 \omega_2 \omega_3, S} = \ket{\omega_2 \omega_1 \omega_3, S} \]

   A general symmetric 3-particle state is then a linear combination of the six states obtained by swapping two indices at a time.
   \begin{align*}
      \ket{\omega_1 \omega_2 \omega_3, S} = A \left[ \ket{\omega_1 \omega_2 \omega_3} + \ket{\omega_1 \omega_3 \omega_2} + \ket{\omega_2 \omega_3 \omega_1} + \ket{\omega_2 \omega_1 \omega_3} + \ket{\omega_3 \omega_2 \omega_1} + \ket{\omega_3 \omega_1 \omega_2}  \right]
   \end{align*}
   where $A$ is the overall normalization factor.

   We can find $A$ using the normalization condition 
   \[ \inner{\omega_1 \omega_2 \omega_3, S}{\omega_1 \omega_2 \omega_3, S} = 1 \]

   That is,
   \begin{align*}
      1 = |A^2| \left(\text{A number of inner products} \right)
   \end{align*}

   Writing out each inner product would be a pain. We can skip that by recalling that only inner products of the form 
   \[ \inner{\omega_i \omega_j \omega_k}{\omega_i \omega_j \omega_k} = \delta{ii}\delta_{jj}\delta{kk} = 1\]
   will contribute, while the other inner products will vanish.

   Thus, we have 
   \begin{align}
      6 \cdot |A|^2 = 1
   \end{align}
   Thus,
   \[ \boxed{A = \frac{1}{\sqrt{6}}} \]
   
   \item Similarly, we can find the totally antisymmetric state as a linear combination of the states obtained by swapping two indices at a time (introducing a minus sign each time we swap!):
   
   \begin{align*}
      \ket{\omega_1 \omega_2 \omega_3, A} = A \left[ \ket{\omega_1 \omega_2 \omega_3} - \ket{\omega_1 \omega_3 \omega_2} + \ket{\omega_2 \omega_3 \omega_1} - \ket{\omega_2 \omega_1 \omega_3} + \ket{\omega_3 \omega_1 \omega_2} - \ket{\omega_3 \omega_2 \omega_1}  \right]
   \end{align*}

   Once again, we use the normalization condition,
   \[ \inner{\omega_1 \omega_2 \omega_3, A}{\omega_1 \omega_2 \omega_3, A} = 1 \]

   Again, the only inner products that survive are those in which the ordering is the exact same in the first and second argument. Additionally, since any state which has a $(-1)$ takes an inner product with itself, also having a factor of $(-1)$, the negative factors cancel each other and once again we arrive at 

   \[ 6 \cdot |A|^2 = 1 \]

   or 

   \[ \boxed{A = \frac{1}{\sqrt{6}}} \]

   \item The totally anti-symmetric state is a linear combination of states such that swapping any two indices introduces a negative sign. So, for example,
   \[ \ket{\omega_1 \omega_2 \omega_3, A} = - \ket{\omega_2 \omega_1 \omega_3, A} \]

   Or, more generally, we have 
   \[ \ket{\omega_i \omega_j \omega_k, A} = - \ket{\omega_j \omega_i \omega_k, A}  \]

   and so on.

   But this is exactly the kind of behavior which is encapsulated by the \textbf{Levi-Civita symbol} defined as 
   \[ \epsilon_{ijk} = \begin{cases}
      +1,\;\; \text{symmetric permutation of ijk}\\
      0,\;\; \text{otherwise}\\
      -1,\;\;\text{anti-symmetric permutation of ijk}
      \end{cases} \]

   Thus, the totally anti-symmetric three-particle state is 
   \[ \boxed{\ket{\omega_i \omega_j \omega_k} = \sum_{i \neq j \neq k} \frac{1}{\sqrt{6}}\epsilon_{ijk} \ket{\omega_i \omega_j \omega_k, A} } \]

\end{enumerate}

\vskip 0.5cm
\hrule
\vskip 0.5cm



%%%%%%%%%%%%%%%%%%%%%%%%%%%%%%%%%%%%%%%%%
%%%%%%%%%%%%% Question 3
%%%%%%%%%%%%%%%%%%%%%%%%%%%%%%%%%%%%%%%%%
\underline{\textbf{Problem 3:}}

\begin{enumerate}
   \item \textbf{Using the representations of the position and momentum operators, we want to verify the canonical commutation relations:}
   
   The position and momentum operators are represented as 
   \[ X_i \rightarrow x_i, \;\;\;\;  P_i \rightarrow -i\hbar\frac{\partial}{\partial x_i} \]

   where $i,j = 1,2,3$ correspond to the x,y,z directions.
   \vskip 1cm


   First off, the commutator between $X_i$ and $P_j$ is
   \begin{align*}
      [X_i, P_j] &= X_i P_j - P_j X_i 
   \end{align*}
   So, applying the commutator to some ket $\ket{\psi}$ gives us 
   \begin{align*}
      [X_i, P_j]\ket{\psi} &= (X_i P_j - P_j X_i ) \ket{\psi} \\
   \end{align*}
   and projecting this equation onto position space, we have
   \begin{align*}
      [X_i, P_j]\psi(x) &= (X_i P_j - P_j X_i ) \psi(x) \\
      &= x_i \cdot \left(-i\hbar \frac{\partial \psi(x)}{\partial x_j} \right) - \left(-i\hbar \frac{\partial}{\partial x_j} \right)(x_i\psi(x))
   \end{align*}
   We apply the product rule on the second term, giving us 
   \begin{align*}
      [X_i, P_j]\psi(x) &= (X_i P_j - P_j X_i ) \psi(x) \\
      &= -i\hbar x \frac{\partial \psi(x)}{\partial x_j}  - \left(-i\hbar \frac{\partial x_i}{\partial x_j} \psi(x) - i\hbar x_i \frac{\partial \psi(x)}{\partial x_j}  \right) \\
   \end{align*}
   The first and third terms cancel and we are left with the basis independent result
   \[ \boxed{[X_i, P_j] = i\hbar \delta_{ij}} \]

   The commutators of the different position operators amongst themselves, and the momentum operators amongst themselves, are simpler.

   Following the same idea, but writing it without reference to the basis in the interest of time, we have

   \begin{align*}
      [X_i, X_j] &= X_i X_j - X_j X_i \\
      &= x_i x_j - x_j x_i \\
      &= 0
   \end{align*}

   and 
   
   \begin{align*}
      [P_i, P_j] &= P_i P_j - P_j P_i \\
      &= \left(-i\hbar \frac{\partial}{\partial x_i}\right)\left(-i\hbar \frac{\partial}{\partial x_j}\right) - \left(-i\hbar \frac{\partial}{\partial x_j}\right)\left(-i\hbar \frac{\partial}{\partial x_i}\right) \\
      &= \left(-i\hbar \frac{\partial}{\partial x_i x_j}\right) - \left(-i\hbar \frac{\partial}{\partial x_j x_i}\right) \\
      &= 0 \;\;\text{For appropriate wavefunctions, according to Clairaut's Thm.}
   \end{align*}
   \vskip 1cm

   \item The Angular Momentum operator is defined to be 
   \[ L_i = \sum_{j,k} \epsilon_{ijk} X_j P_k \]

   Now, the commutators of angular momentum with position and momentum are

   \begin{align*}
      [L_i, X_l] &= \left[\epsilon_{ijk} X_j P_k,\; X_l\right] \\
      &= -\left[ X_, \epsilon_{ijk} X_j P_k\right]
   \end{align*}
   where we are using \textbf{einstein summation notation} and suppressing the sum.

   Then, we can use the following commutator identity:
   \[ [A, BC] = [A, B]C + B[A, C] \]

   Thus, we get 
   \begin{align*}
      \left[ X_l, \epsilon_{ijk} X_j P_k\right] &= \epsilon_{ijk}[X_l, X_j]P_k + \epsilon_{ijk} X_j[X_l, P_k] 
   \end{align*}
   Now, since $[X_l, X_j] = 0$ and $[X_l, P_k] = i\hbar \delta_{lk}$, we have
   \begin{align*}
      \left[ X_l, \epsilon_{ijk} X_j P_k\right] &= \epsilon_{ijk}i\hbar X_j \delta_{lk} \\
      &= i\hbar \epsilon_{ijl} X_j
   \end{align*}
   Therefore,
   \[ \boxed{[L_i, X_l] = -i\hbar \epsilon_{ijl} X_j} \]
   or, being explicit with the summation, 
   \[ \boxed{[L_i, X_l] = -i\hbar \sum_j \epsilon_{ijl} X_j} \]

   Carrying out the same procedure but with $[L_i, P_l]$, we find that 
   \[ \boxed{[L_i, X_l] = -i\hbar \sum_j \epsilon_{ijl} P_j}\]
   \vskip 1cm

   \item Using the above results, we want to find the commutators amongst the different angular momentum operators:
   
   We have (using different indices less prone to being mistaken for one another)
   \begin{align*}
      [L_i, L_j] &= L_i L_j - L_j L_i \\
      &= \epsilon{iab} \epsilon_{jcd} [X_a P_b, X_C, P_d] \\
      &= \epsilon_{iab}\epsilon_{jcd} \left( X_a [P_b, X_c]P_d + X_c[X_a, P_d]P_b\right)
   \end{align*}
   where the last equality follows from commutator identities.

   Then, using the commutator rules between the position and momentum operators, we can simplify this to be

   \begin{align*}
      [L_i, L_j] &= i\hbar \left( \epsilon_{iab} \epsilon_{bjd} + \epsilon_{dib} \epsilon_{bja}\right) X_a P_d 
   \end{align*}
   where we are summing over the indices other than $i, j$, so
   
   \begin{align*}
      [L_i, L_j] &= i\hbar \left( \delta_{ij}\delta_{ad} - \delta_{id}\delta_{aj} + \delta_{dj}\delta_{ia} - \delta_{da}\delta_{ij} \right) X_a P_d \\
      &= i\hbar (X_i P_j - X_j P_i) \\
      &= i\hbar \epsilon_{ijk} L_k
   \end{align*}
   \vskip 1cm

   \item The commutators $[L_z, r^2]$ and $[L_z, p^2]$ where $r^2 = X^2 + Y^2 + Z^2$ and $p^2 = P_x^2 + P_y^2 + P_z^2$ are each zero.
   \vskip 1cm

   \item 

\end{enumerate}

\vskip 0.5cm
\hrule
\vskip 0.5cm


\end{document}





