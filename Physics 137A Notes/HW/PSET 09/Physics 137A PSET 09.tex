%
% This is the LaTeX template file for lecture notes for CS294-8,
% Computational Biology for Computer Scientists.  When preparing 
% LaTeX notes for this class, please use this template.
%
% To familiarize yourself with this template, the body contains
% some examples of its use.  Look them over.  Then you can
% run LaTeX on this file.  After you have LaTeXed this file then
% you can look over the result either by printing it out with
% dvips or using xdvi.
%
% This template is based on the template for Prof. Sinclair's CS 270.

\documentclass[twoside]{article}
\usepackage{graphics}
\usepackage{mathtools}
\usepackage[]{mdframed}
\usepackage{amsmath}
\usepackage{amsfonts}
\usepackage[shortlabels]{enumitem}
\usepackage{bbm}

%\usepackage{asmfonts}
\setlength{\oddsidemargin}{0.25 in}
\setlength{\evensidemargin}{-0.25 in}
\setlength{\topmargin}{-0.6 in}
\setlength{\textwidth}{6.5 in}
\setlength{\textheight}{8.5 in}
\setlength{\headsep}{0.75 in}
\setlength{\parindent}{0 in}
\setlength{\parskip}{0.1 in}

%
% The following commands set up the lecnum (lecture number)
% counter and make various numbering schemes work relative
% to the lecture number.
%
\newcounter{lecnum}
\renewcommand{\thepage}{\thelecnum-\arabic{page}}
\renewcommand{\thesection}{\thelecnum.\arabic{section}}
\renewcommand{\theequation}{\thelecnum.\arabic{equation}}
\renewcommand{\thefigure}{\thelecnum.\arabic{figure}}
\renewcommand{\thetable}{\thelecnum.\arabic{table}}

%
% The following macro is used to generate the header.
%
\newcommand{\lecture}[4]{
   \pagestyle{myheadings}
   \thispagestyle{plain}
   \newpage
   \setcounter{lecnum}{#1}
   \setcounter{page}{1}
   \noindent
   \begin{center}
   \framebox{
      \vbox{\vspace{2mm}
    \hbox to 6.28in { {\bf Physics 137A: Quantum Mechanics
                        \hfill Fall 2023} }
       \vspace{4mm}
       \hbox to 6.28in { {\Large \hfill PSET #1, Due #2  \hfill} }
       \vspace{2mm}
       \hbox to 6.28in { {\it Lecturer: #3 \hfill #4} }
      \vspace{2mm}}
   }
   \end{center}
   \markboth{Lecture #1: #2}{Lecture #1: #2}
   {\bf Disclaimer}: {\it LaTeX template courtesy of the UC Berkeley EECS Department.}
   \vspace*{4mm}
}

%
% Convention for citations is authors' initials followed by the year.
% For example, to cite a paper by Leighton and Maggs you would type
% \cite{LM89}, and to cite a paper by Strassen you would type \cite{S69}.
% (To avoid bibliography problems, for now we redefine the \cite command.)
% Also commands that create a suitable format for the reference list.
\renewcommand{\cite}[1]{[#1]}
\def\beginrefs{\begin{list}%
        {[\arabic{equation}]}{\usecounter{equation}
         \setlength{\leftmargin}{2.0truecm}\setlength{\labelsep}{0.4truecm}%
         \setlength{\labelwidth}{1.6truecm}}}
\def\endrefs{\end{list}}
\def\bibentry#1{\item[\hbox{[#1]}]}

%Use this command for a figure; it puts a figure in wherever you want it.
%usage: \fig{NUMBER}{SPACE-IN-INCHES}{CAPTION}
\newcommand{\fig}[3]{
			\vspace{#2}
			\begin{center}
			Figure \thelecnum.#1:~#3
			\end{center}
	}
% Use these for theorems, lemmas, proofs, etc.
\newtheorem{theorem}{Theorem}[lecnum]
\newtheorem{lemma}[theorem]{Lemma}
\newtheorem{proposition}[theorem]{Proposition}
\newtheorem{claim}[theorem]{Claim}
\newtheorem{corollary}[theorem]{Corollary}
\newtheorem{definition}[theorem]{Definition}
\newenvironment{proof}{{\bf Proof:}}{\hfill\rule{2mm}{2mm}}

% **** IF YOU WANT TO DEFINE ADDITIONAL MACROS FOR YOURSELF, PUT THEM HERE:

\begin{document}
%FILL IN THE RIGHT INFO.
%\lecture{**LECTURE-NUMBER**}{**DATE**}{**LECTURER**}{**SCRIBE**}
%\footnotetext{These notes are partially based on those of Nigel Mansell.}


%%%%%%%%%%%%%%%%%%%%%%%%%%%%%%%%%%%%%%%%%%
%Additional commands

\newcommand{\ket}[1]{\mid#1\rangle}
\newcommand{\bra}[1]{\langle#1\mid}
\newcommand{\R}{\mathbb{R}}
\newcommand{\Prob}[1]{\mathbb{P}(#1)}
\newcommand{\mean}[1]{\left\langle #1 \right\rangle}
\newcommand{\inner}[2]{\left\langle #1 \bigg\rvert #2 \right\rangle}
\newcommand{\ham}{\hat{H}}
\newcommand{\mom}{\hat{P}}



%% Install amsfonts or amssymb package so that below command can be defined 
%\newcommand{\R}{\mathbb{R}

%%%%%%%%%%%%%%%%%%%%%%%%%%%%%%%%%%%%%%%%%%

% **** YOUR NOTES GO HERE:

%%%%%%%%%%%%%%%%%%%%%%%%%%%%%%%%%%%%%%%%%%
%%%%%         HOMEWORK 1
%%%%%%%%%%%%%%%%%%%%%%%%%%%%%%%%%%%%%%%%%%
\lecture{09}{November 15}{Chien-I Chiang}{Keshav Deoskar}


% Some general latex examples and examples making use of the
% macros follow.  
%**** IN GENERAL, BE BRIEF. LONG SCRIBE NOTES, NO MATTER HOW WELL WRITTEN,
%**** ARE NEVER READ BY ANYBODY.


%%%%%%%%%%%%%%%%%%%%%%%%%%%%%%%%%%%%%%%%%
%%%%%%%%%%%%% Question 1
%%%%%%%%%%%%%%%%%%%%%%%%%%%%%%%%%%%%%%%%%
\underline{\textbf{Problem 1:}}

We have a Quantum Harmonic Oscillator in the state
   \[ \ket{\Psi(t)} = c_1 e^{-i E_1 t / \hbar} \ket{1} + c_2 e^{-i E_2 t / \hbar} \ket{2} \]

   and we know that the expectation value of energy is 
   \[ \inner{\Psi(t)}{\hat{H}|\Psi(t)} = 2\hbar\omega \]

\begin{enumerate}
   \item First, we want to find the coefficients $c_1$ and $c_2$.
   To do this, let's think about $\inner{\Psi(t)}{\hat{H}|\Psi(t)} = 2\hbar\omega$. 

   This quantity is just the expectation value of the energy of the state, which can be expressed as 
   \begin{align*}
      \inner{\Psi(t)}{\hat{H}|\Psi(t)} &= \Prob{E_1}E_1 + \Prob{E_2}E_2 \\
                                       &= |c_1|^2 E_1 + |c_2|^2 E_2
   \end{align*}
   The energy of a state $\ket{n}$ in a QHO is given by 
   \[ E_n = \hbar\omega (n + \frac{1}{2}) \]

   So,
   \[  E_1 = \frac{3}{2} \hbar\omega \text{   and   } E_2 = \frac{5}{2} \hbar\omega \]

   Therefore,
   \begin{align*}
      \inner{\Psi(t)}{\hat{H}|\Psi(t)} &= \Prob{E_1}E_1 + \Prob{E_2}E_2 \\
                 \implies 2\hbar\omega &= |c_1|^2 \frac{3}{2} \hbar\omega+ |c_2|^2 \frac{5}{2} \hbar\omega \\
                 \implies 4 &= 3|c_1|^2 + 5|c_2|^2 
   \end{align*}
   But, we also know at 
   \begin{align*}
      &\inner{\Psi(t)}{\Psi(t)} = 1 \\
      \implies& |c_1|^2 + |c_2|^2 = 1
   \end{align*}

   So, we have a system of linear equations for $|c_1|^2$ and $|c_2|^2$. Solving the system of linear equations, we find
   \begin{align*}
      &|c_1|^2 = \frac{1}{2} \text{   and   } |c_2|^2 = \frac{1}{2} \\
      \implies& c_1 = \frac{1}{\sqrt{2}} \text{   and   } c_2 = \frac{1}{\sqrt{2}}
   \end{align*}
   where, by convention, we assume $c_1, c_2 \in \mathbb{R}$.

   Therefore, the state is 
   \[ \boxed{
      \ket{\Psi(t)} = \frac{1}{\sqrt{2}}e^{-i \frac{3}{2} \omega t} \ket{1} +  \frac{1}{\sqrt{2}} e^{-i \frac{5}{2} \omega t} \ket{2} 
   } \]

   \vskip 0.5cm

   \item Now, for $\mean{\hat{X}} = \inner{\Psi(t)}{\hat{X}|\Psi(t)}$, we want to show that 
   \[ \frac{d}{dt}\mean{\hat{X}} = -\omega^2 \mean{X} \] 

   Let's start by getting a more computationally useful expression for $\mean{\hat{X}}$.

   In a QHO, we have 
   \[ \hat{X} = \sqrt{\frac{\hbar}{2m\omega}}(\hat{a} + \hat{a}^{\dagger}) \]
   
   where $\hat{a}$ and $\hat{a}^{\dagger}$ are the lowering and raising operators respectively.

   To find $\mean{\hat{X}} = \inner{\Psi(t)}{\hat{X}\biggr\rvert\Psi(t)}$, we must first find $\mean{X(0)} = \inner{\Psi(0)}{\hat{X}\biggr\rvert\Psi(0)}$

   This is given by 
   \begin{align*}
      \mean{X(0)} &= \frac{1}{2} \left( \bra{1} + \bra{2} \right) | \hat{X} | \left( \ket{1} + \ket{2} \right) \\
      &= \frac{1}{2} \left( \bra{1} + \bra{2} \right)|\sqrt{\frac{\hbar}{2m\omega}} (\hat{a}^{\dagger} + \hat{a}) | \left( \ket{1} + \ket{2} \right) \\
      &= \frac{1}{2} \sqrt{\frac{\hbar}{2m\omega}} \left( \bra{1} + \bra{2} \right)| (\hat{a}^{\dagger} + \hat{a}) | \left( \ket{1} + \ket{2} \right) \\
      &= \frac{1}{2} \sqrt{\frac{\hbar}{2m\omega}} \left( \bra{1} + \bra{2} \right)| (\hat{a}^{\dagger} + \hat{a}) | \left( \ket{1} + \ket{2} \right) 
   \end{align*}
   Raising $\ket{1}$ to $\ket{2}$ but taking its inner product with $\ket{1}$ will just return zero, and vice versa. So, writing only the non-zero terms, we have
   \begin{align*}
      \mean{X(0)} &= \frac{1}{2} \sqrt{\frac{\hbar}{2m\omega}} \left( \inner{1}{\hat{a}|2} + \inner{2}{\hat{a}^{\dagger}|1} \right) \\
      &= \frac{1}{2} \sqrt{\frac{\hbar}{2m\omega}} \left( 2 
      \cdot \inner{1}{1} + 2 \cdot \inner{2}{2} \right) \\
      &= \frac{1}{2} \sqrt{\frac{\hbar}{2m\omega}} \left( 2 + 2 \right)
   \end{align*}

   So, 
   \[ \boxed{\mean{X(0)} = 2\sqrt{\frac{\hbar}{2m\omega}}} \]


   Now, 
   \begin{align*}
      \mean{X(t)} &= 2\sqrt{\frac{\hbar}{2m\omega}} \cdot \left( \bra{1}e^{i\frac{3}{2}\omega t} + \bra{2}e^{i\frac{5}{2}\omega t}\right)|(\hat{a} + \hat{a}^{\dagger})|\left( e^{-i\frac{3}{2}\omega t} \ket{1} + e^{-i\frac{5}{2}\omega t} \ket{2} \right) \\
   \end{align*}
   Again, the only terms which are non-zero are those where the ket is raised/lowered to match the bra. So,
   \begin{align*}
      \mean{X(t)} &= 2\sqrt{\frac{\hbar}{2m\omega}} \left( \bra{1 e^{i\frac{3}{2}\omega t}} \hat{a}\ket{ e^{-i\frac{5}{2}\omega t} 2} + \bra{2 e^{i\frac{5}{2}\omega t}}\hat{a}^{\dagger}\ket{e^{-i\frac{3}{2}\omega t} 1} \right) \\
      &= 2\sqrt{\frac{\hbar}{2m\omega}} \cdot 2 \left( e^{-i\omega t} + e^{ i \omega t} \right) \\
      &= 8\sqrt{\frac{\hbar}{2m\omega}} \cos(\omega t)
   \end{align*}

   Now that we've found $\mean{X(t)}$ we can simply differentiate twice and verify that 
   \[ \boxed{\frac{d}{dt}\mean{\hat{X}} = -\omega^2 \mean{X}} \] 
\end{enumerate}

\vskip 0.5cm
\hrule
\vskip 0.5cm

%%%%%%%%%%%%%%%%%%%%%%%%%%%%%%%%%%%%%%%%%
%%%%%%%%%%%%% Question 2
%%%%%%%%%%%%%%%%%%%%%%%%%%%%%%%%%%%%%%%%%

\underline{\textbf{Problem 2:}}

In this problem, we consider several properties of the quantum harmonic oscillator:

\begin{enumerate}
   \item The annihilation and creation operators, $\hat{a}$ and $\hat{a}^{\dagger}$ are defined as 
   
   \begin{align*}
      \hat{a} &= \left(\frac{m\omega}{2\hbar}\right)^{1/2} \hat{X} + i\left( \frac{1}{2m\omega\hbar} \right)^{1/2} \hat{P} \\
      &\text{and} \\
      \hat{a}^{\dagger} &= \left(\frac{m\omega}{2\hbar}\right)^{1/2} \hat{X} - i\left( \frac{1}{2m\omega\hbar} \right)^{1/2} \hat{P}
   \end{align*}

   So, we can express the position and momentum operators in terms of the annihilation and creation operators as 
   \begin{align*}
      \hat{X} &= \frac{1}{2} \left(\frac{m\omega}{2\hbar} \right)^{-1/2} \left( \hat{a} + \hat{a}^{\dagger} \right) = \sqrt{\frac{\hbar}{2 m \omega}} \left( \hat{a} + \hat{a}^{\dagger} \right) \\
      &\text{and} \\
      \hat{P} &= \frac{1}{2i} \left(\frac{1}{2m\omega\hbar} \right)^{-1/2} \left( \hat{a} - \hat{a}^{\dagger} \right) = -i \sqrt{\frac{m\omega\hbar}{2}} \left( \hat{a} - \hat{a}^{\dagger} \right)
   \end{align*}
   
   \item Now we want to find the expectation values $\mean{\hat{X}}$, $\mean{\hat{X^2}}$, and $\mean{\hat{V}}$ for the $n^{th}$ energy eigenstate $\ket{n}$, where $V(\hat{X}) = \frac{1}{2}m\omega^2 \hat{X}^2$ is the potential energy.
   
   \begin{align*}
      \mean{X} = \inner{n}{\hat{X}|n} &= \sqrt{\frac{\hbar}{2m\omega}} \inner{n}{(\hat{a} + \hat{a}^{\dagger})|n} \\
      &= \sqrt{\frac{\hbar}{2m\omega}} \left[ \inner{n}{\hat{a}|n} + \inner{n}{\hat{a}^{\dagger}|n} \right] \\
      &= \sqrt{\frac{\hbar}{2m\omega}} \left[ \sqrt{n}\inner{n}{n-1} + \sqrt{n+1}\inner{n}{n+1} \right] \\
      &= \sqrt{\frac{\hbar}{2m\omega}} \left[ \sqrt{n}\delta_{n, n-1} + \sqrt{n+1}\delta_{n, n+1} \right] \\
      &= 0
   \end{align*}

   So, 
   \[ \boxed{ \mean{\hat{X}} = 0 } \]

   % Similarly, we have 
   % \begin{align*}
   %    \mean{\hat{P}} = \inner{n}{\hat{P}|n} &= i\sqrt{\frac{m\omega\hbar}{2}} \inner{n}{\hat{a}^{\dagger} - \hat{a}|n} \\
   %    &= i\sqrt{\frac{m\omega\hbar}{2}} \left[ \sqrt{n+1}\delta_{n, n+1} - \sqrt{n}\delta_{n, n-1} \right] \\
   %    &= 0 
   % \end{align*}

   For $\mean{\hat{X^2}}$, we have
   \begin{align*}
      \hat{X^2} &= \hat{X} \cdot \hat{X} \\
      &= \frac{\hbar}{2m\omega}\left[ \hat{a}\hat{a} + \hat{a}\hat{a}^{\dagger} + \hat{a}^{\dagger}\hat{a} + \hat{a}^{\dagger}\hat{a}^{\dagger} \right]
   \end{align*}

   So, the expected value is calculated as 
   \[ \mean{\hat{X^2}} = \frac{\hbar}{2m\omega} \left[ \inner{n}{ \hat{a}\hat{a} |n } + \inner{n}{ \hat{a}\hat{a}^{\dagger} |n } + \inner{n}{ \hat{a}^{\dagger}\hat{a}|n } + \inner{n}{\hat{a}^{\dagger}\hat{a}^{\dagger} |n } \right]  \]

   But we notice that applying $\hat{a}\hat{a}$ on $\ket{n}$ will give us $\sqrt{n \cdot (n-1)}\ket{n-2}$ and due to the orthogonality of the different energy states, we have $\inner{n}{n-2} = 0$ so the entire term is zero. 

   The same argument applies for $\hat{a}^{\dagger}\hat{a}^{\dagger}$, since that gives us $\sqrt{(n+1)(n+2)} \ket{n+2}$. 

   So, the only non-zero terms are the cross terms.


   Now, 
   \[ \inner{n}{\hat{a}\hat{a}^{\dagger}|n} = \inner{\hat{a}^{\dagger}n}{\hat{a}^{\dagger}n} = (\sqrt{n})^{*} (\sqrt{n}) \cdot \inner{n-1}{n-1} = n \]

   and similarly, 
   \[ \inner{n}{\hat{a}^{\dagger}\hat{a}|n} = \inner{\hat{a} n}{\hat{a}n} = (\sqrt{n+1})^{*} (\sqrt{n+1}) \cdot \inner{n+1}{n+1} = n+1 \]

   So, plugging these in, 
   \begin{align*}
      \mean{\hat{X^2}} &= \frac{\hbar}{2m\omega} \left[ n + n+1 \right] \\
      &= \frac{\hbar}{2m\omega} \cdot (2n+1)
   \end{align*}
   Thus,
   \[ \boxed{\mean{\hat{X^2}} = \frac{\hbar}{m\omega}  \cdot \left(n + \frac{1}{2}\right)} \]
   \vskip 1cm

   Lastly, we want to find the expectation value $\mean{V}$ where $V(\hat{X}) = \frac{1}{2}m \omega^2 \hat{X}^2$. Since we are just multiplying $\hat{X^2}$ by a constant, we can immediately find the mean value to be 

   \[ \boxed{ \mean{V} = \frac{\hbar\omega}{2}\cdot \left( n + \frac{1}{2} \right)  } \]
   \vskip 1cm

   \item In this part, we want to find the expectation values of $\mean{\hat{P}}$, $\mean{\hat{P^2}}$, and $\mean{T}$ where $T = \frac{\hat{P^2}}{2m}$ is the kinetic energy, for the $n^{th}$ energy eigenstates.

   \begin{align*}
      \mean{\hat{P}} = \inner{n}{\hat{P}|n} &= i\sqrt{\frac{m\omega\hbar}{2}} \inner{n}{\hat{a}^{\dagger} - \hat{a}|n} \\
      &= i\sqrt{\frac{m\omega\hbar}{2}} \left[ \sqrt{n+1}\delta_{n, n+1} - \sqrt{n}\delta_{n, n-1} \right] \\
      &= 0 
   \end{align*}
   So, 
   \[ \boxed{ \mean{\hat{P}} = 0 } \]
   \vskip 1cm

   We can express $\hat{P^2}$ as
   \begin{align*}
      \hat{P^2} &= \hat{P} \cdot \hat{P} \\
      &= \left( i\sqrt{\frac{m\omega\hbar}{2}}(\hat{a}^{\dagger} - \hat{a}) \right) \cdot \left( i\sqrt{\frac{m\omega\hbar}{2}}(\hat{a}^{\dagger} - \hat{a}) \right) \\
      &= (-1) \cdot (\frac{m\omega\hbar}{2}) \left[ \hat{a}^{\dagger}\hat{a}^{\dagger} - \hat{a}^{\dagger}\hat{a} - \hat{a}\hat{a}^{\dagger} + \hat{a}\hat{a} \right]
   \end{align*}
   
   So, the expectation value is 
   \begin{align*}
      \mean{\hat{P^2}} &= \frac{-m\omega\hbar}{2} \left[ \inner{n}{ \hat{a}^{\dagger}\hat{a}^{\dagger} |n } - \inner{n}{  \hat{a}^{\dagger}\hat{a} |n } - \inner{n}{ \hat{a}\hat{a}^{\dagger} |n } + \inner{n}{\hat{a}\hat{a}|n } \right] 
   \end{align*}

   Once again, the only contributing terms are the cross terms, so we find
   \begin{align*}
      \mean{\hat{P^2}} &= \frac{-m\omega\hbar}{2}\cdot\left(- \left[ (n+1) + n \right] \right) \\
      &= \frac{m\omega\hbar}{2} \cdot \left( 2n + 1 \right)
   \end{align*}
   So, 
   \[ \boxed{\mean{\hat{P^2}} = m\omega\hbar \cdot \left( n + \frac{1}{2} \right) } \]

   And to obtain the kinetic Energy, we just divide by $2m$, so again, we can directly find $\mean{K}$ to be 

   \[ \boxed{\mean{K} = \frac{\hbar\omega}{2} \cdot \left( n + \frac{1}{2} \right)} \]

   So, the relation between the expected Kinetic Energy and Potental is 
   \[ \boxed{\mean{\hat{K}} =  \mean{\hat{V}} }\]

   The expected values for Kinetic and Potential Energy are the same!
   \vskip 1cm

   \item Our state is a generic combination of the $0^{th}$ and $1^{st}$ states:
   \[ \ket{\psi} = a\ket{0} + be^{i\phi}\ket{1} \]
   where $a$, $b$, $\phi$ are real and $a^2 + b^2 = 1$.

   Using the results from parts (b) and (c) of this question, the expected values of kinetic and potential energy are
   \begin{align*}
      \mean{K} &= |c_0|^2 K_0 + |c_1|^2 K_1 \\
      &= a^2 \cdot \frac{\hbar\omega}{2} \left(0 + \frac{1}{2}\right) + |b \cdot e^{-i\phi}|^2 \frac{\hbar\omega}{2} \left(1 + \frac{1}{2}\right) \\
      &= \frac{\hbar\omega}{2} \left( \frac{a^2}{2} + \frac{3b^2}{2} \right)
   \end{align*}

   Therefore,
   \begin{align*}
      &\mean{V} = \mean{K} = \frac{\hbar}{2}\left( \frac{a^2}{2} + \frac{3b^2} {2} \right) \\
      \implies&\mean{V} = \mean{K} = \frac{\hbar}{4} (a^2 + 3b^2)
   \end{align*}

   No, \underline{the result will not change} if we consider the time evolution of the state because the coefficients $c_0^{'} = c_0 e^{-iEt/ \hbar}$ and $c_1^{'} = c_1 e^{-iEt/ \hbar}$ will still have the same squared magnitudes 
   \[ |c_0^{'}|^2 = |c_0|^2\text{   and   }|c_1^{'}|^2 = |c_1|^2 \]
   because the complex exponential $e^{-iEt/\hbar}$ has modulus one.
\end{enumerate}

\vskip 0.5cm
\hrule
\vskip 0.5cm

%%%%%%%%%%%%%%%%%%%%%%%%%%%%%%%%%%%%%%%%%
%%%%%%%%%%%%% Question 3
%%%%%%%%%%%%%%%%%%%%%%%%%%%%%%%%%%%%%%%%%

\underline{\textbf{Problem 3:}}

In this problem, we find the relation between $\mean{V}$ and $\mean{T}$ using another method:

\begin{enumerate}
   \item Recalling that the hamiltonian for a quantum harmonic oscillator reads as
   \[ \hat{H} = \frac{\hat{P^2}}{2m} + \frac{1}{2} m\omega^2 \hat{X^2} \]

   Let us first calculate the commutator $\left[\hat{H}, \hat{P}\hat{X} \right]$.

   We can express each of these operators in terms of the annihilation and creation operators:

   \[ \hat{X} = \sqrt{\frac{\hbar}{2 m \omega}} \left( \hat{a}^{\dagger} + \hat{a} \right)\] 

   \[ \hat{P} = i \sqrt{\frac{m\omega\hbar}{2}} \left( \hat{a}^{\dagger} - \hat{a} \right) \]

   \[ \hat{H} = \hbar\omega\left(\hat{a}^{\dagger}\hat{a} + \frac{1}{2}\right)\]

   Now, the product $\hat{P}\hat{X}$ is 
   \begin{align*}
      \hat{P}\hat{X} &= i \sqrt{\frac{m\omega\hbar}{2}} \left( \hat{a}^{\dagger} - \hat{a} \right) \cdot \sqrt{\frac{\hbar}{2 m \omega}} \left( \hat{a}^{\dagger} + \hat{a} \right) \\
      &= \frac{i\hbar}{2} \left( \hat{a}^{\dagger} - \hat{a} \right) \left( \hat{a}^{\dagger} + \hat{a} \right) \\
      &= \frac{i\hbar}{2} \left( \hat{a}^{\dagger}\hat{a}^{\dagger} + \hat{a}^{\dagger}\hat{a} - \hat{a}\hat{a}^{\dagger} + \hat{a}\hat{a}\right)
   \end{align*}

   But recall that $[\hat{a}^{\dagger}, \hat{a}] = \hat{a}^{\dagger}\hat{a} - \hat{a}\hat{a}^{\dagger} = 1$, So

   \begin{align*}
      \hat{P}\hat{X} &= \frac{i\hbar}{2} \left( \hat{a}^{\dagger}\hat{a}^{\dagger} + \hat{a}\hat{a} + 1 \right)
   \end{align*}


   % So, the products $\hat{H}(\hat{P}\hat{X})$ and $(\hat{P}\hat{X})\hat{H}$ are 

   % \begin{align*}
   %    \hat{H}(\hat{P}\hat{X}) &= \hbar\omega\left(\hat{a}^{\dagger}\hat{a} + \frac{1}{2}\right) \cdot \frac{i\hbar}{2} \left( \hat{a}^{\dagger}\hat{a}^{\dagger} + \hat{a}\hat{a} + 1 \right) \\
   %    &= 
   % \end{align*}

   We can now calucate the commutator as 
   \begin{align*}
      \left[\hat{H}, \hat{P}\hat{X} \right] &= \hat{H}\left( \hat{P}\hat{X} \right) - \left( \hat{P}\hat{X} \right)\hat{H} 
   \end{align*} 

   Another way we can calculate the commutator is 
   \begin{align*}
      \left[ \hat{H}, \hat{P}\hat{X} \right] &= \left[ \hat{H}, \hat{P}\right]\hat{X} + \hat{P}\left[ \hat{H}, \hat{X}  \right]
   \end{align*}

   \vskip 1cm

   We'll not try to evaluate each of the commutators on the RHS, using the following properties:
   
   \begin{align*}
      \left[ \hat{a}^{\dagger}, \hat{H} \right] &= \left[ \hat{a}^{\dagger}, \hat{H} \right] \\
      &= \left[ \hat{a}^{\dagger}, \hbar\omega\left(\hat{a}^{\dagger}\hat{a} + \frac{1}{2}\right) \right] \\
      &= \left[ \hat{a}^{\dagger}, \hbar\omega\left(\hat{a}^{\dagger}\hat{a} \right) \right] \\
      &= \hbar\omega\left[ \hat{a}^{\dagger}, \hat{a}^{\dagger}\hat{a} \right] \\
      &= \hbar\omega (\left[\hat{a}^{\dagger}, \hat{a}^{\dagger} \right]\hat{a} + \hat{a}^{\dagger}\left[ \hat{a}^{\dagger}, \hat{a}\right]) \\
      &= \hbar\omega (0 \cdot \hat{a} - \hat{a}^{\dagger}\left[ \hat{a}, \hat{a}^{\dagger}\right]) \\
      &= -\hbar\omega \hat{a}^{\dagger}
   \end{align*}
   So, in conclusion, 
   \[ \boxed{\left[ \hat{a}^{\dagger}, \hat{H} \right] = -\hbar\omega \hat{a}^{\dagger}} \]

   Doing a similar calculation for $\left[ \hat{a}, \hat{H} \right]$, we find that 
   \[ \boxed{\left[ \hat{a}, \hat{H} \right] = \hbar\omega \hat{a}} \]
   \vskip 2cm

   \begin{enumerate}
      \item Okay now let's actually compute the commutators. First, the one with momentum and the Hamiltonian:
      \begin{align*}
         \left[ \hat{H}, \hat{P} \right] &= -\left[\hat{P}, \hat{H} \right] \\
         &= -\left[i \sqrt{\frac{m\omega\hbar}{2}} \left( \hat{a}^{\dagger} - \hat{a} \right), \hat{H} \right] \\
         &= -i \sqrt{\frac{m\omega\hbar}{2}} \left[\left( \hat{a}^{\dagger} - \hat{a} \right), \hat{H} \right] \\
         &= -i \sqrt{\frac{m\omega\hbar}{2}} ( \left[ \hat{a}^{\dagger}, \hat{H} \right] - \left[ \hat{a}, \hat{H} \right] ) \\
         &= -i\sqrt{\frac{m\omega\hbar}{2}} \left( -\hbar\omega \hat{a}^{\dagger} - \hbar\omega \hat{a} \right) \\
         &= (i\hbar\omega) \sqrt{\frac{m\omega\hbar}{2}} \left(\hat{a}^{\dagger} + \hat{a} \right) \\
         &= \sqrt{\frac{\hbar}{2m\omega}} \left(\hat{a}^{\dagger} + \hat{a} \right) \cdot \sqrt{\frac{2m\omega}{\hbar}} \cdot (i\hbar\omega)\sqrt{\frac{m\omega\hbar}{2}} \\
         &= \left( \sqrt{\frac{\hbar}{2m\omega}} \left(\hat{a}^{\dagger} + \hat{a} \right) \right) \cdot i\hbar\omega m\\
         &= i\hbar \omega m \hat{X}
      \end{align*}

      So, 
      \[ \boxed{\left[ \hat{H}, \hat{P} \right] = i\hbar m^2 \omega^2 \hat{X}  } \]
      \vskip 1cm

      \item Next, the commutator between the position operator and the Hamiltonian:
      \begin{align*}
         \left[ \hat{H}, \hat{X} \right] &= - \left[ \hat{X}, \hat{H} \right] \\
         &= -\left[ \sqrt{\frac{\hbar}{2 m \omega}} \left( \hat{a}^{\dagger} + \hat{a} \right) , \hat{H} \right] \\
         &= -\sqrt{\frac{\hbar}{2 m \omega}} \left[ \left( \hat{a}^{\dagger} + \hat{a} \right) , \hat{H} \right] \\
         &= -\sqrt{\frac{\hbar}{2 m \omega}} \left( \left[ \hat{a}^{\dagger}, \hat{H} \right] + \left[ \hat{a}, \hat{H} \right] \right) \\
         &= -\sqrt{\frac{\hbar}{2m\omega}} \left(- \hbar\omega \hat{a}^{\dagger} + \hbar\omega \hat{a} \right) \\ 
         &= \sqrt{\frac{\hbar}{2m\omega}} \cdot \hbar\omega \left( \hat{a}^{\dagger} - \hat{a} \right) \\
         &= \left( \sqrt{\frac{\hbar}{2m\omega}} \cdot \hbar\omega \right) \cdot \left(\frac{1}{i} \sqrt{\frac{2}{m\omega\hbar}}\right) \cdot \left(i\sqrt{\frac{m\omega\hbar}{2}}\right) \left( \hat{a}^{\dagger} - \hat{a} \right) \\
         &= -i \frac{\hbar\omega}{m\omega} \hat{P}
      \end{align*}

      So, 
      \[ \boxed{\left[ \hat{H}, \hat{X} \right] = -i \frac{\hbar}{m} \hat{P}} \]
   \end{enumerate}

   So, finally, let's tackle the original commutator we were trying to evaluate: $\left[ \hat{H}, \hat{P}\hat{X} \right]$

   We have 
   \begin{align*}
      \left[ \hat{H}, \hat{P}\hat{X} \right] &= \left[ \hat{H}, \hat{P} \right] \hat{X} + \hat{P} \left[ \hat{H}, \hat{X} \right] \\
      &= \left(i \hbar\omega m \hat{X}\right) \hat{X} + \hat{P} \left( \frac{-i\hbar}{m} \hat{P} \right) \\
      &= i\hbar\omega m \hat{X^2} - \frac{i\hbar}{m} \hat{P^2} \\
      &= \frac{2 i \hbar}{\omega} \cdot \left( \frac{1}{2} m \omega^2 \hat{X^2} - \frac{1}{2m} \hat{P^2} \right)
   \end{align*}

   \[ \boxed{\left[ \hat{H}, \hat{P}\hat{X} \right] = \frac{2 i \hbar}{\omega} \cdot \left( \frac{1}{2} m \omega^2 \hat{X^2} - \frac{1}{2m} \hat{P^2} \right)} \]
   \vskip 1cm

   \item We found one expression for the commutator $\left[ \hat{H}, \hat{P}\hat{X} \right]$ in the previous section, but another way to find it is as 
   
   \[ \left[ \hat{H}, \hat{P}\hat{X} \right] = \hat{H} \left( \hat{P}\hat{X}\right) - \left( \hat{P}\hat{X}\right) \hat{H} \]

   and we found earlier that $\hat{P}\hat{X} = \frac{i\hbar}{2} \left( \hat{a}^{\dagger}\hat{a}^{\dagger}  + \hat{a}\hat{a} + 1 \right)$

   So, 
   \begin{align*}
      \left[ \hat{H}, \hat{P}\hat{X} \right] &= \left[ \hat{H}, \frac{i\hbar}{2} \left( \hat{a}^{\dagger}\hat{a}^{\dagger}  + \hat{a}\hat{a} + 1 \right) \right] \\
      &= \frac{i\hbar}{2}  \left[ \hat{H}, \left( \hat{a}^{\dagger}\hat{a}^{\dagger}  + \hat{a}\hat{a} + 1 \right) \right]  \\
      &= \frac{i\hbar}{2}  \left[ \hat{H}, \left( \hat{a}^{\dagger}\hat{a}^{\dagger}  + \hat{a}\hat{a} \right) \right] \\
      &= \frac{i\hbar}{2}  \left[ \hat{H}, \hat{a}^{\dagger}\hat{a}^{\dagger} \right] + \frac{i\hbar}{2}  \left[ \hat{H}, \hat{a}\hat{a} \right] 
   \end{align*}

   Let's look at each of these commutators separately:

   \begin{enumerate}
      \item For the first one, 
      \begin{align*}
         \left[ \hat{H}, \hat{a}^{\dagger}\hat{a}^{\dagger} \right] &= \left[ \hat{H}, \hat{a}^{\dagger} \right]\hat{a}^{\dagger} + \hat{a}^{\dagger}\left[ \hat{H}, \hat{a}^{\dagger}\right] \\
         &= -\left[\hat{a}^{\dagger}, \hat{H}  \right]\hat{a}^{\dagger} - \hat{a}^{\dagger}\left[ \hat{a}^{\dagger}, \hat{H} \right] \\
         &= (\hbar\omega \hat{a}^{\dagger}) \hat{a}^{\dagger} - \left(\hat{a}^{\dagger}(\hbar\omega \hat{a}^{\dagger}) \right) \\
         &= \hbar\omega \hat{a}^{\dagger}\hat{a}^{\dagger} - \hbar\omega \hat{a}^{\dagger}\hat{a}^{\dagger} \\
         &= 0
      \end{align*}

      \item For the second one, 
      \begin{align*}
         \left[ \hat{H}, \hat{a}\hat{a} \right] &= \left[ \hat{H}, \hat{a} \right]\hat{a} + \hat{a}\left[ \hat{H}, \hat{a}\right] \\
         &= -\left[\hat{a}, \hat{H}  \right]\hat{a} - \hat{a}\left[ \hat{a}, \hat{H} \right] \\
         &= -(\hbar\omega \hat{a}) \hat{a} - \left(-\hat{a}(\hbar\omega \hat{a}) \right) \\
         &= -\hbar\omega \hat{a}\hat{a} + \hbar\omega \hat{a}\hat{a} \\
         &= 0
      \end{align*}
   \end{enumerate}

   So, we find that $\left[ \hat{H}, \hat{P}\hat{X} \right] = 0$!

   This combined with the result from the first part of the question tells us that, for a quantum harmonic oscillator, we have 
   \[ \frac{2 i \hbar}{\omega} \cdot \left( \frac{1}{2} m \omega^2 \hat{X^2} - \frac{1}{2m} \hat{P^2} \right) = 0 \]

   Which means, 
   \[ \left( \frac{1}{2} m \omega^2 \hat{X^2} - \frac{1}{2m} \hat{P^2} \right) = 0\]

   So, 
   \[ \boxed{\frac{1}{2m} \hat{P^2} = \frac{1}{2} m \omega^2 \hat{X^2}} \]

   Thus, we expect the Kinetic and Potential Energies to be the same!
\end{enumerate}

\vskip 0.5cm
\hrule
\vskip 0.5cm

%\section*{References}
%\beginrefs
%\bibentry{AGM97}{\sc N.~Alon}, {\sc Z.~Galil} and {\sc O.~Margalit},
%On the Exponent of the All Pairs Shortest Path Problem,
%{\it Journal of Computer and System Sciences\/}~{\bf 54} (1997),
%pp.~255--262.

%\bibentry{F76}{\sc M. L. ~Fredman}, New Bounds on the Complexity of the 
%Shortest Path Problem, {\it SIAM Journal on Computing\/}~{\bf 5} (1976), 
%pp.~83-89.
%\endrefs



\end{document}





