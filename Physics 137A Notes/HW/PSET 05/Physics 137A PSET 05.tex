%
% This is the LaTeX template file for lecture notes for CS294-8,
% Computational Biology for Computer Scientists.  When preparing 
% LaTeX notes for this class, please use this template.
%
% To familiarize yourself with this template, the body contains
% some examples of its use.  Look them over.  Then you can
% run LaTeX on this file.  After you have LaTeXed this file then
% you can look over the result either by printing it out with
% dvips or using xdvi.
%
% This template is based on the template for Prof. Sinclair's CS 270.

\documentclass[twoside]{article}
\usepackage{graphics}
\usepackage{mathtools}
\usepackage[]{mdframed}
\usepackage{amsmath}
\usepackage{amsfonts}
\usepackage{enumitem}
\usepackage{bbm}

%\usepackage{asmfonts}
\setlength{\oddsidemargin}{0.25 in}
\setlength{\evensidemargin}{-0.25 in}
\setlength{\topmargin}{-0.6 in}
\setlength{\textwidth}{6.5 in}
\setlength{\textheight}{8.5 in}
\setlength{\headsep}{0.75 in}
\setlength{\parindent}{0 in}
\setlength{\parskip}{0.1 in}

%
% The following commands set up the lecnum (lecture number)
% counter and make various numbering schemes work relative
% to the lecture number.
%
\newcounter{lecnum}
\renewcommand{\thepage}{\thelecnum-\arabic{page}}
\renewcommand{\thesection}{\thelecnum.\arabic{section}}
\renewcommand{\theequation}{\thelecnum.\arabic{equation}}
\renewcommand{\thefigure}{\thelecnum.\arabic{figure}}
\renewcommand{\thetable}{\thelecnum.\arabic{table}}

%
% The following macro is used to generate the header.
%
\newcommand{\lecture}[4]{
   \pagestyle{myheadings}
   \thispagestyle{plain}
   \newpage
   \setcounter{lecnum}{#1}
   \setcounter{page}{1}
   \noindent
   \begin{center}
   \framebox{
      \vbox{\vspace{2mm}
    \hbox to 6.28in { {\bf Physics 137A: Quantum Mechanics
                        \hfill Fall 2023} }
       \vspace{4mm}
       \hbox to 6.28in { {\Large \hfill PSET #1, Due #2  \hfill} }
       \vspace{2mm}
       \hbox to 6.28in { {\it Lecturer: #3 \hfill #4} }
      \vspace{2mm}}
   }
   \end{center}
   \markboth{Lecture #1: #2}{Lecture #1: #2}
   {\bf Disclaimer}: {\it LaTeX template courtesy of the UC Berkeley EECS Department.}
   \vspace*{4mm}
}

%
% Convention for citations is authors' initials followed by the year.
% For example, to cite a paper by Leighton and Maggs you would type
% \cite{LM89}, and to cite a paper by Strassen you would type \cite{S69}.
% (To avoid bibliography problems, for now we redefine the \cite command.)
% Also commands that create a suitable format for the reference list.
\renewcommand{\cite}[1]{[#1]}
\def\beginrefs{\begin{list}%
        {[\arabic{equation}]}{\usecounter{equation}
         \setlength{\leftmargin}{2.0truecm}\setlength{\labelsep}{0.4truecm}%
         \setlength{\labelwidth}{1.6truecm}}}
\def\endrefs{\end{list}}
\def\bibentry#1{\item[\hbox{[#1]}]}

%Use this command for a figure; it puts a figure in wherever you want it.
%usage: \fig{NUMBER}{SPACE-IN-INCHES}{CAPTION}
\newcommand{\fig}[3]{
			\vspace{#2}
			\begin{center}
			Figure \thelecnum.#1:~#3
			\end{center}
	}
% Use these for theorems, lemmas, proofs, etc.
\newtheorem{theorem}{Theorem}[lecnum]
\newtheorem{lemma}[theorem]{Lemma}
\newtheorem{proposition}[theorem]{Proposition}
\newtheorem{claim}[theorem]{Claim}
\newtheorem{corollary}[theorem]{Corollary}
\newtheorem{definition}[theorem]{Definition}
\newenvironment{proof}{{\bf Proof:}}{\hfill\rule{2mm}{2mm}}

% **** IF YOU WANT TO DEFINE ADDITIONAL MACROS FOR YOURSELF, PUT THEM HERE:

\begin{document}
%FILL IN THE RIGHT INFO.
%\lecture{**LECTURE-NUMBER**}{**DATE**}{**LECTURER**}{**SCRIBE**}
%\footnotetext{These notes are partially based on those of Nigel Mansell.}


%%%%%%%%%%%%%%%%%%%%%%%%%%%%%%%%%%%%%%%%%%
%Additional commands

\newcommand{\ket}[1]{\mid#1\rangle}
\newcommand{\bra}[1]{\langle#1\mid}
\newcommand{\R}{\mathbb{R}}
\newcommand{\Prob}[1]{\mathbb{P}(#1)}
\newcommand{\mean}[1]{\langle #1 \rangle}
\newcommand{\inner}[2]{\langle #1 | #2 \rangle}
\newcommand{\ham}{\hat{H}}
\newcommand{\mom}{\hat{P}}



%% Install amsfonts or amssymb package so that below command can be defined 
%\newcommand{\R}{\mathbb{R}

%%%%%%%%%%%%%%%%%%%%%%%%%%%%%%%%%%%%%%%%%%

% **** YOUR NOTES GO HERE:

%%%%%%%%%%%%%%%%%%%%%%%%%%%%%%%%%%%%%%%%%%
%%%%%         HOMEWORK 1
%%%%%%%%%%%%%%%%%%%%%%%%%%%%%%%%%%%%%%%%%%
\lecture{05}{October 12}{Chien-I Chiang}{Keshav Deoskar}


% Some general latex examples and examples making use of the
% macros follow.  
%**** IN GENERAL, BE BRIEF. LONG SCRIBE NOTES, NO MATTER HOW WELL WRITTEN,
%**** ARE NEVER READ BY ANYBODY.


%%%%%%%%%%%%%%%%%%%%%%%%%%%%%%%%%%%%%%%%%
%%%%%%%%%%%%% Question 1
%%%%%%%%%%%%%%%%%%%%%%%%%%%%%%%%%%%%%%%%%
\underline{\textbf{Problem 1:}}

\begin{enumerate}
   \item We want to find the normalized position space wavefunction for a momentum eigenstate $\ket{p}$ i.e. we want to find $\inner{x}{p} \equiv \psi_p (x)$.
   
   Since $\ket{p}$ is an eigenvector of the momentum operator $\hat{P}$, we have 
   \[ \hat{P}\ket{p} = p\ket{p} \]

   Now, we know how the momentum operator acts on a ket $\ket{\psi}$ in position space:
   \[ \inner{x}{\hat{P}|\psi} = -i\hbar \frac{d\psi}{dx}\]

   So, then, fro an eigenvector $\ket{p}$, we have $\inner{x}{\hat{P}|\psi} = \inner{x}{(p|p}) = p\inner{x}{p} = p\psi_p(x)$

   \[ p\psi_p = -i\hbar \frac{d\psi_p}{dx} \]
   
   Since we are dealing with just one variable, we can carry out a simple separation of variables to get 
   \[ \frac{ip}{\hbar} \int dx = \int \frac{d\psi_p}{\psi_p}\] (since $\frac{1}{(-i)} = i$) which gives us 
   \begin{align*}
      \frac{ip}{\hbar} x &= \ln(\psi_p) + C_1 \\
                         &= \ln{C_0 \psi_p} 
   \end{align*} where $C_0 = \ln(C_1)$
   So, exponentiating both sides, we obtain
   \[ \psi_p = \frac{1}{C_0} e^{\frac{ipx}{\hbar}}\]
   But again, we can write $\frac{1}{C_0}$ more simply as some other constant $C$. So,
   \begin{equation}
      \boxed{\psi_p(x) = \inner{x}{p} = C e^{\frac{ipx}{\hbar}}} 
   \end{equation}
   
   We can find the constant $C$ using the normalization condition $\inner{p}{p'} = \delta(p - p')$ as 

   \begin{align*}
      \inner{p}{p'} &= \inner{p}{\mathbbm{1}|p'} \\
                    &= \int dx\; \inner{p}{x} \inner{x}{p'} \\
                    &= \int dx\; \inner{x}{p}^{*} \inner{x}{p'} \\
                    &= \int dx\; C^{*}e^{-\frac{ipx}{\hbar}} \cdot Ce^{\frac{ipx}{\hbar}} \\
                    &= |C|^2 \int dx\; e^{i(p'-p)x/h}
   \end{align*}
   To proceed, we use the following mathematical identity:
   \[ \int dk\; e^{ik(x-x')} = 2\pi\delta(x-x') \]
   or, in the notation we will apply it to,
   \[ \boxed{\int dx\; e^{ik(p-p')} = 2\pi\delta(p-p')} \]

   So, we have 
   \begin{align*}
      \inner{p}{p'} &= |C|^2 \int dx\; e^{i(p'-p)x/h} \\
                    &= \hbar|C|^2 \int \left(\frac{dx}{\hbar}\right)\;e^{i\frac{x}{h}(p-p')} \\
                    &= \hbar|C|^2 \cdot (2\pi\delta(p-p')) \\
                    &= (2\pi\hbar)\cdot|C|^2 \delta(p-p')
   \end{align*}
   So, 
   \[ \delta(p-p') = (2\pi\hbar)\cdot|C|^2 \delta(p-p') \]
   Thus, we obtain the result
   \[ |C| = \frac{1}{\sqrt{2\pi\hbar}} \]

   By convention, we choose $C$ to be a real number, so we simply have 
   \[ C = \frac{1}{\sqrt{2\pi\hbar}} \]

   Finally, the expression for the momentum wavefunction in position space is 
   \[ \boxed{\inner{x}{p} \equiv \psi_p(x) = \frac{1}{\sqrt{2\pi\hbar}} e^{ipx/\hbar}} \]

   \item Now, we want to find the matrix elements of the position operator in the momentum space. 
   
   We know that in position space, the position operator looks like
   \[ \inner{x}{\hat{X}|x'} = x\delta(x-x')  \]

   and we want to carry out a change of basis from the position eigenvectors to the momentum eigenvectors.

   We can do this by kind of going in the opposite direction and starting off with $\inner{p}{\hat{X}|p'}$ and inserting completenes i.e. $\mathbbm{1} = \int dx\; \ket{x}\bra{x}$

   We have 
   \begin{align*}
      \inner{p}{\hat{X}|p'} &= \inner{p}{\hat{X}|p'} \\
                            &= \int dx\; \inner{p}{X|x}\inner{x}{p'} \\
                            &= \int dx\;x\inner{p}{x}\inner{x}{p'} \\
                            &= \int dx\; x \left( \frac{1}{\sqrt{2\pi\hbar}}e^{-ipx/\hbar} \right) \cdot \left( \frac{1}{\sqrt{2\pi\hbar}}e^{ip'x/\hbar} \right) \\
                            &= \frac{1}{2\pi\hbar} \int dx\; x e^{i\frac{x}{\hbar}(p'-p)} \\
                            &= \frac{1}{2\pi\hbar} \int dx\;(i\hbar) \cdot \left( \frac{d}{dp} e^{i\frac{x}{\hbar}(p-p)} \right) \\
                            &= \frac{i\hbar}{2\pi\hbar} \left( \frac{d}{dp} \int dx\; e^{i\frac{x}{h}(p'-p)} \right) \\
   \end{align*}
   Now, using the same identity we used in part(a), we know that 
   \[ \int dx\; e^{i\frac{x}{\hbar}(p-p')} =  \int dx\; e^{ix(\frac{p-p'}{\hbar})} = 2\pi\delta\left(\frac{p-p'}{\hbar}\right) \]

   So, our postion operator in momentum space is 
   \begin{align*}
      \inner{p}{\hat{X}|p'} &= \frac{i\hbar}{2\pi\hbar} \cdot \frac{d}{dp} \left( 2\pi\delta\left(\frac{p-p'}{\hbar}\right) \right) \\
                            &= \frac{i\hbar}{2\pi\hbar} \cdot \frac{d}{dp} \left[ 2\pi\hbar\delta(p-p') \right] \\
                            &= i\hbar \frac{d}{dp} \delta(p-p')
   \end{align*}
   So, the representation of our position operator in momentum space is 
   \[ \boxed{ \inner{p}{\hat{X}|p'} = i\hbar \frac{d}{dp} \delta(p-p') } \]

   This means that for a general state $\ket{\psi}$ we have 
   \[ \inner{p}{\hat{X}|\psi} = i\hbar \frac{d}{dp} \psi(p) \]

   So, in position space, applying the Position operator looks like 
   \[ \inner{p}{\hat{X}|\psi} = \hat{\mathcal{X}}\psi(p) \]

   where $\hat{\mathcal{X}} = i\hbar \frac{d}{dp}$ and $\psi(p) \equiv \inner{p}{\psi}$

   \item Now, we had originally \emph{defined} the position operator $\hat{P}$ to be the generator of spatial translations. i.e. 
   \[ e^{iaP/\hbar}\ket{x} = \ket{x+a} \]

   And considering $a$ to be some infinitessimal translation $\epsilon$, we found that $\mom$ in position space could be expressed as 
   \[ \inner{x}{\mom|x'} = -i\hbar\frac{d}{dx} \delta(x-x') \]
   or 
   \[ \inner{x}{\mom|\psi} = -i\hbar\frac{d}{dx} \psi(x) \]

   We found a VERY similar expression for the \underline{\emph{postiion operator}} in \underline{\emph{momentum space}} as 
   \[ \inner{p}{\hat{X}|\psi} = i\hbar \frac{d}{dp} \psi(p) \]

   This suggests to us that the position operator plays a similar role in momentum space i.e. \textbf{The postiion operator is the generator of momentum translation.}

   % In fact, consider a momentum translation $\ket{p} \rightarrow \ket{p+\epsilon}$

   % Before the translation, i.e. when we have momentum $\ket{p}$, the position can be expressed as 
   % \[ \psi(x) = \frac{1}{\sqrt{2\pi\hbar}} \int dp\; e^{ipx/\hbar} \tilde{\psi}(p) \]

   % where $\psi(x) \equiv \inner{x}{\psi}$ and $\tilde{\psi}(p) \equiv \inner{p}{\psi}$

   % Once the momentum translation $\ket{p} \rightarrow \ket{p + \epsilon}$ has occured, the position wavefunction is given by 
   % \[ \psi(x) = \frac{1}{\sqrt{2\pi\hbar}} \int dp\; e^{i(p+\epsilon)x/\hbar} \tilde{\psi}(p+\epsilon) \]

\end{enumerate}

\vskip 0.5cm
\hrule
\vskip 0.5cm

\underline{\textbf{Problem 2:}}

\begin{enumerate}
   \item Suppose we have two states $\ket{\psi_1}$ and $\ket{\psi_2}$. Then, we can shift into the position space to express their inner product as
   \begin{align*}
      \inner{\psi_2}{\psi_1} &= \int dx\; \inner{\psi_2}{x} \inner{x}{\psi_1} \\
                             &= \int dx\; \inner{x}{\psi_2}^{*} \inner{x}{\psi_1} \\
                             &= \int dx\; \psi_2(x)^{*} \psi_1(x)
   \end{align*}
   Therefore, 
   \[ \boxed{\inner{\psi_2}{\psi_1} = \int dx\; \psi_2(x)^{*} \psi_1(x)} \]

   \item A state $\ket{\psi}$ is normalized if $\inner{\psi}{\psi} = 1$. Using part (a), we have 
   \[ \inner{\psi}{\psi} = \int dx\; \psi(x)^{*} \psi(x)  \]

   But, since $\psi(x)$ is just some complex number, we know that $\psi(x)^{*} \psi(x) = |\psi(x)|^2$

   Thus, in the position basis, the normalization condition looks like
   \[ \boxed{\int dx\; |\psi(x)|^2 = 1} \]

   \item We know that, in position space, the position operator $\hat{X}$ acts on a state $\ket{\psi}$ by multiplying the (position-space) wavefunction with $x$. That is,
   \[ \hat{X}\ket{\psi} \rightarrow x\psi(x) \]

   Now, the expectation value for the operator is found as 
   \begin{align*}
      \inner{\psi}{\hat{X}|\psi} &= \inner{\psi}{\mathbbm{1}\hat{X}\mathbbm{1}|\psi} \\
                                 &= \int dx\;dx'\; \inner{\psi}{x} \inner{x}{\hat{X}|x'} \inner{x'}{\psi}
   \end{align*}
   We know, from studying the eigenvalue problem of $\hat{X}$, that 
   \[ \inner{x}{\hat{X}|x'} = x\delta(x-x') \]

   So, we have 
   \begin{align*}
      \inner{\psi}{\hat{X}|\psi} &= \int dx\;dx'\; \inner{x}{\psi}^{*} x\delta(x-x') \inner{x'}{\psi} \\
                                 &= \int dx\;dx'\;\delta(x-x') \psi(x)^{*} x \psi(x') \\
                                 &= \int dx\; \psi(x)^{*} x \psi(x)
   \end{align*}
   Therefore, the expecation value of $\hat{X}$ is 
   \[ \boxed{\inner{\psi}{\hat{X}|\psi} = \int dx\; \psi(x)^{*} x \psi(x)} \]

   \item In momentum space, the momentum operator $\hat{P}$ acts on a state $\ket{\psi}$ as 
   \[ \hat{P}\ket{\psi} \rightarrow -i\hbar \frac{d\psi(x)}{dx} \]
   Or more precisely,
   \[ \inner{x}{\hat{P}|\psi} = -i\hbar \frac{d\psi(x)}{dx} \]

   Now, the expectation value for the momentum operator is given by 
   \begin{align*}
      \inner{\psi}{\hat{P}|\psi} &= \inner{\psi}{\mathbbm{1}\hat{P}\mathbbm{1}|\psi} \\
                                 &= \int dx\;dx'\;\inner{\psi}{x} \inner{x}{\hat{P}|x'} \inner{x'}{\psi} \\
                                 &= \int dx\;dx'\; \inner{x}{\psi}^{*} \inner{x}{\hat{P}|x'} \inner{x'}{\psi}
   \end{align*}
   From our earlier studies of the momentum operator, we know that the matrix elements of the operator in the position space are given by 
   \[ \inner{x}{\hat{P}|x'} = -i\hbar\frac{d}{dx}\psi(x) \]

   So, we have 
   \begin{align*}
      \inner{\psi}{\hat{P}|\psi} &= \int dx\;dx'\; \psi(x)^{*} \left( -i\hbar \frac{d}{dx} \delta(x-x') \right) \psi(x') \\
                                 &= \int dx\; \psi(x)^{*} \left( -i\hbar\frac{d}{dx} \right) \psi(x)
   \end{align*}
   Therefore, the expecation value of the momentum operator can be found in position space as 
   \[ \boxed{\inner{\psi}{\hat{P}|\psi} = \int dx\; \psi(x)^{*} \left( -i\hbar\frac{d}{dx} \right) \psi(x)} \]

   \item The operator $\hat{P}^2$ can be thought as 
   \begin{align*}
      \hat{P}^2 &= \hat{P} \cdot \hat{P} \\
                &= \left( -i\hbar\frac{d}{dx} \right) \cdot \left( -i\hbar\frac{d}{dx} \right) \\ 
                &= -\hbar^2 \frac{d^2}{dx^2}
   \end{align*}
   and the matrix elements of this operator should be given by 
   \[ \inner{x}{\hat{P}^2|x'} = -\hbar^2 \frac{d^2}{dx^2} \delta(x-x') \]

   So, the expectation value of the squared momentum operator can be found as 
   \begin{align*}
      \inner{\psi}{\hat{P}^2|\psi} &= \inner{\psi}{\mathbbm{1}\hat{P}^2\mathbbm{1}|\psi} \\
                                   &= \int dx\;dx'\; \inner{\psi}{x} \inner{x}{\hat{P}^2|x'} \inner{x'}{\psi} \\
                                   &= \int dx\;dx'\; \psi(x')^{*} \left( -\hbar^2 \frac{d^2}{dx^2} \delta(x-x') \right) \psi(x) \\ 
                                   &= \int dx\; \psi(x) \left( -\hbar^2 \frac{d^2}{dx^2} \right) \psi(x)
   \end{align*}
   Therefore, the expectation value for $\hat{P}^2$ is 
   \[ \boxed{\inner{\psi}{\hat{P}^2|\psi} = \int dx\; \psi(x) \left( -\hbar^2 \frac{d^2}{dx^2}  \right) \psi(x)} \]

   \item We have a system with potential $V(x)$ and hamiltonian $\ham = \frac{\mom^2}{2m} + V$.
   
   \underline{TISE:}
   The Time Independent Schrödinger Equation is 
   \[ \ham\ket{E} = E\ket{E} \]

   In position space, the left hand side becomes 
   \begin{align*}
      \inner{x}{\ham|E} &= \bra{x} \left( \frac{\mom^2}{2m} + V \right) \ket{E} \\
                        &= \inner{x}{\left[ \left( \frac{\mom^2}{2m} + V \right) |E} \right] \\
                        &= \frac{1}{2m}\inner{x}{\mom^2|E} + \inner{x}{V|E}
   \end{align*}
   From our earlier studies, we know that 
   \[ \inner{x}{\mom^2|E} = -\hbar \frac{d^2}{dx^2}\psi_E(x)\;\;\;\text{and}\;\;\;\inner{x}{V|E} = V(x)\]
   
   Thus, the left hand side of the TISE is 
   \begin{equation}
      \ham\ket{E} = -\frac{\hbar}{2m} \frac{d^2}{dx^2}\psi_E(x) + V(x) 
   \end{equation}

   The expression on the right hand, $E\ket{E}$, in the position-space is simply
   \begin{equation}
      \inner{x}{E|E} = E\inner{x}{E} = E\psi_E(x)
   \end{equation}

   So combining our results from equations (5.2) and (5.3), we find that the TISE in position-space is 
   \[ \boxed{-\frac{\hbar}{2m} \frac{d^2}{dx^2}\psi_E(x) + V(x) = E\psi_E(x)} \]

   \underline{TDSE:}
   The Time-Dependent Schrödinger Equation says 
   \[ i\hbar \frac{d}{dt} \ket{\psi(t)} = \ham \ket{\psi(t)} \]

   Now, the left-hand expression in position space is 
   \begin{align*}
      \inner{x}{\left( i\hbar \frac{d}{dt} \right)|\psi(t)} &= i\hbar\frac{d}{dt} \inner{x}{\psi(t)} = i\hbar\frac{\partial}{\partial t} \psi_E(x,t)
   \end{align*}
   where we are able to pull out the linear operator $\left( i\hbar \frac{d}{dt} \right)$ because it is a unitary operator, and inner products are invariant under unitary translations.

   The right hand expression, expressed in position-space, is 
   \begin{align*}
      \inner{x}{\ham|\psi(t)} &= \inner{x}{\left( \frac{\mom^2}{2m} + V\right)|\psi(t)} \\
                              &= \frac{1}{2m}\inner{x}{\mom^2|\psi(t)} + \inner{x}{V|\psi(t)} \\
                              &= 
                              \frac{-\hbar}{2m} \frac{\partial^2 }{\partial x^2 } \psi_E(x,t) + V(x)\psi_E(x,t) \\
                              &= \left[ \frac{-\hbar}{2m} \frac{\partial^2 }{\partial x^2 } + V(x) \right]\psi_E(x,t)
   \end{align*}

   So, in position-space, the TDSE is expressed as 
   \[ \boxed{i\hbar\frac{\partial}{\partial t} \psi_E(x,t) = \left[ \frac{-\hbar}{2m} \frac{\partial^2 }{\partial x^2 } + V(x) \right]\psi_E(x,t)} \]
\end{enumerate}
\vskip 0.5cm
\hrule
\vskip 0.5cm

\underline{Problem 3:}

\begin{enumerate}
   \item We have a particle confined to the x-axis whose wavefunction is described by 
   \[ \psi = \begin{cases} 
      0 & x\leq -a \\
      C & -a\leq x\leq a \\
      0 & c \geq a
   \end{cases}
   \]
   where $C$ is a normalization constant.

   We can find $C$ using the normalization condition 
   \[ \int_{-\infty}^{\infty} dx\; |\psi(x)|^2 = 1\]

   We have 
   \begin{align*}
      &0 + \int_{-a}^{a} dx\; |C|^2 + 0 = 1 \\
      \implies& C^2 \int_{-a}^{a} dx = 1 \;\;\;(\text{By Convention, } C \in \mathbb{R} \text{ so } |C| = C) \\
      \implies& C^2 \cdot (2a) = 1 
   \end{align*}
   Thus,
   \[ \boxed{C = \frac{1}{\sqrt{2a}}} \]


   \item The expectation value for $\hat{X}$ is given by 
   \begin{align*}
      \langle \hat{X} \rangle &= \int_{-\infty}^{\infty} \psi(x)^{*} x \psi(x) \\
                              &= 0 + \int_{-a}^{a} dx\; \left(\frac{1}{\sqrt{2a}}\right)^{*} x \left(\frac{1}{\sqrt{2a}}\right) + 0 \\
                              &= \frac{1}{2a} \int_{-a}^{a} dx\; x \\
                              &= \frac{1}{2a} \cdot \left[ \frac{x^2}{2} \Biggr|_{-a}^{a} \right] \\
                              &= \frac{1}{2a} \cdot \left( \frac{a^2}{2} - \frac{a^2}{2} \right) \\
                              &= 0
   \end{align*}
   So, the expected position is $\boxed{ \langle \hat{X} \rangle = 0}$.


   The expectation value for $\hat{X}^2$ is given by 
   \begin{align*}
      \langle \hat{X} \rangle &= \int_{-\infty}^{\infty} \psi(x)^{*} x^2 \psi(x) \\
                              &= 0 + \int_{-a}^{a} dx\; \left(\frac{1}{\sqrt{2a}}\right)^{*} x^2 \left(\frac{1}{\sqrt{2a}}\right) + 0 \\
                              &= \frac{1}{2a} \int_{-a}^{a} dx\; x^2 \\
                              &= \frac{1}{2a} \cdot \left[ \frac{x^3}{3} \Biggr|_{-a}^{a} \right] \\
                              &= \frac{1}{2a} \cdot \left( \frac{a^3}{3} + \frac{a^3}{3} \right) \\
                              &= \frac{1}{2a} \cdot a^3 \\
                              &= \frac{a^2}{2}
   \end{align*}
   So, the expected position is $\boxed{ \langle \hat{X}^2 \rangle = \frac{a^2}{2}}$.

   \item There are some values of momentum $p_x$ which the probability to find the particle in is zero. Our goal is to find these momenta.
   
   Recall that  the probability of having a particular momentum $p$ is given by $|\inner{p}{\psi}|^2$. 
   
   Inserting completeness (in the position-space), we can write
   \begin{align*}
      \inner{p}{\psi} &= \int dx\; \inner{p}{x} \inner{x}{\psi} \\
                          &= \int_{-\infty}^{\infty} dx\; \inner{x}{p}^{*} \inner{x}{\psi} \\
                          &= \int_{-\infty}^{\infty} dx\; (\psi_p(x)^{*})(\psi(x)) \\
                          &= \int_{-\infty}^{\infty} \left( \frac{1}{\sqrt{2\pi\hbar}} e^{-\frac{ipx}{\hbar}} \right) \psi(x) \\
                          &= 0 + \int_{-a}^{a} dx\; \left( \frac{1}{\sqrt{2\pi\hbar}} e^{-\frac{ipx}{\hbar}} \right) \cdot \left(\frac{1}{\sqrt{2a}}\right) + 0 \\
                          &= \frac{1}{\sqrt{2a}} \cdot \frac{1}{\sqrt{2\pi\hbar}} \int_{-a}^{a} e^{-\frac{ipx}{\hbar}} \\
                          &= \frac{1}{\sqrt{2a}} \cdot \frac{1}{\sqrt{2\pi\hbar}} \left[ \frac{-\hbar}{ip} e^{-\frac{ipx}{\hbar}} \right]_{-a}^{a} \\
                          &= \frac{1}{\sqrt{2a}} \cdot \frac{1}{\sqrt{2\pi\hbar}} \left[ \frac{i\hbar}{p} e^{-\frac{ipx}{\hbar}} \right]_{-a}^{a} \\
                          &= \frac{1}{\sqrt{2a}} \cdot \frac{1}{\sqrt{2\pi\hbar}} \cdot  \frac{i\hbar}{p} \left[e^{-\frac{ipx}{\hbar}} \right]_{-a}^{a} \\
                          &= \frac{i}{2p} \cdot \sqrt{\frac{\hbar}{\pi a}} \left( e^{-\frac{ipa}{\hbar}} -  e^{\frac{ipa}{\hbar}} \right) \\
                          &= \frac{i}{2p} \cdot \sqrt{\frac{\hbar}{\pi a}}\cdot (-2)\sinh\left(\frac{ipa}{h}\right)
   \end{align*}
   That is 
   \[ \boxed{\inner{p}{\psi} =  \frac{-i}{p} \cdot \sqrt{\frac{\hbar}{\pi a}}\cdot \sinh\left(\frac{ipa}{h}\right) } \]

   Further, we can note that $\sinh(z) = -i\sin(iz)\;\; \forall z \in \mathbb{C}$ which means
   \begin{align*}
      \frac{-i}{p} \cdot \sqrt{\frac{\hbar}{\pi a}}\cdot \sinh\left(\frac{ipa}{h}\right) &= \frac{-i}{p} \cdot \sqrt{\frac{\hbar}{\pi a}}\cdot (-i)\sin\left(-\frac{pa}{h}\right) \\
              &= \frac{-i \cdot i}{p} \cdot \sqrt{\frac{\hbar}{\pi a}}\cdot \sin\left(\frac{pa}{h}\right) \\
              &= \frac{1}{p} \cdot \sqrt{\frac{\hbar}{\pi a}}\cdot \sin\left(\frac{pa}{h}\right) 
   \end{align*}
   so, 
   \[ \boxed{\inner{p}{\psi} = \frac{1}{p} \cdot \sqrt{\frac{\hbar}{\pi a}}\cdot \sin\left(\frac{pa}{h}\right)  } \]

   Now, since this is a real number, we have 
   \[ \inner{\psi}{p} = \inner{p}{\psi}^{*} = \inner{p}{\psi} \]

   Which means the probability of the particle posessing the momentum $p$ is 
   \[ \mathcal{P}(p) = \left[ \frac{1}{p} \cdot \sqrt{\frac{\hbar}{\pi a}}\cdot \sin\left(\frac{pa}{h}\right) \right]^2 \]


   The momenta whose probabilities of ocurring are given by 
   \[ \mathcal{P}(p) = 0 \]

   In order for this to be the case, we must have \[ \sin\left(\frac{pa}{\hbar}\right) = 0 \]

   So, we have 
   \begin{align*}
      &\frac{pa}{\hbar} = \pi n \\
      \implies &p = \frac{n\pi\hbar}{a}
   \end{align*}
   Therefore, the momenta of zero probability are given by 
   \[ \boxed{p = \frac{n\pi\hbar}{a},\;\;n \in \mathbb{Z}} \]
\end{enumerate}

\hrule

%\section*{References}
%\beginrefs
%\bibentry{AGM97}{\sc N.~Alon}, {\sc Z.~Galil} and {\sc O.~Margalit},
%On the Exponent of the All Pairs Shortest Path Problem,
%{\it Journal of Computer and System Sciences\/}~{\bf 54} (1997),
%pp.~255--262.

%\bibentry{F76}{\sc M. L. ~Fredman}, New Bounds on the Complexity of the 
%Shortest Path Problem, {\it SIAM Journal on Computing\/}~{\bf 5} (1976), 
%pp.~83-89.
%\endrefs



\end{document}





