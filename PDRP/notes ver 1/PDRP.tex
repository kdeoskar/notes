\documentclass{article}

% Language setting
% Replace `english' with e.g. `spanish' to change the document language
\usepackage[english]{babel}

% Set page size and margins
% Replace `letterpaper' with`a4paper' for UK/EU standard size
\usepackage[letterpaper,top=2cm,bottom=2cm,left=3cm,right=3cm,marginparwidth=1.75cm]{geometry}

% Useful packages
\usepackage{amsmath}
\usepackage{amssymb}
\usepackage{mathtools}
\usepackage{graphicx}
\usepackage{enumitem}
\usepackage[colorlinks=true, allcolors=blue]{hyperref}

\usepackage{hyperref}
\hypersetup{
    colorlinks=true,
    linkcolor=blue,
    filecolor=magenta,      
    urlcolor=cyan,
    pdftitle={Overleaf Example},
    pdfpagemode=FullScreen,
    }

\urlstyle{same}

\usepackage{tikz-cd}

%%%%%%%%%%% Box pacakges and definitions %%%%%%%%%%%%%%
\usepackage[most]{tcolorbox}
\usepackage{xcolor}

% Define the colors
\definecolor{boxheader}{RGB}{0, 51, 102}  % Dark blue
\definecolor{boxfill}{RGB}{173, 216, 230}  % Light blue

% Define the tcolorbox environment
\newtcolorbox{mathdefinitionbox}[2][]{%
    colback=boxfill,   % Background color
    colframe=boxheader, % Border color
    fonttitle=\bfseries, % Bold title
    coltitle=white,     % Title text color
    title={#2},         % Title text
    enhanced,           % Enable advanced features
    attach boxed title to top left={yshift=-\tcboxedtitleheight/2}, % Center title
    boxrule=0.5mm,      % Border width
    sharp corners,      % Sharp corners for the box
    #1                  % Additional options
}
%%%%%%%%%%%%%%%%%%%%%%%%%

\newtcolorbox{dottedbox}[1][]{%
    colback=white,    % Background color
    colframe=white,    % Border color (to be overridden by dashrule)
    sharp corners,     % Sharp corners for the box
    boxrule=0pt,       % No actual border, as it will be drawn with dashrule
    boxsep=5pt,        % Padding inside the box
    enhanced,          % Enable advanced features
    overlay={\draw[dashed, thin, black, dash pattern=on \pgflinewidth off \pgflinewidth, line cap=rect] (frame.south west) rectangle (frame.north east);}, % Dotted line
    #1                 % Additional options
}

\usepackage{biblatex}
\addbibresource{sample.bib}


%%%%%%%%%%% New Commands %%%%%%%%%%%%%%
\newcommand*{\T}{\mathcal T}
\newcommand*{\cl}{\text cl}


\newcommand{\ket}[1]{|#1 \rangle}
\newcommand{\bra}[1]{\langle #1|}
\newcommand{\inner}[2]{\langle #1 | #2 \rangle}
\newcommand{\R}{\mathbb{R}}
\newcommand{\C}{\mathbb{C}}
\newcommand{\V}{\mathbb{V}}
\newcommand{\Hilbert}{\mathcal{H}}
\newcommand{\oper}{\hat{\Omega}}
\newcommand{\lam}{\hat{\Lambda}}
\newcommand{\defeq}{\vcentcolon=}

\newcommand{\bigslant}[2]{{\raisebox{.2em}{$#1$}\left/\raisebox{-.2em}{$#2$}\right.}}
\newcommand{\restr}[2]{{% we make the whole thing an ordinary symbol
  \left.\kern-\nulldelimiterspace % automatically resize the bar with \right
  #1 % the function
  \vphantom{\big|} % pretend it's a little taller at normal size
  \right|_{#2} % this is the delimiter
  }}
%%%%%%%%%%%%%%%%%%%%%%%%%%%%%%%%%%%%%%%


\tcbset{theostyle/.style={
    enhanced,
    sharp corners,
    attach boxed title to top left={
      xshift=-1mm,
      yshift=-4mm,
      yshifttext=-1mm
    },
    top=1.5ex,
    colback=white,
    colframe=blue!75!black,
    fonttitle=\bfseries,
    boxed title style={
      sharp corners,
    size=small,
    colback=blue!75!black,
    colframe=blue!75!black,
  } 
}}

\newtcbtheorem[number within=section]{Theorem}{Theorem}{%
  theostyle
}{thm}

\newtcbtheorem[number within=section]{Definition}{Definition}{%
  theostyle
}{def}



\title{PDRP Notes}
\author{Keshav Balwant Deoskar}

\begin{document}
\maketitle

% \vskip 0.5cm
% These are notes taken from lectures on Topology, Geometry, and Algebra delivered by EVibhu Ravindran, Pablo Castano, and Michelle Dong for UC Berekley's Physics 198 class (DeCal) in the Sprng 2024 semester. Any errors that may have crept in are solely my fault.
% \pagebreak 

\tableofcontents

\pagebreak

%%%%%%%%%%%%%%%%%%%%%%%%%%%%%%%%%%%%%%%%%%%%%%%%%%%%%%%%%%%%%%%%%%
\section{February 8 - First meeting}
%%%%%%%%%%%%%%%%%%%%%%%%%%%%%%%%%%%%%%%%%%%%%%%%%%%%%%%%%%%%%%%%%

\subsection{Game Plan}
The plan is to cover the following topics in roughly descending order

\vskip 0.5cm
\begin{itemize}
  \item Susy and Morse Theory 
  \item Witten TQFT's : Cohomology
  \item Schwarz : Path Integration 
  \item Chern-Simons theory : Related to condensed matter theory
\end{itemize}

\vskip 0.5cm
Some math we'll need for \emph{all} of these Homology, Bundles, and Morse Theory.

Stuff we want to review before the meat:
\begin{itemize}
  \item QFT, particularly Canonical Quantization
  \item Path Integrals
\end{itemize}

\vskip 1cm
\subsection{Review of Topology}

-Refer to Physics 198 notes lmao

\subsection*{Why is the definition of a topology useful?}

\pagebreak


\section{Path integrals and Fractional Quantization}

To consider the time evolution of a state, we need to calculate the 


\vskip 0.5cm
Why do we need fields? To preseve unitary while being able to talk about particle creation and annihilation.

\vskip 0.5cm
How do we evaluate 
\[ \langle do \rangle = \int \mathcal{D A} e^{i\mathcal{S}[\mathcal{A}]} \]
?

\vskip 0.5cm
This is an infinite dimensional integral and further the action is now a functional. The same issues come up when we try to evaluate correlation functions $\inner{0}{T\{ \phi(x_1), \dots, \phi(x_2) \}|0}$. \textbf{This is quite the conundrum}.

\vskip 0.5cm
We're going to try and use Feynman's Trick. To replicate feynman's trick, we perturb the action a bit 
\[ \mathcal{S}[\mathcal{A}(x)] =  \]

\emph{Note:} We call $\mathcal J$ a source, and it is a generator of the partition function $Z[\mathcal{J}]$

We say 
\[ \mathcal{Z}[\mathcal{J}] = \int \mathcal{D} \mathcal{A} e^{i \mathcal{S}[x]} \]

Then, we get 
\begin{align*}
  \frac{d}{d\mathcal{J}(x)} \restr{\int \mathcal{D} \mathcal{A} e^{i \mathcal{S}[x]}}{\mathcal{J} = 0} &= 
\end{align*}

\vskip 0.5cm
Usually, we have to deal with \emph{time ordering} to account for causality. When looking at a two point function, we have time ordering if 
\[ \langle \mathcal{A}(x_1), \mathcal{A}(x_2) \rangle \] 


\vskip 0.5cm
But in the path integral above, these commute. But we get time ordering for free [explain why].


\end{document}
