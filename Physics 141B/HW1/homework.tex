\documentclass[11pt]{article}

% basic packages
\usepackage[margin=1in]{geometry}
\usepackage[pdftex]{graphicx}
\usepackage{amsmath,amssymb,amsthm}
\usepackage{custom}
\usepackage{lipsum}

\usepackage{xcolor}
\usepackage{tikz}

\usepackage[most]{tcolorbox}
\usepackage{xcolor}
\usepackage{mdframed}

% page formatting
\usepackage{fancyhdr}
\pagestyle{fancy}

\renewcommand{\sectionmark}[1]{\markright{\textsf{\arabic{section}. #1}}}
\renewcommand{\subsectionmark}[1]{}
\lhead{\textbf{\thepage} \ \ \nouppercase{\rightmark}}
\chead{}
\rhead{}
\lfoot{}
\cfoot{}
\rfoot{}
\setlength{\headheight}{14pt}

\linespread{1.03} % give a little extra room
\setlength{\parindent}{0.2in} % reduce paragraph indent a bit
\setcounter{secnumdepth}{2} % no numbered subsubsections
\setcounter{tocdepth}{2} % no subsubsections in ToC


%%%%%%%%%%%%%%%%%%%%%%%%%%%%%%%%%%%%%%%%%%%%%%%%%%%%%%%%%%%%%%%%%
% CUSTOM BOXES AND STUFF
\newtcolorbox{redbox}{colback=red!5!white,colframe=red!75!black, breakable}
\newtcolorbox{bluebox}{colback=blue!5!white,colframe=blue!75!black,breakable}

\newtcolorbox{dottedbox}[1][]{%
    colback=white,    % Background color
    colframe=white,    % Border color (to be overridden by dashrule)
    sharp corners,     % Sharp corners for the box
    boxrule=0pt,       % No actual border, as it will be drawn with dashrule
    boxsep=5pt,        % Padding inside the box
    enhanced,          % Enable advanced features
    breakable,         % Enables it to span multiple pages
    overlay={\draw[dashed, thin, black, dash pattern=on \pgflinewidth off \pgflinewidth, line cap=rect] (frame.south west) rectangle (frame.north east);}, % Dotted line
    #1                 % Additional options
}

% Define the colors
\definecolor{boxheader}{RGB}{0, 51, 102}  % Dark blue
\definecolor{boxfill}{RGB}{173, 216, 230}  % Light blue


% Define the tcolorbox environment
\newtcolorbox{mathdefinitionbox}[2][]{%
    colback=boxfill,   % Background color
    colframe=boxheader, % Border color
    fonttitle=\bfseries, % Bold title
    coltitle=white,     % Title text color
    title={#2},         % Title text
    enhanced,           % Enable advanced features
    breakable,
    attach boxed title to top left={yshift=-\tcboxedtitleheight/2}, % Center title
    boxrule=0.5mm,      % Border width
    sharp corners,      % Sharp corners for the box
    #1                  % Additional options
}
%%%%%%%%%%%%%%%%%%%%%%%%%


\definecolor{lightblue}{RGB}{173,216,230} % Light blue color
\definecolor{darkblue}{RGB}{0,0,139} % Dark blue color

% Define the custom proof environment
\newtcolorbox{ex}[2][Example]{
  colback=red!5!white, % Light blue background
  colframe=red!75!black, % Darker blue border
  coltitle=white, % Title color
  fonttitle=\bfseries, % Title font style
  title={{#2}},
  arc=1mm, % Rounded corners with 4mm radius,
  boxrule=0.5mm,
  left=2mm, right=2mm, top=2mm, bottom=2mm, % Padding inside the box
  breakable, % Allow box to be broken across pages
  before=\vspace{10pt}, % Padding above the box
  after=\vspace{10pt}, % Padding below the box
  before upper={\parindent15pt} % Ensure indentation
}

% Define the custom proof environment
\newtcolorbox{defn}[2][Definition]{
  colback=green!5!white, % Light blue background
  colframe=green!75!black, % Darker blue border
  coltitle=white, % Title color
  fonttitle=\bfseries, % Title font style
  title={{#2}},
  arc=1mm, % Rounded corners with 4mm radius,
  boxrule=0.5mm,
  left=2mm, right=2mm, top=2mm, bottom=2mm, % Padding inside the box
  breakable, % Allow box to be broken across pages
  before=\vspace{10pt}, % Padding above the box
  after=\vspace{10pt}, % Padding below the box
  before upper={\parindent15pt} % Ensure indentation
}


%%%%%%%%%%%%%%%%%%%%%%%%%%%%%%%%%%%%%%%%%%%%%%%%%%%%%%%%%%%%%%%%%


\begin{document}

% make title page
\thispagestyle{empty}
\bigskip \
\vspace{0.1cm}

\begin{center}
{\fontsize{14}{14} \selectfont Instructor: James Analytis}
\vskip 16pt
{\fontsize{28}{28} \selectfont \bf \sffamily Physics 141B: Solid State Physics II}
\vskip 24pt
{\fontsize{14}{14} \selectfont \rmfamily Homework 1} 
\vskip 6pt
{\fontsize{14}{14} \selectfont \ttfamily kdeoskar@berkeley.edu} 
\vskip 24pt
\end{center}

% {\parindent0pt \baselineskip=15.5pt \lipsum[1-4]} 

% make table of contents
% \newpage


\subsection*{Question 1: SSH Model}

\begin{enumerate}[(a).]
  \item We have a unitary operator $\Theta$ such that, for hamiltonian $H$, $$ \Theta H \Theta^{-1} = -H $$ i.e. $$ \Theta H = -H\Theta $$ Consider an energy eigenstate $$ H\ket{\psi} = E \ket{\psi} $$ and then consider the state $\Theta \ket{\psi}$. Then,
  \begin{align*}
    H \left( \Theta \ket{\psi} \right) &= \left(H \Theta\right)\ket{\psi} \\
    &= -\Theta H \ket{\psi} \\
    &= -\Theta\left(E \ket{\psi}\right) \\
    &= -E \left( \Theta \ket{\psi} \right)
  \end{align*}
  Thus, for eigenstate $\ket{\psi}$ with energy $E$ there exists eigenstate $\Theta \ket{\psi}$ with energy $-E$.
  
  \item The SSH Model with nearest-neighbor-only hopping has Chiral symmetry $$ \sigma_z H \sigma_z^{-1} = -H $$ because its hamiltonian can be expressed as \begin{align*}
    H(k) &= \vec{\sigma} \cdot \vec{b}(k)
  \end{align*}
  where \begin{align*}
    &\vec{b}_x = \begin{pmatrix}
      \Delta_1(k) \\ 0 \\ 0
    \end{pmatrix}, ~\vec{b}_y = \begin{pmatrix}
      0 \\ \Delta_2(k) \\ 0
    \end{pmatrix}, ~\vec{b}_z = \begin{pmatrix}
      0\\ 0 \\ 0
    \end{pmatrix} \\
    \\
    &\text{ and } \vec{b}(k) = b_x \hat{x} + b_y \hat{y} + b_z \hat{z}
  \end{align*} More specifically, the $z-$component of $\vec{b}$ is zero and so our hamiltonian has no $\sigma_z$ terms and only terms with $\sigma_x, \sigma_y$. This allows it to anticommute with $\sigma_z$ from the canonical commutation rules of the Pauli matrices. The statement that $H$ anticommutes with $\sigma_z$ is equivalent to the Chiral symmetry statement:
  \begin{align*}
    &\{\sigma_z, H \} = 0 \\
    \implies& \sigma_z H + H \sigma_z = 0 \\
    \implies& \sigma_z H = - H \sigma_z \\
    \implies& \sigma_z H \sigma_z^{-1} = - H
  \end{align*}  \\
  If we allowed for second-neighbor hoppings as well, then we would have non-zero $\sigma_z$ terms and our hamiltonian would no longer anticommute with $\sigma_z$. Thus, it would also no longer have Chiral Symmetry.
\end{enumerate}

\vskip 1cm
\hrule
\vskip 1cm


\subsection*{Question 2}

For the 10 site ring (zero-indexed), we have 
\begin{center}
  \includegraphics*[scale=0.7]{plots/Q2plot.png}
\end{center}

\vskip 1cm
\hrule
\vskip 1cm


\subsection*{Question 3}

\begin{enumerate}[(i).]
  \item For the five smallest eigenenergy states:
  \begin{center}
    \includegraphics*[scale=0.7]{plots/Q3part1.png}
  \end{center}

  \item Breaking the ring:
  \begin{center}
    \includegraphics*[scale=0.7]{plots/Q3part2.png}
  \end{center}

  \item Switching the hopping parameters:
  \begin{center}
    \includegraphics*[scale=0.7]{plots/Q3part3.png}
  \end{center}
\end{enumerate}

\vskip 1cm
\hrule
\vskip 1cm


\subsection*{Question 4}

\begin{enumerate}[(i).]
  \item For the 11 site ring:
  \begin{center}
    \includegraphics*[scale=0.7]{plots/Q4part1.png}
  \end{center}

  \item Switching the hopping parameters:
  \begin{center}
    \includegraphics*[scale=0.7]{plots/Q4part2.png}
  \end{center}
\end{enumerate}


\vskip 1cm
\hrule




















% \subsection*{Question 1}


% \vskip 1cm
% \hrule
% \pagebreak



% \begin{bluebox}
%   \textbf{Question 1:} 
% \end{bluebox}

% \vskip 0.5cm
% \textbf{\underline{Solution:}}
% \\
% \\
% text
% \vskip 0.5cm
% \hrule
% \pagebreak



% %%%%%%%%%%%%%%%%%%%%%%%%%%%%%%%%%%%%%%%%%%%%%%
% \newpage
% % \section{References}
% %%%%%%%%%%%%%%%%%%%%%%%%%%%%%%%%%%%%%%%%%%%%%%
% \vskip 0.5cm
% \bibliographystyle{plain} % We choose the "plain" reference style
% \bibliography{citation}




\end{document}










