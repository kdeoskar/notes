%
% This is the LaTeX template file for lecture notes for CS294-8,
% Computational Biology for Computer Scientists.  When preparing 
% LaTeX notes for this class, please use this template.
%
% To familiarize yourself with this template, the body contains
% some examples of its use.  Look them over.  Then you can
% run LaTeX on this file.  After you have LaTeXed this file then
% you can look over the result either by printing it out with
% dvips or using xdvi.
%
% This template is based on the template for Prof. Sinclair's CS 270.

\documentclass[twoside]{article}
\usepackage{graphics}
\usepackage{mathtools}
\usepackage[]{mdframed}
\usepackage{amsfonts}
\usepackage{amsmath}
\setlength{\oddsidemargin}{0.25 in}
\setlength{\evensidemargin}{-0.25 in}
\setlength{\topmargin}{-0.6 in}
\setlength{\textwidth}{6.5 in}
\setlength{\textheight}{8.5 in}
\setlength{\headsep}{0.75 in}
\setlength{\parindent}{0 in}
\setlength{\parskip}{0.1 in}

%
% The following commands set up the lecnum (lecture number)
% counter and make various numbering schemes work relative
% to the lecture number.
%
\newcounter{lecnum}
\renewcommand{\thepage}{\thelecnum-\arabic{page}}
\renewcommand{\thesection}{\thelecnum.\arabic{section}}
\renewcommand{\theequation}{\thelecnum.\arabic{equation}}
\renewcommand{\thefigure}{\thelecnum.\arabic{figure}}
\renewcommand{\thetable}{\thelecnum.\arabic{table}}

%
% The following macro is used to generate the header.
%
\newcommand{\lecture}[4]{
   \pagestyle{myheadings}
   \thispagestyle{plain}
   \newpage
   \setcounter{lecnum}{#1}
   \setcounter{page}{1}
   \noindent
   \begin{center}
   \framebox{
      \vbox{\vspace{2mm}
    \hbox to 6.28in { {\bf Knot 176: Introduction to Low-Dimensional Topology
                        \hfill Fall 2023} }
       \vspace{4mm}
       \hbox to 6.28in { {\Large \hfill Lecture #1: #2  \hfill} }
       \vspace{2mm}
       \hbox to 6.28in { {\it Lecturers: #3 \hfill Scribe: #4} }
      \vspace{2mm}}
   }
   \end{center}
   \markboth{Lecture #1: #2}{Lecture #1: #2}
   {\bf Disclaimer}: {\it LaTeX template courtesy of the UC Berkeley EECS Department.}
   \vspace*{4mm}
}

%
% Convention for citations is authors' initials followed by the year.
% For example, to cite a paper by Leighton and Maggs you would type
% \cite{LM89}, and to cite a paper by Strassen you would type \cite{S69}.
% (To avoid bibliography problems, for now we redefine the \cite command.)
% Also commands that create a suitable format for the reference list.
\renewcommand{\cite}[1]{[#1]}
\def\beginrefs{\begin{list}%
        {[\arabic{equation}]}{\usecounter{equation}
         \setlength{\leftmargin}{2.0truecm}\setlength{\labelsep}{0.4truecm}%
         \setlength{\labelwidth}{1.6truecm}}}
\def\endrefs{\end{list}}
\def\bibentry#1{\item[\hbox{[#1]}]}

%Use this command for a figure; it puts a figure in wherever you want it.
%usage: \fig{NUMBER}{SPACE-IN-INCHES}{CAPTION}
\newcommand{\fig}[3]{
			\vspace{#2}
			\begin{center}
			Figure \thelecnum.#1:~#3
			\end{center}
	}
% Use these for theorems, lemmas, proofs, etc.
\newtheorem{theorem}{Theorem}[lecnum]
\newtheorem{lemma}[theorem]{Lemma}
\newtheorem{proposition}[theorem]{Proposition}
\newtheorem{claim}[theorem]{Claim}
\newtheorem{corollary}[theorem]{Corollary}
\newtheorem{definition}[theorem]{Definition}
\newenvironment{proof}{{\bf Proof:}}{\hfill\rule{2mm}{2mm}}

% **** IF YOU WANT TO DEFINE ADDITIONAL MACROS FOR YOURSELF, PUT THEM HERE:

\begin{document}
%FILL IN THE RIGHT INFO.
%\lecture{**LECTURE-NUMBER**}{**DATE**}{**LECTURER**}{**SCRIBE**}
\lecture{3}{August 28}{Vatatmaja, Huang, and Lideros}{Keshav Deoskar}
%\footnotetext{These notes are partially based on those of Nigel Mansell.}


%%%%%%%%%%%%%%%%%%%%%%%%%%%%%%%%%%%%%%%%%%
%Additional commands

\newcommand{\ket}[1]{\mid #1 \rangle}
\newcommand{\bra}[1]{\langle #1 \mid}

%% Install amsfonts or amssymb package so that below command can be defined 
%\newcommand{\R}{\mathbb{R}

%%%%%%%%%%%%%%%%%%%%%%%%%%%%%%%%%%%%%%%%%%

% **** YOUR NOTES GO HERE:

% Some general latex examples and examples making use of the
% macros follow.  
%**** IN GENERAL, BE BRIEF. LONG SCRIBE NOTES, NO MATTER HOW WELL WRITTEN,
%**** ARE NEVER READ BY ANYBODY.

%The firist three lectures will review Point-Set Topology, with this one convering Metric Spaces.

%When we think of spaces like $\R$ o,r more relatably, the surface of the earth, there is a reasonable sense of \emph{distance} between two points. For example, the distance between Berkeley and San Francisco is 13.6 miles. Similarly. the distance between the numbers $5$ and $7$ is $2$. But what if one is asked to consider two elements or "points" in the set of functions $f : \R \mapsto \R$ and

\lecture{21}{September 27}{Vatatmaja, Huang, and Lideros}{Keshav Deoskar}

\section*{CW Complexes}

\begin{align*}
   L_z &= x_1 p_2 - x_2 p_1 \\
       &= \epsilon_{3ij}x_i p_j
\end{align*}


%\section*{References}
%\beginrefs
%\bibentry{AGM97}{\sc N.~Alon}, {\sc Z.~Galil} and {\sc O.~Margalit},
%On the Exponent of the All Pairs Shortest Path Problem,
%{\it Journal of Computer and System Sciences\/}~{\bf 54} (1997),
%pp.~255--262.

%\bibentry{F76}{\sc M. L. ~Fredman}, New Bounds on the Complexity of the 
%Shortest Path Problem, {\it SIAM Journal on Computing\/}~{\bf 5} (1976), 
%pp.~83-89.
%\endrefs

\lecture{0000}{October 9}{Vatatmaja, Huang, and Lideros}{Keshav Deoskar}

\section*{Review of some basic Knot Theory:}
What is a knot?

\underline{def:} A Knot is a smooth embedding from $K\;:\;S^1 \rightarrow S^3$ (note: should be curly arrow for embedding).
\\
\\
We consider knots to be embedded in $S^3$ because it is a compact space and that's nice. We study these objects up to smooth isotopy i.e. maps $S^1 \times I \rightarrow S^3$.
\\
\\
insert figure
\\
\\
How do we tell that both of these knots are really the same (i.e. the unknot?)
\\
\\
Further, how can we decide whether these two are both the trefoil? 
\\
\\
insert image
\\
\\
More generally, is there some machinery/algorithm that we can use to tell if two knots are the same?
\\
\\
Well, in fact, there is machinery which allos us to tell if two knots are really projections of the same knot. These are called \underline{\textbf{Reidemeister Moves}}
\\
\\
insert image of R1, R2, R3 -- Reidemeister moves.
\\
\\
\underline{\textbf{Theorem (essential):}} If two knots $K_1$ amd $K_2$ are isotopic, then there exists a sequence of Reidemeister moves from the projection of $K_1$ to the projection of $K_2$.

\textbf{Given a projection, Is there some sort of algorithm that tells us whether or not it is the unknot?}
\\
\\
There is one algorithm by Haken, but it's time complexity is super bad so there's no effective way to write a computer program to do this.

\section*{More Morse Theory:}
\begin{enumerate}
   \item One very basic \emph{Knot Invariant} is that given a knot, what is the minimum number of maximums.
   \item So, for example, anything with only $1$ maximum is the unknot.
   \item This invariant is called the \textbf{Bridge Number} of a knot.
\end{enumerate}

\section*{Seifert Genus:}
\begin{enumerate}
   \item Recall that a Seifert Surface of $K$ is an orientable, embedded surface $\Sigma \subset S^3$ such that the boundary of the surface is $K$ i.e. $\partial \Sigma = K$.
   
   \item There can be multiple different Seifert surfaces for the \underline{same knot} which have different genus, however there is a minimal genus.
   
   \item This \emph{minimal genus} is the \underline{Siefert Genus}.
\end{enumerate}
We can define the Genus of a surface $g(\Sigma)$ in terms of the \underline{\textbf{Euler Characteristic}, $\chi$}, as 
\[\boxed{ 2g(\Sigma) - 2 - \#b = \chi }\]

\textbf{Claim:} The Unknot is the only knot of genus zero.\\
\textbf{Proof:} Write later... Insert figure (note: Inverse function theorem used :sad:)

Can we classify genus 1 knots? There is no hope.

Insert image of the pretzel knot $P(3,-3,1)$.

\section*{One more stupid invariant: Tricolorability}

Note: This extends to p-colorability where $p$ is a prime.

We say a Knot $K$ is Tricolorable if we can color arcs of $K$ with exactly three colors. (arc of a knot if a segment that "goes behind") such that
\begin{enumerate}
   \item All three colors are used at least once.
   \item At each crossing, either all colors are the same or all colors are different.
\end{enumerate}

\begin{enumerate}
   \item Tricolorability is a Knot Invariant.
   \item This one allows us to differentiate between the trefoil knot and the unknot.
\end{enumerate}

\underline{\textbf{Claim:}} The Trefoil Knot is Tricolorable.

\underline{\textbf{Proof:}} Insert picture. Proof by image lmao. 

The trefoil knot is tricolorable while the unknot is not, so they cannot be the same knot.

\section*{A Knot Invariant That's Good at Detecting the Unknot:}
The invariant $v(K)$ is defined as 
\[ v(k) = 1 if unknot, 0 otherwise \]

Another invariant that is equally hard to compute is the Fundamental Group of the Knot Complement.
\[ \pi_1(S^2 \setminus k, x_0) = \mathbb{Z} if k is unknot, something else otherwise \]

This actually classifies all knots, but it gives us a presentation and determining whether two presentations are the same is really hard.

\lecture{0000}{October 11}{Vatatmaja, Huang, and Lideros}{Keshav Deoskar}

\subsection*{Review:}

The siefert genus of a knot is given by 
\[ g(k) = min\{genus(F) : F \text{is a siefert surface.} \} \]

\textbf{Qn: Do all knots have Siefert Surfaces?}

The answer is yes.

(Missed first few minutes and then was lost during this lecture; try to catch up after midterms.)

Recall that the Euler Characteristic of a given CW Complex is 
(Write from picture)
\vskip 0.5cm

\textbf{Note:} Composite knots are \underline{sums} of prime knots.

\textbf{What is a Knot Sum?}

Given $K_1, K_2$, look at the "unknotted spannimare".

\subsection*{Knot Genus is Additive}
That is, $g(K_1 + K_2) = g(K_1) + g(K_2)$

\textbf{Proof:}

First, we'll want to show that $g(K_1 + K_2) \leq g(K_1) + g(K_2)$ and then show the inequality in the other direction.

\underline{Forward Direction:} Construct Seifert Surface for $(K_1 + K_2)$ from minimal surfaces from $K_1$, $K_2$, $F_1$ seifer for $K_1$ and minimal, $F_2$ seifert for $K_2$ and minimal. (Attach images and expand on this direction of the prrof.)

\underline{Backward Direction:} Now we move to show that $g(K_1) + g(K_2) \leq g(K_1 + K_1)$

(Insert image)

We want to pick out a sphereical surface $\Sigma$ which separates the knots and then pick out any arc $\beta$ on the sphere's surface which intersects $K_1 + K_2$ only at two points and this intersection is "transverse".

Now, who's to say the Seifert surface $F$ doesn't run up and intersect the sphere? Perhaps it does. So, what we can say is 
\[ \Sigma \cap F = \beta + \text{a bunch of simple closed curves} \]

(Note that since all of our surfaces are embedded in $R^3$, we can perturb them slightly to ensure that all intersections are transverse -- this is why the intersections of the seifert surface are simple closed curves.)

lost for the rest of the lecture. understand it later.

\lecture{0001}{October 13}{Vatatmaja, Huang, and Lideros}{Keshav Deoskar}
Today we will cover another Knot invariant, which is a bit more useful/easy to work with than the invariants discussed thus far. This invariant is called the 
\emph{Alexander Polynomial, $\Delta_k(T)$}.

\section*{Some definitions:}
The Alexander Polynomial has multiple equivalent definitions
\begin{enumerate}
   \item Skein relation
   \item Seifert form
   \item Kauffman States
   \item $\pi_1(S^3 / K)$
   \item Cyclinc Branched Cover
\end{enumerate}

\section*{Skein Relation:}
This is a tool that helps us compute the Alexander Polynomial of a knot.
\vskip 0.5cm
insert figure
\vskip 0.5cm

We can define the Alexander Polynomial via the (recursive) Skein Relation with 
\begin{align*}
   &\Delta(L_{+}) - \Delta(L_{-}) = (t^{1/2} - t^{-1/2})\Delta(L_0) \\
   &\Delta_u(t) = 1
\end{align*}
where $\Delta_u(t)$ represents the Alexander Polynomial of the Unknot. 

For example,
\vskip 0.5cm
insert picture
\vskip 0.5cm

And another more complicated example:
\vskip 0.5cm
insert picture
\vskip 0.5cm
(In this example, $L_{-}$ is the unknot and $L_0$ is the Hopf Link.)

For this knot, the Alexander Polynomial is given by the relation
\begin{align*}
   \Delta(L_+) &= -(t^{1/2} - t^{-1/2})^2 + 1 \\
               &= -t - t^{-1} + 3
\end{align*}
So far, in our examples, we've been able to obtain the Alexander Polynomial by considering just one crossing but there are cases where we may need to consider more. (look up such examples)
\\
\\
However, \emph{showing} that just considering one crossing was \emph{enough} is a difficult matter.
\\
\\
\textbf{Note:} Alexander Polynomials are well defined once we are allowed to multiply by $t^n$. Sort of. More on this next lecture. 

\section*{Seifert Forms:}
(For more rigor, see Lickorish Ch. 6)

Consider some knot $K$ and its Seifert Surface $F$. We're going to look at \emph{loops} in the Seifert Surface to obtain \emph{linking numbers} which give us the \emph{Seifert Form} -- from which we can finally obtain the Alexander Polynomial.

\subsection*{Linking Number:}
Given two oriented Links $L$, $L'$ the linking number if the sum of \underline{\emph{linkings with sign}.}

For example, the linking number of the two unknots below is $+2$:
\vskip 0.5cm
insert picture
\vskip 0.5cm

\textbf{Def:}

Given a seifert surface $F$ with generators $f_i$ of $H_1(F, \mathbb{Z})$, the seifert form $A_{ij}$ is the matrix with entries
\[ lk(f_i, f_j^+) = A_ij \]

where $lk$ represents the linking number.

Had to leave early; get notes from someone else for last 10 minutes.
\\
\\
Missed lecture on monday and wednesday(15, 17 Oct)
\lecture{4343}{October 20}{Vatatmaja, Huang, and Lideros}{Keshav Deoskar}

\section*{The Jones Polynomial:}

Similarly to the Alexander Polynomail, Jones polynomials can be defined usnig Seni Relations. More specifcially, we use something called the \emph{Kauffman Bracket}


\subsection*{The Kauffman Bracket:}
A Kauffman Bracekt is a function from an unorentated links diagram to a Laurent Polynomial $K[t, t^{-1}]$.

For a diagram $D$, the Kauffman Bracket $ \langle D \rangle$ is tis polynomal.

\begin{enumerate}
   \item $\langle 0 \rangle$ = 1 where $0$ represents the unknot.
\end{enumerate}

read the book and add notes later.

Note: Jones polynomial distinguishes between knot and its mirror image, but the Alexander polynomial does not.

\subsection*{Jones Polynomials in TQFTs:}
Knots can be generalized to 3-manifolds -- we keep the Kauffman Bracket around but with some conditions modified/relaxed.

Also, 3-manifolds can be built from Knots.

\lecture{4344}{October 23}{Vatatmaja, Huang, and Lideros}{Keshav Deoskar}
Not paying attention today; figure out what the lecture was later.

\lecture{4343}{October 25}{Vatatmaja, Huang, and Lideros}{Keshav Deoskar}
\\
Today, we cover  Surgery Theory:

\subsection*{Topics for today:}
\begin{enumerate}
   \item Surgery
   \item Dehn Surgery
   \item Lens Surgery
\end{enumerate}

Notation: $S^{-1} = \emptyset = \partial D^0$

\section*{Surgery:}
We want to systtematcally build new manifolds using existing manifolds. \emph{Surgery} allows us to do this by creating cuts and sewing thnigs back together.

\textbf{Def:}

-Let $M$ be an n-dimensioinal manifold. An $S^k$ sphere ($-1 \leq k \leq n$) specify a \textbf{\emph{framing}} f.

The pair $(\phi, f)$ determnies an embeddnig $\hat{\phi} : S^k \times D^{n-k} \rightarrow M$

(Look up the technical definition of framing; essentially it tells us if we have twists in our manifold.)

-Surgery on this pair $(\phi, f)$ is the process of removing $\hat{\phi}(S^k \times D^{n-k})$ and glue back a copy of $D^{k+1} \times S^{n-k-1}$ via $\phi$. (The thing we removed and the thing we glued back have the same boundary by construction).

\subsection*{Example:} $S^1$ surgery on a torus

Insert picture and explain

Now, within the realm of Knot theory, there is a more specific type of surgery we can employ calld \emph{Dehn Surgery}.

\section*{Dehn Surgery:}
Given a Knot (or Link) $K \subset S^3$, take the (tubular) neighborhood $N(k)$ of the knot, remove from $S^3$ the solid torus $Q = S^3 - N(k)$. Then, we glue back a copy of $D^2 \times S^1$ by some homeomorphism $h : T^2 \rightarrow T^2$. i.e. specify where the meridian of the torus goes as $m \mapsto p\mu + q\lambda$ 

Here, $m$ is the meridian of the original $T^2$, $\lambda$ is the unique longitude such that $lk(\lambda, k) = 0$ for the image $T^2$.
(insert picture)

Call $N = Q \cup_h D^2 \times S^1 = S^{3}_{p/q} (k)$ where $S^{3}_{p/q} (k)$ is called the Dehn Surgery on $k$ with slope $p/q \in \mathbb{Q} \cup \{\infty\}$.

\section*{Lens Space:}

On friday, we'll talk about the following BIG theorem by Lickorish and Wallace:

\textbf{Any closed, orientable 3-manifold is obtained by Dehn Surgery on a link $\mathcal{L} \in S^3$ with $\pm 1$ slope.}


This theorem classifies \emph{all closed, orientable 3-manifolds} based on results of surgery. This is super deep!! 

\subsection*{Cosmetic Surgery:} 
We say a knot $K$ admits cosmetic surgeries if $S^3_r (k) \cong S^3_{r'} (k)$ for $r \neq r'$. 

Conjecture: No cosmetic surgeries.

\lecture{4345}{October 27}{Vatatmaja, Huang, and Lideros}{Keshav Deoskar}

Didn't pay attention in lecture today but it was about the Fundamental Theorem of Lcikorish and Wallace.

Try to figure out the material later.

Things that were mentioned:

- Twist Equivalence\\
- 'Separating' curves

\lecture{434565}{October 30}{Vatatmaja, Huang, and Lideros}{Keshav Deoskar}
For the rest of the semester, no set topics. We'll see how it goes.

\textbf{Dependence Chart}:

insert picture


Today, we'll discuss Handlebody theory:
\section*{Handlebody Theory}

\textbf{Def:}

Let $0 \leq k \leq n$. Let $X$ be an n-manifold. An n-dimensional handle is a copy of $D^ \times D^{n-k}$ attached to $\partial X$ along $\partial D^k \times D^{n-k} \cong $ complete these notes later.

\end{document}





