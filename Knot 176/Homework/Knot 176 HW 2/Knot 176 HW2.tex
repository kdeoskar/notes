%
% This is the LaTeX template file for lecture notes for CS294-8,
% Computational Biology for Computer Scientists.  When preparing 
% LaTeX notes for this class, please use this template.
%
% To familiarize yourself with this template, the body contains
% some examples of its use.  Look them over.  Then you can
% run LaTeX on this file.  After you have LaTeXed this file then
% you can look over the result either by printing it out with
% dvips or using xdvi.
%
% This template is based on the template for Prof. Sinclair's CS 270.

\documentclass[twoside]{article}
\usepackage{graphics}
\usepackage{mathtools}
\usepackage[]{mdframed}
\usepackage{amsmath}
\usepackage{amsfonts}
\usepackage{enumitem}
%\usepackage{asmfonts}
\setlength{\oddsidemargin}{0.25 in}
\setlength{\evensidemargin}{-0.25 in}
\setlength{\topmargin}{-0.6 in}
\setlength{\textwidth}{6.5 in}
\setlength{\textheight}{8.5 in}
\setlength{\headsep}{0.75 in}
\setlength{\parindent}{0 in}
\setlength{\parskip}{0.1 in}

%
% The following commands set up the lecnum (lecture number)
% counter and make various numbering schemes work relative
% to the lecture number.
%
\newcounter{lecnum}
\renewcommand{\thepage}{\thelecnum-\arabic{page}}
\renewcommand{\thesection}{\thelecnum.\arabic{section}}
\renewcommand{\theequation}{\thelecnum.\arabic{equation}}
\renewcommand{\thefigure}{\thelecnum.\arabic{figure}}
\renewcommand{\thetable}{\thelecnum.\arabic{table}}

%
% The following macro is used to generate the header.
%
\newcommand{\lecture}[4]{
   \pagestyle{myheadings}
   \thispagestyle{plain}
   \newpage
   \setcounter{lecnum}{#1}
   \setcounter{page}{1}
   \noindent
   \begin{center}
   \framebox{
      \vbox{\vspace{2mm}
    \hbox to 6.28in { {\bf Knot 176: Introduction to Low-Dimensional Topology
                        \hfill Fall 2023} }
       \vspace{4mm}
       \hbox to 6.28in { {\Large \hfill Homework #1 \hfill} }
       \vspace{2mm}
       \hbox to 6.28in { {\it Lecturers: #3 \hfill #4} }
      \vspace{2mm}}
   }
   \end{center}
   \markboth{Lecture #1: #2}{Lecture #1: #2}
   {\bf Disclaimer}: {\it LaTeX template courtesy of the UC Berkeley EECS Department.}
   \vspace*{4mm}
}

%
% Convention for citations is authors' initials followed by the year.
% For example, to cite a paper by Leighton and Maggs you would type
% \cite{LM89}, and to cite a paper by Strassen you would type \cite{S69}.
% (To avoid bibliography problems, for now we redefine the \cite command.)
% Also commands that create a suitable format for the reference list.
\renewcommand{\cite}[1]{[#1]}
\def\beginrefs{\begin{list}%
        {[\arabic{equation}]}{\usecounter{equation}
         \setlength{\leftmargin}{2.0truecm}\setlength{\labelsep}{0.4truecm}%
         \setlength{\labelwidth}{1.6truecm}}}
\def\endrefs{\end{list}}
\def\bibentry#1{\item[\hbox{[#1]}]}

%Use this command for a figure; it puts a figure in wherever you want it.
%usage: \fig{NUMBER}{SPACE-IN-INCHES}{CAPTION}
\newcommand{\fig}[3]{
			\vspace{#2}
			\begin{center}
			Figure \thelecnum.#1:~#3
			\end{center}
	}
% Use these for theorems, lemmas, proofs, etc.
\newtheorem{theorem}{Theorem}[lecnum]
\newtheorem{lemma}[theorem]{Lemma}
\newtheorem{proposition}[theorem]{Proposition}
\newtheorem{claim}[theorem]{Claim}
\newtheorem{corollary}[theorem]{Corollary}
\newtheorem{definition}[theorem]{Definition}
\newenvironment{proof}{{\bf Proof:}}{\hfill\rule{2mm}{2mm}}

% **** IF YOU WANT TO DEFINE ADDITIONAL MACROS FOR YOURSELF, PUT THEM HERE:

\begin{document}
%FILL IN THE RIGHT INFO.
%\lecture{**LECTURE-NUMBER**}{**DATE**}{**LECTURER**}{**SCRIBE**}
%\footnotetext{These notes are partially based on those of Nigel Mansell.}


%%%%%%%%%%%%%%%%%%%%%%%%%%%%%%%%%%%%%%%%%%
%Additional commands

\newcommand{\ket}[1]{\mid #1 \rangle}
\newcommand{\bra}[1]{\langle #1 \mid}
\newcommand{\R}{\mathbb{R}}

%% Install amsfonts or amssymb package so that below command can be defined 
%\newcommand{\R}{\mathbb{R}

%%%%%%%%%%%%%%%%%%%%%%%%%%%%%%%%%%%%%%%%%%

% **** YOUR NOTES GO HERE:

%%%%%%%%%%%%%%%%%%%%%%%%%%%%%%%%%%%%%%%%%%
%%%%%         HOMEWORK 1
%%%%%%%%%%%%%%%%%%%%%%%%%%%%%%%%%%%%%%%%%%
\lecture{2}{September 08}{Vatatmaja, Huang, and Lideros}{Keshav Deoskar}


% Some general latex examples and examples making use of the
% macros follow.  
%**** IN GENERAL, BE BRIEF. LONG SCRIBE NOTES, NO MATTER HOW WELL WRITTEN,
%**** ARE NEVER READ BY ANYBODY.

\section*{\underline{Compactness}}

\textbf{Q2.1.1:} Suppose $(X, \tau_{X})$ and $(Y, \tau_{Y})$ are two compact topological spaces. Show that $X \times Y$ equipped with the product topology is also compact. 
\\
\\
\textbf{Proof:} 
Given two topological spaces $(X, \tau_{X})$ and $(Y, \tau_{Y})$, their product $X \times Y$ is a topological space when endowed with the product topology, which has a basis $\mathcal{B}$ given as
\[ \mathcal{B} = \{ U_i \times V_j \;\;|\;\; U_i \subset \tau_X, V_j \subset \tau_Y \} \]
\\
\\
In order to show that $X \times Y$ inherits compactness from $X$ and $Y$, it would be useful to study how $X \times Y$ can be "decomposed" in terms of the orginial spaces, so let's think about projections. 
\\
\\
\subsection*{Projection Maps}
Consider the projection map 
$\Pi_{X} : X \times Y \rightarrow X $ given by $(x,y) \mapsto y$. 
\\
\\
Then, if $Z$ is some open set in $X \times Y$, 
\[ \Pi_{X}(Z) = U \]
where $U$ is some open set in $X$. 
\\
\\
Why is this? Because any open set (such as) $Z \in X \times Y$ can be written as a (arbitrary) union or (finite) intersection of the basis sets, and the basis sets of the product topology for $X \times Y$ look like $U_{i} \times V_{j}$ where $U_i \in \tau_X$ and $V_i \in \tau_Y$.

For any sets $U_i$ open in $X$ and $V_j$ open in $Y$, we have $\Pi_X(U_i \times V_j) = U_i$, and clearly, the map respects unions and intersection in that 
\begin{align*}
   \Pi_X\left(\bigcup_{i \in I} U_i \times V_i \right) &= \Pi_X (\{ (x \times y) : \text{x and y are in some } U_i \text{ and } V_i \}) \\
   &= \{x : x \in U_i \text{ for some } i \in I\} \\
   &= \bigcup_{i \in I} U_i
\end{align*}
and 
\begin{align*}
   \Pi_X\left(\bigcap_{i \in I} U_i \times V_i \right) &= \Pi_X (\{ (x \times y) : \text{x and y are in all } U_i \text{ and } V_i, i \in I \}) \\
   &= \{x : x \in U_i \text{ for all } i \in I\} \\
   &= \bigcap_{i \in I} U_i
\end{align*}

The open set $Z$ in $X \times Y$ can be expressed as some union or finite intersection of the basis sets $U_i \times V_j$, so its projection $\Pi_X(Z)$ simply extracts the sets $U_i$ from the basis sets $U_i \times V_j$ which are used to "build" Z via intersections and unions. After extracting these sets, it unions / finitely intersects them together, giving us some open set in $X$.
\\
\\
So, \underline{the projection map $\Pi_X((x,y)) = x$ is an open map.}
\\
\\
Similarly, \underline{the map $\Pi_Y((x,y)) = y$ is also an open map.}
\\
\\
\subsection*{Showing $X \times Y$ is compact}
Now, suppose $\{Z_{\alpha}\}$ is an open cover of $X \times Y$. Pick some $x_0 \in X$ and let 
\[ A_{x_0} = \{ \alpha \;\;|\;\; (x_0, y) \in Z_{\alpha} \} \]

Since $\{Z_{\alpha}\}$ covers $X \times Y$, we must have $\Pi_Y(\{Z_{\alpha}\})_{\alpha \in \mathcal{A}_{x_0}}$ covering $Y$. So, there is some finite subcover $\Pi_Y(\{Z_{\alpha}\})_{\alpha \in I_{x_0}}$ where $I_{x_0}$ is finite. Then, there is some $W_{x_0} \in X$ such that $W_{x_{0}} \times Y$ is covered by  $\{Z_{\alpha}\}_{\alpha \in I_{x_0}}$.
\\
\\
The sets $\{W_{x_0}\}_{x_0 \in X}$ form an open cover of $X$, hence there is a finite subcover of $\{W_{x_0}\}_{x_0 \in F}$ where $F \subset X$ is finite.
\\
\\
So, the collection $\{Z_{\alpha}\}_{\alpha \in \mathcal{A}_{x_0}, x_0 \in F}$ is finite and covers the entire space $X \times Y$. Thus, $X \times Y$ equipped with the product topology is compact!
\vskip 0.25cm
\hrule
\vskip 1cm 

\textbf{Q2.1.1:} The interval $I = [0, 1]$ is a compact space.
\\
\\
\textbf{Proof:} 

\vskip 0.25cm
\hrule
\vskip 1cm 

% \textbf{Q1.2.2:} Let $X$ be a set, and $\sigma \subset \mathcal{P}(X)$ be a basis. Show that the set of arbitrary unions of elements of $\sigma$ form a topology on $X$. 
% \\
% \\
% \textbf{Proof:} We know that $\sigma$ is a base of $X$. That is, for each $x \in X$
% \begin{enumerate}[label=(\alph*)]
%    \item There exists $A \in \sigma$ such that $x \in A$.
%    \item If $x \in A \cap B$, for $A,B \in \sigma$, then there exists a $C \in \sigma$ such that $x \in C$ and $C \subset A \cap B$. 
% \end{enumerate}
% Let's denote an arbitrary union of elements of $\sigma$ as 
% \[ U_{\alpha} = \bigcup_{i \in I_{\alpha}} U_i\] where each $U_i \in \sigma$.

% We want to show that the \underline{\textbf{set of arbitrary unions}}, $\{U_\alpha\}_{\alpha \in A} = \tau$, forms a topology on $X$. That is,
% \begin{enumerate}[label=(\alph*)]
%    \item Arbitrary unions of elements in $\tau$ are also in $\tau$.
%    \item Finite intersections of elements in $\tau$ are also in $\tau$.
%    \item $X$ and $\emptyset$ are in $\tau$.
% \end{enumerate}
% Let's show that the conditions hold.
% \begin{enumerate}[label=(\alph*)]
%    \item Elements of $\tau$ (which we denote as $U_{\alpha}$) are themselves arbitrary unions of elements of $\sigma$. So, arbitrary unions of these $U_{\alpha}$'s are arbitrary unions of arbitrary unions of elements of $\sigma$ -- which are of course, arbitrary unions of elements of $\sigma$! So, the first condition holds.
   
%    \item Now consider a finite intersection of elements in $\tau$, 
%    \begin{align*}
%       V &= U_{\alpha_1} \cap \cdots \cap U_{\alpha_n} \\
%         &= \left( \bigcup_{i \in I_{\alpha_{1}}} U_i \right) \cap \left( \bigcup_{i \in I_{\alpha_{2}}} U_i \right) \cap \cdots \cap \left( \bigcup_{i \in I_{\alpha_{n-1}}} U_i \right) \cap \left( \bigcup_{i \in I_{\alpha_{n}}} U_i \right) \\
%    \end{align*}
%    where each $U_i \in \sigma$.

%    Then, by distributivity of intersection of sets, we can express $V$ as a union of many intersections the $U_i$'s i.e. as a union of intersections of elements of $\sigma$. 
%    i.e.

%    Consider an arbitrary intersection of elements $A_j \in \sigma$, $j \in J$
%    \begin{align*}
%       V' &= \bigcap_{j \in J} A_j \\
%          &= A_{j_1} \cap A_{j_2} \cap A_{j_2} \cdots \\
%          &= (A_{j_1} \cap A_{j_2}) \cap A_{j_2} \cdots
%    \end{align*} 
   

% \end{enumerate}

% \vskip 0.25cm
% \hrule
% \vskip 1cm

%\section*{References}
%\beginrefs
%\bibentry{AGM97}{\sc N.~Alon}, {\sc Z.~Galil} and {\sc O.~Margalit},
%On the Exponent of the All Pairs Shortest Path Problem,
%{\it Journal of Computer and System Sciences\/}~{\bf 54} (1997),
%pp.~255--262.

%\bibentry{F76}{\sc M. L. ~Fredman}, New Bounds on the Complexity of the 
%Shortest Path Problem, {\it SIAM Journal on Computing\/}~{\bf 5} (1976), 
%pp.~83-89.
%\endrefs


\end{document}





