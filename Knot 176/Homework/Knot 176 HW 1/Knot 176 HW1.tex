%
% This is the LaTeX template file for lecture notes for CS294-8,
% Computational Biology for Computer Scientists.  When preparing 
% LaTeX notes for this class, please use this template.
%
% To familiarize yourself with this template, the body contains
% some examples of its use.  Look them over.  Then you can
% run LaTeX on this file.  After you have LaTeXed this file then
% you can look over the result either by printing it out with
% dvips or using xdvi.
%
% This template is based on the template for Prof. Sinclair's CS 270.

\documentclass[twoside]{article}
\usepackage{graphics}
\usepackage{mathtools}
\usepackage[]{mdframed}
\usepackage{amsmath}
\usepackage{amsfonts}
\usepackage{enumitem}
%\usepackage{asmfonts}
\setlength{\oddsidemargin}{0.25 in}
\setlength{\evensidemargin}{-0.25 in}
\setlength{\topmargin}{-0.6 in}
\setlength{\textwidth}{6.5 in}
\setlength{\textheight}{8.5 in}
\setlength{\headsep}{0.75 in}
\setlength{\parindent}{0 in}
\setlength{\parskip}{0.1 in}

%
% The following commands set up the lecnum (lecture number)
% counter and make various numbering schemes work relative
% to the lecture number.
%
\newcounter{lecnum}
\renewcommand{\thepage}{\thelecnum-\arabic{page}}
\renewcommand{\thesection}{\thelecnum.\arabic{section}}
\renewcommand{\theequation}{\thelecnum.\arabic{equation}}
\renewcommand{\thefigure}{\thelecnum.\arabic{figure}}
\renewcommand{\thetable}{\thelecnum.\arabic{table}}

%
% The following macro is used to generate the header.
%
\newcommand{\lecture}[4]{
   \pagestyle{myheadings}
   \thispagestyle{plain}
   \newpage
   \setcounter{lecnum}{#1}
   \setcounter{page}{1}
   \noindent
   \begin{center}
   \framebox{
      \vbox{\vspace{2mm}
    \hbox to 6.28in { {\bf Knot 176: Introduction to Low-Dimensional Topology
                        \hfill Fall 2023} }
       \vspace{4mm}
       \hbox to 6.28in { {\Large \hfill Homework #1: Due Date: #2  \hfill} }
       \vspace{2mm}
       \hbox to 6.28in { {\it Lecturers: #3 \hfill Scribe: #4} }
      \vspace{2mm}}
   }
   \end{center}
   \markboth{Lecture #1: #2}{Lecture #1: #2}
   {\bf Disclaimer}: {\it LaTeX template courtesy of the UC Berkeley EECS Department.}
   \vspace*{4mm}
}

%
% Convention for citations is authors' initials followed by the year.
% For example, to cite a paper by Leighton and Maggs you would type
% \cite{LM89}, and to cite a paper by Strassen you would type \cite{S69}.
% (To avoid bibliography problems, for now we redefine the \cite command.)
% Also commands that create a suitable format for the reference list.
\renewcommand{\cite}[1]{[#1]}
\def\beginrefs{\begin{list}%
        {[\arabic{equation}]}{\usecounter{equation}
         \setlength{\leftmargin}{2.0truecm}\setlength{\labelsep}{0.4truecm}%
         \setlength{\labelwidth}{1.6truecm}}}
\def\endrefs{\end{list}}
\def\bibentry#1{\item[\hbox{[#1]}]}

%Use this command for a figure; it puts a figure in wherever you want it.
%usage: \fig{NUMBER}{SPACE-IN-INCHES}{CAPTION}
\newcommand{\fig}[3]{
			\vspace{#2}
			\begin{center}
			Figure \thelecnum.#1:~#3
			\end{center}
	}
% Use these for theorems, lemmas, proofs, etc.
\newtheorem{theorem}{Theorem}[lecnum]
\newtheorem{lemma}[theorem]{Lemma}
\newtheorem{proposition}[theorem]{Proposition}
\newtheorem{claim}[theorem]{Claim}
\newtheorem{corollary}[theorem]{Corollary}
\newtheorem{definition}[theorem]{Definition}
\newenvironment{proof}{{\bf Proof:}}{\hfill\rule{2mm}{2mm}}

% **** IF YOU WANT TO DEFINE ADDITIONAL MACROS FOR YOURSELF, PUT THEM HERE:

\begin{document}
%FILL IN THE RIGHT INFO.
%\lecture{**LECTURE-NUMBER**}{**DATE**}{**LECTURER**}{**SCRIBE**}
%\footnotetext{These notes are partially based on those of Nigel Mansell.}


%%%%%%%%%%%%%%%%%%%%%%%%%%%%%%%%%%%%%%%%%%
%Additional commands

\newcommand{\ket}[1]{\mid #1 \rangle}
\newcommand{\bra}[1]{\langle #1 \mid}
\newcommand{\R}{\mathbb{R}}

%% Install amsfonts or amssymb package so that below command can be defined 
%\newcommand{\R}{\mathbb{R}

%%%%%%%%%%%%%%%%%%%%%%%%%%%%%%%%%%%%%%%%%%

% **** YOUR NOTES GO HERE:

%%%%%%%%%%%%%%%%%%%%%%%%%%%%%%%%%%%%%%%%%%
%%%%%         HOMEWORK 1
%%%%%%%%%%%%%%%%%%%%%%%%%%%%%%%%%%%%%%%%%%
\lecture{1}{September 08}{Vatatmaja, Huang, and Lideros}{Keshav Deoskar}


% Some general latex examples and examples making use of the
% macros follow.  
%**** IN GENERAL, BE BRIEF. LONG SCRIBE NOTES, NO MATTER HOW WELL WRITTEN,
%**** ARE NEVER READ BY ANYBODY.

\section*{\underline{Metric Spaces}}

\textbf{Q1.1.1:} Prove that for any metric space $(Z, d_Z)$, a function between the metric spaces $f : (\R^n, d_E) \rightarrow (Z, d_Z)$ is continuous if and only if $f : (\R^n, d_T) \rightarrow (Z, d_Z)$ is continuous.
\\
\\
\textbf{Proof:} Recall that a function between two metric spaces $f : (X, d_X) \rightarrow (Y, d_Y)$ is said to be continuous if for any open set $V \subseteq Y$, the pre-image $f^{-1}(V)$ is open in $X$. 

\underline{Forward Direction:} Suppose we have a continuous function $f : (\R^n, d_E) \rightarrow (Z, d_Z)$. That means, for any open set $V \subset Z$ (wrt $d_Z$), the pre-image $f^{-1}(V) \subset \R^n$ is open with respect to the Euclidean Metric. 

That is, for any point $p \in f^{-1}(V)$, there exists an open ball $B(p, r_p)$ centered around $p$ with radius $r_p$ such that $B(p, r_p) \subset f^{-1}(V)$ where 
\[ B(p, r_p) = \left\{ y \in \R^n : \sqrt{\sum_{i = 1}^{n} (p_i - y_i)^2} < r_p  \right\} \]

We have a set of points $y$ for which
\begin{align*}
   \sqrt{\sum_{i = 1}^{n} (p_i - y_i)^2} &< r_p \\
   \implies \sum_{i = 1}^{n} (p_i - y_i)^2 &< r_p^2 \\ 
\end{align*}
But notice that $(p_i - y_i)^2 = | p_i - y_i |^2$. So,
\begin{align*}
   \sum_{i = 1}^{n} |p_i - y_i|^2 &< r_p^2 
\end{align*}
But again, notice that $|p_i - y_i| \leq |p_i - y_i|^2$. So,
\[ \boxed{\sum_{i = 1}^{n} |p_i - y_i| < r_p^2 } \]

So, around any point $p \in f^{-1}(V)$, if there exists an open ball $B_{E} (p, r_p)$ then there also exists an open ball $B_{T} (p, r_p^2)$ -- which means that if a set $f^{-1}(V)$ is open with respect to the Euclidean Metric, it is also open with respect to the Taxicab Metric.

Thus, if $f : (\R^n, d_E) \rightarrow (Z, d_Z)$ is continuous, then $f : (\R^n, d_T) \rightarrow (Z, d_Z)$ is also continuous.
\vskip 0.5cm 

\underline{Reverse Direction:} Now suppose we have a continuous function $f : (\R^n, d_T) \rightarrow (Z, d_Z)$. That means, for any open set $V \subset Z$ (wrt $d_Z$), the pre-image $f^{-1}(V) \subset \R^n$ is open with respect to the Taxicab Metric. 

That is, for any point $q \in f^{-1}(V)$, there exists an open ball $B(q, r_q)$ centered around $q$ with radius $r_q$ such that $B(q, r_q) \subset f^{-1}(V)$ where 
\[ B(q, r_q) = \left\{ y \in \R^n : \sum_{i = 1}^{n} |q_i - y_i| < r_q  \right\} \]

So we have a set of points $y$ for which
\begin{align*}
   &\sum_{i = 1}^{n} |q_i - y_i| < r_q \\
   \implies &\left( \sum_{i = 1}^{n} |q_i - y_i| \right)^2 < r_q^2 \\
   \implies &\left(|q_1 - y_1| + \dots + |q_n - y_n| \right)^2 < r_q^2 \\
   \implies &\sum_{i = 1}^{n} |q_i - y_i|^2 + \sum_{i,j < n} |q_i - y_i|\cdot|q_j - y_j| < r_q^2 \\
   \implies &\sum_{i = 1}^{n} |q_i - y_i|^2 < r_q^2
\end{align*}
But once again, $|q_i - y_i|^2 = (q_i - y_i)^2$. So,
\[ \boxed{\sum_{i = 1}^{n} (q_i - y_i)^2 < r_q^2} \]

Therefore, for any point $q \in f^{-1}(V)$, if there is an open ball with respect to the Taxicab metric $B_T(q, r_q)$ there is also an open ball with respect to the Euclidean metric $B_E(q, r_q^2)$. That is, if a set is open in the Taxicab Metric it is also open in the Euclidean Metric.
\\
\\
As a result, if a function is continuous with respect to the Taxicab metric, it is also continuous with respect to the Euclidean Metric.
\vskip 0.25cm
\hrule
\vskip 1cm
 
\textbf{Q1.1.2:} Consider the following definition of continuity: a map between metric spaces $f : X \rightarrow Y$ is continuous if for every $x \in X$ and every $\epsilon > 0$, there exists a $\delta > 0$ such that $f(B(x, \delta)) \subset B(f(x), \epsilon)$. Prove that a map is continuous in this sense if and only if for every open set $V \subset Y$ we have $f^{-1}(V)$ is open in $X$.
\\
\\
\textbf{Proof:}
\\
\\
\underline{\textbf{"Forward" direction:}} Suppose we have a function which is continuous in the $\epsilon-\delta$ sense. Now, suppose for contradiction that we have an open set $V \subset Y$ such that $f^{-1}(V)$ is not open in $X$. 
\\
\\
That means, there is some point $f(x_0) \in V$ whose pre-image $x_0 \in f^{-1}(V)$ does not have an open ball around it which is contained in $f^{-1}(V)$. That is, there is no $\delta$ such that $B(x_0, \delta) \subset f^{-1}(V)$, so there is no $\delta$ such that $f(B(x_0,\delta)) \subset V$, which means that there exist $\epsilon > 0$ for which there are no $\delta$ such that $f(B(x_0, \delta)) \subset B(f(x_0), \epsilon)$ -- but this contradicts our assumption! So, the $\epsilon-\delta$ definition of continuity implies the Open Pre-image definition of continuity.
\\
\\
\underline{\textbf{"Reverse" direction:}} Suppose the function $f : X \rightarrow Y$ is continuous in the sense that for every open set $V \in Y$, the pre-image $f^{-1}(V)$ is open in $X$. 
\\
\\
Now, let's fix a point $x \in X$ and some $\epsilon > 0$. Then, $B_Y (f(x_0), \epsilon)$ -- the epsilon ball around $f(x_0)$ -- is an open subset of $Y$.

Since $f$ is continuous, if $B_Y (f(x_0), \epsilon)$ is open in $Y$ then $f^{-1}(B_Y (f(x_0), \epsilon))$ is open in $X$. 

Now, we also know that $x_0 \in f^{-1}(B_Y (f(x_0), \epsilon))$ and since $f^{-1}(B_Y (f(x_0), \epsilon))$ is open -- so by the definition of an open set in a metric space, there exists an open ball with some  radius $\delta > 0$ containing $x$ such that $B_X(x, \delta) \subset f^{-1}(B_Y (f(x_0), \epsilon))$
\\
\\
That is, 
\[ \boxed{f(B_X(x, \delta)) \subset B_Y (f(x_0), \epsilon)} \]

So, the open-set definition of continuity implies the $\epsilon-\delta$ definition of continuity for maps between metric spaces.
\vskip 0.25cm
\hrule
\vskip 1cm

\section*{\underline{Topologies}}

\textbf{Q1.2.1:} Let $\{ \tau_i \}_{i \in I}$ be a collection of topologies on $X$. Show that $\cap_{i \in I} \tau_i$ is a topology on $X$.
\\
\\
\textbf{Proof:} In order to show that $\mathcal{O} = \cap_{i \in I} \tau_i$ is a topology on $X$, we need to show that 
\begin{enumerate}[label=(\alph*)]
   \item Arbitrary unions of sets in $\mathcal{O}$ are also in $\mathcal{O}$.
   \item Finite intersections of sets in $\mathcal{O}$ are in $\mathcal{O}$.
   \item $X$ and $\emptyset$ are in $\mathcal{O}$.
\end{enumerate}
Let's show that these conditions are satisfied:
\begin{enumerate}[label=(\alph*)]
   \item The collection $\mathcal{O}$ consists of all the sets that are common to all of $\{\tau_i\}_{i \in I}$. Consider an arbitrary union of open sets $ A_j \in \mathcal{O}$.

   Now, each of these sets $A_j$ is open in every one of $\tau_i$, the union $\cup_{j} A_j$ is also open in each one of $\tau_i$ (since each $\tau_i$ is itself a topology).
   
   Therefore, the union $\cup_{j} A_j$ is an open set which is common to all of $\tau_i$. Therefore,
   \[ \bigcup_{j} A_j \in \mathcal{O} \]

   \item Consider a finite intersection of open sets in the intersection $V = A_1 \cap A_2 \cap \cdots \cap A_n$, where $A_j \in \mathcal{O}$ i.e. each of $A_j$ is common to every $\tau_i$. Then, since each $\tau_i$ is a topology and $A_j$ are open sets in each $\tau_i$, their finite intersection $V$ is also an open set in each $\tau_i$. Now, since $V \in \tau_{i}$ for all $i \in I$, we have 
   \[ V \in \bigcap_{i \in I} \tau_i = \mathcal{O} \]

   \item The empty set is, by definition, an element of every set so 
   \[ \emptyset \in \mathcal{O} \]

   And since every $\tau_i$ is a topology, we have $X \in \tau_i$ for all $i \in I$. Thus,
   \[ X \in \bigcap_{i \in I} \tau_i = \mathcal{O}\]
\end{enumerate}
Hence, the intersection of a family of topologies on $X$ is itself a topology on $X$.

\vskip 0.25cm
\hrule
\vskip 1cm

% \textbf{Q1.2.2:} Let $X$ be a set, and $\sigma \subset \mathcal{P}(X)$ be a basis. Show that the set of arbitrary unions of elements of $\sigma$ form a topology on $X$. 
% \\
% \\
% \textbf{Proof:} We know that $\sigma$ is a base of $X$. That is, for each $x \in X$
% \begin{enumerate}[label=(\alph*)]
%    \item There exists $A \in \sigma$ such that $x \in A$.
%    \item If $x \in A \cap B$, for $A,B \in \sigma$, then there exists a $C \in \sigma$ such that $x \in C$ and $C \subset A \cap B$. 
% \end{enumerate}
% Let's denote an arbitrary union of elements of $\sigma$ as 
% \[ U_{\alpha} = \bigcup_{i \in I_{\alpha}} U_i\] where each $U_i \in \sigma$.

% We want to show that the \underline{\textbf{set of arbitrary unions}}, $\{U_\alpha\}_{\alpha \in A} = \tau$, forms a topology on $X$. That is,
% \begin{enumerate}[label=(\alph*)]
%    \item Arbitrary unions of elements in $\tau$ are also in $\tau$.
%    \item Finite intersections of elements in $\tau$ are also in $\tau$.
%    \item $X$ and $\emptyset$ are in $\tau$.
% \end{enumerate}
% Let's show that the conditions hold.
% \begin{enumerate}[label=(\alph*)]
%    \item Elements of $\tau$ (which we denote as $U_{\alpha}$) are themselves arbitrary unions of elements of $\sigma$. So, arbitrary unions of these $U_{\alpha}$'s are arbitrary unions of arbitrary unions of elements of $\sigma$ -- which are of course, arbitrary unions of elements of $\sigma$! So, the first condition holds.
   
%    \item Now consider a finite intersection of elements in $\tau$, 
%    \begin{align*}
%       V &= U_{\alpha_1} \cap \cdots \cap U_{\alpha_n} \\
%         &= \left( \bigcup_{i \in I_{\alpha_{1}}} U_i \right) \cap \left( \bigcup_{i \in I_{\alpha_{2}}} U_i \right) \cap \cdots \cap \left( \bigcup_{i \in I_{\alpha_{n-1}}} U_i \right) \cap \left( \bigcup_{i \in I_{\alpha_{n}}} U_i \right) \\
%    \end{align*}
%    where each $U_i \in \sigma$.

%    Then, by distributivity of intersection of sets, we can express $V$ as a union of many intersections the $U_i$'s i.e. as a union of intersections of elements of $\sigma$. 
%    i.e.

%    Consider an arbitrary intersection of elements $A_j \in \sigma$, $j \in J$
%    \begin{align*}
%       V' &= \bigcap_{j \in J} A_j \\
%          &= A_{j_1} \cap A_{j_2} \cap A_{j_2} \cdots \\
%          &= (A_{j_1} \cap A_{j_2}) \cap A_{j_2} \cdots
%    \end{align*} 
   

% \end{enumerate}

% \vskip 0.25cm
% \hrule
% \vskip 1cm

%\section*{References}
%\beginrefs
%\bibentry{AGM97}{\sc N.~Alon}, {\sc Z.~Galil} and {\sc O.~Margalit},
%On the Exponent of the All Pairs Shortest Path Problem,
%{\it Journal of Computer and System Sciences\/}~{\bf 54} (1997),
%pp.~255--262.

%\bibentry{F76}{\sc M. L. ~Fredman}, New Bounds on the Complexity of the 
%Shortest Path Problem, {\it SIAM Journal on Computing\/}~{\bf 5} (1976), 
%pp.~83-89.
%\endrefs


\end{document}





