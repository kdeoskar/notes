\documentclass[11pt]{article}

% basic packages
\usepackage[margin=1in]{geometry}
\usepackage[pdftex]{graphicx}
\usepackage{amsmath,amssymb,amsthm}
\usepackage{custom}
\usepackage{lipsum}

\usepackage{xcolor}
\usepackage{tikz}

\usepackage[most]{tcolorbox}

% page formatting
\usepackage{fancyhdr}
\pagestyle{fancy}

\renewcommand{\sectionmark}[1]{\markright{\textsf{\arabic{section}. #1}}}
\renewcommand{\subsectionmark}[1]{}
\lhead{\textbf{\thepage} \ \ \nouppercase{\rightmark}}
\chead{}
\rhead{}
\lfoot{}
\cfoot{}
\rfoot{}
\setlength{\headheight}{14pt}

\linespread{1.03} % give a little extra room
\setlength{\parindent}{0.2in} % reduce paragraph indent a bit
\setcounter{secnumdepth}{2} % no numbered subsubsections
\setcounter{tocdepth}{2} % no subsubsections in ToC


%%%%%%%%%%%%%%%%%%%%%%%%%%%%%%%%%%%%%%%%%%%%%%%%%%%%%%%%%%%%%%%%%
% CUSTOM BOXES AND STUFF
\newtcolorbox{redbox}{colback=red!5!white,colframe=red!75!black}
\newtcolorbox{bluebox}{colback=blue!5!white,colframe=blue!75!black}
%%%%%%%%%%%%%%%%%%%%%%%%%%%%%%%%%%%%%%%%%%%%%%%%%%%%%%%%%%%%%%%%%


\begin{document}

% make title page
\thispagestyle{empty}
\bigskip \
\vspace{0.1cm}

\begin{center}
{\fontsize{22}{22} \selectfont Physics 198 Term Paper}
\vskip 16pt
{\fontsize{36}{36} \selectfont \bf \sffamily Topological Order}
\vskip 24pt
{\fontsize{18}{18} \selectfont \rmfamily Keshav Balwant Deoskar} 
\vskip 6pt
{\fontsize{14}{14} \selectfont \ttfamily kdeoskar@berkeley.edu} 
\vskip 24pt
\end{center}

% {\parindent0pt \baselineskip=15.5pt \lipsum[1-4]} 

% make table of contents
% \newpage
\microtoc
% \tableofcontents 
\newpage

% main content
\section{Introduction}
In every-day life, we often talk about the different "phases" of matter such as solids, liquids, and gases. In condensed matter physics, there is a more precise meaning to the phrase "phase of matter" where materials can be classified based on their atomic arrangements. In the '80s, there emerged a powerful system for classifying the different orders: the Landau-Ginzburg-Wilson theory of Symmetry Breaking. 

\vskip 0.5cm

For example, the atoms of a gas in a container are uniformly distributed so that if we choose a point and then translate continuously in any direction, the density of particles at the new point is the same. The gas displays \textbf{\emph{continuous symmetry}}. Now if we reduce the temperature and/or apply pressure, causing the gas to form a crystalline solid, we no longer have continuous symmetry; Instead we have repeating patterns at regular intervals i.e. Crystals display \emph{\textbf{discrete symmetry}}. In carrying out the phase transition from gas to solid, we broke the continuous symmetry and obtained discrete symmetry.

\vskip 0.5cm
In this regime, the hamiltonian has some symmetry group $G$ with subgroup $H \subset G$ which serves as the symmetry group of the system after phase change.

\vskip 0.5cm
\subsection{Fractional Quantum Hall States}
\vskip 0.5cm 
For a while it seemed as though this theory was robust enough to classify all orders of matter. But it turns out life is not that simple. With the advent of Semi-conductor technology, physicists were able to confine electrons to the interface of two semiconductors, thus creating 2DEG (2 Dimensional Electron Gas).

\vskip 0.5cm 
In 1982 Tsui, Stomer, and Gossard found that if a 2DEG is placed under strong magnetic field and cooled to very low temperatures, then the 2DEG forms a new kind of state, called a \emph{\textbf{Fractional Quantum Hall (FQH) State}}.

\vskip 0.5cm Due to the low temperatures and strong interactions between electrons, these were expected to form strongly correlated states, like in a crystal. However, they turned out to form a unique kind of material called a \emph{\textbf{quantum spin liquid}} as the strong quantum fluctuation due to the low mass of the electrons prevented crystal formation.

\vskip 0.5cm 
These FQH States display many interesting properties, with the most unique being a quantization of the transverse conductance when a current passes through them, famously known as the \textbf{\emph{Quantum Hall Effect}}. Another quantized properly of hall states is their electron density. Quantum Hall Liquids are rigid materials in that they cannot be compressed - they have fixed densities. Bizarrely, it was found that if we measure the electron density in terms of a filling factor $\nu$ defined as 
\[ \nu = \frac{nhc}{eB} = \frac{\text{density of electrons}}{\text{density of magnetic flux quanta}} \]
then the densites of quantum hall states correspond to exactly rational filling factors $\nu = 1, 2/3, 1/3, \cdots$ with exact integer values of $\nu$ corresponding to the \emph{\textbf{Integer Quantum Hall (IQH) effect}} and others to the FQH effect.

\vskip 0.5cm 
Through many theoretical studies, it was found that FQH states have internal "patterns" which differentiate them, but are \emph{not} associated with the symmetries (or breaking of symmetries) of the FQH liquid. Thus, Landau-theory is not enough to describe them. We need a new kind of order called \emph{\textbf{Topological Order}} to describe them.

\vskip 0.5cm
\section{A hint of Topological Order: Lattice model}

Consider a many-body quantum system composed of a lattice described by a pair $(\mathcal{V}_N, \mathcal{H}_N)$ where
\[ \mathcal{V}_N = \bigotimes_{i = 1}^N \mathcal{V}_i \]
for $\mathcal{V}_i$ denoting the hilbert space at the $i^{th}$ lattice point and $\mathcal{H}_N$ being a local hamiltonian acting as 
\[  \mathcal{H}_n = \sum_{i} O_i + \sum_{i < j} O_{ij}  \]
where $O_i$ acts on $\mathcal{V}_i$ and $O_{ij}$ acts on $\mathcal{V}_i \otimes \mathcal{V}_j$. Suppose our system is also \emph{\textbf{gapped}} i.e. there is a finite gap $(\Delta)$ in the energy spectrum between the ground-state subspace (of width $\varepsilon$) and the excited states. 

\vskip 0.5cm
Mathematically, this can be expressed in the following:
\begin{redbox}
    A \emph{\textbf{Gapped Quantum System}} consists of a sequence of pairs $\{ (\mathcal{V}_{N_i}, \mathcal{H}_{N_i}) \}$ describing the system with $N_i$ lattice points satisfying the property that each $H_{N_i}$ has a gapped energy spectrum i.e. In the limit $N_i \rightarrow \infty$ we have $\Delta_{N_i}\rightarrow \Delta_{N_{\infty}}$ (finite) and $\varepsilon_{N_i} \rightarrow \infty$.
\end{redbox}

\begin{bluebox}
    An example of such a gapped system is a semi-conductor.
\end{bluebox}

\vskip 0.5cm
Two gapped systems i.e. two sequences $\{H_N\}$ and $\{H'_N\}$ are equivalent if one can be deformed into the other without closing the gap $\Delta$. This is reminiscent of how spaces which can be continuously deformed into each other are topologically equivalent. In fact, this correspondence will allow us to differentiate between FQH liquids. 

\vskip 0.5cm
FQH states have different phases even if there is no symmetry $G = 1$ and no symmetry breaking $H = G$. However, these newly defined equivalence classes of $\{H_N\}$ give rise to new \emph{\textbf{topological invariants}}.

\vskip 0.5cm
So, how do we extract these invariants from a many-body state?  Put the gapped system on a space with various topologies and measure the ground state degeneracy!

\subsection{Chracterization of Topological Order}

The ground state degeneracy is an invariant up to the equivalence classes we've defined, because in the $N \rightarrow \infty$ limit, it is robust against small perturbations that can break symmetry. In order to change the ground state degeneracy, one would need to close the gap - which would require a large change to the hamiltonian. This degeneracy is called \emph{\textbf{Topological Degeneracy}}.


\vskip 0.5cm
But topological degeneracy only partially characterizes topological order. Studying the Ground state structure of FQH States on a Torus, Wen and Keski-Vakkuri conjectured that $n$d ($n+1$D) topological order is fully characterized by the vector bundle structure on the moduli space of Hamiltonians.



\section*{References}
\begin{enumerate}
    \item Xiao-Gang Wen, \emph{"An Introduction to Topological Orders"}
    \item Xiao-Gang Wen, \emph{Higher Structures and Field Theory (MIT)}
    \item Wen IJMPB 4, 239 (90); Keski-Vakkuri Wen IJMPB 7, 4227 (93)
    \item Ravi Karki, \emph{"Topological Order in Physics"}, The Himalayan Physics Vol 6 \& 7, April 2017 (108-111)
\end{enumerate}

\end{document}