\documentclass[11pt]{article}

% basic packages
\usepackage[margin=1in]{geometry}
\usepackage[pdftex]{graphicx}
\usepackage{amsmath,amssymb,amsthm}
\usepackage{custom}
\usepackage{lipsum}

\usepackage{xcolor}
\usepackage{tikz}

\usepackage[most]{tcolorbox}

% page formatting
\usepackage{fancyhdr}
\pagestyle{fancy}

\renewcommand{\sectionmark}[1]{\markright{\textsf{\arabic{section}. #1}}}
\renewcommand{\subsectionmark}[1]{}
\lhead{\textbf{\thepage} \ \ \nouppercase{\rightmark}}
\chead{}
\rhead{}
\lfoot{}
\cfoot{}
\rfoot{}
\setlength{\headheight}{14pt}

\linespread{1.03} % give a little extra room
\setlength{\parindent}{0.2in} % reduce paragraph indent a bit
\setcounter{secnumdepth}{2} % no numbered subsubsections
\setcounter{tocdepth}{2} % no subsubsections in ToC


%%%%%%%%%%%%%%%%%%%%%%%%%%%%%%%%%%%%%%%%%%%%%%%%%%%%%%%%%%%%%%%%%
% CUSTOM BOXES AND STUFF
\newtcolorbox{redbox}{colback=red!5!white,colframe=red!75!black}
\newtcolorbox{bluebox}{colback=blue!5!white,colframe=blue!75!black}
%%%%%%%%%%%%%%%%%%%%%%%%%%%%%%%%%%%%%%%%%%%%%%%%%%%%%%%%%%%%%%%%%


\begin{document}

% make title page
\thispagestyle{empty}
\bigskip \
\vspace{0.1cm}

\begin{center}
{\fontsize{22}{22} \selectfont Physics 198 Term Paper}
\vskip 16pt
{\fontsize{36}{36} \selectfont \bf \sffamily Topological Order, Insulators, and Semiconductors}
\vskip 24pt
{\fontsize{18}{18} \selectfont \rmfamily Keshav Balwant Deoskar} 
\vskip 6pt
{\fontsize{14}{14} \selectfont \ttfamily kdeoskar@berkeley.edu} 
\vskip 24pt
\end{center}


This expository paper was written as an outgrowth of my final project for the UC Berkeley DeCal "Topology and Geometry in Physics" which took place in the Spring '24 semester. I'd like to thank the facilitators of the Decal and the Faculty Sponsor for the time and effort spent in making the experience educational and enjoyable. This is content I am actively learning at the moment, so if there are any errors or misguided explanations, I would appreciate corrections and feedback!
% {\parindent0pt \baselineskip=15.5pt \lipsum[1-4]} 

% make table of contents
% \newpage
\microtoc
% \tableofcontents 
\newpage

% main content
\section{Introduction}
In every-day life, we often talk about different phases of matter such as "solids, liquids, and gases" because each of them display different properties. But we know that, for instance, a solid ferromagnet behaves differently from a solid insulator. How do we classify materials that display similar characteristics, in a more nuanced way? 
\\
\\
One method to classify them based on their \textbf{\emph{symmetries}}. Intuitively it makes sense that different symmetries would imply different properties. For instnce, a gas at equilibrium has translational symmetry because if we move from one point to any other the properties of the material (pressure, density, temperature, etc.) remain the same. In contrast to this, a crystal like $NaCl$ composed of a lattice of $Na^+, Cl^-$ does not have translational symmetry because if we move from an $Na^+$ site to a $Cl^-$ site, the electric charge of the site is different. This incredibly successful paradigm was introduced in the 1950s and is known as the Landau-Ginzburg theory of Symmetry-Breaking Phases.
\\
\\
Separate from this, but following closely after in terms of chronology, topological effects began showing up in physics. In the 1980s, the Integer Quantum Hall Effect (IQHE) was discovered, and it was found that the origin for the amazing adherence to quantization of the hall conductance is topological in origin. Soon after, the Fractional Quantum Hall Effect in which the hall conductance is quantized with fractional values rather than only integers, was discovered, and with it the first example of a phenomenon \emph{beyond} the Landau-Ginzburg paradigm. 
\\
\\
Fractional Quantum Hall Liquids cannot be understood solely based on their symmetries, but rather require a new \textbf{\emph{topological order}} to be fully described. Fractional Quantum Hall Effect (QHE) states are an example of Thouless-type topological order, but another important one is Wen-type topological order. 
\\
\\
More recently, since the early 2000s, new and fascinating classes of materials known as Topological Insulators and Superconductors have been the garnering more and more research interest across the globe. In Topological Insulators, rather than some quantized transport quantity, the thing that is quantized is a kind of magnetoelectric effect \cite{MooreMoessner21}. Aside from being interesting purely from a scholarly point of view,these materials have great potential for applications in new types of low energy electroniscs. 
\\
\\
The goal of this expository paper is to try and summarize some of the important topological concepts in this amazing field lying at the intersection of Physics, Mathematics, and (possibly sometime soon) Computing \& Low Level Electronics.
\\
\section{Symmetry-Breaking Phases and the Order Parameter}
[Complete this section soon]

\section{Berry Connection, Curvature, and Phase}

\section{Integer Quantum Hall Effect (IQHE)}

\subsection{Landau Levels}

\subsection{TKNN Invariant = $1^{\text{st}}$ Chern Number}

\subsection{Chiral Edge States}



% \section{Beyond the Landau-Ginzburg Paradigm: Fractional Quantum Hall States and Topological Order}



\section{Topological Insulators}



% \section{Topological Superconductors}


% \section*{References}
% \begin{enumerate}
%     \item Xiao-Gang Wen, \emph{"An Introduction to Topological Orders"}
%     \item Xiao-Gang Wen, \emph{Higher Structures and Field Theory (MIT)}
%     \item Wen IJMPB 4, 239 (90); Keski-Vakkuri Wen IJMPB 7, 4227 (93)
%     \item Ravi Karki, \emph{"Topological Order in Physics"}, The Himalayan Physics Vol 6 \& 7, April 2017 (108-111)
% \end{enumerate}

%%%%%%%%%%%%%%%%%%%%%%%%%%%%%%%%%%%%%%%%%%%%%%
\newpage
% \section{References}
%%%%%%%%%%%%%%%%%%%%%%%%%%%%%%%%%%%%%%%%%%%%%%
\vskip 0.5cm
\bibliographystyle{plain} % We choose the "plain" reference style
\bibliography{ref} % Entries are in the refs.bib file

\end{document}