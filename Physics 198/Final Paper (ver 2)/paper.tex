\documentclass[11pt]{article}

% basic packages
\usepackage[margin=1in]{geometry}
\usepackage[pdftex]{graphicx}
\usepackage{amsmath,amssymb,amsthm}
\usepackage{custom}
\usepackage{lipsum}

\usepackage{xcolor}
\usepackage{tikz}

\usepackage[most]{tcolorbox}

% page formatting
\usepackage{fancyhdr}
\pagestyle{fancy}

\renewcommand{\sectionmark}[1]{\markright{\textsf{\arabic{section}. #1}}}
\renewcommand{\subsectionmark}[1]{}
\lhead{\textbf{\thepage} \ \ \nouppercase{\rightmark}}
\chead{}
\rhead{}
\lfoot{}
\cfoot{}
\rfoot{}
\setlength{\headheight}{14pt}

\linespread{1.03} % give a little extra room
\setlength{\parindent}{0.2in} % reduce paragraph indent a bit
\setcounter{secnumdepth}{2} % no numbered subsubsections
\setcounter{tocdepth}{2} % no subsubsections in ToC


%%%%%%%%%%%%%%%%%%%%%%%%%%%%%%%%%%%%%%%%%%%%%%%%%%%%%%%%%%%%%%%%%
% CUSTOM BOXES AND STUFF
\newtcolorbox{redbox}{colback=red!5!white,colframe=red!75!black}
\newtcolorbox{bluebox}{colback=blue!5!white,colframe=blue!75!black}
%%%%%%%%%%%%%%%%%%%%%%%%%%%%%%%%%%%%%%%%%%%%%%%%%%%%%%%%%%%%%%%%%


\begin{document}

% make title page
\thispagestyle{empty}
\bigskip \
\vspace{0.1cm}

\begin{center}
{\fontsize{22}{22} \selectfont Physics 198 Term Paper}
\vskip 16pt
{\fontsize{36}{36} \selectfont \bf \sffamily Topological Order, Insulators, and Semiconductors}
\vskip 24pt
{\fontsize{18}{18} \selectfont \rmfamily Keshav Balwant Deoskar} 
\vskip 6pt
{\fontsize{14}{14} \selectfont \ttfamily kdeoskar@berkeley.edu} 
\vskip 24pt
\end{center}

% {\parindent0pt \baselineskip=15.5pt \lipsum[1-4]} 

% make table of contents
% \newpage
\microtoc
% \tableofcontents 
\newpage

% main content
\section{Introduction}
In every-day life, we often talk about the different "phases" of matter such as solids, liquids, and gases. In condensed matter physics, there is a more precise meaning to the phrase "phase of matter" where materials can be classified based on their atomic arrangements. In the '80s, there emerged a powerful system for classifying the different orders: the Landau-Ginzburg-Wilson theory of Symmetry Breaking. 

\vskip 0.5cm

For example, the atoms of a gas in a container are uniformly distributed so that if we choose a point and then translate continuously in any direction, the density of particles at the new point is the same. The gas displays \textbf{\emph{continuous symmetry}}. Now if we reduce the temperature and/or apply pressure, causing the gas to form a crystalline solid, we no longer have continuous symmetry; Instead we have repeating patterns at regular intervals i.e. Crystals display \emph{\textbf{discrete symmetry}}. In carrying out the phase transition from gas to solid, we broke the continuous symmetry and obtained discrete symmetry.

\vskip 0.5cm
In this regime, the hamiltonian has some symmetry group $G$ with subgroup $H \subset G$ which serves as the symmetry group of the system after phase change.


\section{Landau-Ginzburg Theory of Symmetry-Breaking Phases}

\section{Beyond the Landau-Ginzburg Paradigm: Fractional Quantum Hall States and Topological Order}

\section{Topological Insulators}

\section{Topological Superconductors}


% \section*{References}
% \begin{enumerate}
%     \item Xiao-Gang Wen, \emph{"An Introduction to Topological Orders"}
%     \item Xiao-Gang Wen, \emph{Higher Structures and Field Theory (MIT)}
%     \item Wen IJMPB 4, 239 (90); Keski-Vakkuri Wen IJMPB 7, 4227 (93)
%     \item Ravi Karki, \emph{"Topological Order in Physics"}, The Himalayan Physics Vol 6 \& 7, April 2017 (108-111)
% \end{enumerate}

%%%%%%%%%%%%%%%%%%%%%%%%%%%%%%%%%%%%%%%%%%%%%%
\newpage
% \section{References}
%%%%%%%%%%%%%%%%%%%%%%%%%%%%%%%%%%%%%%%%%%%%%%
\vskip 0.5cm
\bibliographystyle{plain} % We choose the "plain" reference style
\bibliography{refs} % Entries are in the refs.bib file



\end{document}