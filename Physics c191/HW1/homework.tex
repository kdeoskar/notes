\documentclass[11pt]{article}

% basic packages
\usepackage[margin=1in]{geometry}
\usepackage[pdftex]{graphicx}
\usepackage{amsmath,amssymb,amsthm}
\usepackage{custom}
\usepackage{lipsum}

\usepackage{xcolor}
\usepackage{tikz}

\usepackage[most]{tcolorbox}
\usepackage{xcolor}
\usepackage{mdframed}

% page formatting
\usepackage{fancyhdr}
\pagestyle{fancy}

\renewcommand{\sectionmark}[1]{\markright{\textsf{\arabic{section}. #1}}}
\renewcommand{\subsectionmark}[1]{}
\lhead{\textbf{\thepage} \ \ \nouppercase{\rightmark}}
\chead{}
\rhead{}
\lfoot{}
\cfoot{}
\rfoot{}
\setlength{\headheight}{14pt}

\linespread{1.03} % give a little extra room
\setlength{\parindent}{0.2in} % reduce paragraph indent a bit
\setcounter{secnumdepth}{2} % no numbered subsubsections
\setcounter{tocdepth}{2} % no subsubsections in ToC


%%%%%%%%%%%%%%%%%%%%%%%%%%%%%%%%%%%%%%%%%%%%%%%%%%%%%%%%%%%%%%%%%
% CUSTOM BOXES AND STUFF
\newtcolorbox{redbox}{colback=red!5!white,colframe=red!75!black, breakable}
\newtcolorbox{bluebox}{colback=blue!5!white,colframe=blue!75!black,breakable}

\newtcolorbox{dottedbox}[1][]{%
    colback=white,    % Background color
    colframe=white,    % Border color (to be overridden by dashrule)
    sharp corners,     % Sharp corners for the box
    boxrule=0pt,       % No actual border, as it will be drawn with dashrule
    boxsep=5pt,        % Padding inside the box
    enhanced,          % Enable advanced features
    breakable,         % Enables it to span multiple pages
    overlay={\draw[dashed, thin, black, dash pattern=on \pgflinewidth off \pgflinewidth, line cap=rect] (frame.south west) rectangle (frame.north east);}, % Dotted line
    #1                 % Additional options
}

% Define the colors
\definecolor{boxheader}{RGB}{0, 51, 102}  % Dark blue
\definecolor{boxfill}{RGB}{173, 216, 230}  % Light blue


% Define the tcolorbox environment
\newtcolorbox{mathdefinitionbox}[2][]{%
    colback=boxfill,   % Background color
    colframe=boxheader, % Border color
    fonttitle=\bfseries, % Bold title
    coltitle=white,     % Title text color
    title={#2},         % Title text
    enhanced,           % Enable advanced features
    breakable,
    attach boxed title to top left={yshift=-\tcboxedtitleheight/2}, % Center title
    boxrule=0.5mm,      % Border width
    sharp corners,      % Sharp corners for the box
    #1                  % Additional options
}
%%%%%%%%%%%%%%%%%%%%%%%%%


\definecolor{lightblue}{RGB}{173,216,230} % Light blue color
\definecolor{darkblue}{RGB}{0,0,139} % Dark blue color

% Define the custom proof environment
\newtcolorbox{ex}[2][Example]{
  colback=red!5!white, % Light blue background
  colframe=red!75!black, % Darker blue border
  coltitle=white, % Title color
  fonttitle=\bfseries, % Title font style
  title={{#2}},
  arc=1mm, % Rounded corners with 4mm radius,
  boxrule=0.5mm,
  left=2mm, right=2mm, top=2mm, bottom=2mm, % Padding inside the box
  breakable, % Allow box to be broken across pages
  before=\vspace{10pt}, % Padding above the box
  after=\vspace{10pt}, % Padding below the box
  before upper={\parindent15pt} % Ensure indentation
}

% Define the custom proof environment
\newtcolorbox{defn}[2][Definition]{
  colback=green!5!white, % Light blue background
  colframe=green!75!black, % Darker blue border
  coltitle=white, % Title color
  fonttitle=\bfseries, % Title font style
  title={{#2}},
  arc=1mm, % Rounded corners with 4mm radius,
  boxrule=0.5mm,
  left=2mm, right=2mm, top=2mm, bottom=2mm, % Padding inside the box
  breakable, % Allow box to be broken across pages
  before=\vspace{10pt}, % Padding above the box
  after=\vspace{10pt}, % Padding below the box
  before upper={\parindent15pt} % Ensure indentation
}


%%%%%%%%%%%%%%%%%%%%%%%%%%%%%%%%%%%%%%%%%%%%%%%%%%%%%%%%%%%%%%%%%


\begin{document}

% make title page
\thispagestyle{empty}
\bigskip \
\vspace{0.1cm}

\begin{center}
{\fontsize{14}{14} \selectfont Instructors: Birgit K Whaley, Alp Sipahigil, Geoffrey Pennington}
\vskip 16pt
{\fontsize{30}{30} \selectfont \bf \sffamily Physics c191: Introduction to Quantum Computing}
\vskip 24pt
{\fontsize{14}{14} \selectfont \rmfamily Homework 1} 
\vskip 6pt
{\fontsize{14}{14} \selectfont \ttfamily kdeoskar@berkeley.edu} 
\vskip 24pt
\end{center}

% {\parindent0pt \baselineskip=15.5pt \lipsum[1-4]} 

% make table of contents
% \newpage


\subsection*{Question 1:}
We want to show that a single cubit pure state can be described by only two real parameters. 
\\
\\
An arbitrary single cubit pure state is an element of $\mathcal{H} = \C^2$. We can write it as a linear combination of the basis states $\ket{0}, \ket{1}$:
$$ \ket{\psi} = \alpha \ket{0} + \beta \ket{1} $$ Writing the complex coefficients in polar form, $ \alpha = r_1 e^{i\phi_1}, ~\beta = r_2 e^{i\phi_2} $ we can write 
\begin{align*}
  \ket{\psi} &= r_1 e^{i\phi} \ket{0} + r_2 e^{i\phi} \ket{1} \\
  &= e^{i\phi_1} \left[ r_1 \ket{0} + r_2 e^{i(\phi_2 - \theta_1)} \ket{1} \right]
\end{align*} Physically, any two states related by a global phase are equivalent, because only the amplitude squared matter when it comes to measurements and $|e^{i\phi}|^2 = 1$ for any angle $\phi$. Thus, we can ignore the global phase $\phi_1$ in our expression for $\ket{\psi}$.
\\
\\
We also have the normalization condition 
\begin{align*}
  &|r_1|^2 + |r_2 e^{i(\phi_2 - \phi_1)} |^2 \\
  \implies& r_1^2 + r_2^2 = 1
\end{align*} Thus, an arbitrary single cubit pure state (up to global phase) is completely described by the parameters $r_1, r_2$ such that $r_1^2 + r_2^2 = 1$ i.e. $(r_1, r_2) \in \sph^1$ $\phi \text{:}= (\phi_2 - \phi_1) \in [0, 2\pi]$. 
\\
\\
But since $(r_1, r_2)$ must be a point on the unit circle, we can instead just use the angle corresponding to the point $(r_1, r_2)$ given by $\phi = \arctan(r_2/r_1) \in [-\pi/2, \pi/2]$.
\\
\\
So, our single cubit pure state is completely described by two angles, $\phi \in [0, 2\pi]$ and $\theta \in [-\pi/2, \pi/2]$ i.e. states correspond to points on the unit sphere $\sph^2$.
\\
\\
Equivalently, working with $\theta \in [0, \pi]$ (same interval as earlier), we can imagine using the usual spherical coordinates to describe points on $\sph^2$. Taking $\ket{0}$ and $\ket{1}$ to be the North and South poles respectively i.e. \begin{align*}
  \ket{0} &= \begin{pmatrix}
    x = 0 \\ y = 0 \\ z = 1
  \end{pmatrix} \\
  \ket{1} &= \begin{pmatrix}
    x = 0 \\ y = 0 \\ z = -1
  \end{pmatrix}
\end{align*} we can write an arbitrary state $\ket{\psi}$ as a linear point on the sphere.
\\
\begin{note}
  {Finish this.}
\end{note}
\\
Thus a general single cubit pure state can be written as $$ \ket{\psi} = e^{i\gamma} \left[ \cos(\theta/2) \ket{0} + e^{i\phi} \sin(\theta/2) \ket{1} \right] $$


\vskip 1cm
\hrule
\pagebreak


\subsection*{Question 1}

\begin{enumerate}
  \item We want to find the eigenvectors, eigenvalues, and diagonal representations of the Pauli matrices $I, X, Y, Z$ and show that $X^2 = Y^2 = Z^2 = I$
\end{enumerate}


\vskip 1cm
\hrule
\pagebreak























% \subsection*{Question 1}


% \vskip 1cm
% \hrule
% \pagebreak



% \begin{bluebox}
%   \textbf{Question 1:} 
% \end{bluebox}

% \vskip 0.5cm
% \textbf{\underline{Solution:}}
% \\
% \\
% text
% \vskip 0.5cm
% \hrule
% \pagebreak



% %%%%%%%%%%%%%%%%%%%%%%%%%%%%%%%%%%%%%%%%%%%%%%
% \newpage
% % \section{References}
% %%%%%%%%%%%%%%%%%%%%%%%%%%%%%%%%%%%%%%%%%%%%%%
% \vskip 0.5cm
% \bibliographystyle{plain} % We choose the "plain" reference style
% \bibliography{citation}




\end{document}










