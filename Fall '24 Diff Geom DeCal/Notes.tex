\documentclass[11pt]{article}

% basic packages
\usepackage[margin=1in]{geometry}
\usepackage[pdftex]{graphicx}
\usepackage{amsmath,amssymb,amsthm}
\usepackage{custom}
\usepackage{lipsum}

\usepackage{xcolor}
\usepackage{tikz}

\usepackage[most]{tcolorbox}
\usepackage{xcolor}
\usepackage{mdframed}

% page formatting
\usepackage{fancyhdr}
\pagestyle{fancy}

\renewcommand{\sectionmark}[1]{\markright{\textsf{\arabic{section}. #1}}}
\renewcommand{\subsectionmark}[1]{}
\lhead{\textbf{\thepage} \ \ \nouppercase{\rightmark}}
\chead{}
\rhead{}
\lfoot{}
\cfoot{}
\rfoot{}
\setlength{\headheight}{14pt}

\linespread{1.03} % give a little extra room
\setlength{\parindent}{0.2in} % reduce paragraph indent a bit
\setcounter{secnumdepth}{2} % no numbered subsubsections
\setcounter{tocdepth}{2} % no subsubsections in ToC


% %%%%%%%%%%%%%%%%%%%%%%%%%%%%%%%%%%%%%%%%%%%%%%%%%%%%%%%%%%%%%%%%%
% % CUSTOM BOXES AND STUFF
% \newtcolorbox{redbox}{colback=red!5!white,colframe=red!75!black}
% \newtcolorbox{bluebox}{colback=blue!5!white,colframe=blue!75!black}

% \definecolor{lightblue}{RGB}{173,216,230} % Light blue color
% \definecolor{darkblue}{RGB}{0,0,139} % Dark blue color

% % Define the custom proof environment
% \newtcolorbox{ex}[2][Example]{
%   colback=red!5!white, % Light blue background
%   colframe=red!75!black, % Darker blue border
%   coltitle=white, % Title color
%   fonttitle=\bfseries, % Title font style
%   title={{#2}},
%   arc=1mm, % Rounded corners with 4mm radius,
%   boxrule=0.5mm,
%   left=2mm, right=2mm, top=2mm, bottom=2mm, % Padding inside the box
%   breakable, % Allow box to be broken across pages
% }

% % % Define the custom proof environment
% % \newtcolorbox{defn}[2][Definition]{
% %   colback=blue!5!white, % Light blue background
% %   colframe=blue!75!black, % Darker blue border
% %   coltitle=white, % Title color
% %   fonttitle=\bfseries, % Title font style
% %   title={{#2}},
% %   arc=1mm, % Rounded corners with 4mm radius,
% %   boxrule=0.5mm,
% %   left=2mm, right=2mm, top=2mm, bottom=2mm, % Padding inside the box
% %   breakable, % Allow box to be broken across pages
% % }

% % \definecolor{boxheader}{RGB}{0, 51, 102}  % Dark blue
% % \definecolor{boxfill}{RGB}{173, 216, 230}  % Light blue

% % % Define the tcolorbox environment
% % \newtcolorbox{mathdefinitionbox}[2][]{%
% %     colback=boxfill,   % Background color
% %     colframe=boxheader, % Border color
% %     fonttitle=\bfseries, % Bold title
% %     coltitle=white,     % Title text color
% %     title={#2},         % Title text
% %     enhanced,           % Enable advanced features
% %     attach boxed title to top left={yshift=-\tcboxedtitleheight/2}, % Center title
% %     boxrule=0.5mm,      % Border width
% %     sharp corners,      % Sharp corners for the box
% %     #1                  % Additional options
% % }
% % %%%%%%%%%%%%%%%%%%%%%%%%%


%%%%%%%%%%%%%%%%%%%%%%%%%%%%%%%%%%%%%%%%%%%%%%%%%%%%%%%%%%%%%%%%%


\begin{document}

% make title page
\thispagestyle{empty}
\bigskip \
\vspace{0.1cm}

\begin{center}
{\fontsize{22}{22} \selectfont Physics 198: Differential Geometry and Lie Groups}
\vskip 16pt
{\fontsize{36}{36} \selectfont \bf \sffamily Notes}
\vskip 24pt
{\fontsize{18}{18} \selectfont \rmfamily Keshav Balwant Deoskar} 
\vskip 6pt
{\fontsize{14}{14} \selectfont \ttfamily kdeoskar@berkeley.edu} 
\vskip 24pt
\end{center}

% {\parindent0pt \baselineskip=15.5pt \lipsum[1-4]} 

% make table of contents
% \newpage

These are a compilation of notes based on which (some of the) lectures for 'Physics 198: Differential Geometry and Lie Groups for Physics Students' will be given. These notes are largely based on the primary reference to the class, namely "\textit{Differential Geometry and Lie Groups for Physicists}" by Marián Fecko, and cover topics in roughly the same order as in the book.

% \vskip 0.5cm
% \begin{redbox}
%     Remark: I am not an expert! Please point out and errors / suggestions if you'd like via email.
% \end{redbox}

% \microtoc
\tableofcontents 

% main 

%%%%%%%%%%%%%%%%%%%%%%%%%%%%%%%%%%%%%%%%%%%%%%
\newpage
\section{Lie Groups}
%%%%%%%%%%%%%%%%%%%%%%%%%%%%%%%%%%%%%%%%%%%%%%

\vskip 0.5cm
In physics, we can learn a great deal from studying the symmetries of \emph{continuous} systems. 


%%%%%%%%%%%%%%%%%%%%%%%%%%%%%%%%%%%%%%%%%%%%%%
\newpage
\section{Representations of Lie Groups and Lie Algebras}
%%%%%%%%%%%%%%%%%%%%%%%%%%%%%%%%%%%%%%%%%%%%%%


\subsection*{What is a Representation?}
\begin{itemize}
    \item To understand what a group really \emph{is}, it can be very enlightening to study what the group \emph{does} i.e. to study \emph{\textbf{group actions}}. This refers to how the elements of a group act on some other space of objects (could be another group, a vector space, a topological space, smooth manifold, etc.)
    \item The case of vector spaces is and specifically the group actions which act \emph{linearly} on a \emph{linear space} is particularly important. Essentially, such maps which characterize the behavior of a group are called its \emph{\textbf{Representations}}.
\end{itemize}


% \begin{definition}
%   Given a group $G$ and vector space $V$, a group homomorphism $\rho : G $ is called a \textbf{representation} of $G$ in $V$.
% \end{definition}

% \begin{definition}
  
% \end{definition}


% \begin{ex}{Fecko, Exercise 12.1.5}
    
% \end{ex}

\end{document}










\end{document}