\documentclass[11pt]{article}

% basic packages
\usepackage[margin=1in]{geometry}
\usepackage[pdftex]{graphicx}
\usepackage{amsmath,amssymb,amsthm}
\usepackage{custom}
\usepackage{lipsum}

\usepackage{xcolor}
\usepackage{tikz}

\usepackage[most]{tcolorbox}
\usepackage{xcolor}
\usepackage{mdframed}

% page formatting
\usepackage{fancyhdr}
\pagestyle{fancy}

\renewcommand{\sectionmark}[1]{\markright{\textsf{\arabic{section}. #1}}}
\renewcommand{\subsectionmark}[1]{}
\lhead{\textbf{\thepage} \ \ \nouppercase{\rightmark}}
\chead{}
\rhead{}
\lfoot{}
\cfoot{}
\rfoot{}
\setlength{\headheight}{14pt}

\linespread{1.03} % give a little extra room
\setlength{\parindent}{0.2in} % reduce paragraph indent a bit
\setcounter{secnumdepth}{2} % no numbered subsubsections
\setcounter{tocdepth}{2} % no subsubsections in ToC


%%%%%%%%%%%%%%%%%%%%%%%%%%%%%%%%%%%%%%%%%%%%%%%%%%%%%%%%%%%%%%%%%
% CUSTOM BOXES AND STUFF
\newtcolorbox{redbox}{colback=red!5!white,colframe=red!75!black}
\newtcolorbox{bluebox}{colback=blue!5!white,colframe=blue!75!black}

\definecolor{lightblue}{RGB}{173,216,230} % Light blue color
\definecolor{darkblue}{RGB}{0,0,139} % Dark blue color

% Define the custom proof environment
\newtcolorbox{ex}[2][Example]{
  colback=red!5!white, % Light blue background
  colframe=red!75!black, % Darker blue border
  coltitle=white, % Title color
  fonttitle=\bfseries, % Title font style
  title={{#2}},
  arc=1mm, % Rounded corners with 4mm radius,
  boxrule=0.5mm,
  left=2mm, right=2mm, top=2mm, bottom=2mm, % Padding inside the box
  breakable, % Allow box to be broken across pages
  before=\vspace{10pt}, % Padding above the box
  after=\vspace{10pt}, % Padding below the box
  before upper={\parindent15pt} % Ensure indentation
}

% Define the custom proof environment
\newtcolorbox{defn}[2][Definition]{
  colback=green!5!white, % Light blue background
  colframe=green!75!black, % Darker blue border
  coltitle=white, % Title color
  fonttitle=\bfseries, % Title font style
  title={{#2}},
  arc=1mm, % Rounded corners with 4mm radius,
  boxrule=0.5mm,
  left=2mm, right=2mm, top=2mm, bottom=2mm, % Padding inside the box
  breakable, % Allow box to be broken across pages
  before=\vspace{10pt}, % Padding above the box
  after=\vspace{10pt}, % Padding below the box
  before upper={\parindent15pt} % Ensure indentation
}


%%%%%%%%%%%%%%%%%%%%%%%%%%%%%%%%%%%%%%%%%%%%%%%%%%%%%%%%%%%%%%%%%


\begin{document}

% make title page
\thispagestyle{empty}
\bigskip \
\vspace{0.1cm}

\begin{center}
{\fontsize{22}{22} \selectfont Physics 198: Differential Geometry and Lie Groups}
\vskip 16pt
{\fontsize{36}{36} \selectfont \bf \sffamily Notes}
\vskip 24pt
{\fontsize{18}{18} \selectfont \rmfamily Keshav Balwant Deoskar} 
\vskip 6pt
{\fontsize{14}{14} \selectfont \ttfamily kdeoskar@berkeley.edu} 
\vskip 24pt
\end{center}

% {\parindent0pt \baselineskip=15.5pt \lipsum[1-4]} 

% make table of contents
% \newpage

These are some short notes that will be used to complement the lectures for the UC Berkeley DeCal 'Physics 198: Differential Geometry and Lie Groups for Physics Students'. The primary reference for the class is "\textit{Differential Geometry and Lie Groups for Physicists}" by Marián Fecko, and so this document will cover topics in roughly the same order - though with some additional and alternative explanations.

% \vskip 0.5cm
% I try to explain every concept we will come across in less than 100 words.

\vskip 0.5cm
These notes assume the reader is familiar with results from Linear Algebra and Multivariable Calculus, at the level of Math 54 and Math 53 at UC Berkeley.

\vskip 0.5cm
This template is based heavily off of the one produced by \href{https://knzhou.github.io/}{Kevin Zhou}.

% \microtoc
\tableofcontents 

% main 

%%%%%%%%%%%%%%%%%%%%%%%%%%%%%%%%%%%%%%%%%%%%%%
\newpage
\section{Lie Groups}
%%%%%%%%%%%%%%%%%%%%%%%%%%%%%%%%%%%%%%%%%%%%%%

\vskip 0.5cm
In physics, we can learn a great deal from studying the symmetries of \emph{continuous} systems. 


%%%%%%%%%%%%%%%%%%%%%%%%%%%%%%%%%%%%%%%%%%%%%%
\newpage
\section{Representations of Lie Groups and Lie Algebras}
%%%%%%%%%%%%%%%%%%%%%%%%%%%%%%%%%%%%%%%%%%%%%%
\subsection{What is a Representation?}
\begin{itemize}
    \item To understand what a group really \emph{is}, it can be very enlightening to study what the group \emph{does} i.e. to study \emph{\textbf{group actions}}. Further, linear algebra is easier than abstract algebra, so if we can study the action of a group in terms of linear algebraic objects, we'll get a lot more mileage.
    \item How exactly do we do this? We can use some sort of map to assign a \emph{linear operator} over that vector space to each group element to describe (or represent) the action of each group element on the vector space elements. The map that we use is a \textbf{representation} of the group.
\end{itemize}

\vskip 0.5cm
\begin{redbox}
  Recall that the space of all linear operators $\rho \text{ : } V \rightarrow V$ is denoted $\homend{V}$. The subset of these operators which are invertible (isomorphisms on $V$) is denoted $\aut{V}$. 
  
  \vskip 0.5cm
  Notably, $\aut{V}$ has a group structure! \begin{thought}{Check this!} \end{thought} On the other hand, $\homend{V}$ becomes an (Associative, and later Lie) Algebra if we define the commutator for $A, B \in \homend{V}$ as
  \[ [A, B] = AB - BA \] 
  
  % \begin{thought}{Check this!} \end{thought}
\end{redbox}

\vskip 0.5cm
Now the formal definition.

\vskip 0.5cm
\begin{definition}{Group Representation}
  Given a group $G$ and vector space $V$, a group homomorphism \[ \rho \text{ : } G \rightarrow \mathrm{Aut}(V) \] is called a \textbf{representation} of $G$ in $V$.
\end{definition}

\begin{example}
  Complete this later
\end{example}

\vskip 0.5cm

We can use the same idea to define the representation of an algebra, but this time with $\homend{V}$.

\begin{definition}{Lie Algebra Representation}
  Given a Lie algebra $\mathcal{G}$ and vector space $V$, an algebra homomorphism \[ \rho' \text{ : } \mathcal{G} \rightarrow \homend{V} \] is called a representation of the Lie algebra $\mathcal{G}$ over $V$.
\end{definition}

\begin{remark}{The representations $\rho$ and $\rho'$ of a lie group and its lie algebra are related! so $\rho'$ is called the \textbf{derived representation}.}
\end{remark}

\begin{ex}{Fecko, Exercise 12.1.4}
    Consider a Lie algebra $\mathcal{G}$ whose basis elements $\{E_i\}$ satisfy the commutation relations 
    \[ [E_i, E_j] = c^k_{ij} E_k  \]
    and a representation $f \tcolon \mathcal{G} \rightarrow \homend{V}$. Then, define $\mathcal{E}_i \equiv f(E_i)$. Since $f$ is a homomorphisms between algebra, it is linear and respects the commutator i.e. for $A, B \in \mathcal{G}$
    \[ f\left( [A, B] \right) = \left[f(A), f(B)\right] \] 

    Thus, 
    \begin{align*}
      \left[ \mathcal{E}_i, \mathcal{E}_j \right] &= \left[f(E_i), f(E_j) \right] \\
      &= f\left([E_i, E_j]\right) \\
      &= f\left(c^k_{ij} E_k\right) \\
      &= c^{k}_{ij} f(E_k) \\
      &= c^{k}_{ij} \mathcal{E}_k 
    \end{align*}

    \begin{thought}{The basis elements of the representation satisfy the same commutation relation as those of the Lie Algebra!}
    \end{thought}
\end{ex}

\begin{ex}{Fecko, Exercise 12.1.5}
    Do this one later
\end{ex}

\vskip 0.5cm
\begin{itemize}
  \item The assignment from Lie Group to Lie Algebra $G \mapsto \mathcal{G}$ is nice and unique, but the other way around can get messy.
  \item Similarly, given a Lie group representation $\rho$ there is a unique Lie algebra represenation $\rho'$, but not necessarily the other way around. 
\end{itemize}


\begin{ex}{Fecko, Exercise 12.1.6}
  \begin{enumerate}[label=(\roman*)]
    \item Consider the Lie Group $H = \aut{V} \equiv \mathrm{GL}(V)$. 
    Recall that the Lie Algebra of $H$ is 
  \end{enumerate}
\end{ex}

\vskip 0.5cm
Write about $\rho$-invariant inner products.

\vskip 1cm
\subsection{Reducible and Irreducible Representations}




\vskip 1cm
\subsection*{References for the chapter}
%%%%%%%%%%%%%%%%%%%%%%%%%%%%%%%%%%%%%%%%%%%%%%%%%%%%%%%%%


%%%%%%%%%%%%%%%%%%%%%%%%%%%%%%%%%%%%%%%%%%%%%%%%%%%%%%%%%
\newpage
\section{Connections and Parallel Transport on a Manifold}

\vskip 1cm
\subsection*{References for the chapter}
%%%%%%%%%%%%%%%%%%%%%%%%%%%%%%%%%%%%%%%%%%%%%%%%%%%%%%%%%




% %%%%%%%%%%%%%%%%%%%%%%%%%%%%%%%%%%%%%%%%%%%%%%%%%%%%%%%%%
% \newpage
% \section*{Chapter name}

% \vskip 1cm
% \subsection*{References for the chapter}
% %%%%%%%%%%%%%%%%%%%%%%%%%%%%%%%%%%%%%%%%%%%%%%%%%%%%%%%%%


\end{document}









