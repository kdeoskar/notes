\documentclass{article}

% Language setting
% Replace `english' with e.g. `spanish' to change the document language
\usepackage[english]{babel}

% Set page size and margins
% Replace `letterpaper' with`a4paper' for UK/EU standard size
\usepackage[letterpaper,top=2cm,bottom=2cm,left=3cm,right=3cm,marginparwidth=1.75cm]{geometry}

% Useful packages
\usepackage{amsmath}
\usepackage{amssymb}
\usepackage{bbm}
\usepackage{graphicx}
\usepackage{enumitem}
\usepackage{cancel}
\usepackage{tensor}
\usepackage[colorlinks=true, allcolors=blue]{hyperref}

\usepackage{hyperref}
\hypersetup{
    colorlinks=true,
    linkcolor=blue,
    filecolor=magenta,      
    urlcolor=cyan,
    pdftitle={137B HW 8 - KDEOSKAR},
    pdfpagemode=FullScreen,
    }

\urlstyle{same}

\usepackage{tikz-cd}

%%%%%%%%%%% Box pacakges and definitions %%%%%%%%%%%%%%
\usepackage[most]{tcolorbox}
\usepackage{xcolor}

% Define the colors
\definecolor{boxheader}{RGB}{0, 51, 102}  % Dark blue
\definecolor{boxfill}{RGB}{173, 216, 230}  % Light blue

% Define the tcolorbox environment
\newtcolorbox{mathdefinitionbox}[2][]{%
    colback=boxfill,   % Background color
    colframe=boxheader, % Border color
    fonttitle=\bfseries, % Bold title
    coltitle=white,     % Title text color
    title={#2},         % Title text
    enhanced,           % Enable advanced features
    attach boxed title to top left={yshift=-\tcboxedtitleheight/2}, % Center title
    boxrule=0.5mm,      % Border width
    sharp corners,      % Sharp corners for the box
    #1                  % Additional options
}
%%%%%%%%%%%%%%%%%%%%%%%%%

\newtcolorbox{dottedbox}[1][]{%
    colback=white,    % Background color
    colframe=white,    % Border color (to be overridden by dashrule)
    sharp corners,     % Sharp corners for the box
    boxrule=0pt,       % No actual border, as it will be drawn with dashrule
    boxsep=5pt,        % Padding inside the box
    enhanced,          % Enable advanced features
    overlay={\draw[dashed, thin, black, dash pattern=on \pgflinewidth off \pgflinewidth, line cap=rect] (frame.south west) rectangle (frame.north east);}, % Dotted line
    #1                 % Additional options
}

\usepackage{biblatex}
\addbibresource{sample.bib}


%%%%%%%%%%% New Commands %%%%%%%%%%%%%%
\newcommand*{\T}{\mathcal T}
\newcommand*{\cl}{\text cl}


\newcommand{\ket}[1]{|#1 \rangle}
\newcommand{\bra}[1]{\langle #1|}
\newcommand{\inner}[2]{\langle #1 | #2 \rangle}
\newcommand{\mean}[1]{\langle #1 \rangle}
\newcommand{\R}{\mathbb{R}}
\newcommand{\C}{\mathbb{C}}
\newcommand{\V}{\mathbb{V}}
\newcommand{\Hilbert}{\mathcal{H}}
\newcommand{\oper}{\hat{\Omega}}
\newcommand{\lam}{\hat{\Lambda}}

\newcommand{\bigslant}[2]{{\raisebox{.2em}{$#1$}\left/\raisebox{-.2em}{$#2$}\right.}}
\newcommand{\restr}[2]{{% we make the whole thing an ordinary symbol
  \left.\kern-\nulldelimiterspace % automatically resize the bar with \right
  #1 % the function
  \vphantom{\big|} % pretend it's a little taller at normal size
  \right|_{#2} % this is the delimiter
  }}
%%%%%%%%%%%%%%%%%%%%%%%%%%%%%%%%%%%%%%%


\tcbset{theostyle/.style={
    enhanced,
    sharp corners,
    attach boxed title to top left={
      xshift=-1mm,
      yshift=-4mm,
      yshifttext=-1mm
    },
    top=1.5ex,
    colback=white,
    colframe=blue!75!black,
    fonttitle=\bfseries,
    boxed title style={
      sharp corners,
    size=small,
    colback=blue!75!black,
    colframe=blue!75!black,
  } 
}}

\newtcbtheorem[number within=section]{Theorem}{Theorem}{%
  theostyle
}{thm}

\newtcbtheorem[number within=section]{Definition}{Definition}{%
  theostyle
}{def}



\title{Physics 137B Homework 8}
\author{Keshav Balwant Deoskar}

\begin{document}
\maketitle


%%%%%%%%%%%%%%%%%%%%%%%%%%%%%%%%%%%%%%%%%%%%%%%%%%%%%%%%%%%%%%%%%
% \textbf{Question :} 
\section*{Question 1: Fermi's Golden Rule for Three Body Decays} 
\begin{enumerate}[label=(\alph*)]
  \item \textbf{Calculating the total rate:} Using the set up laid out above, we can integrate to get an expression for the total decay.
  \item \textbf{The lifetime of the neutron:} Assuming that $\mathcal{M} \sim 1$, calculate the lifetime of the neutron. Comment on the order of magnitude compared to the experimental value, which is $\tau \sim 887.7$s.
  \item \textbf{The lifetime of the muon:} Do the same with the muon, whose experimentaly measured lifetime is $\tau \sim 2.197 \times 10^{-6}$s.
\end{enumerate}
%%%%%%%%%%%%%%%%%%%%%%%%%%%%%%%%%%%%%%%%%%%%%%%%%%%%%%%%%%%%%%%%%

\vskip 0.5cm
\underline{\textbf{Solution:}} 

\begin{enumerate}[label=(\alph*)]
  \item The inital particle (neutron) is at rest, so by concservation of momentum we have 
  \[ 0 = \mathbf{P_f} + \mathbf{p_1} + \mathbf{p_2}  \]
  where $P_f, p_1, p_2$ denote the momenta of the proton, electron, and anti-neutrino after decay.

  \vskip 0.5cm
  The protons is effectively static and we ignore its momentum, whereas we treat the proton and anti-neutrino relativistically. For simplicity, we asssume that $m_{\overline{\nu_{e}}} = 0$ since the mass of the anti-neutrino is much smaller than even that of the electron.

  \vskip 0.5cm
  Now, to get the Decay Rate, we can integrate over the momenta of the electron and anti-neutrino:
  \[ W = \frac{2\pi}{\hbar} \left|\frac{G_F \mathcal{M}}{V}\right|^2 \int \frac{V d^3 \mathbf{p_1}}{(2\pi \hbar)^3} \frac{V d^3 \mathbf{p_2}}{(2\pi \hbar)^3} \delta\left(E_0 - E_1 - E_2\right)  \]

  % The $\delta\left(E_0 - E_1 - E_2\right)$ selects the contribution from $\left(E_0 - E_1 - E_2\right) = 0$ i.e. $E_0 = E_1 + E_2 = c^2\left(m_n - m_p\right)$.

  Note that $d^3 \mathbf{p_1} \rightarrow 4\pi p_1^2 \mathrm{d}p_1$ and similarly $d^3 \mathbf{p_2} \rightarrow 4\pi p_2^2 \mathrm{d} p_2$. And since we're assuming $m_{\overline{\nu}} = 0$, 
  \begin{align*}
    &E_2^2 = (p_2 c)^2 + (\underbrace{m}_{=0} c^2)^4 \\
    \implies& E_2^2 = c^2p_2^2 \\
    \implies& 2E_2 dE_2 = 2c^2p_2 dp_2 \\
    \implies& E_2 dE_2 = c^2 p_2 dp_2 \\
    \implies& p_2 dp_2 = \frac{E_2}{c^2} dE_2 \\
    \implies& p_2^2 dp_2 = \frac{E_2 p_2}{c^2} dE_2 = \frac{E_2 \cdot \left( \frac{E_2}{c} \right)}{c^2} dE_2 \\
    \implies& p_2^2 \mathrm{d}p_2 = \frac{(E_2)^2}{c^3} \mathrm{d} E_2
  \end{align*}

  Now, 
  \begin{align*}
    W &= \frac{2\pi}{\hbar} \left|\frac{G_F \mathcal{M}}{V}\right|^2 \iint \frac{V d^3 \mathbf{p_1}}{(2\pi \hbar)^3} \frac{V d^3 \mathbf{p_2}}{(2\pi \hbar)^3} \delta\left(E_0 - E_1 - E_2\right) \\
    &= \frac{2\pi}{\hbar} \frac{\left|G_F \mathcal{M}\right|^2}{(2\pi \hbar)^6} \iint d^3 \mathbf{p_1}d^3 \mathbf{p_2}  \delta\left(E_0 - E_1 - E_2\right) \\
    &= \frac{2\pi}{\hbar} \frac{\left|G_F \mathcal{M}\right|^2}{(2\pi \hbar)^6} \iint 4\pi p_1^2 \mathrm{d}p_1 \cdot  4\pi p_2^2 \mathrm{d}p_2   \delta\left(E_0 - E_1 - E_2\right) \\
    &= \frac{2\pi \cdot (4\pi)^2}{\hbar} \frac{\left|G_F \mathcal{M}\right|^2}{(2\pi \hbar)^6} \iint p_1^2 \mathrm{d}p_1 \cdot p_2^2 \mathrm{d}p_2 \;\; \delta\left(E_0 - E_1 - E_2\right) \\
    &= \frac{\left|G_F \mathcal{M}\right|^2}{2 \pi^3 \hbar^7} \iint p_1^2 \mathrm{d}p_1 \cdot \frac{(E_2)^2}{c^3} \mathrm{d} E_2 \;\; \delta\left(E_0 - E_1 - E_2\right) \\
  \end{align*}

  The $\delta(E_0 - E_1 - E_2)$ factor selects only the contribution in which $E_0 - E_1 - E_2 = 0 \iff E_0 = E_1 + E_2$ so we get 
  \begin{align*}
    W &= \frac{\left|G_F \mathcal{M}\right|^2}{2 \pi^3 \hbar^7} \int p_1^2 \frac{(E_2)^2}{c^3} \mathrm{d}p_1 \\
    &= \frac{\left|G_F \mathcal{M}\right|^2}{2 \pi^3 \hbar^7 c^3} \int p_1^2 (E_0 - E_1)^2 \mathrm{d}p_1 \\
  \end{align*}

  In the relativistic limit ($E_1 \approx p_1c$), we have 
  \begin{align*}
    \int_0^{{p_1^{max}}} p_1^2 (E_0 - E_1)^2 \mathrm{d}p_1 &= \int_{0}^{p_1^{max}} p_1^2(E_0 - p_1c)^2 \mathrm{d}p_1 \\
    &= \int_{0}^{p_1^{max}} p_1^2(E_0^2 + p_1^2c^2 - 2E_0p_1c)^2 \mathrm{d}p_1 \\
    &= \int{0}^{p_1^{max}} E_0^2 p_1^2 + p_1^4 c^2 - 2E_0 p_1^3 c \mathrm{d}p_1 \\
    &= \left[\frac{E_0^2 \left(p_1^{max}\right)^3}{3} + \frac{\left(p_1^{max}\right)^5 c^2}{5} - 2E_0 \frac{\left(p_1^{max}\right)^4 c}{4} \right] \\
  \end{align*}

  And recall that $p_1^{max} = \frac{E_0}{c}$. Thus,
  \[ \int_0^{{p_1^{max}}} p_1^2 (E_0 - E_1)^2 \mathrm{d}p_1 = \frac{E_0^5}{30c^3} \]

  so, 
  \[ \boxed{W = \frac{\left|G_F \mathcal{M}\right|^2}{60 \pi^3 \hbar^7 c^6} \cdot E_0^5}  \]

  or, in terms of the neutron and proton masses, 
  \begin{align*}
    &W = \frac{\left|G_F \mathcal{M}\right|^2}{60 \pi^3 \hbar^7 c^6} \cdot [c^2\cdot(m_n - n_p)]^5 \\
    \implies& \boxed{W = \frac{\left|G_F \mathcal{M}\right|^2}{60 \pi^3 \hbar^7 c^4} \cdot(m_n - n_p)^5} \\
  \end{align*}

  \vskip 0.5cm
  \item Assuming $\mathcal{M} \sim 1$, the lifetime of a neutron, $\tau$, is given by
  \[ \tau = \frac{1}{W} = \frac{60 \pi^3 \hbar^7 c^6}{\left|G_F \mathcal{M}\right|^2} \cdot \frac{1}{(E_0)^5} = \frac{60 \pi^3 \hbar^7 c^4}{\left|G_F \mathcal{M}\right|^2} \cdot \frac{1}{(m_n - m_p)^5} \sim 2500s = 2.5 \times 10^3s \]

  The experimentally measured value is $\tau \sim 887.7s$, so our estimate is nearly on the same order of magnitude. It's still a very crude approximation because we're off by a factor of about 3.

  \vskip 0.5cm
  \item One way in which the muon decays is
  \[ \mu \rightarrow e + \overline{\nu}_e + \nu_{\mu}  \]
  This decay follows mechanics similar to the beta decay we studied earlier, except in this case the light particles are $\overline{\nu}_{e}, \nu_{\mu}$ while the heavy particles are $\mu, e$. Thus in this case $E_0 = c^2\left(m_{\mu} - m_e\right)$. Then, using the formula found earlier, we calculate the lifetime of the muon to be 
  \begin{align*}
    \tau &= \frac{60 \pi^3 \hbar^7 c^4}{\left|G_F \mathcal{M}\right|^2} \cdot \frac{1}{(m_{\mu} - m_e)^5} \\
    &\sim 6.5 \times 10^{-7}s
  \end{align*}

  This is one order of magnitude off from the experimental value of $\tau \approx 2.197 \times 10^{-6}s$.

  \end{enumerate}

\vskip 0.5cm 
\hrule 
\vskip 0.5cm
% \pagebreak




%%%%%%%%%%%%%%%%%%%%%%%%%%%%%%%%%%%%%%%%%%%%%%%%%%%%%%%%%%%%%%%%%
% \textbf{Question :} 
\section*{Question 2: Sudden Approximation} 
A particle of mass $m$ is in the ground state of a harmonic oscillator with the standard Hamiltonian $\hat{H}_{HO} = \frac{\hat{p}^2}{2m} + \frac{k\hat{x}^2}{2}$. At time $t = 0$, the value of $k$ changes suddenly to $k' = 4k$. Find the probability that the oscillator remains in its ground state. 
%%%%%%%%%%%%%%%%%%%%%%%%%%%%%%%%%%%%%%%%%%%%%%%%%%%%%%%%%%%%%%%%%

\vskip 0.5cm
\underline{\textbf{Solution:}} 

\begin{align*}
  &\underline{\text{Before time $t = 0$}:} \;\hat{H} \ket{\psi_n} = \hbar\omega \left(n + \frac{1}{2}\right)\ket{\psi_n} \text{ with } \hat{H} = \frac{\hat{p}^2}{2m} + \frac{k\hat{x}^2}{2} \\
  &\underline{\text{After time $t = 0$}:} \;\hat{H}' \ket{\phi_{n'}} = \hbar\omega' \left(n'+ \frac{1}{2}\right)\ket{\phi_{n'}} \text{ with } \hat{H} = \frac{\hat{p}^2}{2m} + \frac{4k\hat{x}^2}{2} \\
\end{align*}

where $\phi_n$ is the $n^{\text{th}}$ state of the new harmonic oscillator with spring constant $4k$.

The particle starts off in the ground state, and the probability that we find it \emph{still} in the ground state after the sudden change is 

\begin{align*}
  P_{(n' = 0)} &= \left| \inner{\phi_n(t)}{\psi(t)} \right|^2 \\
  &= \left|d_{(n' = 0)}\right|^2 \\
\end{align*}

In the position basis, 
\begin{align*}
  \psi_0(x) &= \inner{x}{\psi_0} \\
  &= \left(\frac{m\omega}{\pi \hbar}\right)^{1/4} e^{-m\omega x^2 / 2\hbar} \\
  &= \left(\frac{m}{\pi \hbar}\right)^{1/4} \left(\frac{k}{m}\right)^{1/8} e^{-x^2\sqrt{km}/2\hbar}
\end{align*}

and 

\begin{align*}
  \phi_{0}(x) &= \inner{x}{\phi_0} \\
  &= \left(\frac{m\omega'}{\pi \hbar}\right)^{1/4} e^{-m\omega' x^2 / 2\hbar} \\
  &= \left(\frac{m}{\pi \hbar}\right)^{1/4} \left(\frac{4k}{m}\right)^{1/8} e^{-x^2\sqrt{4km}/2\hbar}
\end{align*}

\begin{align*}
  \inner{\phi_0}{\psi_0} &= \int_{-\infty}^{\infty} dx \inner{\phi_0}{x} \inner{x}{\psi_0} \\
  &= \int_{-\infty}^{\infty} dx \phi_0^*(x) \psi_0(x) \\
  &= \left(\frac{m}{\pi \hbar}\right)^{1/4} \left(\frac{m}{\pi \hbar}\right)^{1/4} \left(\frac{k}{m}\right)^{1/8} \left(\frac{4k}{m}\right)^{1/8} \int_{-\infty}^{\infty} dx \exp\left(-\frac{x^2 \sqrt{km}}{2\hbar}\right) \exp\left(-\frac{x^2 \sqrt{4km}}{2\hbar}\right) \\
  &= \left(\frac{m}{\pi \hbar}\right)^{1/2} \cdot \underbrace{\left(\frac{k}{m}\right)^{1/4}}_{\sqrt{\omega}} \cdot (4)^{1/8} \int_{-\infty}^{\infty} dx \;e^{-\frac{3x^2 \sqrt{km}}{2\hbar}} \\
  &= 2^{1/4} \left(\frac{m\omega}{\pi \hbar}\right)^{1/2} \sqrt[root]{\frac{2}{2} \frac{\pi \hbar}{m \omega}} \\
  &= 2^{1/4} \left(\frac{2}{3}\right)^{1/2} \left(\frac{m\omega}{\pi \hbar}\right)^{1/2} \cdot \left(\frac{m\omega}{\pi \hbar}\right)^{-1/2} \\
  &= \frac{2^{3/4}}{3^{1/2}} 
\end{align*}

So, 
\begin{align*}
  d_{(n'=0)} &= \left|\inner{\phi_0}{\psi_0}\right|^2 \\
  &= \frac{2^{3/2}}{3} \\
  &= 2 \frac{\sqrt{2}}{3}
\end{align*}

\vskip 0.5cm 
\hrule 
\vskip 0.5cm
% \pagebreak


%%%%%%%%%%%%%%%%%%%%%%%%%%%%%%%%%%%%%%%%%%%%%%%%%%%%%%%%%%%%%%%%%
% \textbf{Question :} 
\section*{Question 3: More Sudden Approximation} 
If the atom is initially in the groun state, what is the probability that the $^3\text{He}^+$ ion remains in the ground state after the transition?
%%%%%%%%%%%%%%%%%%%%%%%%%%%%%%%%%%%%%%%%%%%%%%%%%%%%%%%%%%%%%%%%%

\vskip 0.5cm
\underline{\textbf{Solution:}} 

\vskip 0.5cm
The effect of the nuclear decay is to change the nuclear charge at $t = 0$ without affecting the orbital electrons. We're interested in the probability that the ion remains in the ground state after the transition.

\begin{align*}
  &\underline{\text{Before time $t = 0$}: } \hat{H}\ket{\psi_{nlm}} = -\frac{E_1}{n^2} \ket{\psi_{nlm}} \\
  &\underline{\text{After time $t = 0$}: } \hat{H}' \ket{\phi_{nlm}} = -\frac{2^2E_1}{n^2} \ket{\phi_{nlm}} \\
\end{align*}

\vskip 0.5cm
where $\hat{H}$ is the hydrogen atom hamiltonian wherein the nuclear charge is $Z = 1$ and $\hat{H}'$ is the same hamiltonian modified with $Z = 2$ in this case.

\vskip 0.5cm
The ground state solution for the a Hydrogenic Hamiltonian is 
\[ \psi_{100}(r) = \left(\frac{2}{\sqrt{4\pi a_0^3}}\right) e^{-r/a_0}  \]
with the Bohr radius being
\[ a_0 = \frac{\hbar}{mZe^2} \]

So, the Bohr radii for Tritium is 
\[ a_0^t = \frac{\hbar}{me^2} \]

and since the Helium ion has $Z = 2$, its bohr radius is
\[ a_0^{+} = \frac{\hbar}{2me^2} = \frac{a_0^t}{2}  \]

\vskip 0.5cm
the probability that the ion remains in the ground state after the sudden change is given by $\left|\inner{\phi_{100}}{\psi_{100}}\right|^2$.

\begin{align*}
  \inner{\phi_{100}}{\psi_{100}} &= \iiint \mathrm{d}^3\vec{r}  \phi_{100}^*(r) \psi_{100}(r) \\
  &= 4\pi \int_{0}^{\infty} r^2\mathrm{d}r \; \phi_{100}^*(r) \psi_{100}(r) \\
  &= 4\pi \int_{0}^{\infty} r^2\mathrm{d}r \; \left(\frac{2}{\sqrt{4\pi (a_0^t/2)^3}} e^{-2r/a_0^t}\right)^* \left( \frac{2}{\sqrt{4\pi (a_0^t)^3}} e^{-r/a_0^t} \right) \\
  &= 4\pi \int_{0}^{\infty} r^2\mathrm{d}r \left(\frac{4}{4\pi} \cdot \frac{2^{3/2}}{(a_0^t)^3}\right) e^{-3r/a_0^t} \\
  &= 4\pi \frac{\sqrt{8}}{(a_0^t)^3 \pi} \int_{0}^{\infty} r^2\mathrm{d}r \; e^{-3r/a_0^t} \\
\end{align*}

Let $y = r/a_0^t$. Then,
\begin{align*}
  \inner{\phi_{100}}{\psi_{100}} &= \frac{4\sqrt{8}}{(a_0^t)^3} \int_0^{\infty} (a_0^t y)^2 (a_0^t \mathrm{d}y) e^{-3y} \\
  &= 4 \sqrt{8} \int_{0}^{\infty} \mathrm{d}y \; y^2 e^{-3y} \\
  &= 4\sqrt{8} \cdot \left(\frac{2}{27}\right) \text{ Using Integral calculator} \\
  &= \frac{8 \sqrt{8}}{27} \\
  &= \frac{16 \sqrt{2}}{27} \\
  &= 0.838
\end{align*}

Thus the probability that we find the ion still in the ground state is $\left|\inner{\phi_{100}}{\psi_{100}}\right|^2 = (0.838)^2 \approx 0.703$ 

\vskip 0.5cm 
\hrule 
\vskip 0.5cm
% \pagebreak




% %%%%%%%%%%%%%%%%%%%%%%%%%%%%%%%%%%%%%%%%%%%%%%%%%%%%%%%%%%%%%%%%%
% % \textbf{Question :} 
% \section*{Question : } 
% \begin{enumerate}[label=(\alph*)]
  % \item
% \end{enumerate}
% %%%%%%%%%%%%%%%%%%%%%%%%%%%%%%%%%%%%%%%%%%%%%%%%%%%%%%%%%%%%%%%%%

% \vskip 0.5cm
% \underline{\textbf{Solution:}} 


% \vskip 0.5cm 
% \hrule 
% \vskip 0.5cm
% % \pagebreak




\end{document}
