\documentclass{article}

% Language setting
% Replace `english' with e.g. `spanish' to change the document language
\usepackage[english]{babel}

% Set page size and margins
% Replace `letterpaper' with`a4paper' for UK/EU standard size
\usepackage[letterpaper,top=2cm,bottom=2cm,left=3cm,right=3cm,marginparwidth=1.75cm]{geometry}

% Useful packages
\usepackage{amsmath}
\usepackage{amssymb}
\usepackage{bbm}
\usepackage{graphicx}
\usepackage{enumitem}
\usepackage{cancel}
\usepackage{tensor}
\usepackage[colorlinks=true, allcolors=blue]{hyperref}

\usepackage{hyperref}
\hypersetup{
    colorlinks=true,
    linkcolor=blue,
    filecolor=magenta,      
    urlcolor=cyan,
    pdftitle={137B HW 9 - KDEOSKAR},
    pdfpagemode=FullScreen,
    }

\urlstyle{same}

\usepackage{tikz-cd}

%%%%%%%%%%% Box pacakges and definitions %%%%%%%%%%%%%%
\usepackage[most]{tcolorbox}
\usepackage{xcolor}

% Define the colors
\definecolor{boxheader}{RGB}{0, 51, 102}  % Dark blue
\definecolor{boxfill}{RGB}{173, 216, 230}  % Light blue

% Define the tcolorbox environment
\newtcolorbox{mathdefinitionbox}[2][]{%
    colback=boxfill,   % Background color
    colframe=boxheader, % Border color
    fonttitle=\bfseries, % Bold title
    coltitle=white,     % Title text color
    title={#2},         % Title text
    enhanced,           % Enable advanced features
    attach boxed title to top left={yshift=-\tcboxedtitleheight/2}, % Center title
    boxrule=0.5mm,      % Border width
    sharp corners,      % Sharp corners for the box
    #1                  % Additional options
}
%%%%%%%%%%%%%%%%%%%%%%%%%

\newtcolorbox{dottedbox}[1][]{%
    colback=white,    % Background color
    colframe=white,    % Border color (to be overridden by dashrule)
    sharp corners,     % Sharp corners for the box
    boxrule=0pt,       % No actual border, as it will be drawn with dashrule
    boxsep=5pt,        % Padding inside the box
    enhanced,          % Enable advanced features
    overlay={\draw[dashed, thin, black, dash pattern=on \pgflinewidth off \pgflinewidth, line cap=rect] (frame.south west) rectangle (frame.north east);}, % Dotted line
    #1                 % Additional options
}

\usepackage{biblatex}
\addbibresource{sample.bib}


%%%%%%%%%%% New Commands %%%%%%%%%%%%%%
\newcommand*{\T}{\mathcal T}
\newcommand*{\cl}{\text cl}


\newcommand{\ket}[1]{|#1 \rangle}
\newcommand{\bra}[1]{\langle #1|}
\newcommand{\inner}[2]{\langle #1 | #2 \rangle}
\newcommand{\mean}[1]{\langle #1 \rangle}
\newcommand{\R}{\mathbb{R}}
\newcommand{\C}{\mathbb{C}}
\newcommand{\V}{\mathbb{V}}
\newcommand{\Hilbert}{\mathcal{H}}
\newcommand{\oper}{\hat{\Omega}}
\newcommand{\lam}{\hat{\Lambda}}

\newcommand{\bigslant}[2]{{\raisebox{.2em}{$#1$}\left/\raisebox{-.2em}{$#2$}\right.}}
\newcommand{\restr}[2]{{% we make the whole thing an ordinary symbol
  \left.\kern-\nulldelimiterspace % automatically resize the bar with \right
  #1 % the function
  \vphantom{\big|} % pretend it's a little taller at normal size
  \right|_{#2} % this is the delimiter
  }}
%%%%%%%%%%%%%%%%%%%%%%%%%%%%%%%%%%%%%%%


\tcbset{theostyle/.style={
    enhanced,
    sharp corners,
    attach boxed title to top left={
      xshift=-1mm,
      yshift=-4mm,
      yshifttext=-1mm
    },
    top=1.5ex,
    colback=white,
    colframe=blue!75!black,
    fonttitle=\bfseries,
    boxed title style={
      sharp corners,
    size=small,
    colback=blue!75!black,
    colframe=blue!75!black,
  } 
}}

\newtcbtheorem[number within=section]{Theorem}{Theorem}{%
  theostyle
}{thm}

\newtcbtheorem[number within=section]{Definition}{Definition}{%
  theostyle
}{def}



\title{Physics 137B Homework 9}
\author{Keshav Balwant Deoskar}

\begin{document}
\maketitle



%%%%%%%%%%%%%%%%%%%%%%%%%%%%%%%%%%%%%%%%%%%%%%%%%%%%%%%%%%%%%%%%%
% \textbf{Question :} 
\section*{Question 1: Equations of motion:} 
\begin{enumerate}[label=(\alph*)]
  \item Using the generalized Ehrenfest theorem, show that 
  \[  \frac{d\mean{\vec{r}}}{dt} = \frac{1}{m} \mean{\left(\vec{p} - q\vec{A}\right)}  \] where $\mean{\vec{r}}$ is the expectation value of the position of the particle.

  \item Using the fact that $\mean{\vec{v}} \equiv d\mean{\vec{r}}/dt$, show that 
  \[ m\frac{d\mean{\vec{v}}}{dt} = q \mean{\vec{E}} + \frac{q}{2m} \mean{\left(\vec{p} \times \vec{B} - \vec{B} \times \vec{p}\right)} - \frac{q^2}{m} \mean{\vec{A} \times \vec{B}} \]

  \item Assuming a unfiorm electrinc and magnetic field, show that the expectation value of $d\mean{\vec{r}}/dt$ is consistent with the Lorentz force law,
  \[ m\frac{d\mean{\vec{r}}}{dt} = q \mean{\vec{E}} + q\mean{\vec{v} \times \vec{B}}  \]
\end{enumerate}
%%%%%%%%%%%%%%%%%%%%%%%%%%%%%%%%%%%%%%%%%%%%%%%%%%%%%%%%%%%%%%%%%

\vskip 0.5cm
\underline{\textbf{Solution:}} 


\vskip 0.5cm 
\hrule 
\vskip 0.5cm
% \pagebreak



%%%%%%%%%%%%%%%%%%%%%%%%%%%%%%%%%%%%%%%%%%%%%%%%%%%%%%%%%%%%%%%%%
% \textbf{Question :} 
\section*{Question 2: Selection Rules} 
\begin{enumerate}[label=(\alph*)]
  \item \textbf{Commutation relations:} Given the commutation relation $[\hat{L}_a, \hat{V}_b] = i\hbar \epsilon_{abc} \hat{V}_c$, show that these can be written as 
  \begin{align*}
    \text{fill these in later}
  \end{align*} where $\hat{V}_{\pm} \equiv \left(\hat{V}_{x} \pm i \hat{V}_y\right)$ and similarly for $\hat{L}_{\pm}$.

  \item \textbf{Selection rules:} Evaluate the value of these six commutators sandwiches by the $\ket{nlm}$ state. Show that these are consistent with the condition
  \begin{align*}
    \text{Fill condition in later}
  \end{align*}

  \item \textbf{Selection Rules continued:} Using the properties of the Clebsh-Gordon coefficient, explain the constraints places on $\delta l = l' - l$ and $\delta m = m' - m$.
  
  \item \textbf{Parity:} Let $\vec{V}_+$ and $\vec{V}_{-}$ be vectors that are even and odd under the parity transformations. What additional constraints are placed on $\delta l$ and $\delta m$ for matrix elements of the form 
  \[  \inner{n'l'm'}{\hat{V}_{\pm}|nlm}  \] due to parity, if any.
\end{enumerate}
%%%%%%%%%%%%%%%%%%%%%%%%%%%%%%%%%%%%%%%%%%%%%%%%%%%%%%%%%%%%%%%%%

\vskip 0.5cm
\underline{\textbf{Solution:}} 

\begin{enumerate}[label=(\alph*)]
  \item Tedious. Do later.
  \item 
\end{enumerate}


\vskip 0.5cm 
\hrule 
\vskip 0.5cm
% \pagebreak


%%%%%%%%%%%%%%%%%%%%%%%%%%%%%%%%%%%%%%%%%%%%%%%%%%%%%%%%%%%%%%%%
% \textbf{Question :} 
\section*{Question 3: Dipole Appximation and Selection Rules} 

What are the diplole selection rules f or a one-dimensional harmonic oscillator potential?
%%%%%%%%%%%%%%%%%%%%%%%%%%%%%%%%%%%%%%%%%%%%%%%%%%%%%%%%%%%%%%%%%

\vskip 0.5cm
\underline{\textbf{Solution:}} 


\vskip 0.5cm 
\hrule 
\vskip 0.5cm
% \pagebreak



% %%%%%%%%%%%%%%%%%%%%%%%%%%%%%%%%%%%%%%%%%%%%%%%%%%%%%%%%%%%%%%%%%
% % \textbf{Question :} 
% \section*{Question : } 
% \begin{enumerate}[label=(\alph*)]
  % \item
% \end{enumerate}
% %%%%%%%%%%%%%%%%%%%%%%%%%%%%%%%%%%%%%%%%%%%%%%%%%%%%%%%%%%%%%%%%%

% \vskip 0.5cm
% \underline{\textbf{Solution:}} 


% \vskip 0.5cm 
% \hrule 
% \vskip 0.5cm
% % \pagebreak




\end{document}
