\documentclass{article}

% Language setting
% Replace `english' with e.g. `spanish' to change the document language
\usepackage[english]{babel}

% Set page size and margins
% Replace `letterpaper' with`a4paper' for UK/EU standard size
\usepackage[letterpaper,top=2cm,bottom=2cm,left=3cm,right=3cm,marginparwidth=1.75cm]{geometry}

% Useful packages
\usepackage{amsmath}
\usepackage{amssymb}
\usepackage{bbm}
\usepackage{graphicx}
\usepackage{enumitem}
\usepackage{cancel}
\usepackage{tensor}
\usepackage[colorlinks=true, allcolors=blue]{hyperref}

\usepackage{hyperref}
\hypersetup{
    colorlinks=true,
    linkcolor=blue,
    filecolor=magenta,      
    urlcolor=cyan,
    pdftitle={137B HW 3 - KDEOSKAR},
    pdfpagemode=FullScreen,
    }

\urlstyle{same}

\usepackage{tikz-cd}

%%%%%%%%%%% Box pacakges and definitions %%%%%%%%%%%%%%
\usepackage[most]{tcolorbox}
\usepackage{xcolor}

% Define the colors
\definecolor{boxheader}{RGB}{0, 51, 102}  % Dark blue
\definecolor{boxfill}{RGB}{173, 216, 230}  % Light blue

% Define the tcolorbox environment
\newtcolorbox{mathdefinitionbox}[2][]{%
    colback=boxfill,   % Background color
    colframe=boxheader, % Border color
    fonttitle=\bfseries, % Bold title
    coltitle=white,     % Title text color
    title={#2},         % Title text
    enhanced,           % Enable advanced features
    attach boxed title to top left={yshift=-\tcboxedtitleheight/2}, % Center title
    boxrule=0.5mm,      % Border width
    sharp corners,      % Sharp corners for the box
    #1                  % Additional options
}
%%%%%%%%%%%%%%%%%%%%%%%%%

\newtcolorbox{dottedbox}[1][]{%
    colback=white,    % Background color
    colframe=white,    % Border color (to be overridden by dashrule)
    sharp corners,     % Sharp corners for the box
    boxrule=0pt,       % No actual border, as it will be drawn with dashrule
    boxsep=5pt,        % Padding inside the box
    enhanced,          % Enable advanced features
    overlay={\draw[dashed, thin, black, dash pattern=on \pgflinewidth off \pgflinewidth, line cap=rect] (frame.south west) rectangle (frame.north east);}, % Dotted line
    #1                 % Additional options
}

\usepackage{biblatex}
\addbibresource{sample.bib}


%%%%%%%%%%% New Commands %%%%%%%%%%%%%%
\newcommand*{\T}{\mathcal T}
\newcommand*{\cl}{\text cl}


\newcommand{\ket}[1]{|#1 \rangle}
\newcommand{\bra}[1]{\langle #1|}
\newcommand{\inner}[2]{\langle #1 | #2 \rangle}
\newcommand{\mean}[1]{\langle #1 \rangle}
\newcommand{\R}{\mathbb{R}}
\newcommand{\C}{\mathbb{C}}
\newcommand{\V}{\mathbb{V}}
\newcommand{\Hilbert}{\mathcal{H}}
\newcommand{\oper}{\hat{\Omega}}
\newcommand{\lam}{\hat{\Lambda}}

\newcommand{\bigslant}[2]{{\raisebox{.2em}{$#1$}\left/\raisebox{-.2em}{$#2$}\right.}}
\newcommand{\restr}[2]{{% we make the whole thing an ordinary symbol
  \left.\kern-\nulldelimiterspace % automatically resize the bar with \right
  #1 % the function
  \vphantom{\big|} % pretend it's a little taller at normal size
  \right|_{#2} % this is the delimiter
  }}
%%%%%%%%%%%%%%%%%%%%%%%%%%%%%%%%%%%%%%%


\tcbset{theostyle/.style={
    enhanced,
    sharp corners,
    attach boxed title to top left={
      xshift=-1mm,
      yshift=-4mm,
      yshifttext=-1mm
    },
    top=1.5ex,
    colback=white,
    colframe=blue!75!black,
    fonttitle=\bfseries,
    boxed title style={
      sharp corners,
    size=small,
    colback=blue!75!black,
    colframe=blue!75!black,
  } 
}}

\newtcbtheorem[number within=section]{Theorem}{Theorem}{%
  theostyle
}{thm}

\newtcbtheorem[number within=section]{Definition}{Definition}{%
  theostyle
}{def}



\title{Physics 137B Homework 3}
\author{Keshav Balwant Deoskar}

\begin{document}
\maketitle

% \vskip 0.5cm
% \pagebreak 

%%%%%%%%%%%%%%%%%%%%%%%%%%%%%%%%%%%%%%%%%%%%%%%%%%%%%%%%%%%%%%%%%
% \textbf{Question 1: 2D Infinite Square Well Potential} 
\section*{Question 1: 2D Infinite Square Well Potential} 

Consider the 2-D Infitnite Square Well Potential 
\[ V_0(x, y)  = \begin{cases}
  0, -\frac{L}{2} \leq x, y \leq \frac{L}{2} \\
  \infty, \text{ otherwise}
\end{cases}\]

With this, we can write the unperturbed Hamiltonian as 
\[ \hat{H}^{(0)} = \frac{\hat{P}_x^2 + \hat{P}_y^2}{2m} + V_0(\hat{X}, \hat{Y}) \]

which has eigenstates 
\[ \inner{x, y}{n, n'^{(0)}} = \psi_n^{(0)}(x) \psi_{n'}^{(0)}(y) \]

where 
\[ \psi_n(x) = \begin{cases}
  \sqrt{\frac{2}{L}} \cos(\frac{n x \pi}{L}), \text{  $n$ odd} \\
  \sqrt{\frac{2}{L}} \sin(\frac{n x \pi}{L}), \text{  $n$ even} 
\end{cases} \]

with energies given by 
\[ E_{n n'}^{(0)} = \left( \frac{\pi \hbar}{L} \right)^2 \frac{1}{2m}(n^2 + n'^2) = E_0 \cdot (n^2 + n'^2), \;\;\;E_0 = \left(\frac{\pi \hbar}{L}\right) \cdot \frac{1}{2m} \]

Let us consider the perturbation to the Hamiltonian 
\[ \hat{H'}(x, y) = \lambda E_1 \sin \left(\hat{X} \frac{\pi}{L}\right) \sin \left( \hat{Y} \frac{\pi}{L} \right) \]

Calculate the first order correction to the energy of the states with unperturbed energy $E^{(0)} = 5E_0$.
%%%%%%%%%%%%%%%%%%%%%%%%%%%%%%%%%%%%%%%%%%%%%%%%%%%%%%%%%%%%%%%%%

\vskip 0.5cm
\underline{\textbf{Solution:}} 

\vskip 0.5cm
We want to find the first order correction to the energy of the states with unperturbed energy $E^{(0)} = 5E_0$ i.e. to the $\ket{n = 1, n' = 2}$, $\ket{n = 2, n' = 1}$ states.

\vskip 0.25cm
Let's denote these states as 
\begin{align*}
  \psi_{a}^{(0)}(x, y) &= \psi_1(x)\psi_2(y) = \frac{2}{L} \cos\left(x \frac{\pi}{L}\right) \sin\left(2 y \frac{\pi}{L}\right) \\
  \psi_{b}^{(0)}(x, y) &= \psi_2(x)\psi_1(y) = \frac{2}{L} \sin\left(2 x \frac{\pi}{L}\right) \cos\left(y \frac{\pi}{L}\right)
\end{align*}

\vskip 0.25cm

Since we only have two-fold degeneracy, we can directly apply the \textbf{emph
fundamental result of degenerate perturbation theory},

\[ E_{\pm}^{(1)} = \frac{1}{2} \left[ \hat{H}_{aa}' + \hat{H}_{ab}' \pm \sqrt{(\hat{H}_{aa}' - \hat{H}_{bb}')^2 + 4 \left| \hat{H}_{ab}' \right|^2 } \right] \]

where 
\[ H_{ij} = \inner{i}{\hat{H}' | j} \]

So, let's find each  of these matrix elements $H_{ij}$ by expanding them out in the position basis:

\begin{align*}
  H_{aa} &= \int_{-L/2}^{L/2}  \int_{-L/2}^{L/2}  \psi_a^{*}(x, y) \hat{H}'(x, y) \psi_a(x, y) dx dy \\
  &= \frac{4}{L^2} \int_{-L/2}^{L/2}  \int_{-L/2}^{L/2} \left[ \cos\left(x \frac{\pi}{L}\right) \sin\left(2 y \frac{\pi}{L}\right) \right]^{*} \cdot \left[ \lambda E_1 \sin \left(x \frac{\pi}{L}\right) \sin \left( y \frac{\pi}{L} \right)  \right] \cdot \left[ \cos\left(x \frac{\pi}{L}\right) \sin\left(2 y \frac{\pi}{L}\right) \right] dx dy \\
  &= \frac{4\lambda E_1}{L^2} \int_{-L/2}^{L/2} \int_{-L/2}^{L/2} \left[ \cos\left(x \frac{\pi}{L}\right) \sin\left(2 y \frac{\pi}{L}\right) \right]^2 \cdot \left[ \sin \left(x \frac{\pi}{L}\right) \sin \left( y \frac{\pi}{L} \right)  \right] dx dy \\
  &= \frac{4\lambda E_1}{L^2} \int_{-L/2}^{L/2} \int_{-L/2}^{L/2} \sin \left( y \frac{\pi}{L} \right)  \left[ \cos\left(x \frac{\pi}{L}\right) \right]^2 \cdot \sin \left(x \frac{\pi}{L}\right) dx dy \\
  &= \frac{4\lambda E_1}{L^2}  \int_{-L/2}^{L/2} \sin^2\left( 2y\frac{\pi}{L} \right) \sin \left( y \frac{\pi}{L} \right) \left( \int_{-L/2}^{L/2}   \cos^2\left(x \frac{\pi}{L}\right) \cdot \sin \left(x \frac{\pi}{L}\right) dx\right) dy \\
\end{align*}

But notice that the integral over $x$ vanishes because we are integrating an odd function over a symmetrical interval. Thus, the entire term vanishes
\[ \boxed{H_{aa} = 0} \]

\vskip 0.25cm
Next, $H_{bb}$:
\begin{align*}
  H_{bb} &= \int_{-L/2}^{L/2}  \int_{-L/2}^{L/2}  \psi_b^{*}(x, y) \hat{H}'(x, y) \psi_b(x, y) dx dy \\
  &= \int_{-L/2}^{L/2}  \int_{-L/2}^{L/2} \left[ \frac{2}{L} \sin\left(2 x \frac{\pi}{L}\right) \cos\left(y \frac{\pi}{L}\right)\right]^{*} \cdot \left[ \lambda E_1 \sin \left(x \frac{\pi}{L}\right) \sin \left( y \frac{\pi}{L} \right)  \right] \cdot \left[ \frac{2}{L} \sin\left(2 x \frac{\pi}{L}\right) \cos\left(y \frac{\pi}{L}\right) \right] dx dy \\
\end{align*}

However, notice that this is essentially the same integral as $H_{aa}$ but with $x$ and $y$ interchanged. This time, if we separate the double integral into the integral over $y$ and $x$, the integral over $y$ will vanish. So, 
\[ \boxed{H_{bb} = 0} \]

\vskip 0.25cm
Finally, the cross term $H_{ab}$:
\begin{align*}
  H_{ab} &= \int_{-L/2}^{L/2}  \int_{-L/2}^{L/2}  \psi_a^{*}(x, y) \hat{H}'(x, y) \psi_b(x, y) dx dy \\
  &= \int_{-L/2}^{L/2}  \int_{-L/2}^{L/2} \left[ \frac{2}{L} \cos\left(x \frac{\pi}{L}\right) \sin\left(2 y \frac{\pi}{L}\right) \right]^{*} \cdot \left[ \lambda E_1 \sin \left(x \frac{\pi}{L}\right) \sin \left( y \frac{\pi}{L} \right)  \right] \cdot \left[ \frac{2}{L} \sin\left(2 x \frac{\pi}{L}\right) \cos\left(y \frac{\pi}{L}\right) \right] dx dy \\
  &= \frac{4\lambda E_1}{L^2} \left[\int_{-L/2}^{L/2}  \cos\left( x \frac{\pi}{L} \right) \sin \left( x \frac{\pi}{L} \right) \sin \left( 2x \frac{\pi}{L} \right) dx\right] \cdot \left[\int_{-L/2}^{L/2} \sin\left(2y \frac{\pi}{L}\right) \sin\left(y \frac{\pi}{L}\right) \cos \left( y \frac{\pi}{L} \right) dy\right] \\
\end{align*}

This time the two integrals are non-zero. Using Mathematica, we find that each of the integrals has a value of $\frac{L}{4}$.

\begin{align*}
  \implies &H_{ab} = \frac{4 \lambda E_1}{L^2} \cdot \frac{L^2}{4^2} \\
  \implies & \boxed{H_{ab}= \frac{\lambda E_1}{4}}
\end{align*}

Therefore, the first order corrections to the energy are 
\begin{align*}
  E_{\pm}^{(1)} &= \frac{1}{2} \left[ H_{aa} + H_{bb} \pm \sqrt{(H_{aa} - H_{bb})^2 + 4 \left| H_{ab} \right|^2 } \right] \\
  &= \frac{1}{2} \left[ (0 + 0) \pm \sqrt{(0 - 0)^2 + 4\left( \frac{\lambda E_1}{4} \right)^2} \right] \\
  &= \frac{1}{2} \left(\pm \sqrt{\frac{\lambda^2 E_1^2}{2^2}}\right) \\
  &= \pm \frac{1}{2} \cdot \frac{\lambda E_1}{2} \\
  &= \pm \frac{\lambda E_1}{4}
\end{align*}

\[ \boxed{ E_{\pm}^{(1)} = \pm \frac{\lambda E_1}{4} } \]

\vskip 0.5cm 
\hrule 
\vskip 0.5cm
\pagebreak


%%%%%%%%%%%%%%%%%%%%%%%%%%%%%%%%%%%%%%%%%%%%%%%%%%%%%%%%%%%%%%%%%
% \textbf{Question :} 
\section*{Question 2: Degenerate Perturbation Theory for the 2D Harmonic Oscillator} 


\begin{enumerate}[label=(\alph*)]
  \item Derive $\hat{H}'\ket{n, n'}^{(0)}$ for arbitrary $n, n'$.
  \item Using the solution from part (a), write the 3 by 3 $\mathbf{W}$ matrix in the space of $\ket{1, 1}^{(0)}, \ket{2, 0}^{(0)}, \ket{0, 2}^{(0)}$.
  \item Solve for the corrections for the energy of these states at leading order in $\lambda$.
  \item Compare these solutions to the exact result derived in class.
  \item Find the "good" states of this system. In other words, find the linear combination of $\ket{1, 1}^{(0)}, \ket{2, 0}^{(0)}, \ket{0, 2}^{(0)}$ that diagonalizes $\hat{H}$.
\end{enumerate}
%%%%%%%%%%%%%%%%%%%%%%%%%%%%%%%%%%%%%%%%%%%%%%%%%%%%%%%%%%%%%%%%%

\vskip 0.5cm
\underline{\textbf{Solution:}} 

\vskip 0.5cm
The 2D Harmonic Oscillator has unperturbed hamiltonian 
\[ \hat{H}_{HO}^{(0)} = \frac{1}{2m} \left( \hat{p}_x^2 + \hat{p}_y^2 \right)  + \frac{1}{2} m\omega^2 \left( \hat{x}^2 + \hat{y}^2 \right)\]

% \vskip 0.25cm
and the unperturbed states are given by 
\[ \ket{n, n'}^{(0)} = \ket{n^{(0)}}_a \otimes \ket{n'^{(0)}}_b \]

Also recall that the raising and lowering operators in the $x, y$ directions are given by 
\begin{align*}
  \hat{a}_{\pm} &= \frac{1}{\sqrt{2\hbar m \omega}} \left( \mp i\hat{p}_x + m\omega \hat{x} \right) \\
  \hat{b}_{\pm} &= \frac{1}{\sqrt{2\hbar m \omega}} \left( \mp i\hat{p}_y + m\omega \hat{y} \right) 
\end{align*}

We consider the perturbation 
\[ \hat{H}' =  \lambda m \omega^2 \hat{x} \hat{y} \]

\vskip 0.5cm

\begin{enumerate}[label=(\alph*)]
  \item To derive $\hat{H}'\ket{n, n'}^{(0)}$, let's first write the perturbation in terms of the raising and lowering operators.

  We can express the position operators in terms of the lowering and raising operators as 
  \begin{align*}
    \hat{x} &= \sqrt{\frac{\hbar}{2m\omega}} \left( \hat{a}_+ + \hat{a}_- \right) \\
    \hat{y} &= \sqrt{\frac{\hbar}{2m\omega}} \left( \hat{b}_+ + \hat{b}_- \right) \\
  \end{align*}

  Now, 
  \begin{align*}
    \hat{H}' &=  \lambda m \omega^2 \hat{x} \hat{y} \\
    &= \lambda m \omega^2 \cdot \frac{\hbar}{2 m \omega} \left( \hat{a}_+ + \hat{a}_- \right)\left( \hat{b}_+ + \hat{b}_- \right) \\
    &= \frac{\lambda \omega \hbar}{2} \left( \hat{a}_{+} \hat{b}_{+} + \hat{a}_{+} \hat{b}_{-} + \hat{a}_{-} \hat{b}_{+} + \hat{a}_{-} \hat{b}_{-} \right)  
  \end{align*}

  
  So, for arbitrary $n, n'$
  \begin{align*}
    \hat{H}' \ket{n, n'}^{(0)} &= \frac{\lambda \omega \hbar}{2} \left( \hat{a}_{+} \hat{b}_{+} + \hat{a}_{+} \hat{b}_{-} + \hat{a}_{-} \hat{b}_{+} + \hat{a}_{-} \hat{b}_{-} \right) \ket{n, n'}^{(0)} \\
    &= \frac{\lambda \omega \hbar}{2} \left( \hat{a}_{+} \hat{b}_{+} + \hat{a}_{+} \hat{b}_{-} + \hat{a}_{-} \hat{b}_{+} + \hat{a}_{-} \hat{b}_{-} \right) \left( \ket{n^{(0)}}_a \otimes \ket{n'^{(0)}}_b \right) \\
    &= \frac{\lambda \omega \hbar}{2} \left( \hat{a}_{+} \hat{b}_{+} \ket{n^{(0)}}_a \otimes \ket{n'^{(0)}}_b + \hat{a}_{+} \hat{b}_{-} \ket{n^{(0)}}_a \otimes \ket{n'^{(0)}}_b + \hat{a}_{-} \hat{b}_{+} \ket{n^{(0)}}_a \otimes \ket{n'^{(0)}}_b + \hat{a}_{-} \hat{b}_{-} \ket{n^{(0)}}_a \otimes \ket{n'^{(0)}}_b \right)  \\
    &= \frac{\lambda \omega \hbar}{2} [ \sqrt{n+1}\ket{(n+1)^{(0)}} \otimes \sqrt{n'+1}\ket{(n'+1)^{(0)}} + \sqrt{n+1}\ket{(n+1)^{(0)}} \otimes \sqrt{n}\ket{(n'-1)^{(0)}} + \\
    &\sqrt{n}\ket{(n-1)^{(0)}} \otimes \sqrt{n'+1}\ket{(n'+1)^{(0)}} + \sqrt{n}\ket{(n-1)^{(0)}} \otimes \sqrt{n'}\ket{(n'-1)^{(0)}} ] \\
  \end{align*}

  Thus, we obtain The following expression:

  \begin{equation*}
\boxed{    \resizebox{\textwidth}{!}
    {$ \hat{H}' \ket{n, n'}^{(0)} = \frac{\lambda \omega \hbar}{2}  \left[\sqrt{(n+1)(n'+1)}\ket{n+,n'+1}^{(0)} + \sqrt{(n+1)(n)}\ket{n,n'-1}^{(0)} + \sqrt{(n)(n'+1)}\ket{n-1,n'}^{(0)} + \sqrt{(n)(n')}\ket{n-1,n-1'}^{(0)}\right] $
    }}
  \end{equation*}

  \vskip 0.5cm
  \item Now, we want to find the $3 \times 3$ matrix $\mathbf{W}$ in the space of $\ket{1,1}^{(0)}, \ket{2,1}^{(0)}, \ket{0,2}^{(0)}$.
  
  \vskip 0.5cm
  Looking at the result obtained in part (a), we see that the diagonal entries of $\mathbf{W}$ are all zero when we apply the bra $ \tensor[^{(0)}]{\bra{n, n'}}{}$ to $\hat{H}'\ket{n,n'}^{(0)}$, because each of term in the expression obtained in part (a) either raises or lowers the state each of the $x, y$ states.

  \vskip 0.5cm
  The only non-zero entries are 
  \begin{align*}
    \tensor[^{(0)}]{\bra{1,1}}{}\hat{H}'\ket{2,0}^{(0)}, \tensor[^{(0)}]{\bra{1,1}}{}\hat{H}'\ket{0,2}^{(0)}, \tensor[^{(0)}]{\bra{2,0}}{}\hat{H}'\ket{1,1}^{(0)}, \tensor[^{(0)}]{\bra{0,2}}{}\hat{H}'\ket{1,1}^{(0)} 
  \end{align*}
  
  Now, writing out only the non-zero terms, we have
  \begin{align*}
    \tensor[^{(0)}]{\bra{1,1}}{}\hat{H}'\ket{2,0}^{(0)} &= \frac{\lambda \omega 
    \hbar }{2} \left[ \inner{1,1}{\sqrt{2 \cdot 1}|1,1}  \right] \\
    &= \frac{\lambda \omega \hbar}{2} \cdot \sqrt{2} \cdot 1 \\
    &= \frac{\lambda \omega \hbar}{\sqrt{2}}\\
  \end{align*}

  Calculating the other non-zero matrix entires, we find that they're all equal to the same value $\frac{\lambda \omega \hbar}{\sqrt{2}}$.

  Therefore, the $3 \times 3$ matrix is given by 
  \[ \mathbf{W} = \begin{pmatrix}
    0 & \frac{\lambda \omega \hbar}{\sqrt{2}} & 0 \\
    \frac{\lambda \omega \hbar}{\sqrt{2}} & 0 & \frac{\lambda \omega \hbar}{\sqrt{2}} \\
    0 & \frac{\lambda \omega \hbar}{\sqrt{2}} & 0 \\
  \end{pmatrix} \]

  where the columns ( rows ) are along $\ket{2,0}$, $\ket{1,1}$, and $\ket{0,2}$ from left to right ( up to down ) respectively.

  \vskip 0.5cm
  \item Now, let $E^{(1)}$ denote the possible energy corrections. We obtain the value(s) of $E^{(1)}$ and the corresponding states by solving the Eigenvalue problem

  \begin{align*}
    &\begin{pmatrix}
      0 & \frac{\lambda \omega \hbar}{\sqrt{2}} & 0 \\
      \frac{\lambda \omega \hbar}{\sqrt{2}} & 0 & \frac{\lambda \omega \hbar}{\sqrt{2}} \\
      0 & \frac{\lambda \omega \hbar}{\sqrt{2}} & 0 \\
    \end{pmatrix} \begin{pmatrix}
      \alpha \\
      \beta \\
      \gamma 
    \end{pmatrix} = E^{(1)} \begin{pmatrix}
      \alpha \\
      \beta \\
      \gamma 
    \end{pmatrix} 
  \end{align*}
  \[ \implies \left(\mathbf{W} - E^{(1)} \mathbf{I}_{3}\right)\begin{pmatrix}
    \alpha \\
    \beta \\
    \gamma 
  \end{pmatrix} =  0 \]

  \[ \implies \begin{pmatrix}
    -E^{(1)} & \delta & 0 \\
    \delta & -E^{(1)} &  \delta  \\
    0 &  \delta  & -E^{(1)} \\
  \end{pmatrix} \begin{pmatrix}
    \alpha \\
    \beta \\
    \gamma 
  \end{pmatrix} = \mathbf{0}  \]

  where $\delta = \frac{\lambda \omega \hbar}{\sqrt{2}}$. This equation holds only if 
  \[ \text{det}\begin{pmatrix}
    -E^{(1)} & \delta & 0 \\
    \delta & -E^{(1)} &  \delta  \\
    0 &  \delta  & -E^{(1)} \\
  \end{pmatrix} = 0\]
  which means 
  \begin{align*}
    &-E^{(1)}(\delta^2 - {E^{(1)}}^2) - \delta(-\delta E^{(1)}  - 9) + 0 = 0\\
    \implies& - E^{(1)}(\delta^2 - {E^{(1)}}^2) + \delta E^{(1)} = 0 \\
    \implies& E^{(1)} \left[ 2 \delta - {E^{(1)}}^2 \right] = 0 \\
    \implies& \boxed{E^{(1)} = 0, E^{(1)} = + \sqrt{2} \delta, E^{(1)} = - \sqrt{2}\delta}
  \end{align*}

  Thus the energy corrections in the first order of $\lambda$ are 
  \begin{align*}
    E^{(1)} &= +\lambda \omega \hbar \\
    E^{(1)} &= 0 \\
    E^{(1)} &= -\lambda \omega \hbar \\
  \end{align*}

  % We can find the corresponding states to which these corrections apply. Doing so, we find

  \item In class, we derive the \emph{exact energies} to be 
  \[ E_{n,n'} = \left( n + \frac{1}{2} \right) \hbar\omega \left( 1 + \lambda \right)^{1/2} + \left( n' + \frac{1}{2} \right) \hbar\omega \left( 1 - \lambda \right)^{1/2} \]
  and expanding to first order in $\lambda$, we found 
  \[ E_{n,n'} = E_{n,n'}^{(0)} + (n - n') \frac{\hbar \omega \lambda}{2} + \mathcal{O}(\lambda^2) \]

  Then, the exact energies for $\ket{2,0}$, $\ket{1,1}$, and $\ket{0, 2}$ states are 
  \vskip 0.25cm
  \begin{align*}
    E_{2, 0} &= E_{2, 0}^{(0)} + \underbrace{2 \frac{\lambda \omega 
    \hbar}{2}}_{E_{2, 0}^{(1)} = +\lambda \omega \hbar} \\
    E_{1,1} &= E_{1,1}^{(0)} + \underbrace{0}_{E_{1,1}^{(1)} = 0} \\
    E_{0,2} &= E_{0,2}^{(0)} - \underbrace{2 \frac{\lambda \omega 
    \hbar}{2}}_{E_{0,2}^{(1)} = -\lambda \omega \hbar} \\
  \end{align*}
  Therefore, the corrections obtained from our solution to the first-order perturbation match exactly with the exact solution up to leading order in $\lambda$.

  \item Next, we want to find the "good states" for this system i.e. a linear combination of $\ket{1, 1}^{(0)}, \ket{2, 0}^{(0)}, \ket{0, 2}^{(0)}$ that diagonalizes $\hat{H}$.
  
  We do so by finding the corresponding egenstates to the energy corrections found in part (d). The states corresponding to the corrections $0, +, -$ are 

  \begin{align*}
    &\frac{1}{\sqrt{2}} \left( \ket{2, 0} - \ket{0, 2} \right) \\
    &\frac{1}{\sqrt{2}} \left( \ket{2, 0} + \sqrt{2}\ket{1,1} \ket{0, 2} \right) \\
    &\frac{1}{\sqrt{2}} \left( \ket{2, 0} - \sqrt{2}\ket{1,1} + \ket{0, 2} \right) \\
  \end{align*}
  These are the good states. (Didn't have enough time to type in the algebra. apologies.)
\end{enumerate}

\vskip 0.5cm 
\hrule 
\vskip 0.5cm
\pagebreak




%%%%%%%%%%%%%%%%%%%%%%%%%%%%%%%%%%%%%%%%%%%%%%%%%%%%%%%%%%%%%%%%
% \textbf{Question : } 
\section*{Question 3: The Stark Effect} 

We have the following modification to the Hamiltonian 
\[ \hat{H}' = eE_0 \hat{z} \]

and we want to 
\begin{enumerate}[label=(\alph*)]
  \item Find the "good states" and energy corrections for the $n = 2$ states.
  \item Construct a Hermitian operator $\hat{A}$ such that 
  \[ \left[\hat{H}_0, \hat{A}\right] = \left[\hat{H}', \hat{A}\right] = 0  \]
  and reconcile this with the solutions found in part $(a)$.
  \item Explain what would change if we kept track of the electron's spin.
\end{enumerate}
%%%%%%%%%%%%%%%%%%%%%%%%%%%%%%%%%%%%%%%%%%%%%%%%%%%%%%%%%%%%%%%%%

\vskip 0.5cm
\underline{\textbf{Solution:}} 

\vskip 0.5cm
Recall that the unperturbed solutions for the Hydrogen atom can be written in teh $\ket{nlm}$ basis and have unperturbed energies given by 
\[ E_n^{(0)} = \frac{E_1^(0)}{n^2} \]
where $E_1^{(0)} = -\frac{m_e}{2\hbar^2} \left( \frac{e^2}{4\pi \epsilon_0}\right)^2$

\begin{enumerate}[label=(\alph*)]
  \item The $n=2$ state has four-fold degeneracy since there are four eigenstates $\ket{n = 2,l,m}$ corresponding to the same energy $E_2$. Since we're only considering the $n=2$ states, let's just label the states with $l,m$. The degenerate state, written in the position basis, are 
  \begin{align*} 
    &\inner{x \;}{l = 0, m = 0} = \psi_{00}(x) = \\
    &\inner{x \;}{l = 1, m = 1} = \psi_{10}(x) = \\
    &\inner{x \;}{l = 1, m = 0} = \psi_{01}(x) = \\
    &\inner{x \;}{l = 1, m = -1} = \psi_{1-1}(x) = \\
  \end{align*} 

  To calculate the first-order energy corrections, we need to find the $4 \times 4$ matrix with matrix elements $\inner{l,m}{\hat{H}'| l', m'}$. But notice! The Perturbation has azimuthal symmetry since it only has dependence on $r, \theta$ and \emph{not} $\phi$. As a result, $\inner{l,m}{\hat{H}'| l', m'} = 0$ unless $m = m'$. Further, $\hat{H}' = eE_0 \hat{z} = eE_0 (r\cos(\theta))$ has odd parity. Hence, it gives non-zero values only for $l = 0, l = 1$.
 
\vskip 0.5cm 
Therefore, the only non-zero values in the matrix are $\inner{0, 0}{\hat{H}'|1, 0}$ and its conjugate $\inner{1, 0}{\hat{H}'|0, 0}$

Carrying out the calculation, we find 
\begin{align*}
  \inner{2, 0, 0}{r\cos(\theta)|2, 1, 0} &= \frac{\sqrt{3}}{4\pi} \cdot \frac{1}{\sqrt{12}a_0^3} \frac{1}{2a_0}2\pi \int_{0}^{\infty} \left( 1 - \frac{r}{2a_0} \right) r^4 e^{-4/a_0} \int_{0}^{\pi} \cos^2(\theta) \sin(\theta) d\theta \\
  &\text{(Skipping algebra because there's no time to type it out)} \\
  &= -3a_0
\end{align*}

So, the perturbation matrix is 
\[ \mathbf{W} = e E_0 \begin{pmatrix}
  0 & -3a_0 & 0 & 0 \\
  -3a_0 & 0 & 0 & 0 \\
  0 & 0 & 0 & 0 \\
  0 & 0 & 0 & 0 \\
\end{pmatrix} \]

We find the eigenvalues for this operator to be 
\[ \Delta = \pm 3eE_0 a_0 \]

The corresponding eigenstates can be found to be 
\begin{align*}
  \psi_{+} &= \frac{1}{\sqrt{2}} \left( \ket{2, 0, 0} + \ket{2, 1, 0} \right) \\
  \psi_{+} &= \frac{1}{\sqrt{2}} \left( \ket{2, 0, 0} - \ket{2, 1, 0} \right) 
\end{align*}

\item The operator $A$ is the operator 
\[ \frac{1}{\sqrt{2}}\begin{pmatrix}
  1 & 1 \\
  1 & -1 \\
\end{pmatrix} \]
in the basis of $\ket{2,0,0}, \ket{2,1,0}$. This operator is spanned by $\psi_+, \psi_-$ which are eigenstates of the perturbation as found in part (a), but are also eigenstates of the unperturbed hamiltonian because they are linear combinations of $\ket{2,0,0}$ and $\ket{2,1,0}$ which are themselves solutions of the unperturbed hamiltonian with the same energy.

\item So far we ignored the spin. However, the electron is really a spin-1/2 particle so there's an extra level of degeneracy. Each of the unperturbed eigenstates is really two degenerate states - one with spin up and the other with spin down.

In order to account for the effects of spin, we have to add extra terms. In particular, we need to account for Spin-Coupling, the Zeeman Effect, and the Darwin effect.
\end{enumerate}


\vskip 0.5cm 
\hrule 
\vskip 0.5cm
\pagebreak


%%%%%%%%%%%%%%%%%%%%%%%%%%%%%%%%%%%%%%%%%%%%%%%%%%%%%%%%%%%%%%%%%
% \textbf{Question :} 
\section*{Question 4: Feynman-Hellman Theorem} 

Consider a Hamiltonian $\hat{H}$ that depends linearly on an external parameter $\lambda$.

\begin{enumerate}[label=(\alph*)]
  \item Prove the Feynman-Hellman Theorem, which states 
  \[ \restr{\frac{\partial E_n(\lambda)}{\partial \lambda}}{\lambda = 0} =  \langle \psi_n(0) \left| \frac{\partial \hat{H}}{\partial \lambda} \right| \psi_n(0)  \rangle \]
  \item Explain how this can be reconciled with the leading order correction obtained in perturbation theory.
  \item Using a similar trick, derive the second order correction i.e. what we called $E_n^{(2)}$ in class. Clearly explain the assumptions made.
\end{enumerate}
%%%%%%%%%%%%%%%%%%%%%%%%%%%%%%%%%%%%%%%%%%%%%%%%%%%%%%%%%%%%%%%%%

\vskip 0.5cm
\underline{\textbf{Solution:}} 

\vskip 0.5cm
alph\begin{enumerate}[label=(\alph*)]
  \item Consider a hamiltonian $\hat{H}$ that depends linearly on $\lambda$. The Schroedinger equation states 
  \[ \hat{H}(\lambda)  \ket{\psi_n(\lambda)} = E_n(\lambda) \ket{\psi_n(\lambda)} \]
  Taking the inner product on both sides with $\bra{\psi_n(\lambda)}$, we get 
  \begin{align*}
    \bra{\psi_n(\lambda)} \hat{H}(\lambda)  \ket{\psi_n(\lambda)} &= E_n \inner{\psi_n(\lambda)}{\psi_n(\lambda)} \\
    &= E_n
  \end{align*}
  assumng the states are normalized. 

  Then, differentiating with respect to $\lambda$, we get 
  \begin{align*}
    \frac{\partial E_n}{\partial \lambda} &= \frac{\partial }{\partial \lambda} \inner{\psi_n(\lambda)}{\hat{H}(\lambda)|\psi_n(\lambda)} \\ 
    &= \inner{\frac{\partial \psi_n}{\partial \lambda}}{\hat{H}(\lambda) | \psi_n(\lambda)} + \inner{\psi_n}{ \frac{\partial \hat{H}}{\partial \lambda} | \psi_n} + \inner{\psi_n}{\hat{H}(\lambda) | \frac{\partial \psi_n}{\partial \lambda}} \\
    &= E_n \inner{\frac{\partial \psi_n}{\partial \lambda}}{\psi_n} + \inner{\psi_n}{\frac{\partial \hat{H}}{\partial \lambda}|\psi_n} + E_n \inner{\psi_n}{\frac{\partial \psi_n}{\partial \lambda}} \\
    &= \inner{\psi_n}{\frac{\partial \hat{H}}{\partial \lambda}|\psi_n} + E_n \inner{\frac{\partial \psi_n}{\partial \lambda}}{\psi_n} + E_n \inner{\psi_n}{\frac{\partial \psi_n}{\partial \lambda}} \\
    &= \inner{\psi_n}{\frac{\partial \hat{H}}{\partial \lambda}|\psi_n} + E_n\left( \inner{\frac{\partial \psi_n}{\partial \lambda}}{\psi_n} + \inner{\psi_n}{\frac{\partial \psi_n}{\partial \lambda}}\right) \\
    &= \inner{\psi_n}{\frac{\partial \hat{H}}{\partial \lambda}|\psi_n} + E_n \frac{\partial}{\partial \lambda} \left( \inner{\psi_n}{\psi_n} \right) \\
    &= \inner{\psi_n}{\frac{\partial \hat{H}}{\partial \lambda}|\psi_n} + E_n \underbrace{\frac{\partial}{\partial \lambda} \left( 1 \right) }_{ = 0}\\
    &= \inner{\psi_n(\lambda)}{\frac{\partial \hat{H}}{\partial \lambda}|\psi_n(\lambda)}
  \end{align*}
  Therefore, taking the limit $\lambda \rightarrow 0$ we get 
  \[ \boxed{ \restr{\frac{\partial E_n}{\partial \lambda}}{\lambda = 0} = \inner{\psi_n(0)}{\frac{\partial \hat{H}}{\partial \lambda}|\psi_n(0)}} \]

  \vskip 0.5cm
  \item In class, we found the leading order correction to energy as 
  \[ E^{(1)} = \inner{\psi_n^{(0)}}{\hat{H}'|\psi_n^{(0)}} \]

  We reconcile this with the Feynman-Hellman Theorem as if we expand the eigenstate and eigenenergies in terms of $\lambda$
  \begin{align*}
    E_n &= \sum_{k = 0}^{\infty} \lambda^k E^{(k)} \\
    \ket{n} &= \sum_{k = 0}^{\infty} \lambda^k \ket{n^{(k)}} \\
  \end{align*}

  and write the Hamiltonian as 
  \[ \hat{H} = \hat{H}_0 + \lambda \hat{H}' \]
  since it depends linearly on $\lambda$, then 
  \begin{align*}
    \restr{\frac{\partial E_n}{\partial \lambda}}{\lambda = 0} &= \frac{\partial}{\partial \lambda} \restr{\left( E^{(0)} + \lambda E^{(1)} + \lambda^2 E^{(2)} + \cdots \right){\lambda = 0}}{\lambda = 0} \\
    &= E^{(1)}
  \end{align*}

  and the Feynman-Hellman Theorem tells us that 
  \begin{align*}
    E_n^{(0)} = \restr{\frac{\partial E_n}{\partial \lambda}}{\lambda = 0} &= \inner{\psi_n(0)}{\frac{\partial \hat{H}}{\partial \lambda}|\psi_n(0)} \\
    &= \inner{\psi_n(0)}{\frac{\partial }{\partial \lambda} (\hat{H}_0 + \lambda \hat{H}')|\psi_n(0)} \\
    &= \inner{n^{(0)} + 0 \cdot n^{(1)} + 0 \cdot n^{(2)} + \cdots }{\hat{H}'| n^{(0)} + 0 \cdot n^{(1)} + 0 \cdot n^{(2)} + \cdots} \\
    &= \inner{n^{(0)}}{ \hat{H}' | n^{(0)}}
  \end{align*}
  which matches up exactly with the expression we obtained for the first-order correction!
  
  \vskip 0.5cm
  \item The expression for $E_n^{(2)}$ derived in class was 
  \[ E_n^{(2)} = \sum_{k \neq n} \frac{\left|\inner{k^{(0)}}{\hat{H}'|n^{(0)}}\right|^2}{E_n^{(0)} - E_k^{(0)}} \]

  Note that $\partial_{\lambda} \hat{H} = \partial_{\lambda} \left( \hat{H}_0 + \lambda \hat{H}' \right) = \hat{H'}$.

%   Then, if we denote the matrix elements of $\hat{H}'$ by $\hat{V}_{ij} = \inner{i^{(0)}}{\hat{H}'|j^{(0)}}$ we have 
%   \begin{align*}
%     \hat{V}_{ij} &= \inner{i^{(0)}}{\hat{H}'|j^{(0)}} \\
%     &= \inner{i^{(0)}}{\left(\partial_{\lambda} \hat{H}\right)|j^{(0)}} \\
%     &= \bra{i^{(0)}} \partial_{\lambda} \left( \hat{H} \ket{j^{(0)}} \right) - \bra{i^{(0)}} \hat{H}\left(\partial_{\lambda} \ket{j^{(0)}}\right) \\
%     &= \bra{i^{(0)}} \partial_{\lambda} \left( E_j \ket{j^{(0)} }\right) - \bra{ \hat{H} i^{(0)}} \left(\partial_{\lambda} \ket{j^{(0)}}\right) \\
%     &= \partial_{\lambda} E_j \inner{i^{(0)}}{j^{(0)}} - E_i \inner{i^{(0)}}{\partial_{\lambda} j^{(0)}} \\
%   \end{align*}
  
  Now, 
  \begin{align*}
    &\inner{m}{\partial_{\lambda} \hat{H}|n} + \inner{m}{\hat{H}| \partial_{\lambda} n} = \partial_{\lambda} E_n \inner{m}{n} + E_n \inner{m}{\partial_{\lambda} n}  \\
    \implies&\inner{m}{\partial_{\lambda} \hat{H}|n} + \inner{\hat{H}m}{ \partial_{\lambda} n} = \partial_{\lambda} E_n \inner{m}{n} + E_n \inner{m}{\partial_{\lambda} n}  \\
    \implies&\inner{m}{\partial_{\lambda} \hat{H}|n} + E_m \inner{m}{\partial_{\lambda} n} = \partial_{\lambda} E_n \delta_{mn} + E_n \inner{m}{\partial_{\lambda} n}  \\
    \implies&\inner{m}{\partial_{\lambda} \hat{H}|n} = \partial_{\lambda} E_n \delta_{mn} +( E_n - E_m)\inner{m}{\partial_{\lambda} n}  
  \end{align*}
  where $E_n, E_m$ are the \emph{exact} eigenenergies and eigenstates rather than the unperturbed energies and states.

  Thus, the matrix elements of $\hat{H}' = \partial_{\lambda} H$ are given by 
  \[\boxed{ V_{mn} = \inner{m}{V|n} = \inner{m}{\partial_{\lambda} \hat{H}|n} = \partial_{\lambda} E_n \delta_{mn} +( E_n - E_m)\inner{m}{\partial_{\lambda} n }} \]

  Therefore, when $m \neq n$, we have 
  \begin{align*}
    &V_{mn} = (E_n - E_m) \inner{m}{\partial_{\lambda} n} \\
    \implies& \boxed{\inner{m}{\partial_{\lambda} n} = \frac{V_{mn}}{(E_n - E_m)}}
  \end{align*}

  and similarly, 
  \[ \implies \boxed{\inner{\partial_{\lambda} m}{n} = \frac{V_{nm}}{(E_m - E_n)}} \]

  When $n = m$, we simply have $V_nn = \partial_{\lambda} E_n$

  We showed in part (a) that 
  \[ \partial_{\lambda} E_{n} = V_nn = \inner{n}{\partial_{\lambda} \hat{H}|n} \]

  Let's now differentiate the above equation \emph{again} with respect to $\lambda$. Then 
  \begin{align*}
    \partial^2_{\lambda} E_n &= \partial_{\lambda}\inner{n}{\partial_{\lambda} \hat{H}|n} \\
    &= \inner{\partial_{\lambda} n}{\partial_{\lambda} \hat{H}|n} + \inner{n}{\partial^2_{\lambda} \hat{H}|n} + \inner{n}{\partial_{\lambda} \hat{H}| \partial_{\lambda}n }
  \end{align*}
  But recall that $\hat{H} = \hat{H}_0 + \lambda \hat{H}'$ so $\partial^2_{\lambda} \hat{H} = 0$. Then, we have just the two terms remaining.
  \[ \partial^2_{\lambda} E_n  = \inner{\partial_{\lambda} n}{\partial_{\lambda} \hat{H}|n} + \inner{n}{\partial_{\lambda} \hat{H}| \partial_{\lambda}n } \]

  Now, inserting unity, we have 
  \begin{align*}
    \partial^2_{\lambda} E_n  &= \sum_{m} \left(\inner{\partial_{\lambda} n}{m} \inner{m}{\partial_{\lambda} \hat{H} | m} \inner{m}{n} + \inner{n}{m}\inner{m | \partial_{\lambda} \hat{H}}{m} \inner{m}{\partial_{\lambda} n}\right) \\
    &= \sum_{m \neq n} \left(\inner{\partial_{\lambda} n}{m} \inner{m}{\partial_{\lambda} \hat{H} | m} \inner{m}{n} + \inner{n}{m}\inner{m | \partial_{\lambda} \hat{H}}{m} \inner{m}{\partial_{\lambda} n}\right) \\
    &+ \left(\inner{\partial_{\lambda} n}{n} \inner{n}{\partial_{\lambda} \hat{H} | n} \inner{n}{n} + \inner{n}{n}\inner{n | \partial_{\lambda} \hat{H}}{n} \inner{n}{\partial_{\lambda} n}\right) \\
    &= 2 \sum_{m \neq n} \left( \frac{V_{nm}V_{mn}}{E_n - E_m} \right) + V_{nn}\left( \inner{\partial_{\lambda} n}{n} + \inner{n}{\partial_{\lambda} n} \right) \\
    &= 2 \sum_{m \neq n} \left( \frac{V_{nm}V_{mn}}{E_n - E_m} \right)  + V_{nn} \underbrace{\delta_{\lambda} \left( \inner{n}{n} \right)}_{ = 0} \\
    &= 2 \sum_{m \neq n} \left( \frac{V_{nm}V_{mn}}{E_n - E_m} \right) \\  
    &= 2 \sum_{m \neq n} \left( \frac{\left|V_{mn}\right|^2}{E_n - E_m} \right) \\  
  \end{align*}

  Now, since $E_n$ is a power series in terms of $\lambda$ and the energy corrections,
  \[ E_n = \sum_{k} \lambda^k E^{(k)} \]
  we have 
  \[ E^{(2)} = \frac{1}{2} \frac{\partial^2 E_n}{\partial \lambda^2} = \frac{1}{2} \partial^2_{\lambda} E_n \]

  Therefore, the second-order correction is given by 
  \[ E_{n}^{(2)} = \sum_{m \neq n} \frac{\left| \inner{m}{\hat{H}'|n} \right|^2}{E_n - E_m} \]
  
  This matches up with the result obtained in class.
\end{enumerate}

\vskip 0.5cm 
\hrule 
\vskip 0.5cm
\pagebreak


%%%%%%%%%%%%%%%%%%%%%%%%%%%%%%%%%%%%%%%%%%%%%%%%%%%%%%%%%%%%%%%%%
% \textbf{Question :} 
\section*{Question 5: Conserved Quantities with Spin-Orbit Coupling}

In quantum mechanics, when one says that the vector $\vec{J}$ is conserved, one means that for any state the system is in, the expectation values of the $3$ components of $\vec{J}$ are constant. This follows if these $3$ operators all commute with $\hat{H}'$. Show that for the Hydrogen atom with spin-orbit coupling, $\hat{H}' \propto \hat{\vec{L}} \cdot \hat{\vec{S}}, \hat{\vec{L}}$, and $\hat{\vec{S}}$ are not conserved but $\hat{\vec{J}}$ is conserved.
%%%%%%%%%%%%%%%%%%%%%%%%%%%%%%%%%%%%%%%%%%%%%%%%%%%%%%%%%%%%%%%%%

\vskip 0.5cm
\underline{\textbf{Solution:}} 

For a Hydrogen atom with spin-orbit coupling, we have 
\[ \hat{H}' \propto \vec{L} \cdot \vec{S} = \sum_i L_i S_i = L_x S_x + L_y S_y + L_z S_z \]

% We know from earlier studies of the Hydrogen atom that the unperturbed hamiltonian $\hat{H}_0$ commutes with each of 

Now, the operators $\vec{L}$ and $\vec{S}$ can be written as 
\begin{align*}
  \vec{L} &= \hat{L}_x + \hat{L}_y + \hat{L}_z \\ 
  \vec{S} &= \hat{S}_x + \hat{S}_y + \hat{S}_z \\ 
\end{align*}

Let's consider the commutators of these operators with $\hat{H}'$.

\begin{align*}
  [\vec{L}, \hat{H}'] &= [\hat{L}_x + \hat{L}_y + \hat{L}_z, \hat{H}'] \\
  &= [\hat{L}_x, \hat{H}'] + [\hat{L}_y, \hat{H}'] + [\hat{L}_z, \hat{H}'] 
\end{align*}


\subsection*{Commutator of $\vec{L}$ with $\hat{H}'$}
Let's consider just the first commutator for now:
\begin{align*}
  [\hat{L}_x, \hat{H}']  &= [\hat{L}_x, \hat{L}_x\hat{S}_x + \hat{L}_y\hat{S}_y + \hat{L}_z\hat{S}_z] \\
  &= [\hat{L}_x, \hat{L}_x\hat{S}_x] + [\hat{L}_x, \hat{L}_y\hat{S}_y] + [\hat{L}_x, \hat{L}_z\hat{S}_z] \\
\end{align*}

Now using the identity
\[ [A, BC] = [A, B]C + B[A, C] \]

we have 

\begin{align*}
  [\hat{L}_x, \hat{H}']  &= \underbrace{([L_x, L_x]S_x + L_x[L_x, S_x])}_{= 0} + \left( [L_x, L_y]S_y + L_y\underbrace{[L_x, S_y]}_{ = 0} \right) +  \left( [L_x, L_z]S_z + L_z\underbrace{[L_x, S_z]}_{ = 0} \right) \\
  &= [L_x, L_y]S_y + [L_x, L_z]S_z 
\end{align*}

Recalling the commutation relations between Angular Momentum Operators, we have 
\begin{align*}
  [\hat{L}_x, \hat{H}']  &= (i\hbar L_z)S_y  + (-i\hbar L_y)S_z
\end{align*}

Carrying out the same procedure for the other commutators, we find 
\begin{align*}
  [\hat{L}_y, \hat{H}']  &= (-i\hbar L_z)S_x  + (-i\hbar L_x)S_z \\
  [\hat{L}_z, \hat{H}']  &= (i\hbar L_y)S_x  + (i\hbar L_x)S_y \\  
\end{align*}

So, the total commutator is 
\[ \boxed{ [\vec{L}, \hat{H}'] = (i\hbar L_z)S_y  + (-i\hbar L_y)S_z + (-i\hbar L_z)S_x  + (-i\hbar L_x)S_z +  (i\hbar L_y)S_x  + (i\hbar L_x)S_y } \]
This commutator is non-zero, so the operator $\vec{L}$ does not commute with $\hat{H}'$, and thus is not conserved.

\vskip 0.5cm
\subsection*{Commutator of $\vec{S}$ with $\hat{H}'$}

Since the spin operators follow exactly the same algebra as the orbtial angular momentum operators, and $\hat{H}' = \vec{L}\cdot \vec{S}$ is symmetrical in $\vec{L}$ and $\vec{S}$, we can find the commutator with $\vec{S}$ in exactly the same way

\[ \boxed{ [\vec{S}, \hat{H}'] = (i\hbar L_y)S_z  + (-i\hbar L_z)S_y + (-i\hbar L_x)S_z  + (-i\hbar L_z)S_x +  (i\hbar L_x)S_y  + (i\hbar L_y)S_x  } \]

This commutator is also non-zero, so $\vec{S}$ does not commute with $\hat{H}'$,and thus is not conserved.

\vskip 0.5cm
\subsection*{Commutator of $\vec{J} = \vec{L} + \vec{S}$ with $\hat{H}'$}

Since $\vec{J} = \vec{L} + \vec{S}$, we obtain its commutator by simply adding the commutators of $\vec{S}$ and $\vec{L}$ with $\hat{H}$.

\begin{align*}
  [\vec{J}, \hat{H}'] &= [\vec{L}, \hat{H}'] + [\vec{S}, \hat{H}'] \\
  &= \left[  (i\hbar L_z)S_y  + (-i\hbar L_y)S_z + (-i\hbar L_z)S_x  + (-i\hbar L_x)S_z +  (i\hbar L_y)S_x  + (i\hbar L_x)S_y \right] \\
  &+ \left[ (i\hbar L_y)S_z  + (-i\hbar L_z)S_y + (-i\hbar L_x)S_z  + (-i\hbar L_z)S_x +  (i\hbar L_x)S_y  + (i\hbar L_y)S_x  \right] \\
  &= 0
\end{align*}

Thus, the total angular momentum is conserved.

\vskip 0.5cm 
\hrule 
\vskip 0.5cm
\pagebreak




% %%%%%%%%%%%%%%%%%%%%%%%%%%%%%%%%%%%%%%%%%%%%%%%%%%%%%%%%%%%%%%%%%
% % \textbf{Question :} 
% \section*{Question : } 
% %%%%%%%%%%%%%%%%%%%%%%%%%%%%%%%%%%%%%%%%%%%%%%%%%%%%%%%%%%%%%%%%%

% \vskip 0.5cm
% \underline{\textbf{Solution:}} 

% \vskip 0.5cm


% \vskip 0.5cm 
% \hrule 
% \vskip 0.5cm
% % \pagebreak


\end{document}
