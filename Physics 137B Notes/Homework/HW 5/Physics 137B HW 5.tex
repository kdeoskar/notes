\documentclass{article}

% Language setting
% Replace `english' with e.g. `spanish' to change the document language
\usepackage[english]{babel}

% Set page size and margins
% Replace `letterpaper' with`a4paper' for UK/EU standard size
\usepackage[letterpaper,top=2cm,bottom=2cm,left=3cm,right=3cm,marginparwidth=1.75cm]{geometry}

% Useful packages
\usepackage{amsmath}
\usepackage{amssymb}
\usepackage{bbm}
\usepackage{graphicx}
\usepackage{enumitem}
\usepackage{cancel}
\usepackage{tensor}
\usepackage[colorlinks=true, allcolors=blue]{hyperref}

\usepackage{hyperref}
\hypersetup{
    colorlinks=true,
    linkcolor=blue,
    filecolor=magenta,      
    urlcolor=cyan,
    pdftitle={137B HW 5 - KDEOSKAR},
    pdfpagemode=FullScreen,
    }

\urlstyle{same}

\usepackage{tikz-cd}

%%%%%%%%%%% Box pacakges and definitions %%%%%%%%%%%%%%
\usepackage[most]{tcolorbox}
\usepackage{xcolor}

% Define the colors
\definecolor{boxheader}{RGB}{0, 51, 102}  % Dark blue
\definecolor{boxfill}{RGB}{173, 216, 230}  % Light blue

% Define the tcolorbox environment
\newtcolorbox{mathdefinitionbox}[2][]{%
    colback=boxfill,   % Background color
    colframe=boxheader, % Border color
    fonttitle=\bfseries, % Bold title
    coltitle=white,     % Title text color
    title={#2},         % Title text
    enhanced,           % Enable advanced features
    attach boxed title to top left={yshift=-\tcboxedtitleheight/2}, % Center title
    boxrule=0.5mm,      % Border width
    sharp corners,      % Sharp corners for the box
    #1                  % Additional options
}
%%%%%%%%%%%%%%%%%%%%%%%%%

\newtcolorbox{dottedbox}[1][]{%
    colback=white,    % Background color
    colframe=white,    % Border color (to be overridden by dashrule)
    sharp corners,     % Sharp corners for the box
    boxrule=0pt,       % No actual border, as it will be drawn with dashrule
    boxsep=5pt,        % Padding inside the box
    enhanced,          % Enable advanced features
    overlay={\draw[dashed, thin, black, dash pattern=on \pgflinewidth off \pgflinewidth, line cap=rect] (frame.south west) rectangle (frame.north east);}, % Dotted line
    #1                 % Additional options
}

\usepackage{biblatex}
\addbibresource{sample.bib}


%%%%%%%%%%% New Commands %%%%%%%%%%%%%%
\newcommand*{\T}{\mathcal T}
\newcommand*{\cl}{\text cl}


\newcommand{\ket}[1]{|#1 \rangle}
\newcommand{\bra}[1]{\langle #1|}
\newcommand{\inner}[2]{\langle #1 | #2 \rangle}
\newcommand{\mean}[1]{\langle #1 \rangle}
\newcommand{\R}{\mathbb{R}}
\newcommand{\C}{\mathbb{C}}
\newcommand{\V}{\mathbb{V}}
\newcommand{\Hilbert}{\mathcal{H}}
\newcommand{\oper}{\hat{\Omega}}
\newcommand{\lam}{\hat{\Lambda}}

\newcommand{\bigslant}[2]{{\raisebox{.2em}{$#1$}\left/\raisebox{-.2em}{$#2$}\right.}}
\newcommand{\restr}[2]{{% we make the whole thing an ordinary symbol
  \left.\kern-\nulldelimiterspace % automatically resize the bar with \right
  #1 % the function
  \vphantom{\big|} % pretend it's a little taller at normal size
  \right|_{#2} % this is the delimiter
  }}
%%%%%%%%%%%%%%%%%%%%%%%%%%%%%%%%%%%%%%%


\tcbset{theostyle/.style={
    enhanced,
    sharp corners,
    attach boxed title to top left={
      xshift=-1mm,
      yshift=-4mm,
      yshifttext=-1mm
    },
    top=1.5ex,
    colback=white,
    colframe=blue!75!black,
    fonttitle=\bfseries,
    boxed title style={
      sharp corners,
    size=small,
    colback=blue!75!black,
    colframe=blue!75!black,
  } 
}}

\newtcbtheorem[number within=section]{Theorem}{Theorem}{%
  theostyle
}{thm}

\newtcbtheorem[number within=section]{Definition}{Definition}{%
  theostyle
}{def}



\title{Physics 137B Homework 5}
\author{Keshav Balwant Deoskar}

\begin{document}
\maketitle


%%%%%%%%%%%%%%%%%%%%%%%%%%%%%%%%%%%%%%%%%%%%%%%%%%%%%%%%%%%%%%%%%
% \textbf{Question :} 
\section*{Question 1: WKB Approximations in an infinite square well} 
\begin{enumerate}[label=(\alph*)]
  \item Find the WKB spectrum for $f(x) = \kappa x$ where $\kappa$ is a positive constant.
  \item FInd the WKB spectrum for
  \[ f(x) = \begin{cases}
    0 , 0 < x < a/2 \\
    V_0, a/2 < x < a
  \end{cases} \]
  where $V_0$ is a positive constant.
\end{enumerate}
%%%%%%%%%%%%%%%%%%%%%%%%%%%%%%%%%%%%%%%%%%%%%%%%%%%%%%%%%%%%%%%%%

\vskip 0.5cm
\underline{\textbf{Solution:}} 

\vskip 0.5cm
We found in class that the approximate quantization condition for potentials of the form 
\[ V(x) = \begin{cases}
  f(x), 0 < x < a \\
  \infty, \text{otherwise}
\end{cases} \]
can be written as 
\[ \int_{0}^{a} p(x)dx = n \pi \hbar  \]

where $p(x) = \sqrt{2m(E - V(x))}$ and $E > V(x)$. 

\begin{enumerate}[label=(\alph*)]
  \item For $f(x) = \kappa x$, $\kappa \in \R^+$ the quantization condition can be written as 
  \begin{align*}
    &\int_{0}^{a} \sqrt{2m(E - \kappa x)} dx = n \pi \hbar \\
    \implies& \sqrt{2m}\int_{0}^{a} (E - \kappa x)^{1/2} dx = n \pi \hbar \\
    \implies& \sqrt{2m} \restr{\left[ -\frac{2}{3\kappa}(E - \kappa x)^{3/2} \right]}{x=0}^{x=a} = pi \hbar \\
    \implies& \sqrt{2m} \cdot \left(-\frac{2}{3\kappa}\right) \cdot \left[ (E-\kappa a)^{3/2} - E^{3/2} \right] = n\pi \hbar
  \end{align*}
  This quantization condition gives us the eigen-energy spectrum.

  \vskip 0.5cm
  \item For \[ f(x) = \begin{cases}
    0 , 0 < x < a/2 \\
    V_0, a/2 < x < a
  \end{cases} \]
  where $V_0$ is a positive constant, we have the condition
  \begin{align*}
    &\int_{0}^a p(x)dx = n \pi \hbar \\
    \implies& \int_{0}^{a/2} p(x) dx + \int_{a/2}^a p(x) dx = n \pi\hbar \\
    \implies& \int_0^{a/2} \sqrt{2m(E - 0)} dx + \int_{a/2}^a \sqrt{2m(E - V_0)} dx = n \pi \hbar \\
    \implies& \sqrt{2mE} \cdot \frac{a}{2} + \sqrt{2m(E - V_0)} \cdot \frac{a}{2} = n \pi \hbar \\
    \implies& \frac{\sqrt{2m} a}{2} \left[\sqrt{E_n} + \sqrt{E_n - V_0}\right] = n \pi \hbar
  \end{align*}
\end{enumerate}
This is the quantization condition which gives us the eigen-energy spectrum in this case.

\vskip 0.5cm 
\hrule 
\vskip 0.5cm
% \pagebreak


%%%%%%%%%%%%%%%%%%%%%%%%%%%%%%%%%%%%%%%%%%%%%%%%%%%%%%%%%%%%%%%%%
% \textbf{Question :} 
\section*{Question 2: WKB with a finite barrier} 

Consider the finite-barrier potential define by 
\[ V(x) = \begin{cases}
  V_0, \;0 < x < a \\
  0, \text{ otherwise}
\end{cases} \]

Consider the scenario where $V_0 > E$.
\begin{enumerate}[label=(\alph*)]
  \item Calculate the transmission probability $T = |F/A|^2$ exactly.
  \item Calculate the transmission probability using WKB and the approximations made in class, namely $T \sim e^{-2\gamma}$, where $\gamma = \frac{1}{h} \int_{x}^{a} p(x')dx'$
  \item Compare these two results, and comment on which limit the scaling o these two solutions are expected to agree. We are just looking for exponential behavior.
\end{enumerate} 
%%%%%%%%%%%%%%%%%%%%%%%%%%%%%%%%%%%%%%%%%%%%%%%%%%%%%%%%%%%%%%%%%

\vskip 0.5cm
\underline{\textbf{Solution:}}
\begin{center}
  \includegraphics*[scale=0.50]{Q2 potential .png}
\end{center}

We want to think about the bound states of this system ($V_0 > E$).
We have the following wavefunctions in the different regions:
\begin{align*}
  &\text{Region I: } \psi(x) = Ae^{ikx} + Be^{-ikx} \\
  &\text{Region III: } \psi(x) = Fe^{ikx}
\end{align*}
where $k = \sqrt{2mE}/ \hbar$

but we don't know what the function is in Region II. Let's find out what it is, and then use boundary conditions to find the relation between A and F to calculate the transmission coefficient.

\vskip 0.25cm
\begin{enumerate}[label=(\alph*)]
  \item In region II, we have potential $V_0$, so the Schrödinger Equation reads as
  \[ -\frac{\hbar^2}{2m} \frac{d^2 \psi}{dx^2} + V_0 \psi = E \psi \] 
  Or equivalently,
  \begin{align*}
    \frac{d^2 \psi}{dx^2} = -\frac{2m}{\hbar^2}(E-V_0)\psi = \frac{2m}{\hbar^2}(V_0 - E)\psi
  \end{align*}
  If we write $l \equiv \sqrt{\frac{2m}{\hbar^2}(V_0 - E)}$, then the equation has the form
  \begin{align*}
    \frac{d^2 \psi}{dx^2} = l^2 \psi
  \end{align*}
  and thus has general solution of the form 
  \begin{align*}
    \boxed{\psi(x) = Ce^{lx} + De^{-lx}}
  \end{align*}
  
  So, we have 
  \begin{align*}
    \psi(x) = \begin{cases}
      \text{Region I: } Ae^{ikx} + Be^{-ikx} \\
      \text{Region II: } Ce^{lx} + De^{-lx}\\
      \text{Region III: } Fe^{ikx}
    \end{cases}
  \end{align*}
  along with the boundary conditions
  \begin{align*}
    &\text{Continuity of $\psi$ at $0$: }\psi_I(0) = \psi_I(0) \\
    &\text{Continuity of $d\psi/dx$ at $0$: } \frac{d\psi}{dx}(0^{-}) = \frac{d\psi}{dx}(0^{+}) \\
    &\text{Continuity of $\psi$ at $a$: } \psi_{II}(a) = \psi_{III}(a) \\
    &\text{Continuity of $d\psi/dx$ at $a$: } \frac{d\psi}{dx}(a^{-}) = \frac{d\psi}{dx}(a^{+}) 
  \end{align*}

  The first condition tells us 
  \begin{align*}
    &Ae^{ik(0)} + Be^{-ik(0)} = Ce^{l(0)} + De^{-l(0)} \\
    \implies&A + B = C+ D 
  \end{align*}

  the second tells us 
  \begin{align*}
    &ik\left[Ae^{ik(0)} - Be^{-ika(0)}\right] = l\left[  Ce^{l(0)} - De^{-l(0)}\right] \\ 
    \implies& ik(A - B) = l(C - D)
  \end{align*}

  the third tells us 
  \begin{align*}
    & Ce^{l(a)} + De^{-l(a)} = Fe^{ika} 
  \end{align*}

  the fourth tells us 
  \begin{align*}
    l\left[  Ce^{l(a)} - De^{-l(a)}\right] &=  ik Fe^{ika} 
  \end{align*}

  Now, via a whole bunch of algebra, we find that 
  \[ \frac{F}{A} = \frac{e^{-ika}}{\cosh(la) + i(\gamma/2)\sinh(la)}  \]
  where $\gamma \equiv l/k - k/l$.

  Now, the transmission coefficient is given by 
  \begin{align*}
    T &= \left(\frac{F}{A}\right)^* \left(\frac{F}{A}\right) \\
    &= \frac{e^{+ika}}{\cosh(la) - i(\gamma/2)\sinh(la)} \cdot \frac{e^{-ika}}{\cosh(la) + i((\gamma/2)\sinh(la))} \\
    &= \frac{1}{\cosh^2(la) + (\gamma^2/4)\sinh^2(la)}
  \end{align*}

  \[\implies \boxed{T = \frac{1}{\cosh^2(la) + (\gamma^2/4)\sinh^2(la)} } \]

  where 
  \begin{align*}
    \gamma &= \frac{l}{k} - \frac{k}{l} \\
    &= \frac{\sqrt{V_0 - E}}{\sqrt{E}} - \frac{\sqrt{E}}{\sqrt{V_0 - E}} \\
  \end{align*}

  So, 
  \begin{align*}
    \frac{\gamma^2}{4} &= \frac{1}{4} \left( \frac{V_0 - E}{E} + \frac{E}{V_0 - E} - 2\right) \\
    &= \frac{1}{4} \left( \frac{1-E/V_0}{E/V_0} + \frac{E/V_0}{1 - E/V_0} - 2\right)
  \end{align*}

  \vskip 0.5cm
  \item In class, we found that under the WKB approximation, the transmission coefficient is roughly given by
  \[ T \sim e^{-2\gamma},\;\;\; \gamma = \frac{1}{\hbar}\int_{0}^{a} | p(x') | dx' \]
  where $p(x) = \sqrt{2m(E - V_0)}$

  We find $\gamma$ to be 
  \begin{align*}
    \gamma &= \frac{1}{\hbar}\int_{0}^{a} \left| \sqrt{2m(E - V_0)} \right| dx \\
    &= \frac{1}{\hbar}\int_{0}^{a}  \sqrt{2m(V_0 - E)}  dx \\
    &=\frac{\sqrt{2m(V_0 - E)}a}{\hbar} \\
    &= l \cdot a 
  \end{align*}
  where $l \equiv \sqrt{2m(V-0 - E)}/\hbar$

  Thus, the transmission coefficient is given by 
  \begin{align*}
    T \sim e^{-2al} &= e^{-2\frac{\sqrt{2m(V_0 - E)}a}{\hbar}}
  \end{align*}

  \vskip 0.5cm
  \item Let's now compare our two results. We obtained the WKB approximation by making the assumption that the barrier was very high and very broad. This would mean that 
  \begin{align*}
    &\sqrt{2m(V_0 - E)} >> \sqrt{2mE} \\
    \implies&l >> k \\
    \implies& \gamma^2 = \left(\frac{l}{k} - \underbrace{\frac{k}{l}}_{\approx 0}\right)^2 \approx \left(\frac{l}{k}\right)^2
  \end{align*}
  Then, the transmission coefficient is approximately 
  \begin{align*}
    T &\approx \frac{1}{\cosh(la)^2 + \frac{(l/k)^2}{24}\sinh(la)} \\
  \end{align*}

  We find that this matches pretty well with $e^{-la}$ when $l >> k$.
  \begin{center}
    \includegraphics*[scale=.40]{Q2(c).png}
  \end{center}
  
\end{enumerate}

\vskip 0.5cm


\vskip 0.5cm 
\hrule 
\vskip 0.5cm
% \pagebreak




%%%%%%%%%%%%%%%%%%%%%%%%%%%%%%%%%%%%%%%%%%%%%%%%%%%%%%%%%%%%%%%%%
% \textbf{Question :} 
\section*{Question 3: Half Harmonic Oscillator} 

Consider the half-harmonic oscillator potential in 1D,
\[ V(x) = \begin{cases}
  \frac{1}{2}m\omega^2 x^2, x > 0\\
  \infty, \text{ otherwise}
\end{cases}  \]

\begin{enumerate}[label=(\alph*)]
  \item Find the exact eigenvalues to this problem.
  \item Find the quantization condition that the half-harmonic oscillator satisfie according to the WKB approximation and compare the results to what we found in class.
\end{enumerate}
%%%%%%%%%%%%%%%%%%%%%%%%%%%%%%%%%%%%%%%%%%%%%%%%%%%%%%%%%%%%%%%%%

\vskip 0.5cm
\underline{\textbf{Solution:}} 

\begin{enumerate}[label=(\alph*)]
  \item Since the potential is infinite for $x < 0$, the wavefunction must disappear in that region, which means only the normal QHO wavefunctions which vanish at the origin i.e. which are odd functions can survive. 
  
  \vskip 0.5cm
  The $n$-th QHO eigenfunction has the same parity (odd or even) as the parity of $n$ itself. So, the odd $n$ eigenfunctions survive while the even $n$ eigenfunctions are killed off. 

  \vskip 0.5cm
  Thus, the eigenvalues of the half-harmonic oscillator have the form
  \[ E_n = \left(2n - 1 + \frac{1}{2}\right)\hbar \omega = \left(2n - \frac{1}{2}\right)\hbar \omega \]
  for $n = 1,2,3,\dots$

  \item In the case that there are two turning points located at $x = 0$ and $x = x_2$, the wavefunction satisfies
  
  \[ \psi(x) = \begin{cases}
    \frac{2D}{\sqrt{p(x)}} \sin\left[\frac{1}{h} \int_x^{x_2}p(x')dx' + \frac{\pi}{4}  \right], x < x_2 \\
    \frac{D}{\sqrt{p(x)}} \exp\left[-\frac{1}{h} \int_{x_2}^{x}|p(x')|dx'  \right], x > x_2 \\
  \end{cases}  \]
  assuming that $E > V(x)$ for $x < x_2$ and $E < V(x)$ for $x > x_2$.

  For any system whose potential has a vertical wall, we have 
  \[ \frac{1}{\hbar} \int_{0}^{x} p(x)dx + \frac{\pi}{4} = n\pi  \]
  
  In the half-harmonic oscillator potential, we have 
  \[ V(x) = \begin{cases}
    \frac{1}{2}m\omega^2x^2, x > 0 \\
    \infty, \text{ otherwise}
  \end{cases}   \]

  so
  \[ p(x) = \sqrt{2m(E - V(x))} = \sqrt{2m(E - (1/2)m\omega^2x^2)} = m\omega \sqrt{x_2^2 - x^2} \]

  The first turning point is $x_1 = 0$ and the second turning point $x_2$ is $x_2 = \frac{1}{\omega}\sqrt{\frac{2E}{m}}$. So, 

  \begin{align*}
    \int_{0}^{x_2} p(x)dx &= m\omega \int_{0}^{x_2} \sqrt{x_2^2 - x^2} dx \\
    &= \frac{\pi}{4} m\omega x_2^2 \\
    &= \frac{\pi E}{2 \omega}
  \end{align*}
  
  Then, the quantization condition
  \[ \frac{1}{\hbar} \int_{0}^{x} p(x)dx + \frac{\pi}{4} = n\pi \]

  forces the eigen-energies to have the form 
  \[ E_n = \left(2n - \frac{1}{2}\right)\hbar \omega  \]
  for $n = 1,2,3\dots$

  This matches with the exact eigenenergies found in part (a).

\end{enumerate}


\vskip 0.5cm


\vskip 0.5cm 
\hrule 
\vskip 0.5cm
% \pagebreak


\end{document}
