\documentclass{article}

% Language setting
% Replace `english' with e.g. `spanish' to change the document language
\usepackage[english]{babel}

% Set page size and margins
% Replace `letterpaper' with`a4paper' for UK/EU standard size
\usepackage[letterpaper,top=2cm,bottom=2cm,left=3cm,right=3cm,marginparwidth=1.75cm]{geometry}

% Useful packages
\usepackage{amsmath}
\usepackage{amssymb}
\usepackage{graphicx}
\usepackage[colorlinks=true, allcolors=blue]{hyperref}

\usepackage{hyperref}
\hypersetup{
    colorlinks=true,
    linkcolor=blue,
    filecolor=magenta,      
    urlcolor=cyan,
    pdftitle={Overleaf Example},
    pdfpagemode=FullScreen,
    }

\urlstyle{same}

\usepackage{tikz-cd}

%%%%%%%%%%% Box pacakges and definitions %%%%%%%%%%%%%%
\usepackage[most]{tcolorbox}
\usepackage{xcolor}

% Define the colors
\definecolor{boxheader}{RGB}{0, 51, 102}  % Dark blue
\definecolor{boxfill}{RGB}{173, 216, 230}  % Light blue

% Define the tcolorbox environment
\newtcolorbox{mathdefinitionbox}[2][]{%
    colback=boxfill,   % Background color
    colframe=boxheader, % Border color
    fonttitle=\bfseries, % Bold title
    coltitle=white,     % Title text color
    title={#2},         % Title text
    enhanced,           % Enable advanced features
    attach boxed title to top left={yshift=-\tcboxedtitleheight/2}, % Center title
    boxrule=0.5mm,      % Border width
    sharp corners,      % Sharp corners for the box
    #1                  % Additional options
}
%%%%%%%%%%%%%%%%%%%%%%%%%

\usepackage{biblatex}
\addbibresource{sample.bib}


%%%%%%%%%%% New Commands %%%%%%%%%%%%%%
\newcommand*{\T}{\mathcal T}
\newcommand*{\cl}{\text cl}


\newcommand{\ket}[1]{|#1 \rangle}
\newcommand{\bra}[1]{\langle #1|}
\newcommand{\inner}[2]{\langle #1 | #2 \rangle}
\newcommand{\mean}[1]{\langle #1 \rangle}
\newcommand{\R}{\mathbb{R}}
\newcommand{\C}{\mathbb{C}}
\newcommand{\V}{\mathbb{V}}
\newcommand{\Hilbert}{\mathcal{H}}
\newcommand{\oper}{\hat{\Omega}}
\newcommand{\lam}{\hat{\Lambda}}

\newcommand{\bigslant}[2]{{\raisebox{.2em}{$#1$}\left/\raisebox{-.2em}{$#2$}\right.}}
\newcommand{\restr}[2]{{% we make the whole thing an ordinary symbol
  \left.\kern-\nulldelimiterspace % automatically resize the bar with \right
  #1 % the function
  \vphantom{\big|} % pretend it's a little taller at normal size
  \right|_{#2} % this is the delimiter
  }}
%%%%%%%%%%%%%%%%%%%%%%%%%%%%%%%%%%%%%%%


\tcbset{theostyle/.style={
    enhanced,
    sharp corners,
    attach boxed title to top left={
      xshift=-1mm,
      yshift=-4mm,
      yshifttext=-1mm
    },
    top=1.5ex,
    colback=white,
    colframe=blue!75!black,
    fonttitle=\bfseries,
    boxed title style={
      sharp corners,
    size=small,
    colback=blue!75!black,
    colframe=blue!75!black,
  } 
}}

\newtcbtheorem[number within=section]{Theorem}{Theorem}{%
  theostyle
}{thm}

\newtcbtheorem[number within=section]{Definition}{Definition}{%
  theostyle
}{def}



\title{Physics 137B Notes}
\author{Keshav Balwant Deoskar}

\begin{document}
\maketitle

% \vskip 0.5cm
These are notes taken from lectures on Quantum Mechanics delivered by Professor Raúl A. Briceño for UC Berekley's Physics 137B class in the Sprng 2024 semester.
% \pagebreak 

\tableofcontents

\pagebreak

%%%%%%%%%%%%%%%%%%%%%%%%%%%%%%%%%%%%%%%%%%%%%%%%%%%%%%%%%%%%%%%%%%
\section{January 18 - Review and Intro to Symmetries}
%%%%%%%%%%%%%%%%%%%%%%%%%%%%%%%%%%%%%%%%%%%%%%%%%%%%%%%%%%%%%%%%%%

\subsection{Why Quantum?}
\begin{itemize}
  \item It's cool lmao
  \item Computers [Quantum Computing]
  \item Applications such as Condensed Matter Physics
  \item More accurate description of reality than Classical Mechanics
    \begin{itemize}
      \item 3 of the 4 fundamental forces that we know of are Quantum Mechanical.
      \item These are the Strong Nuclear, Weak Nuclear, and Quantum Electrodynamic.
    \end{itemize}
\end{itemize}
\vskip 0.5cm

\subsection{Topics covered in 137A}
\begin{itemize}
  \item Review of Historical events such as the Photoelectric effect and other precursors to Quantum
  \item Postulates of QM
  \item Solve exactly some key examples -- such as Free Particle, Particle in a Quantum Harmonic Oscillator, Hydrogen Atom.
    \begin{itemize}
      \item Unfortunately, most of the problems in nature we want to solve are not solvable exactly. 
      \item In 137B, one of the key concepts we will introduce in that of \emph{\underline{Perturbation Theory}}, which will allow us to approximate solutions and their associated errors.
    \end{itemize}
\end{itemize}

\vskip 0.5cm
\subsection{Review of 1D QM}

\begin{itemize}
  \item A particle is described by its wavefunction $\Psi(x, t)$ and the probability density is given by $P(x, t) = |\Psi(x, t)|^2$. 

  \item The probability of finding the particle in a particular region of space is 
  \[ dx P(x, t) = dx |\Psi(x, t)|^2 = \Psi(x, t)^{*}\Psi(x, t) \]

  \item Physically, we require $\int dx |\Psi(x, t)|^2 = 1$. Such a wavefunction is called \emph{normalizable}.
  
  \item The wavefunction itself is not something we observe. Instead, our observables are the expectation values of operators. Expectation value of $\hat{\Theta}$ is 
  \[ \mean{\hat{\Theta}} = \int dx \Psi(x, t)^{*} \hat{\Theta} \Psi(x, t) \]

  \item \textbf{Principle of Superposition:} If we have a set $\{ \psi_1, \dots, \psi_n \}$ of solutions which solve the Schroedinger equation, then any linear combination of the $\psi_i$'s will also be a solutions
  \[ \Psi(x) = \sum_n c_n \psi(x, t)_n \]

  \item \textbf{Time-Dependent Schroedinger Equation:} 
  \[ \text{write the equation here later} \]

  \item \textbf{Time Independent Schroedinger Equation:} (When the potential does not depend on time)
  \[ \left(\frac{-\hbar^2}{2m} \frac{d^2}{dx^2} + V(x) \right) \psi(x) = E \psi(x) \]

  \item So, if $\{\psi_n\}$ satisfy TISE with wigenenergies $\{E_n\}$, then we know 
      \begin{itemize}
        \item The states $\{\psi_n\}$ are called \emph{Stationary States}.
        
        \item By the Principle of Superposition,
        \[ \Psi(x, 0) = \sum_n c_n \psi_n(x)  \]
        describes the total wavefunction at time $t = 0$.

        \item To verify that $\Psi$ is a valid wavefunction, we can test its normalizability.
        \begin{align*}
          1 &= \int dx \Psi(x)^{*} \Psi(x) \\
          &= \int dx \left( \sum_n c_n \psi_n(x) \right)^{*} \left( \sum_m c_m \psi_m(x) \right) \\
          &= \sum_n \sum_m \int dx c_n^{*} c_m \psi_n(x) \psi(m) \\
          &= \sum_n \sum_m \int dx c_n^{*} c_m \delta_{nm} \\
          &= \sum_n |c_n|^2
        \end{align*}

        \item At $t \neq 0$, we use the propagator to obtain the state 
        \[  \Psi(x, t) = \sum_n c_n e^{-E_n t / h} \psi_n(x)  \]
      \end{itemize}

  \item \textbf{Bra and Ket notation:} 
  \begin{align*}
    \ket{\psi} &= \sum_i \ket{i} \underbrace{\inner{i}{\psi}}_{c_i} \\
    &= \sum_i \ket{i} c_i \\
    \text{Also, recall that} \sum_i \ket{i}\bra{i} &= 1 \\
  \end{align*}

  Or in the continuous case, 
  \begin{align*}
    \ket{\psi} &= \int dx \ket{i} \underbrace{\inner{i}{\psi}}_{c_i} \\
    &= \int \ket{i} \psi(x)
  \end{align*}

\end{itemize}

% \begin{mathdefinitionbox}{Trial}
% \end{mathdefinitionbox}
\pagebreak
\end{document}
