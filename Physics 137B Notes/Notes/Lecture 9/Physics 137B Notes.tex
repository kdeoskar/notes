\documentclass{article}

% Language setting
% Replace `english' with e.g. `spanish' to change the document language
\usepackage[english]{babel}

% Set page size and margins
% Replace `letterpaper' with`a4paper' for UK/EU standard size
\usepackage[letterpaper,top=2cm,bottom=2cm,left=3cm,right=3cm,marginparwidth=1.75cm]{geometry}

% Useful packages
\usepackage{amsmath}
\usepackage{amssymb}
\usepackage{graphicx}
\usepackage[colorlinks=true, allcolors=blue]{hyperref}

\usepackage{hyperref}
\hypersetup{
    colorlinks=true,
    linkcolor=blue,
    filecolor=magenta,      
    urlcolor=cyan,
    pdftitle={137B Briceño Lecture 9},
    pdfpagemode=FullScreen,
    }

\urlstyle{same}

\usepackage{tikz-cd}

%%%%%%%%%%% Box pacakges and definitions %%%%%%%%%%%%%%
\usepackage[most]{tcolorbox}
\usepackage{xcolor}

% Define the colors
\definecolor{boxheader}{RGB}{0, 51, 102}  % Dark blue
\definecolor{boxfill}{RGB}{173, 216, 230}  % Light blue

% Define the tcolorbox environment
\newtcolorbox{mathdefinitionbox}[2][]{%
    colback=boxfill,   % Background color
    colframe=boxheader, % Border color
    fonttitle=\bfseries, % Bold title
    coltitle=white,     % Title text color
    title={#2},         % Title text
    enhanced,           % Enable advanced features
    attach boxed title to top left={yshift=-\tcboxedtitleheight/2}, % Center title
    boxrule=0.5mm,      % Border width
    sharp corners,      % Sharp corners for the box
    #1                  % Additional options
}
%%%%%%%%%%%%%%%%%%%%%%%%%

% \newtcolorbox{dottedbox}[1][]{%
%     colback=white,    % Background color
%     colframe=white,    % Border color (to be overridden by dashrule)
%     sharp corners,     % Sharp corners for the box
%     boxrule=0pt,       % No actual border, as it will be drawn with dashrule
%     boxsep=5pt,        % Padding inside the box
%     enhanced,          % Enable advanced features
%     overlay={\draw[dashed, thin, black, dash pattern=on \pgflinewidth off \pgflinewidth, line cap=rect] (frame.south west) rectangle (frame.north east);}, % Dotted line
%     #1                 % Additional options
% }


\usepackage{biblatex}
\addbibresource{sample.bib}


%%%%%%%%%%% New Commands %%%%%%%%%%%%%%
\newcommand*{\T}{\mathcal T}
\newcommand*{\cl}{\text cl}


\newcommand{\ket}[1]{|#1 \rangle}
\newcommand{\bra}[1]{\langle #1|}
\newcommand{\inner}[2]{\langle #1 | #2 \rangle}
\newcommand{\mean}[1]{\langle #1 \rangle}
\newcommand{\R}{\mathbb{R}}
\newcommand{\C}{\mathbb{C}}
\newcommand{\V}{\mathbb{V}}
\newcommand{\Hilbert}{\mathcal{H}}
\newcommand{\oper}{\hat{\Omega}}
\newcommand{\lam}{\hat{\Lambda}}

\newcommand{\bigslant}[2]{{\raisebox{.2em}{$#1$}\left/\raisebox{-.2em}{$#2$}\right.}}
\newcommand{\restr}[2]{{% we make the whole thing an ordinary symbol
  \left.\kern-\nulldelimiterspace % automatically resize the bar with \right
  #1 % the function
  \vphantom{\big|} % pretend it's a little taller at normal size
  \right|_{#2} % this is the delimiter
  }}
%%%%%%%%%%%%%%%%%%%%%%%%%%%%%%%%%%%%%%%

\newtcolorbox{dottedbox}[1][]{%
    colback=white,    % Background color
    colframe=white,    % Border color (to be overridden by dashrule)
    sharp corners,     % Sharp corners for the box
    boxrule=0pt,       % No actual border, as it will be drawn with dashrule
    boxsep=5pt,        % Padding inside the box
    enhanced,          % Enable advanced features
    overlay={\draw[dashed, thin, black, dash pattern=on \pgflinewidth off \pgflinewidth, line cap=rect] (frame.south west) rectangle (frame.north east);}, % Dotted line
    #1                 % Additional options
}


\tcbset{theostyle/.style={
    enhanced,
    sharp corners,
    attach boxed title to top left={
      xshift=-1mm,
      yshift=-4mm,
      yshifttext=-1mm
    },
    top=1.5ex,
    colback=white,
    colframe=blue!75!black,
    fonttitle=\bfseries,
    boxed title style={
      sharp corners,
    size=small,
    colback=blue!75!black,
    colframe=blue!75!black,
  } 
}}

\newtcbtheorem[number within=section]{Theorem}{Theorem}{%
  theostyle
}{thm}

\newtcbtheorem[number within=section]{Definition}{Definition}{%
  theostyle
}{def}



\title{Physics 137B Lecture 9}
\author{Keshav Balwant Deoskar}

\begin{document}
\maketitle

% \vskip 0.5cm 
These are notes taken from lectures on Quantum Mechanics delivered by Professor Raúl A. Briceño for UC Berekley's Physics 137B class in the Sprng 2024 semester.
% \pagebreak 

\tableofcontents

\pagebreak

\section{February 5 - Hydrogen Atom}

\vskip 1cm
\subsection{Review of the Hydrogen Atom}


\vskip 0.5cm
Super quickly, let's go over what we know about the Hydrogen Atom. If any of this is unfamiliar, refer to a Griffiths Intro to QM Chapter 4 or any other popular textbook.

\vskip 0.5cm
\begin{itemize}
  \item Recall that the Hydrogen Atom is an example of a \emph{\textbf{central potential}} i.e. a Potential with Spherical Symmetry
  \[ V(\vec{r}) = V(r) \]

  \vskip 0.25cm
  \item In general, when we have central potentials, we can assume \emph{\textbf{separable solutions}} 
  \[ \underbrace{\psi(\vec{r})}_{nlm} = \underbrace{R(r)}_{nl} \underbrace{Y(\theta, \phi)}_{lm} \]
  where the Radial Wavefunction is characterized by the quantum numbers $n, l$ while the Spherical Wavefunction is characterized by $l, m$.

  \vskip 0.25cm
  \item The Schrodinger Equation then separates into radial and spherical parts 
  \begin{align*}
    &\text{Radial: } \frac{\partial}{\partial r} \left( r^2 \frac{\partial}{\partial r} R(r) \right) - \frac{2mr^2}{\hbar^2} \left( V(r) - E \right) R(r) = l(l+1)R(r) \\
    &\text{Spherical: } \frac{1}{\sin(\theta)} \frac{\partial}{\partial \theta} \left( \sin(\theta) \frac{\partial}{\partial \theta} Y(\theta, \phi) \right) \frac{1}{\sin^2(\theta)} \frac{\partial^2 Y(\theta, \phi)}{\partial \phi^2} = -l(l+1)Y(\theta, \phi) \\
  \end{align*}

  \vskip 0.25cm
  \item The solutions to the Spherical Equation are called the \emph{\textbf{Spherical Harmonics}} and are given by 
  \[ Y_{lm}(\theta, \phi) = \sqrt{\frac{(l-m)!(2l+1)}{(l+m)!}} \frac{e^{i\phi m}}{\sqrt{4\pi}} P^m_l(\theta, \phi) \]
  where $ P^m_l(\theta, \phi)$  is an associated Legendre Polynomial. They have the following Normalizataion condition 

  \[ \int \mathrm{d}\Omega\; Y_{lm}^{*}(\theta, \phi) Y_{l'm'}(\theta, \phi) = \delta_{l,l'} \cdot \delta_{m,m'} \]

  \vskip 0.25cm
  \item The particular central potential which describes the Hydrogen atom is the \emph{\textbf{Coulombic Potential}}
  \[ V(r) = - \frac{e^2}{4\pi \epsilon_0} \frac{1}{r} \]

  \vskip 0.25cm
  \item Solving the Schrodinger Equation, we find the Eigen-energies are given by 
  \[ E_n = -\frac{n}{2\hbar^2} \left( \frac{e^2}{4\pi \epsilon_0} \right) \frac{1}{n^2} = \frac{E_1}{n^2}\]
  Notice that the energies only depend on $n$, so we have degeneracies due to \emph{both} $l$ and $n$.

  \vskip 0.25cm
  \item The natural lengthscale for this problem isthe \emph{\textbf{Bohr Radius}}, and we can express the energies in terms of the Bohr Radius, $a = 4\pi \epsilon_0 \hbar^2 / (m_e e^2)$ as
  \[ E_1 = - \frac{\hbar^2}{a^2} \frac{1}{n^2} \]

  \vskip 0.25cm
  \item The general solutions to the Schrodinger Equation with the Coulombic Potential are given by 
  \[ \psi_{nlm}(r, \theta, \phi) = R_{nl}(r) Y_{lm}(\theta, \phi) \]
  $m \in [-l, l]$ and $0 \leq l < n$
\end{itemize}

\vskip 1cm
\subsection{Degeneracies}

We label the state $\psi_{nlm}(r, \theta, \phi)$ as $\ket{nlm}$

\end{document}
