\documentclass{article}

% Language setting
% Replace `english' with e.g. `spanish' to change the document language
\usepackage[english]{babel}

% Set page size and margins
% Replace `letterpaper' with`a4paper' for UK/EU standard size
\usepackage[letterpaper,top=2cm,bottom=2cm,left=3cm,right=3cm,marginparwidth=1.75cm]{geometry}

% Useful packages
\usepackage{amsmath}
\usepackage{amssymb}
\usepackage{graphicx}
\usepackage[colorlinks=true, allcolors=blue]{hyperref}

\usepackage{hyperref}
\hypersetup{
    colorlinks=true,
    linkcolor=blue,
    filecolor=magenta,      
    urlcolor=cyan,
    pdftitle={137B Briceño Lecture },
    pdfpagemode=FullScreen,
    }

\urlstyle{same}

\usepackage{tikz-cd}

%%%%%%%%%%% Box pacakges and definitions %%%%%%%%%%%%%%
\usepackage[most]{tcolorbox}
\usepackage{xcolor}

% Define the colors
\definecolor{boxheader}{RGB}{0, 51, 102}  % Dark blue
\definecolor{boxfill}{RGB}{173, 216, 230}  % Light blue

% Define the tcolorbox environment
\newtcolorbox{mathdefinitionbox}[2][]{%
    colback=boxfill,   % Background color
    colframe=boxheader, % Border color
    fonttitle=\bfseries, % Bold title
    coltitle=white,     % Title text color
    title={#2},         % Title text
    enhanced,           % Enable advanced features
    attach boxed title to top left={yshift=-\tcboxedtitleheight/2}, % Center title
    boxrule=0.5mm,      % Border width
    sharp corners,      % Sharp corners for the box
    #1                  % Additional options
}
%%%%%%%%%%%%%%%%%%%%%%%%%

% \newtcolorbox{dottedbox}[1][]{%
%     colback=white,    % Background color
%     colframe=white,    % Border color (to be overridden by dashrule)
%     sharp corners,     % Sharp corners for the box
%     boxrule=0pt,       % No actual border, as it will be drawn with dashrule
%     boxsep=5pt,        % Padding inside the box
%     enhanced,          % Enable advanced features
%     overlay={\draw[dashed, thin, black, dash pattern=on \pgflinewidth off \pgflinewidth, line cap=rect] (frame.south west) rectangle (frame.north east);}, % Dotted line
%     #1                 % Additional options
% }


\usepackage{biblatex}
\addbibresource{sample.bib}


%%%%%%%%%%% New Commands %%%%%%%%%%%%%%
\newcommand*{\T}{\mathcal T}
\newcommand*{\cl}{\text cl}


\newcommand{\ket}[1]{|#1 \rangle}
\newcommand{\bra}[1]{\langle #1|}
\newcommand{\inner}[2]{\langle #1 | #2 \rangle}
\newcommand{\mean}[1]{\langle #1 \rangle}
\newcommand{\R}{\mathbb{R}}
\newcommand{\C}{\mathbb{C}}
\newcommand{\V}{\mathbb{V}}
\newcommand{\Hilbert}{\mathcal{H}}
\newcommand{\oper}{\hat{\Omega}}
\newcommand{\lam}{\hat{\Lambda}}

\newcommand{\bigslant}[2]{{\raisebox{.2em}{$#1$}\left/\raisebox{-.2em}{$#2$}\right.}}
\newcommand{\restr}[2]{{% we make the whole thing an ordinary symbol
  \left.\kern-\nulldelimiterspace % automatically resize the bar with \right
  #1 % the function
  \vphantom{\big|} % pretend it's a little taller at normal size
  \right|_{#2} % this is the delimiter
  }}
%%%%%%%%%%%%%%%%%%%%%%%%%%%%%%%%%%%%%%%

\newtcolorbox{dottedbox}[1][]{%
    colback=white,    % Background color
    colframe=white,    % Border color (to be overridden by dashrule)
    sharp corners,     % Sharp corners for the box
    boxrule=0pt,       % No actual border, as it will be drawn with dashrule
    boxsep=5pt,        % Padding inside the box
    enhanced,          % Enable advanced features
    overlay={\draw[dashed, thin, black, dash pattern=on \pgflinewidth off \pgflinewidth, line cap=rect] (frame.south west) rectangle (frame.north east);}, % Dotted line
    #1                 % Additional options
}


\tcbset{theostyle/.style={
    enhanced,
    sharp corners,
    attach boxed title to top left={
      xshift=-1mm,
      yshift=-4mm,
      yshifttext=-1mm
    },
    top=1.5ex,
    colback=white,
    colframe=blue!75!black,
    fonttitle=\bfseries,
    boxed title style={
      sharp corners,
    size=small,
    colback=blue!75!black,
    colframe=blue!75!black,
  } 
}}

\newtcbtheorem[number within=section]{Theorem}{Theorem}{%
  theostyle
}{thm}

\newtcbtheorem[number within=section]{Definition}{Definition}{%
  theostyle
}{def}



\title{Physics 137B Lecture (Not sure)}
\author{Keshav Balwant Deoskar}

\begin{document}
\maketitle

% \vskip 0.5cm 
These are notes taken from lectures on Quantum Mechanics delivered by Professor Raúl A. Briceño for UC Berekley's Physics 137B class in the Sprng 2024 semester.
% \pagebreak 

\tableofcontents

\pagebreak

\section{March 4 - Applications of the Fermi Gas Model}

\vskip 1cm 

\subsection{Recap - Fermi Gas Model}

\begin{itemize}
  \item Last time, we began studying the Fermi Gas Model, which is the large $N$ limit of a system comprised of $N$ spin-1/2 fermions in a box.
  \item We calculated the number of states corresponding to each momentum $k$ and then integrated over $k$ space:
  \[ dn = 2 \left(\frac{L}{2\pi}\right)^3 d\Omega \; k^2 dk  \]
  \item We found the \emph{\textbf{Fermi Energy, $E_F$}} and the \emph{\textbf{Total Energy, $F_T$}}
  \[ E_T = N \cdot \frac{3}{2} E_F  \]
  where $E_f = \left(\rho 3 \pi^2\right)^{2/3} \cdot \frac{\hbar^2}{2m} $
  \item Notably, however, we only considered the system at $T = 0 K$.
  \item The Fermi surface is the surface which bounds the sphere in $n$-space [Write better explanation later]
\end{itemize}

\vskip 1cm
\subsection{Finite Temperature}
\begin{itemize}
  \item At $T = 0 K$, the system collapses to the ground state.
  \item At $T > 0 K$, particles have enough energy to jump up to higher energy levels aka. jump the Fermo Surface. We find that 
  \[ \mean{n(E)} = \frac{1}{\exp\left[ \frac{E- E_F}{T}\right] - 1}  \]
  \item Finite Temperature stuff is beyond the scope of this course, however, so we'll stick to the $T = 0 K$ case.
\end{itemize}

\vskip 1cm
\subsection{Application - Stellar Evolution}
\begin{itemize}
  \item In the early stages, stars are mainly composed of Hydrogen. 
  \item The thing that drives energy production in the stars is nuclear fusion, and the primary reaction that takes place is 
  \begin{align*}
    &p + p \rightarrow \underbrace{d}_{n,p} + e^+ + \overline{\nu}_e \\
    &d + p \rightarrow He_3 + \gamma \\
    &He_3 + He_3 \rightarrow p + p + He_4 \\
    &\vdots
  \end{align*}
  \item What keeps the Star stable is the gravitational pull due to the immense mass of the star, which conflicts with the radiation pressure due to the nuclear reactions.
  \item Eventually, however,  run out of nuclear fuel and gravity wins out. At that point, depending on the size of the star, various things can happen.
  \item Stars with masses $M < M_{sun} \times 1.5$ can form \emph{white dwarfs} $[e^-]$ when gravity wins out.
  \item For stars with masses much greater than the sun, it is more energetically neutral for the reaction $e^- + p \rightarrow n + \overline{\nu}_e$, hence they are called neutron stars.
\end{itemize}

\subsection{White Dwarfs}
We can crudely model white dwarfs are seas of Fermi Elections. 

\vskip 0.5cm
Energy of the system: $E_T = E_{kin} + E_{grav}$ where $E_{kin} = N\frac{3}{5}E_F$

Write rest of the notes from recording, just pay attention in class.

\end{document}
