\documentclass{article}

% Language setting
% Replace `english' with e.g. `spanish' to change the document language
\usepackage[english]{babel}

% Set page size and margins
% Replace `letterpaper' with`a4paper' for UK/EU standard size
\usepackage[letterpaper,top=2cm,bottom=2cm,left=3cm,right=3cm,marginparwidth=1.75cm]{geometry}

% Useful packages
\usepackage{amsmath}
\usepackage{amssymb}
\usepackage{graphicx}
\usepackage[colorlinks=true, allcolors=blue]{hyperref}

\usepackage{hyperref}
\hypersetup{
    colorlinks=true,
    linkcolor=blue,
    filecolor=magenta,      
    urlcolor=cyan,
    pdftitle={137B Briceño Lecture },
    pdfpagemode=FullScreen,
    }

\urlstyle{same}

\usepackage{tikz-cd}

%%%%%%%%%%% Box pacakges and definitions %%%%%%%%%%%%%%
\usepackage[most]{tcolorbox}
\usepackage{xcolor}

% Define the colors
\definecolor{boxheader}{RGB}{0, 51, 102}  % Dark blue
\definecolor{boxfill}{RGB}{173, 216, 230}  % Light blue

% Define the tcolorbox environment
\newtcolorbox{mathdefinitionbox}[2][]{%
    colback=boxfill,   % Background color
    colframe=boxheader, % Border color
    fonttitle=\bfseries, % Bold title
    coltitle=white,     % Title text color
    title={#2},         % Title text
    enhanced,           % Enable advanced features
    attach boxed title to top left={yshift=-\tcboxedtitleheight/2}, % Center title
    boxrule=0.5mm,      % Border width
    sharp corners,      % Sharp corners for the box
    #1                  % Additional options
}
%%%%%%%%%%%%%%%%%%%%%%%%%

% \newtcolorbox{dottedbox}[1][]{%
%     colback=white,    % Background color
%     colframe=white,    % Border color (to be overridden by dashrule)
%     sharp corners,     % Sharp corners for the box
%     boxrule=0pt,       % No actual border, as it will be drawn with dashrule
%     boxsep=5pt,        % Padding inside the box
%     enhanced,          % Enable advanced features
%     overlay={\draw[dashed, thin, black, dash pattern=on \pgflinewidth off \pgflinewidth, line cap=rect] (frame.south west) rectangle (frame.north east);}, % Dotted line
%     #1                 % Additional options
% }


\usepackage{biblatex}
\addbibresource{sample.bib}


%%%%%%%%%%% New Commands %%%%%%%%%%%%%%
\newcommand*{\T}{\mathcal T}
\newcommand*{\cl}{\text cl}


\newcommand{\ket}[1]{|#1 \rangle}
\newcommand{\bra}[1]{\langle #1|}
\newcommand{\inner}[2]{\langle #1 | #2 \rangle}
\newcommand{\mean}[1]{\langle #1 \rangle}
\newcommand{\R}{\mathbb{R}}
\newcommand{\C}{\mathbb{C}}
\newcommand{\V}{\mathbb{V}}
\newcommand{\Hilbert}{\mathcal{H}}
\newcommand{\oper}{\hat{\Omega}}
\newcommand{\lam}{\hat{\Lambda}}

\newcommand{\bigslant}[2]{{\raisebox{.2em}{$#1$}\left/\raisebox{-.2em}{$#2$}\right.}}
\newcommand{\restr}[2]{{% we make the whole thing an ordinary symbol
  \left.\kern-\nulldelimiterspace % automatically resize the bar with \right
  #1 % the function
  \vphantom{\big|} % pretend it's a little taller at normal size
  \right|_{#2} % this is the delimiter
  }}
%%%%%%%%%%%%%%%%%%%%%%%%%%%%%%%%%%%%%%%

\newtcolorbox{dottedbox}[1][]{%
    colback=white,    % Background color
    colframe=white,    % Border color (to be overridden by dashrule)
    sharp corners,     % Sharp corners for the box
    boxrule=0pt,       % No actual border, as it will be drawn with dashrule
    boxsep=5pt,        % Padding inside the box
    enhanced,          % Enable advanced features
    overlay={\draw[dashed, thin, black, dash pattern=on \pgflinewidth off \pgflinewidth, line cap=rect] (frame.south west) rectangle (frame.north east);}, % Dotted line
    #1                 % Additional options
}


\tcbset{theostyle/.style={
    enhanced,
    sharp corners,
    attach boxed title to top left={
      xshift=-1mm,
      yshift=-4mm,
      yshifttext=-1mm
    },
    top=1.5ex,
    colback=white,
    colframe=blue!75!black,
    fonttitle=\bfseries,
    boxed title style={
      sharp corners,
    size=small,
    colback=blue!75!black,
    colframe=blue!75!black,
  } 
}}

\newtcbtheorem[number within=section]{Theorem}{Theorem}{%
  theostyle
}{thm}

\newtcbtheorem[number within=section]{Definition}{Definition}{%
  theostyle
}{def}



\title{Physics 137B Lecture (Not sure)}
\author{Keshav Balwant Deoskar}

\begin{document}
\maketitle

% \vskip 0.5cm 
These are notes taken from lectures on Quantum Mechanics delivered by Professor Raúl A. Briceño for UC Berekley's Physics 137B class in the Sprng 2024 semester.
% \pagebreak 

\tableofcontents

\pagebreak

\section{March 1 - }

\vskip 1cm

\subsection*{Recap}
\begin{itemize}
  \item Last time we discussed two particle states:
  \begin{align*}
    \ket{a, b}_{B} &= \frac{1}{\sqrt{2}} \left( \ket{a} \otimes \ket{b}+ \ket{b} \otimes \ket{a} \right) \\
    \ket{a, b}_{F} &= \frac{1}{\sqrt{2}} \left( \ket{a} \otimes \ket{b} - \ket{b} \otimes \ket{a} \right)
  \end{align*}
  \item Since the Fermionic wavefunction has to be antisysmmetric, it must be the case that 
  \[  \ket{a,a}_{F} = 0 \]
\end{itemize}

\vskip 0.5cm
\subsection{Slatter Determinant}
\vskip 0.5cm
If, instead of 2, we have $N$ particles auseful tool for constructing antisymmetric states $\inner{\phi_i}{\phi_j} = S_{ij}$ is the \emph{\textbf{Slatter Determinant}}.

\vskip 0.5cm
\begin{align*}
  \ket{\psi}_{A, N} = \frac{1}{\sqrt{N!}} \begin{bmatrix}
    \ket{\phi_1(\vec{r}_1)} & \ket{\phi_2(\vec{r}_1)} & \cdots & \ket{\phi_N(\vec{r}_1)} \\
    \ket{\phi_N(\vec{r}_1)} & \ket{\phi_N(\vec{r}_2)} \cdots \ket{\phi_N(\vec{r}_3)} \\
    \vdots&  & \ddots \\
    \ket{\phi_1(\vec{r}_N)} & \ket{\phi_2(\vec{r}_N)} & \cdots & \ket{\phi_N(\vec{r}_N)}
  \end{bmatrix}
\end{align*}

\vskip 1cm
\subsection{Non-interacting $N$ particles}

[Fill in some stuff from recording]


\vskip 0.5cm
Let's consider : $\begin{matrix}
  \text{distinguishable} \\ \text{spinless bosons} \\ \text{spin-1/2 fermions}
\end{matrix}$

\vskip 0.5cm
\subsubsection{Distinguishable}

[Write later]

\vskip 0.5cm
\subsubsection{Spinless Bosons}

[Fill later]

\emph{Note:} This is the starting point for Bose-Einstein condensates (at low temperatures, all of the bosons neter the same energy state.)

\vskip 0.5cm
\subsubsection{Spin-1/2 fermions}

This time, it depends on the whether $N$ is odd or even.


\vskip 0.5cm
Next, let's take the large $N$ limit of a system with $N$ fermions. This gives us the \emph{Fermi Gas Model}.

\vskip 1cm
\subsection{Fermi Gas Model}
\begin{itemize}
  \item Large $N$ limit of $N$ non-interacting fermions
  \item Metals
  \item Heavy Nuclei
  \item White Dwarfs and Neutron Stars
\end{itemize}

\vskip 0.5cm
Let's assume our fermions live in a cube of side-length $L$ which is so large that our boundary conditions \emph{don't quite matter}. i.e. it doesn't matter much to the behavior in the interior whether we apply periodic boundary conditions or something else.

\vskip 0.5cm
Let's impose preiodic boundary conditions. Then,
\begin{align*}
  &\phi(x+ L, y, z) = \phi(x, y, z) \\
  &\phi(x, y, z) \sim e^{i\left(k_x x + k_y y + k_z z\right)} \text{ where } k_xL = 2\pi n_x
\end{align*}

It's useful to define the vector
\[ \vec{k} = \frac{2\pi}{L} \left(n_x, n_y, n_z\right) \]

\vskip 0.5cm
Let's try to count the number of modes which can fit in this box, with the goal of finding the total number of spin-1/2 particles inside the box.

\vskip 0.5cm
We have 
\begin{align*}
  dn &= 2 \times \mathrm{d}n_x \mathrm{d}n_y \mathrm{d}n_z \\
  &= 2 \times \left(\frac{L}{2\pi}\right)^2 \mathrm{d}k_x \mathrm{d}k_y \mathrm{d}k_z
\end{align*}

\vskip 0.5cm
We're going to engage in a slight \emph{oopsie} by assuming that even though our momenta are discrete, the steps between consecutive momenta are infinitessimally small. We're going to take the large $N$ limit, where this assumption does hurt, so it should work out.

\vskip 0.5cm
[Write motivation for Fermi Energy and Fermi Momentum]

\vskip 0.5cm
\subsubsection*{Fermi Energy and Fermi Momentum}

\begin{mathdefinitionbox}{}
  We define the \emph{\textbf{Fermi Energy}} to be the largest energy that any particle in the gas takes.
\end{mathdefinitionbox}

\vskip 1cm
[Write the rest from recording]

\end{document}
