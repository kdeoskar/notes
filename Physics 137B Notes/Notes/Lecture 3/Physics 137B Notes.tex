\documentclass{article}

% Language setting
% Replace `english' with e.g. `spanish' to change the document language
\usepackage[english]{babel}

% Set page size and margins
% Replace `letterpaper' with`a4paper' for UK/EU standard size
\usepackage[letterpaper,top=2cm,bottom=2cm,left=3cm,right=3cm,marginparwidth=1.75cm]{geometry}

% Useful packages
\usepackage{amsmath}
\usepackage{amssymb}
\usepackage{graphicx}
\usepackage{cancel}
\usepackage[colorlinks=true, allcolors=blue]{hyperref}

\usepackage{hyperref}
\hypersetup{
    colorlinks=true,
    linkcolor=blue,
    filecolor=magenta,      
    urlcolor=cyan,
    pdftitle={Overleaf Example},
    pdfpagemode=FullScreen,
    }

\urlstyle{same}

\usepackage{tikz-cd}

%%%%%%%%%%% Box pacakges and definitions %%%%%%%%%%%%%%
\usepackage[most]{tcolorbox}
\usepackage{xcolor}

% Define the colors
\definecolor{boxheader}{RGB}{0, 51, 102}  % Dark blue
\definecolor{boxfill}{RGB}{173, 216, 230}  % Light blue

% Define the tcolorbox environment
\newtcolorbox{mathdefinitionbox}[2][]{%
    colback=boxfill,   % Background color
    colframe=boxheader, % Border color
    fonttitle=\bfseries, % Bold title
    coltitle=white,     % Title text color
    title={#2},         % Title text
    enhanced,           % Enable advanced features
    attach boxed title to top left={yshift=-\tcboxedtitleheight/2}, % Center title
    boxrule=0.5mm,      % Border width
    sharp corners,      % Sharp corners for the box
    #1                  % Additional options
}
%%%%%%%%%%%%%%%%%%%%%%%%%

\usepackage{biblatex}
\addbibresource{sample.bib}


%%%%%%%%%%% New Commands %%%%%%%%%%%%%%
\newcommand*{\T}{\mathcal T}
\newcommand*{\cl}{\text cl}


\newcommand{\ket}[1]{|#1 \rangle}
\newcommand{\bra}[1]{\langle #1|}
\newcommand{\inner}[2]{\langle #1 | #2 \rangle}
\newcommand{\mean}[1]{\langle #1 \rangle}
\newcommand{\R}{\mathbb{R}}
\newcommand{\C}{\mathbb{C}}
\newcommand{\V}{\mathbb{V}}
\newcommand{\Hilbert}{\mathcal{H}}
\newcommand{\oper}{\hat{\Omega}}
\newcommand{\lam}{\hat{\Lambda}}

\newcommand{\bigslant}[2]{{\raisebox{.2em}{$#1$}\left/\raisebox{-.2em}{$#2$}\right.}}
\newcommand{\restr}[2]{{% we make the whole thing an ordinary symbol
  \left.\kern-\nulldelimiterspace % automatically resize the bar with \right
  #1 % the function
  \vphantom{\big|} % pretend it's a little taller at normal size
  \right|_{#2} % this is the delimiter
  }}
%%%%%%%%%%%%%%%%%%%%%%%%%%%%%%%%%%%%%%%


\newtcolorbox{dottedbox}[1][]{%
    colback=white,    % Background color
    colframe=white,    % Border color (to be overridden by dashrule)
    sharp corners,     % Sharp corners for the box
    boxrule=0pt,       % No actual border, as it will be drawn with dashrule
    boxsep=5pt,        % Padding inside the box
    enhanced,          % Enable advanced features
    overlay={\draw[dashed, thin, black, dash pattern=on \pgflinewidth off \pgflinewidth, line cap=rect] (frame.south west) rectangle (frame.north east);}, % Dotted line
    #1                 % Additional options
}


\tcbset{theostyle/.style={
    enhanced,
    sharp corners,
    attach boxed title to top left={
      xshift=-1mm,
      yshift=-4mm,
      yshifttext=-1mm
    },
    top=1.5ex,
    colback=white,
    colframe=blue!75!black,
    fonttitle=\bfseries,
    boxed title style={
      sharp corners,
    size=small,
    colback=blue!75!black,
    colframe=blue!75!black,
  } 
}}

\newtcbtheorem[number within=section]{Theorem}{Theorem}{%
  theostyle
}{thm}

\newtcbtheorem[number within=section]{Definition}{Definition}{%
  theostyle
}{def}



\title{Physics 137B Lecture 3}
\author{Keshav Balwant Deoskar}

\begin{document}
\maketitle

% \vskip 0.5cm 
These are notes taken from lectures on Quantum Mechanics delivered by Professor Raúl A. Briceño for UC Berekley's Physics 137B class in the Sprng 2024 semester.
% \pagebreak 

\tableofcontents

\pagebreak

\section{January 22 - More about symmetries}

\subsection{Last time}

\begin{itemize}
  \item Symmetries are transformations which leave a system invariant.
  \item Can be continuous or discrete.
  \item Active: $\ket{\psi} \rightarrow \ket{\psi'} = \hat{U} \ket{\psi}$ 
  Passive: $\hat{\Theta} \rightarrow \hat{\theta'} = \hat{U}^{\dagger} \hat{\Theta} \hat{U}$
  \item Unitary operators preserve norm: 
  \begin{align*}
    &\inner{\psi}{\psi} = \inner{\psi'}{\psi'} \\
    \implies& \hat{U}^{\dagger} \hat{U} = 1 = \hat{U} \hat{U}^{\dagger}
  \end{align*}

  \item If $\hat{U}$ transformations are a symmetry, $\mean{\hat{U}}$ is a conserved quantity since 
  \begin{align*}
    \frac{d}{dt}\mean{\hat{U}} &= \frac{i}{\hbar}\mean{\underbrace{\left[ \hat{U}, \hat{H} \right]}_{=0, \text{symmetry}}} + \cancelto{0}{\mean{\frac{\partial \hat{U}}{\partial t}}} \;\;\text{ Ehrenfest's Theorem} \\
  \end{align*}
  \[ \implies \boxed{ \frac{d}{dt}\mean{U} = 0 } \]
\end{itemize}

\vskip 1cm
\section{Parity $\hat{\Pi}$}
This operator "flips" the coordinate system i.e.
\[ \hat{\Pi} \underbrace{\ket{\vec{r}}}_{\ket{x} \otimes \ket{y} \otimes\ket{z} } = \ket{-\vec{r}} = \ket{-x} \otimes \ket{-y} \otimes\ket{-z}   \]

\vskip 0.5cm
Note that \[\hat{\Pi}^2 \ket{\vec{r}} = \hat{\Pi} \ket{-\vec{r}} = \ket{\vec{r}}\]

Thus, $\hat{\Pi}^2 = \mathbf{1} \implies \boxed{\hat{\Pi}^{-1} = \hat{\Pi}}$
This tells us that the \textbf{eigenvalues of $\hat{|pi}$ are $\lambda_{\pm} = \pm 1$}

\vskip 0.5cm
\textbf{Proof:} Suppose 
\[ \hat{\Pi} \ket{n} = \lambda_n \ket{n} \]
Then
\begin{align*}
  &\hat{\Pi}^2 \ket{n} = \lambda_n \hat{\Pi} \ket{n} - \lambda_n^2 \ket{n} \\
  \implies& \lambda_n^2 = 1 \\
  \implies& \lambda_n = \pm 1
\end{align*}

Parity states can only have positive or negatve parity.

\begin{dottedbox}

\textbf{Claim:} $\hat{\Pi}$ is Hermitian. i.e. $\hat{\Pi}^{\dagger} = \hat{\Pi}$.

\vskip 0.5cm 
\textbf{Proof:} Consider some arbitrary position states $\ket{f}, \ket{g}$ and
\begin{align*}
  \inner{f}{\hat{\Pi} | g} &= \int d\vec{r} \inner{f}{\hat{\Pi}|\vec{r}}\inner{\vec{r}}{g} \\
  &= \int d\vec{r} \inner{f}{|-\vec{r}} g(\vec{r}) \\
  &= \int_{\infty}^{\infty} d\vec{r} f^{*}(-\vec{r'}) g(\vec{r'}) \\
  &= -\int_{\infty}^{-\infty} d\vec{r'} f^{*}(\vec{r'})g(-\vec{r'})\;\;\text{(Change of variables $\vec{r'} = -\vec{r}$)} \\
  &= \int_{-\infty}^{\infty} d\vec{r'} f^{*}(\vec{r'})g(-\vec{r'})\;\;\text{(Change of variables $\vec{r'} = -\vec{r}$)} \\
  &= \int d\vec{r'} \inner{f}{\vec{r'}} \inner{-\vec{r'}}{g} \\
  &= \int d\vec{r'} \inner{f}{\vec{r'}} \inner{\vec{r'}|\hat{\Pi}^{\dagger}}{g} \\
  &= \inner{f|\hat{\Pi}^{\dagger}}{g}
\end{align*}
So, we have 

\[ \boxed{\hat{\Pi} = \hat{\Pi}^{\dagger}} \]
\end{dottedbox}


\vskip 1cm
Symmetries make our calculations very simple.
\begin{dottedbox}

\textbf{Example:} Let $\hat{\Theta}$ be odd under parity i.e.
\[ \hat{\Pi} \hat{\Theta} = - \hat{\Theta} \]

Then, $\inner{n}{\hat{\Theta}|n} = 0$.

\vskip 0.5cm
\textbf{Proof:}
\begin{align*}
  \inner{n|\hat{\Theta}}{n} &= \inner{n|\hat{\Pi}^{\dagger}\hat{\Theta} \hat{\Pi} }{n} \\
  &=  \left( \bra{n} \hat{\Pi} \right)\left( \hat{\Pi} \hat{\Theta} \hat{\Pi}\ \right) \left( \hat{\Pi}\ket{n} \right)\\
  &= -\lambda_n^2 \inner{n}{\hat{\Theta}|n} \\
  &= \inner{n}{\hat{\Theta}|n} \\
  &= 0
\end{align*}
\end{dottedbox}

\vskip 1cm
Vectors like Position and Momentum have odd parity, while pseudoscalars like the dot product and pseudovectors like angular momentum have positive parity.

\begin{itemize}
  \item $\hat{\Pi} \vec{r} \hat{\Pi} = -\vec{r}$
  \item $\hat{\Pi} \vec{p} \hat{\Pi} = -\vec{p}$
  \item $\hat{\Pi} \vec{r} \cdot \vec{r} \hat{\Pi} = |\vec{r}|^2$
  \item \begin{align*}
    \hat{\Pi} \vec{L} \hat{\Pi} &= \hat{\Pi} \vec{r} \times \vec{p} \hat{\Pi} \\
    &= (-1)^2 \vec{r} \times \vec{p} \\
    &=\vec{L}
  \end{align*}
  \item $\hat{\Pi} \vec{S} \hat{\Pi} = \vec{S}$
\end{itemize}

\vskip 1cm
\subsection{Continuous Transformations}
One Example of a continuous transformations is \textbf{translation in 1D} whose operator is denoted as $\hat{\T}$.

\vskip 0.5cm
[Insert Figure]

\vskip 0.5cm
The Translation operator is defined by the action
\[ \hat{T}(a) \psi(x) = \psi'(x) = \psi(x - a) \]

Taylor expanding the last expression, we have 
\begin{align*}
  \hat{T}(a)\psi(x) &= \psi(x - a) \\
  &= \psi(x) - a\frac{d}{dx}\psi(x) + \cdots
\end{align*}

We can relate this to the momentum operator since 
\[ \hat{P} = -i\hbar\frac{d}{dx} \iff \frac{d}{dx} = \frac{i}{\hbar}\hat{P} \]

Which gives us 
\begin{align*}
  \hat{T}(a) \psi(x) &\approx \left( \mathbf{1} - \frac{ia\hat{P}}{\hbar} + \cdots \right)\psi(x) \\
  \implies \hat{T}(a) &= \sum_{n = 0}^{\infty} \frac{1}{n!} \left( \frac{-ia\hat{P}}{\hbar}^2 \right) \\
  &= \exp\left( \frac{-ia\hat{P}}{\hbar} \right)
\end{align*}

One thing we can deduce from this is that the \textbf{Translation operator is Unitary}.

\begin{dottedbox}
  \begin{align*}
    (\hat{T}(a))^{\dagger}(\hat{T}(a)) &= \exp\left( \frac{+ia\hat{P}}{\hbar} \right)\exp\left( \frac{-ia\hat{P}}{\hbar} \right) \\
    &= \exp\left( \frac{ia\hat{P - P}}{\hbar} \right) \;\;\text{ (This is valid because $\hat{P}$ commutes with itself)}\\
    &= \mathbf{1}
  \end{align*}
\end{dottedbox}

\vskip 1cm
\subsection{Momentum conservation}
If $\hat{T}(a)$ is a symmetry of the system we have 
\[ [\underbrace{\hat{T}(a)}_{e^{\frac{-ia\hat{P}}{\hbar}}}, \hat{H}] = 0 \] 

\begin{align*}
  \implies& \left[ \exp\left( -\frac{ia \hat{P}}{\hbar} \right), \hat{H} \right] = 0 \\
  \implies& \left[ \hat{P}, \hat{H} \right] = 0
\end{align*}

But then, Ehrenfest's Theorem tells us that $\mean{\hat{P}} = 0$ as  
\begin{align*}
  \frac{d}{dt}\mean{\hat{P}} = \frac{i}{\hbar} \mean{\left[ \hat{P}, \hat{H} \right]}  0
\end{align*}

So, the value of momentum measured is conserved.

\textbf{Momentum conservation is the result of Translation Symmetry.}

\subsection{Time Translation ($\hat{U}(\Delta)$)}
\begin{align*}
  \hat{U}(\Delta)\psi(t) &= \psi(t - \Delta) \\
  &= \left( 1 - \Delta\frac{d}{dt} + \cdots \right) \psi(t) \\
  &=\left( 1 + \frac{i \Delta \cdot \hat{H}}{\hbar} \cdots \right)\psi(t) \\
  &= \exp\left( \frac{i \Delta \cdot \hat{H}}{\hbar} \right) \psi(t)
\end{align*}
where in the third equality, we used that 
\[ \hat{H} = i\hbar\frac{d}{dt} \implies \frac{d}{dt} = -i\frac{\hat{H}}{\hbar} \]

If Time Translation is a symmetry, then this is equivalent to saying that 
the energy of the system is conserved. \textbf{So, Energy Conservation is a result of Time Translation.}

\end{document}
