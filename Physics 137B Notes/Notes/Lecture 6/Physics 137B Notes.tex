\documentclass{article}

% Language setting
% Replace `english' with e.g. `spanish' to change the document language
\usepackage[english]{babel}

% Set page size and margins
% Replace `letterpaper' with`a4paper' for UK/EU standard size
\usepackage[letterpaper,top=2cm,bottom=2cm,left=3cm,right=3cm,marginparwidth=1.75cm]{geometry}

% Useful packages
\usepackage{amsmath}
\usepackage{amssymb}
\usepackage{graphicx}
\usepackage[colorlinks=true, allcolors=blue]{hyperref}

\usepackage{hyperref}
\hypersetup{
    colorlinks=true,
    linkcolor=blue,
    filecolor=magenta,      
    urlcolor=cyan,
    pdftitle={137B Briceño Lecture 6},
    pdfpagemode=FullScreen,
    }

\urlstyle{same}

\usepackage{tikz-cd}

%%%%%%%%%%% Box pacakges and definitions %%%%%%%%%%%%%%
\usepackage[most]{tcolorbox}
\usepackage{xcolor}

% Define the colors
\definecolor{boxheader}{RGB}{0, 51, 102}  % Dark blue
\definecolor{boxfill}{RGB}{173, 216, 230}  % Light blue

% Define the tcolorbox environment
\newtcolorbox{mathdefinitionbox}[2][]{%
    colback=boxfill,   % Background color
    colframe=boxheader, % Border color
    fonttitle=\bfseries, % Bold title
    coltitle=white,     % Title text color
    title={#2},         % Title text
    enhanced,           % Enable advanced features
    attach boxed title to top left={yshift=-\tcboxedtitleheight/2}, % Center title
    boxrule=0.5mm,      % Border width
    sharp corners,      % Sharp corners for the box
    #1                  % Additional options
}
%%%%%%%%%%%%%%%%%%%%%%%%%

% \newtcolorbox{dottedbox}[1][]{%
%     colback=white,    % Background color
%     colframe=white,    % Border color (to be overridden by dashrule)
%     sharp corners,     % Sharp corners for the box
%     boxrule=0pt,       % No actual border, as it will be drawn with dashrule
%     boxsep=5pt,        % Padding inside the box
%     enhanced,          % Enable advanced features
%     overlay={\draw[dashed, thin, black, dash pattern=on \pgflinewidth off \pgflinewidth, line cap=rect] (frame.south west) rectangle (frame.north east);}, % Dotted line
%     #1                 % Additional options
% }


\usepackage{biblatex}
\addbibresource{sample.bib}


%%%%%%%%%%% New Commands %%%%%%%%%%%%%%
\newcommand*{\T}{\mathcal T}
\newcommand*{\cl}{\text cl}


\newcommand{\ket}[1]{|#1 \rangle}
\newcommand{\bra}[1]{\langle #1|}
\newcommand{\inner}[2]{\langle #1 | #2 \rangle}
\newcommand{\mean}[1]{\langle #1 \rangle}
\newcommand{\R}{\mathbb{R}}
\newcommand{\C}{\mathbb{C}}
\newcommand{\V}{\mathbb{V}}
\newcommand{\Hilbert}{\mathcal{H}}
\newcommand{\oper}{\hat{\Omega}}
\newcommand{\lam}{\hat{\Lambda}}

\newcommand{\bigslant}[2]{{\raisebox{.2em}{$#1$}\left/\raisebox{-.2em}{$#2$}\right.}}
\newcommand{\restr}[2]{{% we make the whole thing an ordinary symbol
  \left.\kern-\nulldelimiterspace % automatically resize the bar with \right
  #1 % the function
  \vphantom{\big|} % pretend it's a little taller at normal size
  \right|_{#2} % this is the delimiter
  }}
%%%%%%%%%%%%%%%%%%%%%%%%%%%%%%%%%%%%%%%

\newtcolorbox{dottedbox}[1][]{%
    colback=white,    % Background color
    colframe=white,    % Border color (to be overridden by dashrule)
    sharp corners,     % Sharp corners for the box
    boxrule=0pt,       % No actual border, as it will be drawn with dashrule
    boxsep=5pt,        % Padding inside the box
    enhanced,          % Enable advanced features
    overlay={\draw[dashed, thin, black, dash pattern=on \pgflinewidth off \pgflinewidth, line cap=rect] (frame.south west) rectangle (frame.north east);}, % Dotted line
    #1                 % Additional options
}


\tcbset{theostyle/.style={
    enhanced,
    sharp corners,
    attach boxed title to top left={
      xshift=-1mm,
      yshift=-4mm,
      yshifttext=-1mm
    },
    top=1.5ex,
    colback=white,
    colframe=blue!75!black,
    fonttitle=\bfseries,
    boxed title style={
      sharp corners,
    size=small,
    colback=blue!75!black,
    colframe=blue!75!black,
  } 
}}

\newtcbtheorem[number within=section]{Theorem}{Theorem}{%
  theostyle
}{thm}

\newtcbtheorem[number within=section]{Definition}{Definition}{%
  theostyle
}{def}



\title{Physics 137B Lecture 6}
\author{Keshav Balwant Deoskar}

\begin{document}
\maketitle

% \vskip 0.5cm 
These are notes taken from lectures on Quantum Mechanics delivered by Professor Raúl A. Briceño for UC Berekley's Physics 137B class in the Sprng 2024 semester.
% \pagebreak 

\tableofcontents

\pagebreak

\section{January 29 - Second order and Degenerate Perturbation Theory}

\vskip 1cm
\subsection*{Recap}
\begin{itemize}
  \item We've been trying to develop \textbf{\emph{Perturbation Theory}} in order to solve problems with Hamiltonians of the form:
  \[ \hat{H} = \underbrace{\hat{H}_0}_{\text{known sol.}} + \lambda\hat{H}' \]

  \item We did so by assuming we could parametrize the Eigen-states and Eigen-energies as functions of some small parameter $\lambda$
  \begin{align*}
    \ket{n^{(\lambda)}} &= \sum_{j} \lambda^j \ket{n^{(j)}} \\
    E_n^{(\lambda)} &= \sum_{j} \lambda^j E_n^{(j)} \\
  \end{align*} 

  (If this cannot be done, then the system we are studying is non-perturbative).

  \item So far, we've found the \textbf{\emph{Leading order corrections}} for \emph{\textbf{non-degenerate systems}} to be
  \begin{align*}
    E_n^{(1)} &= \inner{n^{(0)}}{\hat{H}'|n^{(0)}} \\
    \ket{n^{(1)}} &= \sum_{k \neq n} \ket{k^{(0)}} \frac{\inner{k^{(0)}}{\hat{H}'|n^{(0)}}}{E_n^{(0)} - E_k^{(0)}}
  \end{align*}
\end{itemize}

\vskip 1cm
\subsection{Second Order Perturbation Theory}
In second order PT, we also grab the $\lambda^2$ coefficients:

\[ \mathcal{O}(\lambda^2) : \lambda^2 \left[ (\hat{H}_0 - E_n^{(0)}) \ket{n^{(2)}} + (\hat{H}' - E_n^{(1)}) \ket{n^{(1)}} - E_n^{(2)}\ket{n^{(0)}} \right] = 0 \]

\vskip 0.5cm
Once again, we act on the equation with $\ket{n^{(0)}}$:

\begin{align*}
  \implies& \underbrace{\inner{n^{(0)}}{(\hat{H}_0 - E_n^{(0)})|n^{(2)}}}_{=0} + \inner{n^{0}}{\hat{H}' \ket{n^{(1)}}} - E_n^{(1)}\underbrace{\inner{n^{(0)}}{n^{(1)}}}_{=0} = 0 \\
  \implies& \boxed{E_n^{(2)} = \inner{n^{0}}{\hat{H}' \ket{n^{(1)}}}}
\end{align*}
Using the expression we get for $\ket{n^{(1)}}$:
\begin{align*}
  \implies& E_n^{(2)} = \sum_{n \neq k} \frac{\inner{n^{(0)}}{\hat{H}'| k^{(0)}} \inner{k^{(0)}}{\hat{H}'|n^{(0)}} }{E_n^{(0)} - E_k^{(0)}}
\end{align*}

\vskip 0.5cm
[Write about the possible interpretation as the state benig propagated and then returning back above by watching recording]

\vskip 0.5cm
\underline{Interesting fact:} When we were finding the states using 1st order PT we wrote 
\[ \ket{n} = \ket{n^{(0)}} + \lambda \ket{n^{(1)}} \]
and said we wouldn't worry about normalization, but in fact they are normalized up to second order corrections:
\begin{align*}
  \implies \inner{n}{n} &= \inner{n^{(0)}}{n^{(0)}} + \lambda \left( \underbrace{\inner{n^{(0)}}{n^{(1)}}}_{0} + \underbrace{\inner{n^{(1)}}{n^{(0)}}}_{0} \right) + \mathcal{O}(\lambda^2) \\
  &= 1 + \mathcal{O}(\lambda^2)
\end{align*}

This came abbout because we had $\ket{n^{(1)}} = \sim_{k \neq n} (stuff)$ summing over states $k$ \emph{not equal to} $n$.

\vskip 0.5cm
\begin{dottedbox}
  Now, what about $\ket{n^{(2)}}$? We have two choices:

\vskip 0.5cm
\underline{Choice \#1}: Once again sum over states $k \neq n$
\[\ket{n^{(2)}} = \sum_{k \neq n} \ket{k^{(0)}}  c_{nk}^{(2)} \]


In this case, we yet again get normalized states

\vskip 0.5cm
\underline{Choice \#2}: Include $k = n$
\[\ket{n^{(2)}} = \sum_{k} \ket{k^{(0)}}  c_{nk}^{(2)} \]


In this case, we can determine $c_{nn}^{(2)}$ from the normalization condition
\[ \inner{n}{n} = 1 + \mathcal{O}(\lambda^3) \]
\end{dottedbox}


\vskip 0.5cm
For now, we will follow Choice \#1. So, let
\[\ket{n^{(2)}} = \sum_{k \neq n} \ket{^{(0)}}  c_{nk}^{(2)} \]

Then 
\begin{align*}
  \implies \left( \hat{H}_0 - E_n^{(0)} \right)\sum_{k \neq n} \ket{k^{(0)}}  c_{nk}^{(2)} + \left( \hat{H}' - E_n^{(0)} \right) \ket{n^{(1)}} 0 E_n^{(2)}\ket{n^{(0)}} = 0
\end{align*}
Let $l \neq n$. Then,

\begin{align*}
  &\sum_{k \neq n} \inner{l^{(0)}}{k^{(0)}}  c_{nk}^{(2)} \left( E_k^{(0)} - E_n^{(0)} \right) + \inner{l^{(0)}}{\left( \hat{H}' - E_n^{(0)} \right) | k^{(0)}} - E_n^{(0)}\underbrace{\inner{l^{(0)}}{n^{(0)}}}_{0} = 0 \\
  \implies &c_{nk}^{(2)} = \sum_{k \neq n} \frac{E_n^{(1)}}{E_n^{(0)} - E_l^{(0)}} + \sum_{k \neq n} \sum_{l \neq n} \frac{\inner{k^{(0)}}{\hat{H}'| l^{(0)}} \inner{l^{(0)}}{\hat{H}'| n^{(0)}} }{(E_n^{(0)} - E_k^{(0)})(E_n^{(2)} - E_l^{(0)})}
\end{align*}

[THE ABOVE ARE DUBIOUS; DOUBLE-CHECK THE CALCULATIONS LATER]

\vskip 1cm
\subsection{Degenerate Perturbation Theory}
So far, we've assumed that our systems have stationary states which are non-degerate. However, in many situations we have orthogonal states $\ket{a^{(0)}}$ and $\ket{b^{(0)}}$ with $\inner{a^{(0)}}{b^{(0)}} = 0$ but with

\begin{align*}
  \hat{H}^{(0)}\ket{a^{(0)}} &= E^{(0)} \ket{a^{(0)}} \\
  \hat{H}^{(0)}\ket{b^{(0)}} &= E^{(0)} \ket{b^{(0)}} 
\end{align*}

So, we'll have to tweak our approach. We will try to use the parameter $\lambda$ to split the energies.

\end{document}
