\documentclass{article}

% Language setting
% Replace `english' with e.g. `spanish' to change the document language
\usepackage[english]{babel}

% Set page size and margins
% Replace `letterpaper' with`a4paper' for UK/EU standard size
\usepackage[letterpaper,top=2cm,bottom=2cm,left=3cm,right=3cm,marginparwidth=1.75cm]{geometry}

% Useful packages
\usepackage{amsmath}
\usepackage{amssymb}
\usepackage{graphicx}
\usepackage[colorlinks=true, allcolors=blue]{hyperref}

\usepackage{hyperref}
\hypersetup{
    colorlinks=true,
    linkcolor=blue,
    filecolor=magenta,      
    urlcolor=cyan,
    pdftitle={Overleaf Example},
    pdfpagemode=FullScreen,
    }

\urlstyle{same}

\usepackage{tikz-cd}

%%%%%%%%%%% Box pacakges and definitions %%%%%%%%%%%%%%
\usepackage[most]{tcolorbox}
\usepackage{xcolor}

% Define the colors
\definecolor{boxheader}{RGB}{0, 51, 102}  % Dark blue
\definecolor{boxfill}{RGB}{173, 216, 230}  % Light blue

% Define the tcolorbox environment
\newtcolorbox{mathdefinitionbox}[2][]{%
    colback=boxfill,   % Background color
    colframe=boxheader, % Border color
    fonttitle=\bfseries, % Bold title
    coltitle=white,     % Title text color
    title={#2},         % Title text
    enhanced,           % Enable advanced features
    attach boxed title to top left={yshift=-\tcboxedtitleheight/2}, % Center title
    boxrule=0.5mm,      % Border width
    sharp corners,      % Sharp corners for the box
    #1                  % Additional options
}
%%%%%%%%%%%%%%%%%%%%%%%%%

% \newtcolorbox{dottedbox}[1][]{%
%     colback=white,    % Background color
%     colframe=white,    % Border color (to be overridden by dashrule)
%     sharp corners,     % Sharp corners for the box
%     boxrule=0pt,       % No actual border, as it will be drawn with dashrule
%     boxsep=5pt,        % Padding inside the box
%     enhanced,          % Enable advanced features
%     overlay={\draw[dashed, thin, black, dash pattern=on \pgflinewidth off \pgflinewidth, line cap=rect] (frame.south west) rectangle (frame.north east);}, % Dotted line
%     #1                 % Additional options
% }


\usepackage{biblatex}
\addbibresource{sample.bib}


%%%%%%%%%%% New Commands %%%%%%%%%%%%%%
\newcommand*{\T}{\mathcal T}
\newcommand*{\cl}{\text cl}


\newcommand{\ket}[1]{|#1 \rangle}
\newcommand{\bra}[1]{\langle #1|}
\newcommand{\inner}[2]{\langle #1 | #2 \rangle}
\newcommand{\mean}[1]{\langle #1 \rangle}
\newcommand{\R}{\mathbb{R}}
\newcommand{\C}{\mathbb{C}}
\newcommand{\V}{\mathbb{V}}
\newcommand{\Hilbert}{\mathcal{H}}
\newcommand{\oper}{\hat{\Omega}}
\newcommand{\lam}{\hat{\Lambda}}

\newcommand{\bigslant}[2]{{\raisebox{.2em}{$#1$}\left/\raisebox{-.2em}{$#2$}\right.}}
\newcommand{\restr}[2]{{% we make the whole thing an ordinary symbol
  \left.\kern-\nulldelimiterspace % automatically resize the bar with \right
  #1 % the function
  \vphantom{\big|} % pretend it's a little taller at normal size
  \right|_{#2} % this is the delimiter
  }}
%%%%%%%%%%%%%%%%%%%%%%%%%%%%%%%%%%%%%%%

\newtcolorbox{dottedbox}[1][]{%
    colback=white,    % Background color
    colframe=white,    % Border color (to be overridden by dashrule)
    sharp corners,     % Sharp corners for the box
    boxrule=0pt,       % No actual border, as it will be drawn with dashrule
    boxsep=5pt,        % Padding inside the box
    enhanced,          % Enable advanced features
    overlay={\draw[dashed, thin, black, dash pattern=on \pgflinewidth off \pgflinewidth, line cap=rect] (frame.south west) rectangle (frame.north east);}, % Dotted line
    #1                 % Additional options
}


\tcbset{theostyle/.style={
    enhanced,
    sharp corners,
    attach boxed title to top left={
      xshift=-1mm,
      yshift=-4mm,
      yshifttext=-1mm
    },
    top=1.5ex,
    colback=white,
    colframe=blue!75!black,
    fonttitle=\bfseries,
    boxed title style={
      sharp corners,
    size=small,
    colback=blue!75!black,
    colframe=blue!75!black,
  } 
}}

\newtcbtheorem[number within=section]{Theorem}{Theorem}{%
  theostyle
}{thm}

\newtcbtheorem[number within=section]{Definition}{Definition}{%
  theostyle
}{def}



\title{Physics 137B Lecture 4}
\author{Keshav Balwant Deoskar}

\begin{document}
\maketitle

% \vskip 0.5cm 
These are notes taken from lectures on Quantum Mechanics delivered by Professor Raúl A. Briceño for UC Berekley's Physics 137B class in the Sprng 2024 semester.
% \pagebreak 

\tableofcontents

\pagebreak

\section{January 24 - Introduction to Perturbation Theory}

\vskip 0.5cm
In 137A, we could solve $\hat{H} \ket{n} = E_n \ket{n}$ exactly and obtain the stationary states, then build the general solution as 
\[ \ket{\psi(t = 0)} = \sum_{n} c_n \ket{n} \]
and 
\[ \ket{\psi(t)} = \sum_{n} c_n e^{-iE_n t / \hbar} \ket{n} \]

This was possible because we were studying systems with relatively simple/convenient Hamiltonians, but \emph{most} situations that we want to study aren't so simple.

\vskip 1cm
\subsection*{Where can we apply P.T.?}

\begin{itemize}
  \item Perturbation Theory allows us to study Hamiltonians of the form 
  \[ \hat{H} = \hat{H}_0 + \lambda \hat{H}' \]
  
  where $\hat{H}_0$ is a Hamiltonian we can solve 
  \[ \hat{H}_0 \ket{n^{(0)}} = E_n^{(0)} \ket{n^{(0)}} \]
  
  and $\lambda \in (0, 1]$.

  \vskip 0.5cm
  \item The idea is to parametrize our solutions in terms of $\lambda$ and find the $\lambda \rightarrow 1$ solution.
  
  \begin{dottedbox}
    Why avoid $\lambda = 0$? 
    \vskip 0.5cm
    
    We can think of PT as doing 
    \begin{align*}
      \hat{H} &= \hat{H}_0 + \hat{H}_1 \\
              &= \hat{H}_0 + \lambda \left( \frac{\hat{H}_1}{\lambda} \right) \\
              &= \hat{H}_0 + \lambda \hat{H}'
    \end{align*}
    So, $\lambda = 0$ would be a problem.
  \end{dottedbox}
\end{itemize}

\vskip 1cm
\subsection{Time Independent Perturbation Theory}

We want to solve the Hamiltonian
\[ \hat{H}\ket{n} = \left(\hat{H}_0 + \lambda \hat{H}'\right) \ket{n} = E_n \ket{n} \;\;\;\; (1)\]

We assume that our solutions can be parametrized as functions of $\lambda$. Then, we taylor expand them as  
\[ \ket{n} = \sum_{j = 0}^{\infty} \lambda^{j} \ket{n^{(j)}} \]

\[ E_n = \sum_{j = 0}^{\infty} E_n^{(j)} \lambda^j  \]

Now, if we bring the RHS of equation (1) to the left, we have
\begin{align*}
  0 &= \left(\hat{H}_0 + \lambda \hat{H}' - E_n \right) \ket{n} \\
  &= \left(\hat{H}_0 + \lambda \hat{H}' - \sum_{j = 0}^{\infty} E_n^{(j)} \lambda^j \right) \left( \sum_{k = 0}^{\infty} \lambda^{k} \ket{n^{(k)}} \right) \\
  &= \left(\hat{H}_0 + \lambda \hat{H}' - (E_n^{(0)} + \lambda E_n^{(1)} + \lambda^2 E_n^{(2)} + \cdots )\right) \left( \ket{n}^{(0)} + \lambda \ket{n}^{(1)} + \lambda^2 \ket{n}^{(2)} + \cdots  \right) 
\end{align*}

Now, let's see what happens when we trunate the resulting sum at different powers of $\lambda$:

\begin{align*}
  &\mathcal{O}(\lambda^0) \;:\; \left( \hat{H}_0 - E_n^{0} \right) \ket{n^{0}} = 0\;\;\left[ \inner{k^{(0)}}{n^{(0)}} = \delta_{kn} \right] \\
  &\mathcal{O}(\lambda^1) \;:\; \lambda \left( \left( \hat{H}' - E_n^{(1)} \right) \ket{n^{(0)}}  +  \left( \hat{H}_0  - E_n^{(0)} \right)\ket{n^{1}} \right)
\end{align*}

In the $\mathcal{O}(1)$ seres, we don't know what $E_n^{(1)}$ is. To obtain this correction, we simply \textbf{act $\bra{n^{(0)}}$ on the equation:}

\begin{align*}
  0 &= \inner{n^{(0)}}{\left( \hat{H}' - E_n^{(1)} \right) | n^{0}} + \underbrace{\inner{n^{(0)}}{\left( \hat{H}_n^{(0)} - E_n^{(0)} \right) | n^{1} }}_{ = 0, \text{ since } E_n^{(0)} \bra{n^{(0)}} - E_n^{(0)} \bra{n^{(0)}}  = 0} \\
  &= \inner{n^{(0)}}{\hat{H}'|n^{(0)}} - E_n^{(1)} \underbrace{\inner{n^{(0)}}{n^{(0)}}}_{1}
\end{align*}
\[ \implies \boxed{E_n^{1} = \inner{n^{(0)}}{\hat{H}'|n^{(0)}}} \]

\vskip 0.5cm
This gives us the leading order Eigenenergy correction! But we still need to find actual states. So, next, we need to solve for $\ket{n^{(1)}}$.
\vskip 1cm
\begin{dottedbox}
  So far we've been studynig Non-degenerate Perturbation Theory. This only applies for Hamiltonians with no degeneracies i.e. Hamiltonians for which 
  \[ E_n^{(0)} = E_k^{(0)} \iff n = k \]
\end{dottedbox}

\vskip 1cm
We can solve for $\ket{n}^{(0)}$ in terms of the non-perturbative stationary states $\ket{k^{(0)}}$ as 
\begin{align*}
  \ket{n^{(1)}} &= \sum_{k} c_{nk}^{(1)} \ket{k^{(0)}} \\
  &= c_{nn}^{(1)} \ket{n^{(0)}} + \sum_{k \neq n} c_{nk}^{(1)} \ket{k^{(0)}}
\end{align*}

Note that $ \ket{n}$ is not yet normalized, so for now we can assume $\ket{n^{(\lambda)}}$ is some arbitrary linear combination of the $\{ \ket{n^{(0)}}, \ket{n^{(1)}}, \dots\}$ and worry about the norm \emph{later}.

\vskip 0.5cm
So,
\begin{align*}
  \ket{n^{(\lambda)}} &= \ket{n^{(0)}} + \lambda \ket{n^{(1)}} + \cdots \\
  &= \underbrace{\left( 1 + \lambda c_{nn}^{(0)} \right) }_{A} \left( \ket{n^{(0)}} + \underbrace{\left(\frac{\lambda}{A}\right)}_{\lambda^{1}} \sum_{k \neq n} c_{nk}^{(1)} \ket{k^{(0)}} + \cdots \right)
\end{align*}


Let 
\[ \boxed{\ket{n^{(1)}} = \sum_{k \neq n} c_{nk}^{(1)} \ket{k^{(0)}} }  \]

Our current goal, then, is to find $c_{nk}^{(1)}$.

Then, 
\begin{align*}
  0 &= \left( \hat{H}_0 - E_n^{(0)} \right) \ket{n^{(1)}} + \left( \hat{H}' + E_n^{(1)} \right) \ket{n^{(0)}} \\
  &= \left( \hat{H}_0 - E_n^{(0)} \right) \sum_{k \neq n} c_{nk}^{(1)} \ket{k}^{(0)} + \left( \hat{H}' - E_n^{(1)} \right) \ket{n^{(0)}} \\
  &= \sum_{k \neq n} c_{nk}^{(1)} \left( E_k^{(0)} - E_n^{(0)} \right) \ket{k^{(0)}} + \left( \hat{H}' - E_n^{(1)} \right) \ket{n^{(0)}}
\end{align*}

Now, to extract $\ket{n^{(0)}}$, we act using another stationary state $\ket{l^{(0)}}$ where $l \neq n$.
\[ \implies \sum_{k \neq n} c_{nk}^{(1)} \left( E_{k}^{(0)}  - E_n^{(0)} \right) \underbrace{\inner{l^{(0)}}{k^{(0)}}}_{\delta_{lk}} + \inner{l^{(0)}}{\hat{H}'| n^{(0)} } - E_n^{(1)} \underbrace{\inner{l^{(0)}}{n^{(0)}}}_{0} = 0 \]

[Lecture ended, so stopped abruptly. Pick up from here in lec 5 notes.]

\end{document}
