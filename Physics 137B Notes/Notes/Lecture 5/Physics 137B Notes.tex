\documentclass{article}

% Language setting
% Replace `english' with e.g. `spanish' to change the document language
\usepackage[english]{babel}

% Set page size and margins
% Replace `letterpaper' with`a4paper' for UK/EU standard size
\usepackage[letterpaper,top=2cm,bottom=2cm,left=3cm,right=3cm,marginparwidth=1.75cm]{geometry}

% Useful packages
\usepackage{amsmath}
\usepackage{amssymb}
\usepackage{graphicx}
\usepackage[colorlinks=true, allcolors=blue]{hyperref}

\usepackage{hyperref}
\hypersetup{
    colorlinks=true,
    linkcolor=blue,
    filecolor=magenta,      
    urlcolor=cyan,
    pdftitle={Overleaf Example},
    pdfpagemode=FullScreen,
    }

\urlstyle{same}

\usepackage{tikz-cd}

%%%%%%%%%%% Box pacakges and definitions %%%%%%%%%%%%%%
\usepackage[most]{tcolorbox}
\usepackage{xcolor}

% Define the colors
\definecolor{boxheader}{RGB}{0, 51, 102}  % Dark blue
\definecolor{boxfill}{RGB}{173, 216, 230}  % Light blue

% Define the tcolorbox environment
\newtcolorbox{mathdefinitionbox}[2][]{%
    colback=boxfill,   % Background color
    colframe=boxheader, % Border color
    fonttitle=\bfseries, % Bold title
    coltitle=white,     % Title text color
    title={#2},         % Title text
    enhanced,           % Enable advanced features
    attach boxed title to top left={yshift=-\tcboxedtitleheight/2}, % Center title
    boxrule=0.5mm,      % Border width
    sharp corners,      % Sharp corners for the box
    #1                  % Additional options
}
%%%%%%%%%%%%%%%%%%%%%%%%%

% \newtcolorbox{dottedbox}[1][]{%
%     colback=white,    % Background color
%     colframe=white,    % Border color (to be overridden by dashrule)
%     sharp corners,     % Sharp corners for the box
%     boxrule=0pt,       % No actual border, as it will be drawn with dashrule
%     boxsep=5pt,        % Padding inside the box
%     enhanced,          % Enable advanced features
%     overlay={\draw[dashed, thin, black, dash pattern=on \pgflinewidth off \pgflinewidth, line cap=rect] (frame.south west) rectangle (frame.north east);}, % Dotted line
%     #1                 % Additional options
% }


\usepackage{biblatex}
\addbibresource{sample.bib}


%%%%%%%%%%% New Commands %%%%%%%%%%%%%%
\newcommand*{\T}{\mathcal T}
\newcommand*{\cl}{\text cl}


\newcommand{\ket}[1]{|#1 \rangle}
\newcommand{\bra}[1]{\langle #1|}
\newcommand{\inner}[2]{\langle #1 | #2 \rangle}
\newcommand{\mean}[1]{\langle #1 \rangle}
\newcommand{\R}{\mathbb{R}}
\newcommand{\C}{\mathbb{C}}
\newcommand{\V}{\mathbb{V}}
\newcommand{\Hilbert}{\mathcal{H}}
\newcommand{\oper}{\hat{\Omega}}
\newcommand{\lam}{\hat{\Lambda}}

\newcommand{\bigslant}[2]{{\raisebox{.2em}{$#1$}\left/\raisebox{-.2em}{$#2$}\right.}}
\newcommand{\restr}[2]{{% we make the whole thing an ordinary symbol
  \left.\kern-\nulldelimiterspace % automatically resize the bar with \right
  #1 % the function
  \vphantom{\big|} % pretend it's a little taller at normal size
  \right|_{#2} % this is the delimiter
  }}
%%%%%%%%%%%%%%%%%%%%%%%%%%%%%%%%%%%%%%%

\newtcolorbox{dottedbox}[1][]{%
    colback=white,    % Background color
    colframe=white,    % Border color (to be overridden by dashrule)
    sharp corners,     % Sharp corners for the box
    boxrule=0pt,       % No actual border, as it will be drawn with dashrule
    boxsep=5pt,        % Padding inside the box
    enhanced,          % Enable advanced features
    overlay={\draw[dashed, thin, black, dash pattern=on \pgflinewidth off \pgflinewidth, line cap=rect] (frame.south west) rectangle (frame.north east);}, % Dotted line
    #1                 % Additional options
}


\tcbset{theostyle/.style={
    enhanced,
    sharp corners,
    attach boxed title to top left={
      xshift=-1mm,
      yshift=-4mm,
      yshifttext=-1mm
    },
    top=1.5ex,
    colback=white,
    colframe=blue!75!black,
    fonttitle=\bfseries,
    boxed title style={
      sharp corners,
    size=small,
    colback=blue!75!black,
    colframe=blue!75!black,
  } 
}}

\newtcbtheorem[number within=section]{Theorem}{Theorem}{%
  theostyle
}{thm}

\newtcbtheorem[number within=section]{Definition}{Definition}{%
  theostyle
}{def}



\title{Physics 137B Lecture 5}
\author{Keshav Balwant Deoskar}

\begin{document}
\maketitle

% \vskip 0.5cm 
These are notes taken from lectures on Quantum Mechanics delivered by Professor Raúl A. Briceño for UC Berekley's Physics 137B class in the Sprng 2024 semester.
% \pagebreak 

\tableofcontents

\pagebreak

\section{January 26 - Perturbation Theory continued...}
% Let 
% \[ \boxed{\ket{n^{(1)}} = \sum_{k \neq n} c_{nk}^{(1)} \ket{k^{(0)}} }  \]

% Then, 
% \begin{align*}
%   0 &= \left( \hat{H}_0 - E_n^{(0)} \right) \ket{n^{(1)}} + \left( \hat{H}' + E_n^{(1)} \right) \ket{n^{(0)}} \\
%   &= \left( \hat{H}_0 - E_n^{(0)} \right) \sum_{k \neq n} c_{nk}^{(1)} \ket{k}^{(0)} + \left( \hat{H}' - E_n^{(1)} \right) \ket{n^{(0)}} \\
%   &= \sum_{k \neq n} c_{nk}^{(1)} \left( E_k^{(0)} - E_n^{(0)} \right) \ket{k^{(0)}} + \left( \hat{H}' - E_n^{(1)} \right) \ket{n^{(0)}}
% \end{align*}

Last time, we left off while trying to extract $\ket{n^{(0)}}$.

\vskip 0.5cm 

To extract $\ket{n^{(0)}}$, we make use of the fact that $\{\ket{l^{(0)}}\}$ is orthonormal. Act using another stationary state $\ket{l^{(0)}}$ where $l \neq n$.

\[ \implies \sum_{k \neq n} c_{nk}^{(1)} \left( E_{k}^{(0)}  - E_n^{(0)} \right) \underbrace{\inner{l^{(0)}}{k^{(0)}}}_{\delta_{lk}} + \inner{l^{(0)}}{\hat{H}'| n^{(0)} } - E_n^{(1)} \underbrace{\inner{l^{(0)}}{n^{(0)}}}_{0} = 0 \]

\[ \implies \left( E_l^{(0)} - E_n^{(0)} \right) c_{nl}^{(1)} = - \inner{l^{(0)}}{\hat{H'} | n^{(0)}} \]

Again, for emphasis, we are currently dealing with a special case: \textbf{Non-degenerate Perturbation Theory} wherein $E_{n}^{(0)} = E_l^{(0)}$ iff $n=l$. So, rearraning the equation, we finally get 

\[ \boxed{ c_{nl}^{(1)} = \frac{\inner{l^{(0)}}{\hat{H'} | n^{(0)}}}{ \left( E_n^{(0)} - E_l^{(0)} \right) } } \]

This gives us the first order correction for the state i.e.

\[ \boxed{ \ket{n^{(1)}} = \sum_{ k \neq n }  \frac{\inner{l^{(0)}}{\hat{H'} | n^{(0)}}}{ \left( E_n^{(0)} - E_l^{(0)} \right) \ket{k^{(0)}} } } \]


\vskip 0.5cm
Before moving on to second-order PT and degenerate PT, let's see some nice examples.

\vskip 1cm
\subsection{Example: 1D Infinite Square Well with a $\delta-$potential}

\vskip 0.5cm
The known potential is 
\[ \hat{H}_0 = \frac{\hat{P}^2}{2m} + V_0(x); \;\;\;\; V_0(x) = \begin{cases}
  0 \;\; -\frac{L}{2} \leq x \leq \frac{L}{2}\\
  \infty \;\;\text{otherwise}
\end{cases} \]

and we have $\hat{H}' = \alpha\delta(x)$

\vskip 0.5cm
The goal is to find $E_{n}^{(1)}$ and $\ket{n^{(1)}}$. We follow the standard procedure:
\begin{enumerate}
  \item Find $\ket{n^{(0)}}$ and $E_n^{(0)}$
  \item Use $E_n^{(1)} = \inner{n^{(0)}}{\hat{H}'|n^{(0)}}$
  \item Use $\ket{n^{(1)}} = \sum_{ k \neq n }  \frac{\inner{k^{(0)}}{\hat{H'} | n^{(0)}}}{ \left( E_n^{(0)} - E_k^{(0)} \right)} \ket{k^{(0)}}$
  \item "think" : do your solutions make sense?
\end{enumerate}

\vskip 1cm
\underline{\textbf{Step 1:}}
To get the \emph{unperturbed} eigenstates and eigenenergies, we set up the TISE:
\[ -\frac{\hbar^2}{2m} \frac{d^2}{dx^2} \psi_n^{(0)}(x) = E_n^{(0)} \psi_n^{(0)}  \]
and impose the \emph{boundary conditions}.

\vskip 1cm
Doing so, we obtain 

\[ \psi_n^{(0)}(x) = 
\begin{cases}
  \frac{2}{L} \cos\left( \frac{n\pi x}{L} \right) \;\;\;n = 2n' + 1 \\
  \frac{2}{L} \sin\left( \frac{n\pi x}{L} \right) \;\;\;n = 2n'
\end{cases} \]

and 
\[ E_n^{(0)} = \frac{\hbar^2}{2m}\left( \frac{\pi}{L} \right)^2 n^2 \]


\vskip 1cm
\underline{\textbf{Step 2:}}

\vskip 0.5cm
The energy correction is 
\begin{align*}
  E_n^{(1)} &= \inner{n^{(1)}}{\hat{H}'| n^{(0)}} \\
  &= \int dx \; \inner{n^{(0)}}{x} \inner{x}{\hat{H}'|x'}\inner{x'}{n^{(0)}} \\
  &= \int dx \psi(x)^{*} \alpha \delta(x) \psi(x) \\
  &= \alpha |\psi_n(0)|^2 \\
  &= \begin{cases}
    \frac{\alpha 2}{L} \;\;n = \text{ odd} \\
    0 \;\;n = \text{ even}
  \end{cases}
\end{align*}

So, the overall eigenenergy is given by 
\[ E_n^{(0)} = \left( \frac{\pi \hbar}{L} \right)^2 \frac{n^2}{2m} + \begin{cases}
  \alpha \frac{L}{2} \;\;\text{iff }n = \text{ odd} \\
  0 \;\;\text{iff }n = \text{ even} \\
\end{cases} \]

\vskip 0.5cm
Note that this makes sense if and only if 
\begin{align*}
  &\alpha \frac{L}{2} < \left( \frac{\pi \hbar}{L} \right)^2 \frac{n^2}{2m} \\
  \implies& \alpha < \frac{2}{L} \left( \frac{\pi h}{L} \right)^2 \frac{n^2}{2m}
\end{align*}

So, if experimental values of $\alpha$ tell us that the 1st order correction is bigger than the 0-order corrrection, we most definitely have a problem and our system may be \emph{non-perturbative}.

\vskip 1cm
\underline{\textbf{Step 3:}}

\vskip 0.5cm
The eigenstate correction is 

\begin{align*}
  \ket{n^{(1)}} &= \sum_{ k \neq n }  \frac{\inner{k^{(0)}}{\hat{H'} | n^{(0)}}}{ \left( E_n^{(0)} - E_k^{(0)} \right)} \ket{k^{(0)}}  \\
\end{align*}

\end{document}
