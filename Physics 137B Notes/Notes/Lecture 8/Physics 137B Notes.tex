\documentclass{article}

% Language setting
% Replace `english' with e.g. `spanish' to change the document language
\usepackage[english]{babel}

% Set page size and margins
% Replace `letterpaper' with`a4paper' for UK/EU standard size
\usepackage[letterpaper,top=2cm,bottom=2cm,left=3cm,right=3cm,marginparwidth=1.75cm]{geometry}

% Useful packages
\usepackage{amsmath}
\usepackage{amssymb}
\usepackage{graphicx}
\usepackage[colorlinks=true, allcolors=blue]{hyperref}

\usepackage{hyperref}
\hypersetup{
    colorlinks=true,
    linkcolor=blue,
    filecolor=magenta,      
    urlcolor=cyan,
    pdftitle={137B Briceño Lecture 8},
    pdfpagemode=FullScreen,
    }

\urlstyle{same}

\usepackage{tikz-cd}

%%%%%%%%%%% Box pacakges and definitions %%%%%%%%%%%%%%
\usepackage[most]{tcolorbox}
\usepackage{xcolor}

% Define the colors
\definecolor{boxheader}{RGB}{0, 51, 102}  % Dark blue
\definecolor{boxfill}{RGB}{173, 216, 230}  % Light blue

% Define the tcolorbox environment
\newtcolorbox{mathdefinitionbox}[2][]{%
    colback=boxfill,   % Background color
    colframe=boxheader, % Border color
    fonttitle=\bfseries, % Bold title
    coltitle=white,     % Title text color
    title={#2},         % Title text
    enhanced,           % Enable advanced features
    attach boxed title to top left={yshift=-\tcboxedtitleheight/2}, % Center title
    boxrule=0.5mm,      % Border width
    sharp corners,      % Sharp corners for the box
    #1                  % Additional options
}
%%%%%%%%%%%%%%%%%%%%%%%%%

% \newtcolorbox{dottedbox}[1][]{%
%     colback=white,    % Background color
%     colframe=white,    % Border color (to be overridden by dashrule)
%     sharp corners,     % Sharp corners for the box
%     boxrule=0pt,       % No actual border, as it will be drawn with dashrule
%     boxsep=5pt,        % Padding inside the box
%     enhanced,          % Enable advanced features
%     overlay={\draw[dashed, thin, black, dash pattern=on \pgflinewidth off \pgflinewidth, line cap=rect] (frame.south west) rectangle (frame.north east);}, % Dotted line
%     #1                 % Additional options
% }


\usepackage{biblatex}
\addbibresource{sample.bib}


%%%%%%%%%%% New Commands %%%%%%%%%%%%%%
\newcommand*{\T}{\mathcal T}
\newcommand*{\cl}{\text cl}


\newcommand{\ket}[1]{|#1 \rangle}
\newcommand{\bra}[1]{\langle #1|}
\newcommand{\inner}[2]{\langle #1 | #2 \rangle}
\newcommand{\mean}[1]{\langle #1 \rangle}
\newcommand{\R}{\mathbb{R}}
\newcommand{\C}{\mathbb{C}}
\newcommand{\V}{\mathbb{V}}
\newcommand{\Hilbert}{\mathcal{H}}
\newcommand{\oper}{\hat{\Omega}}
\newcommand{\lam}{\hat{\Lambda}}

\newcommand{\bigslant}[2]{{\raisebox{.2em}{$#1$}\left/\raisebox{-.2em}{$#2$}\right.}}
\newcommand{\restr}[2]{{% we make the whole thing an ordinary symbol
  \left.\kern-\nulldelimiterspace % automatically resize the bar with \right
  #1 % the function
  \vphantom{\big|} % pretend it's a little taller at normal size
  \right|_{#2} % this is the delimiter
  }}
%%%%%%%%%%%%%%%%%%%%%%%%%%%%%%%%%%%%%%%

\newtcolorbox{dottedbox}[1][]{%
    colback=white,    % Background color
    colframe=white,    % Border color (to be overridden by dashrule)
    sharp corners,     % Sharp corners for the box
    boxrule=0pt,       % No actual border, as it will be drawn with dashrule
    boxsep=5pt,        % Padding inside the box
    enhanced,          % Enable advanced features
    overlay={\draw[dashed, thin, black, dash pattern=on \pgflinewidth off \pgflinewidth, line cap=rect] (frame.south west) rectangle (frame.north east);}, % Dotted line
    #1                 % Additional options
}


\tcbset{theostyle/.style={
    enhanced,
    sharp corners,
    attach boxed title to top left={
      xshift=-1mm,
      yshift=-4mm,
      yshifttext=-1mm
    },
    top=1.5ex,
    colback=white,
    colframe=blue!75!black,
    fonttitle=\bfseries,
    boxed title style={
      sharp corners,
    size=small,
    colback=blue!75!black,
    colframe=blue!75!black,
  } 
}}

\newtcbtheorem[number within=section]{Theorem}{Theorem}{%
  theostyle
}{thm}

\newtcbtheorem[number within=section]{Definition}{Definition}{%
  theostyle
}{def}



\title{Physics 137B Lecture 8}
\author{Keshav Balwant Deoskar}

\begin{document}
\maketitle

% \vskip 0.5cm 
These are notes taken from lectures on Quantum Mechanics delivered by Professor Raúl A. Briceño for UC Berekley's Physics 137B class in the Sprng 2024 semester.
% \pagebreak 

\tableofcontents

\pagebreak

\section{February 2 - Degenerate Perturbation Theory Continued}

\vskip 1cm
\subsection*{Recap}
\begin{itemize}
  \item Last time we considered a system with two-fold degeneracy i.e. two orthonormal states $\psi_a, \psi_b$ with the same energy $E$. The energies being the same causes our formula for the first-order correction obtained in non-degenerate PT to diverge.
  \item However, we found that if we consider linear combinations of the degenerate states $\ket{\Psi}_{\pm} = \alpha \ket{\psi_a} + \beta \ket{\psi_b}$ and solve the eigenvalue problem 
  \[ \mathbf{W} \cdot \mathbf{\vec{V}} = E^{(0)} \mathbf{\vec{V}} \] 
  where 
  \[ \mathbf{W} = \begin{pmatrix}
    H_{aa}' & H_{ab}' \\
    H_{ba}' & H_{bb}' \\
  \end{pmatrix} \]
  then we obtain $\alpha, \beta$ such that $\ket{\Psi}_{\pm}$ diagonalize the degenerate subspace and allow us to lift the degeneracy.

  \vskip 0.5cm
  \item We saw an example in the 2D Harmonic Oscillator, where we wrote the unperturbed hamiltonian in terms of raising and lowering operators as 
  \[ \hat{H}_0 = \hbar \omega \left( \hat{a}_{+} \hat{a}_{-} + \hat{b}_{+} \hat{b}_{-} + 1 \right) \]
  and then introduced the perturbation 
  \[ \hat{H}' = \lambda m \omega^2 \hat{x} \hat{y}\]
  
  \item We were able to solve this problem exactly by changing to \emph{normal coordinates} and found the energy to be 
  \[ E_{nn'} = \left( n + \frac{1}{2} \right) \hbar \omega \left( 1 + \lambda \right)^{1/2} +  \left( n' + \frac{1}{2} \right) \hbar \omega \left( 1 - \lambda \right)^{1/2}   \]
\end{itemize}

\vskip 0.5cm
Today, we will solve for the first order correction using Perturbation Theory and check that it agrees with the exact solution.

\vskip 1cm
\subsection{Perturbation Theory approach to 2D Harmonic Oscillator}

\vskip 0.5cm
\begin{itemize}
  \item The perturbation $\hat{H}' = \lambda m\omega^2 \hat{x}\hat{y}$ can be written in terms of the raising and lowering operators as

  \begin{align*}
    \hat{H}' = \frac{\lambda \hbar \omega}{2} \left( \hat{a}_{+} + \hat{a}_{-} \right) \left( \hat{b}_{+} + \hat{b}_{-} \right)
  \end{align*}
  
  
  \item The unperturbed solutions can be written as 
  \begin{align*}
    \ket{n, n'}^{(0)} &= \ket{n^{(0)}}_a \otimes \ket{n'}^{(0)}_b \\
    &= \frac{(\hat{a}_{+})^n}{\sqrt{n!}} \frac{(\hat{b}_{+})^{n'}}{\sqrt{(n')!}} \ket{00}^{(0)}
  \end{align*}
  where $\ket{00}^{(0)}$ is the unperturbed ground state.

  \item These are all the tools we need to approach the problem. Let's now calculate the first order energy correction to the first excited state.  
\end{itemize}

\vskip 0.5cm
Let's get to evaluating the $\mathbf{W}$ matrix.
\begin{align*}
  \mathbf{W} &= \begin{pmatrix}
    \inner{10}{\hat{H}'|10} & \inner{10}{\hat{H}'|01} \\
    \inner{01}{\hat{H}'|10} & \inner{01}{\hat{H}'|01} 
  \end{pmatrix}
\end{align*}

\vskip 0.25cm
The algebra here seems tedious but we can notice that the perturbation contains raising and lowering operators, which means the diagonal elements are all going to be zero.

\vskip 0.25cm
Further,
\begin{align*}
  \hat{H}' \ket{10} &= \frac{\hbar \omega}{2} \left( \hat{a}_{+} + \hat{a}_{-}  \right) \left( \hat{b}_{+} + \hat{b}_{-}  \right) \ket{10} \\
  &= \frac{\hbar \omega}{2} \left( \hat{b}_{+} + \hat{b}_{-}  \right) \left( \sqrt{2}\ket{2 0} + \ket{00} \right) \\
  &= \frac{\hbar \omega}{2} \left(\sqrt{2} \ket{2 1} + \ket{0 1}\right) 
\end{align*}

So, 
\begin{align*}
  \inner{10}{\hat{H}'|10} &= 0 \\
  \inner{01}{\hat{H}'|10} &= \frac{\hbar \omega}{2} 
\end{align*}

Calculating the other column values, we find 
\[ \mathbf{W} = \frac{\hbar \omega}{2} \begin{pmatrix}
  0 & 1 \\
  1 & 0
\end{pmatrix}  \]

Now that we've found $\mathbf{W}$, we can find the first-order energy corrections:

\begin{align*}
  E_0^{(1)} &= \frac{1}{2} \left[ H_{aa}' + H_{bb}' + \sqrt{(H_{aa}' - H_{bb}') + 4|E_{ab}'|^2}  \right] \\
  &= \pm \frac{1}{2} \sqrt{4 \cdot \left( \frac{\hbar \omega}{2} \right)^2} \\
  &= \pm \frac{\hbar \omega}{2} \lambda
\end{align*}


So the energy corrections to the first excited state are 
\[ \boxed{E_{\pm}^{(1)} = \pm \frac{\hbar \omega}{2} \lambda} \]

\vskip 0.5cm
Let's now compare this result to the exact solution, which is given by 

\[ E_{nn'} = \left( n + \frac{1}{2} \right) \hbar \omega \left( 1 + \lambda \right)^{1/2} +  \left( n' + \frac{1}{2} \right) \hbar \omega \left( 1 - \lambda \right)^{1/2}   \]

\vskip 0.5cm
So, the exact $\ket{01}$ energy is 
\begin{align*}
  E_{0,1} &= \left( 0 + \frac{1}{2} \right) \hbar \omega \underbrace{\left( 1 + \lambda \right)^{1/2}}_{=\left( 1 + \frac{\lambda}{2} + \cdots \right)} + \left( 1 + \frac{1}{2} \right) \hbar \omega \underbrace{\left( 1 - \lambda \right)^{1/2}}_{=\left( 1 - \frac{\lambda}{2} + \cdots \right)} \\
  &= \left( 0 + \frac{1}{2} + 1 + \frac{1}{2} \right)\hbar \omega + \lambda \left( \frac{1}{4} - \frac{1}{2} - \frac{1}{4} \right) \hbar \omega + \mathcal{O}(\lambda^2) \\
  &\approx \underbrace{2\hbar \omega}_{=E_{01}^{(0)}} - \underbrace{\frac{\hbar \omega}{2} \lambda }_{E_{01,(-)}^{(1)}}
\end{align*}

and this matches up with the result we using the $\mathbf{W}$ matrix. Similiarly, we can check that the results match up for the $\ket{10}$ energy correction, or more generally the $\ket{n,n'}$ correction.

\vskip 0.25cm
For the $\ket{n,n'}$ case, we have 
\begin{align*}
  E_{n,n'} &= \left( n + \frac{1}{2} \right) \hbar \omega \underbrace{\left( 1 + \lambda \right)^{1/2}}_{=\left( n' + \frac{\lambda}{2} + \cdots \right)} + \left( 1 + \frac{1}{2} \right) \hbar \omega \underbrace{\left( 1 - \lambda \right)^{1/2}}_{=\left( 1 - \frac{\lambda}{2} + \cdots \right)} \\
  &= \left( n + \frac{1}{2} + n' + \frac{1}{2} \right)\hbar \omega + \lambda \left( \frac{n}{2} \frac{1}{4} - \frac{n'}{2} - \frac{1}{4} \right) \hbar \omega + \mathcal{O}(\lambda^2) \\
  &\approx \underbrace{(n + n' + 1)\hbar \omega}_{=E_{n,n'}^{(0)}} - \underbrace{(n - n')\frac{\hbar \omega}{2} \lambda }_{E_{n,n'}^{(1)}}
\end{align*}

\begin{dottedbox}
  \emph{Note:} If the states $\ket{a}, \ket{b}$ are degenerate \emph{but still} give us a $\mathbf{W}$ matrix whose off diagonals are trivial, 
  \[ \mathbf{W} = \begin{pmatrix}
    H_{aa}' & 0 \\
    0 & H_{bb}'\\
  \end{pmatrix} \]

  then the eigenvalue problem is solved by default! The $\ket{a}-$state energy correction and the $\ket{b}-$state energy correction are simply $H_{aa}'$ and $H_{bb}'$
  \begin{align*}
    E_{a}^{(1)} &= H_{aa}' \\
    E_{b}^{(1)} &= H_{bb}' 
  \end{align*}

  In other words, the basis we started off with was already the "good" basis! We didn't need to solve the Eigenvalue problem to find a different degenerate-eigenspace-basis to diagonalize it.
\end{dottedbox}

\vskip 1cm
\subsection{Good States}

\begin{dottedbox}
  \emph{\textbf{Theorem:}} If $\ket{a}, \ket{b}$ are degenerate with respect to $\hat{H}$ and there exists a hermitian operator $\hat{A}$ such that 
  \begin{enumerate}
    \item The states $\ket{a}, \ket{b}$ are eigenstates of $\hat{A}$ with distinct eigenvalues ($a \neq b$)
    \begin{align*}
      \hat{A} \ket{a} = a \ket{a} \\
      \hat{A} \ket{b} = b \ket{b} 
    \end{align*}
    and $\hat{A}$ commutes with the unperturbed hamiltonian as well as the perturbation,
    \item 
    \[ [\hat{A}, \hat{H}'] = [\hat{A}, \hat{H}_{0}] = [\hat{A}, \hat{H}] = 0 \]
  \end{enumerate}
  then the eigenvectors of $\hat{A}$ form a "good basis" to use in the perturbation theory.
\end{dottedbox}

\vskip 0.5cm
\emph{\textbf{Proof:}} Let $\hat{H}(\lambda) = \hat{H}_0 + \lambda \hat{H}'$ and $\hat{A}$ commute.

[Finish later]

\end{document}
