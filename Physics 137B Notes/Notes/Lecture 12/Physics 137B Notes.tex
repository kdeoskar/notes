\documentclass{article}

% Language setting
% Replace `english' with e.g. `spanish' to change the document language
\usepackage[english]{babel}

% Set page size and margins
% Replace `letterpaper' with`a4paper' for UK/EU standard size
\usepackage[letterpaper,top=2cm,bottom=2cm,left=3cm,right=3cm,marginparwidth=1.75cm]{geometry}

% Useful packages
\usepackage{amsmath}
\usepackage{amssymb}
\usepackage{graphicx}
\usepackage[colorlinks=true, allcolors=blue]{hyperref}

\usepackage{hyperref}
\hypersetup{
    colorlinks=true,
    linkcolor=blue,
    filecolor=magenta,      
    urlcolor=cyan,
    pdftitle={137B Briceño Lecture 12},
    pdfpagemode=FullScreen,
    }

\urlstyle{same}

\usepackage{tikz-cd}

%%%%%%%%%%% Box pacakges and definitions %%%%%%%%%%%%%%
\usepackage[most]{tcolorbox}
\usepackage{xcolor}

% Define the colors
\definecolor{boxheader}{RGB}{0, 51, 102}  % Dark blue
\definecolor{boxfill}{RGB}{173, 216, 230}  % Light blue

% Define the tcolorbox environment
\newtcolorbox{mathdefinitionbox}[2][]{%
    colback=boxfill,   % Background color
    colframe=boxheader, % Border color
    fonttitle=\bfseries, % Bold title
    coltitle=white,     % Title text color
    title={#2},         % Title text
    enhanced,           % Enable advanced features
    attach boxed title to top left={yshift=-\tcboxedtitleheight/2}, % Center title
    boxrule=0.5mm,      % Border width
    sharp corners,      % Sharp corners for the box
    #1                  % Additional options
}
%%%%%%%%%%%%%%%%%%%%%%%%%

% \newtcolorbox{dottedbox}[1][]{%
%     colback=white,    % Background color
%     colframe=white,    % Border color (to be overridden by dashrule)
%     sharp corners,     % Sharp corners for the box
%     boxrule=0pt,       % No actual border, as it will be drawn with dashrule
%     boxsep=5pt,        % Padding inside the box
%     enhanced,          % Enable advanced features
%     overlay={\draw[dashed, thin, black, dash pattern=on \pgflinewidth off \pgflinewidth, line cap=rect] (frame.south west) rectangle (frame.north east);}, % Dotted line
%     #1                 % Additional options
% }


\usepackage{biblatex}
\addbibresource{sample.bib}


%%%%%%%%%%% New Commands %%%%%%%%%%%%%%
\newcommand*{\T}{\mathcal T}
\newcommand*{\cl}{\text cl}


\newcommand{\ket}[1]{|#1 \rangle}
\newcommand{\bra}[1]{\langle #1|}
\newcommand{\inner}[2]{\langle #1 | #2 \rangle}
\newcommand{\mean}[1]{\langle #1 \rangle}
\newcommand{\R}{\mathbb{R}}
\newcommand{\C}{\mathbb{C}}
\newcommand{\V}{\mathbb{V}}
\newcommand{\Hilbert}{\mathcal{H}}
\newcommand{\oper}{\hat{\Omega}}
\newcommand{\lam}{\hat{\Lambda}}

\newcommand{\bigslant}[2]{{\raisebox{.2em}{$#1$}\left/\raisebox{-.2em}{$#2$}\right.}}
\newcommand{\restr}[2]{{% we make the whole thing an ordinary symbol
  \left.\kern-\nulldelimiterspace % automatically resize the bar with \right
  #1 % the function
  \vphantom{\big|} % pretend it's a little taller at normal size
  \right|_{#2} % this is the delimiter
  }}
%%%%%%%%%%%%%%%%%%%%%%%%%%%%%%%%%%%%%%%

\newtcolorbox{dottedbox}[1][]{%
    colback=white,    % Background color
    colframe=white,    % Border color (to be overridden by dashrule)
    sharp corners,     % Sharp corners for the box
    boxrule=0pt,       % No actual border, as it will be drawn with dashrule
    boxsep=5pt,        % Padding inside the box
    enhanced,          % Enable advanced features
    overlay={\draw[dashed, thin, black, dash pattern=on \pgflinewidth off \pgflinewidth, line cap=rect] (frame.south west) rectangle (frame.north east);}, % Dotted line
    #1                 % Additional options
}


\tcbset{theostyle/.style={
    enhanced,
    sharp corners,
    attach boxed title to top left={
      xshift=-1mm,
      yshift=-4mm,
      yshifttext=-1mm
    },
    top=1.5ex,
    colback=white,
    colframe=blue!75!black,
    fonttitle=\bfseries,
    boxed title style={
      sharp corners,
    size=small,
    colback=blue!75!black,
    colframe=blue!75!black,
  } 
}}

\newtcbtheorem[number within=section]{Theorem}{Theorem}{%
  theostyle
}{thm}

\newtcbtheorem[number within=section]{Definition}{Definition}{%
  theostyle
}{def}



\title{Physics 137B Lecture 12}
\author{Keshav Balwant Deoskar}

\begin{document}
\maketitle

% \vskip 0.5cm 
These are notes taken from lectures on Quantum Mechanics delivered by Professor Raúl A. Briceño for UC Berekley's Physics 137B class in the Sprng 2024 semester.
% \pagebreak 

\tableofcontents

\pagebreak

\section{February 12 - Spin-Orbit Coupling continued}

\vskip 1cm

\subsection*{Recap}
\begin{itemize}
  \item Earlier, we discussed the three corrections to the Hydrogen-atom Hamiltonian, the second one being the \emph{spin-orbit coupling correction}
  \[ \hat{H}_{SO} = \frac{e^2}{8m^2 \epsilon_0 c^2} \frac{\vec{S} \cdot \vec{L}}{r^3} \]
  
  \item We wrote $2 \vec{S} \cdot \vec{L} = \left( J^2 - S^2 - L^2 \right)$ and found the first-order energy correction to be 
  \[ E_{SO}^{(1)} = \frac{e^2}{8m^2 \epsilon_0 c^2} \mean{\frac{1}{r^3}}_{nl} \cdot \left[ j(j+1) - l(l+1) -s(s+1) \right] \]
  and we can derive 
  \[ \mean{\frac{1}{r^3}}_{nl} = \frac{1}{l(l + \frac{1}{2})(l - 1)n^3 a^3} \]

  \item Note that if $l = 0$ then we have $j = s$, the energy correction is just zero. So, Spin-Orbit coupling doesn't impact the energies of the zero-angular momentum states.
\end{itemize}

\vskip 1cm
\subsection{Darwin Term}

\vskip 0.5cm
Today, we'll find the correction due to the Darwin Term
\[ \hat{H}_{DAR}' = \frac{\pi \hbar^2}{2m^2 c^2} \cdot \frac{e^2}{4\pi \epsilon_0} \delta^3(\vec{r}) \]
which \emph{only} contributes to the $l = 0$ states.

Why is this? One way to think about it is to recall that when we solved the Schrodinger Equation for the Coulomb potential, we had 
\[ V_{eff}(r) = V_{coulomb}(r) + \underbrace{\frac{\hbar}{2m} \frac{l(l+1)}{r^2}}_{centrifugal} \]

[Include picture]

\vskip 0.5cm
At lower $r$, the centrifugal potential dominates and is a large positive number whereas for large $r$ the coulomb potential dominates and is a negative number close to zero (negative since it's attractive). 

\vskip 0.5cm
As a result, we required our ormalizable states to go to satisfy $\psi_{nlm}(0) \rightarrow 0$ for all $l > 0$.

[Explain more clearly; watch recording]

\vskip 0.5cm
Now, let's calculate the Darwin Correction. 

\begin{align*}
  E_{DAR}^{(1)} &= \frac{\pi \hbar^2}{2m^3 c^2} \frac{e^2}{4\pi \epsilon_0} \inner{n00}{\delta^3(\vec{r})|n00} \\
  &=  \frac{\pi \hbar^2}{2m^3 c^2} \frac{e^2}{4\pi\epsilon_0} \frac{1}{\pi n^3 a^3}
\end{align*}

Note that 
\begin{align*}
  E_n^{(0)} &= -\frac{m}{2\hbar^2} \left( \frac{e^2}{4\pi \epsilon_0} \right)^2 \frac{1}{n} \\
  &= -\frac{mc^2}{2} \underbrace{\left( \frac{e^2}{4\pi \epsilon_0} \frac{1}{c\hbar} \right)^2}_{\alpha^2} \frac{1}{n^2}
\end{align*}
and since $\alpha \approx \frac{1}{137} << 1$ we have 

\[ E_n^{(0)} \approx \frac{E_1^{(0)}}{n^2} \]

\subsection{Fine Structure corrections}

\vskip 0.5cm
Now that we've calculate each of the corrections individually, the total Fine Structure Corrections are given by 
\begin{align*}
  E_n^{(1)} &= E_{rel}^{(1)} + E_{SO}^{(1)} + E_{DAR}^{(1)} \\
  &= E_n^{(0)} \frac{\alpha^2}{n^2} \left( \frac{n}{j + 1/2} - \frac{3}{4} \right)
\end{align*} 

Thus, 
\begin{align*}
  E^{(0)} &\sim \mathcal{O}(\alpha^2) \\
  E^{(1)} &\sim \mathcal{O}(\alpha^4) 
\end{align*}

\subsection{Splitting of Energy Levels}

[Include Image]

\subsection{External Magnetic Fields - Zeeman Effect}
Everything we've done earlier has been \emph{internal}. Now, we're going to submerge the entire atom i.e. \emph{both} the electron and proton in an external magnetic field $\vec{B}_{ext}$.

\vskip 0.5cm
Then, we have a perturbation to the Hamiltonian given by 
\[ \hat{H}_z  =  -(\mu_s + \mu_L) \cdot \vec{B}_{ext} \]

where the $z$-subscript stands for "Zeeman", \emph{not} $z-$axis.

Earlier, we discussed how the magnetic moment of a charged particle is inversely proportional to its mass:
\[ \mu_s = \frac{q}{m} \vec{S}, \;\;\; \mu_L = \frac{q}{2m} \vec{L} \]

Because the proton is on the order of $1000$ times heavier than the electron, its magnetic moment $\mu_p$ is negligible compared to $\mu_e$. Thus, we will neglect the effect due to the proton and only worry about the electron.

\vskip 0.5cm
\[ \implies \hat{H}_z = \frac{e}{2m} \left( \vec{L}  + 2\vec{S} \right) \cdot \vec{B} \]

Since we have a uniform magnetic field $\vec{B}(\vec{r}) = \vec{B}_0$. Let's consider a \emph{weak field} $\vec{B}_{ext} = B_0 \hat{z}$ where $B_0$ is small relative to the fine-structure corrections.

All we need to do is use $J = L + S$ since it's a good quantum number.

\begin{align*}
  E_{zeeman}^{(1)} &= \frac{e}{2m} B_0 \mean{ (L_z + 2S_z) }_{rel} \\
  &= \frac{e}{2m} B_0 \mean{(J_z + S_z)}_{rel} \\
  &= \frac{e}{2m} B_0 \left( m_j \hbar + \mean{S_z}_{rel} \right)
\end{align*}

We can then evaluate the expectation value using the $\ket{l, m}$-space representation of the stat e $\ket{j = l \pm \frac{1}{2}, m_s}$ 

\[ \ket{j = l \pm \frac{1}{2}, m_s} =  \]

We find the expectation value to be 
\[ \mean{S_z}_{rel} = \pm\frac{m_j \hbar}{2l+1} \]

Thus, 
\[ \boxed{ E_{zeeman}^{(1)} = \frac{eB_0}{2m} m_j \hbar \left( 1 \pm \frac{1}{2l+1} \right) } \]

\end{document}
