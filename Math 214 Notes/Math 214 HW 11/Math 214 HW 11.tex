\documentclass{article}

% Language setting
% Replace `english' with e.g. `spanish' to change the document language
\usepackage[english]{babel}

% Set page size and margins
% Replace `letterpaper' with`a4paper' for UK/EU standard size
\usepackage[letterpaper,top=2cm,bottom=2cm,left=3cm,right=3cm,marginparwidth=1.75cm]{geometry}

% Useful packages
\usepackage{amsmath}
\usepackage{amssymb}
\usepackage{mathtools}
\usepackage{graphicx}
\usepackage{enumitem}
\usepackage{bbm}
% \usepackage{tensor}
\usepackage[colorlinks=true, allcolors=blue]{hyperref}

\usepackage{hyperref}
\hypersetup{
    colorlinks=true,
    linkcolor=blue,
    filecolor=magenta,      
    urlcolor=cyan,
    pdftitle={Math 214 HW 11},
    pdfpagemode=FullScreen,
    }

\urlstyle{same}

\usepackage{tikz-cd}

%%%%%%%%%%% Box pacakges and definitions %%%%%%%%%%%%%%
\usepackage[most]{tcolorbox}
\usepackage{xcolor}

% Define the colors
\definecolor{boxheader}{RGB}{0, 51, 102}  % Dark blue
\definecolor{boxfill}{RGB}{173, 216, 230}  % Light blue

% Define the tcolorbox environment
\newtcolorbox{mathdefinitionbox}[2][]{%
    colback=boxfill,   % Background color
    colframe=boxheader, % Border color
    fonttitle=\bfseries, % Bold title
    coltitle=white,     % Title text color
    title={#2},         % Title text
    enhanced,           % Enable advanced features
    attach boxed title to top left={yshift=-\tcboxedtitleheight/2}, % Center title
    boxrule=0.5mm,      % Border width
    sharp corners,      % Sharp corners for the box
    #1                  % Additional options
}
%%%%%%%%%%%%%%%%%%%%%%%%%

\newtcolorbox{dottedbox}[1][]{%
    colback=white,    % Background color
    colframe=white,    % Border color (to be overridden by dashrule)
    sharp corners,     % Sharp corners for the box
    boxrule=0pt,       % No actual border, as it will be drawn with dashrule
    boxsep=5pt,        % Padding inside the box
    enhanced,          % Enable advanced features
    overlay={\draw[dashed, thin, black, dash pattern=on \pgflinewidth off \pgflinewidth, line cap=rect] (frame.south west) rectangle (frame.north east);}, % Dotted line
    #1                 % Additional options
}

\usepackage{biblatex}
\addbibresource{sample.bib}


%%%%%%%%%%% New Commands %%%%%%%%%%%%%%
\newcommand*{\T}{\mathcal T}
\newcommand*{\cl}{\text cl}
\newcommand{\bP}{\mathbb{P}}
\newcommand{\bS}{\mathbb{S}}


\newcommand{\ket}[1]{|#1 \rangle}
\newcommand{\bra}[1]{\langle #1|}
\newcommand{\inner}[2]{\langle #1 | #2 \rangle}
\newcommand{\R}{\mathbb{R}}
\newcommand{\C}{\mathbb{C}}
\newcommand{\A}{\mathbb{A}}
\newcommand{\sphere}{\mathbb{S}}
\newcommand{\V}{\mathbb{V}}
\newcommand{\Hilbert}{\mathcal{H}}
\newcommand{\oper}{\hat{\Omega}}
\newcommand{\lam}{\hat{\Lambda}}
\newcommand{\defeq}{\vcentcolon=}

\newcommand{\bigslant}[2]{{\raisebox{.2em}{$#1$}\left/\raisebox{-.2em}{$#2$}\right.}}
\newcommand{\restr}[2]{{% we make the whole thing an ordinary symbol
  \left.\kern-\nulldelimiterspace % automatically resize the bar with \right
  #1 % the function
  \vphantom{\big|} % pretend it's a little taller at normal size
  \right|_{#2} % this is the delimiter
  }}
%%%%%%%%%%%%%%%%%%%%%%%%%%%%%%%%%%%%%%%


\tcbset{theostyle/.style={
    enhanced,
    sharp corners,
    attach boxed title to top left={
      xshift=-1mm,
      yshift=-4mm,
      yshifttext=-1mm
    },
    top=1.5ex,
    colback=white,
    colframe=blue!75!black,
    fonttitle=\bfseries,
    boxed title style={
      sharp corners,
    size=small,
    colback=blue!75!black,
    colframe=blue!75!black,
  } 
}}

\newtcbtheorem[number within=section]{Theorem}{Theorem}{%
  theostyle
}{thm}

\newtcbtheorem[number within=section]{Definition}{Definition}{%
  theostyle
}{def}



\title{Math 214 Homework 11}
\author{Keshav Balwant Deoskar}

\begin{document}
\maketitle



%%%%%%%%%%%%%%%%%%%%%%%%%%%%%%%%%%%%%%%%%%%%%%%%%%%%%%%%%%%%%%%%%
\textbf{Q11-14.} Consider the following two covector fields on $\R^3$:
\begin{align*}
  \omega &= -\frac{4z dx}{(x^2 + 1)^2} + \frac{2ydy}{y^2 + 1} + \frac{2xdz}{x^2 + 1} \\
  \eta &= -\frac{4xz dx}{(x^2 + 1)^2} + \frac{2y dy}{y^2 + 1} + \frac{2dz}{x^2 + 1}
\end{align*} 

\begin{enumerate}[label=(\alph*)]
  \item Set up and evaluate the line integral of each covector field along the straight line segment from $(0,0,0)$ to $(1,1,1)$.
  \item Determine whether either of these covector fields is exact.
  \item For each one that is exact, find a potential function and use it to recompute the line integral.
\end{enumerate}
%%%%%%%%%%%%%%%%%%%%%%%%%%%%%%%%%%%%%%%%%%%%%%%%%%%%%%%%%%%%%%%%%

\vskip 0.5cm
\textbf{Proof:}

\begin{enumerate}[label=(\alph*)]
  \item Let $\gamma : \R \rightarrow \R^3$ be the following parametrization of the straight line from $(0,0,0)$ to $(1,1,1)$:
  \[ \gamma(t) = (t,t,t), \;\;\; t \in [0, 1] \]
  So, 
  \begin{align*}
    \int_{\gamma} \omega &= \int_{0}^{1} -\frac{4tdt}{(t^2+1)^2} + \frac{2tdt}{t^2 + 1} + \frac{2tdt}{t^2 + 1} \\
    &= -\int_{0}^{1} \frac{4tdt}{(t^2+1)^2} + \int_{0}^{1} \frac{2tdt}{t^2 + 1} + \int_{0}^{1} \frac{2tdt}{t^2 + 1} 
  \end{align*}
  We can solve each of these as in ordinary calculus to get 
  \begin{align*}
    \int_{\gamma} \omega &= -\int_{1}^{2} \frac{2 du}{u^2} + 2 \times \int_{1}^{2} \frac{du}{u} \\
    &= -2\left[\frac{-1}{u}\right]_1^2 + 2 \left[\ln(u)\right]_1^2 \\
    &= 2 \left[\frac{1}{2} - \frac{1}{1}\right] + 2\left[ \ln(2) - \ln(1) \right] \\
    \implies \int_{\gamma} \omega &= 2\ln(2) - 1
  \end{align*}

  We can similarly calculate the integral of $\eta$:
  \vskip 0.5cm
  \begin{align*}
    \int_{\gamma} \eta &= \int_{0}^{1} \frac{-4t^2dt}{(t^2+1)^2} + \frac{2tdt}{t^2 + 1} + \frac{2dt}{t^2 + 1} \\
    &= \int_{0}^{1} \left(\frac{-4t^2}{(t^2+1)^2} + \frac{2t}{t^2 + 1} + \frac{2}{t^2 + 1} \right) dt\\
    &= \int_{0}^{1} \frac{2\left(t^3 - t^2 + t + 1\right)}{(t^2 + 1)^2} dt \\
    &= \ln(2) + 1
  \end{align*}

  \vskip 0.5cm
  \item We can check that $\omega$ is closed by verifying all of the mixed second derivatives are equal:
  \begin{align*}
    &\frac{\partial^2 \omega}{\partial y \partial x} = \frac{\partial \left( \frac{4z}{(x^2 + 1)^2} \right)}{\partial y} = 0 = \frac{\partial \left(\frac{2y}{y^2 + 1}\right)}{\partial x} = \frac{\partial^2 \omega}{\partial x \partial y} \\
    &\frac{\partial^2 \omega}{\partial z \partial x} = \frac{\partial \left( \frac{4z}{(x^2 + 1)^2} \right)}{\partial z} = -\frac{4}{(x^2 + 1)^2} = \frac{\partial \left(\frac{2x}{x^2 + 1}\right)}{\partial x} = \frac{\partial^2 \omega}{\partial x \partial z} \\
    &\frac{\partial^2 \omega}{\partial z \partial y} = \frac{\partial \left( \frac{2y}{y^2 + 1} \right)}{\partial z} = 0 = \frac{\partial \left(\frac{2x}{x^2 + 1}\right)}{\partial y} = \frac{\partial^2 \omega}{\partial y \partial z} 
  \end{align*}
  Then, since the line segment from $(0,0,0)$ to $(1,1,1)$ is a star-shaped subset of $\R^3$, the Poincare Lemma (Theorem 11.49) tells us that $\omega$ is closed $\implies$ $\omega$ is exact.

  \vskip 0.25cm
  On the other hand, for $\eta$ we see that 
  \begin{align*}
    &\frac{\partial^2 \eta}{\partial z \partial x} = \frac{\partial \left( \frac{4xz}{(x^2 + 1)^2} \right)}{\partial z} = \frac{4x}{(x^2 + 1)^2} \\
    &\frac{\partial^2 \eta}{\partial x \partial z} = \frac{\partial \left( \frac{2}{x^2 + 1} \right)}{\partial x} = -\frac{4}{(x^2 + 1)^2} \\
    \implies & \frac{\partial^2 \eta}{\partial z \partial x} \neq \frac{\partial^2 \eta}{\partial x \partial z}
  \end{align*}
  Thus $\eta$ is not closed, but every exact form must be closed so $\eta$ is not exact.


  \vskip 0.5cm
  \item 

\end{enumerate}


\vskip 0.5cm
\hrule 
\vskip 0.5cm



%%%%%%%%%%%%%%%%%%%%%%%%%%%%%%%%%%%%%%%%%%%%%%%%%%%%%%%%%%%%%%%%%
\textbf{Q11-17.} Let $\mathbb{T}^n = \mathbb{S}^1 \times \cdots \times \mathbb{S}^1 \subseteq \C^n$ denote the $n$-torus. For each $j = 1, \cdots, n$, let $\gamma_j : [0, 1] \rightarrow \mathbb{T}^n$ be the curve segment
\[ \gamma_j(t) = \left(1, \cdots, e^{2\pi i t}, \cdots, 1\right) \;\;\;\;\text{(with $e^{2\pi i t}$ in the $j^{th}$ place)} \] Show that a closed covector field $\omega$ on $\mathbb{T}^n$ is exact if and only if $\int_{{\gamma_j}} \omega = 0$ for $j = 1, \cdots, n$
%%%%%%%%%%%%%%%%%%%%%%%%%%%%%%%%%%%%%%%%%%%%%%%%%%%%%%%%%%%%%%%%%

\vskip 0.5cm
\textbf{Proof:}

\underline{"$\implies$":} Suppose closed covector field $\omega \in \mathfrak{X}^* \left(\mathbb{T}^2\right) $ is exact. Then, there exists a potential function $f : M \rightarrow \R$ such that 
\begin{align*}
  \int_{{\gamma_j}} \omega &= \int_{{\gamma_j}} df  \\
  &= f(\gamma_j(1)) - f(\gamma_j(1)) \text{    (By the fundamental theorem of line integrals)} \\
  &= f\left(1, \cdots, e^{2\pi i \cdot 1}, \cdots, 1\right) - f\left(1, \cdots, e^{2\pi i \cdot 0}, \cdots, 1\right) \\
  &= f\left(1, \cdots, 1, \cdots, 1\right) - f\left(1, \
  \cdots, 1, \cdots, 1\right) \\
  &= 0
\end{align*}

\vskip 0.5cm
\underline{"$\impliedby$":} 

For the reverse direction, suppose that for each $j = 1, 2, \cdots, n$ we have \[ \int_{{\gamma_j}} \omega = 0 \] i.e. the integral of $\omega$ along each circle in the decomposition $\mathbb{T}^n = \mathbb{S}^1 \times \cdots \times \mathbb{S}^1$ is zero. If we can somehow show that $\int_{\gamma} \omega$ for \emph{any} piece-wise smooth closed curve, then $\omega$ will be conservative and thus exact.

\vskip 0.5cm
We have the smooth covering map $\varepsilon^n : \R^n \rightarrow \mathbb{T}^n$ defined by \[ \left(x^1, \cdots, x^n \right) \mapsto \left(e^{2\pi i x^1}, \cdots, e^{2\pi i x^n}\right) \] Consider any piece-wise smooth closed curve segment $\gamma : [0, 1] \rightarrow \mathbb{T}^n$ and let $\tilde{\gamma} : [0, 1] \rightarrow \R^n$ be a lift of $\gamma$ such that \[  \gamma = \varepsilon^n \circ \tilde{\gamma}   \] Since $\varepsilon^n$ is smooth, surjective and $\gamma$ is also smooth, exercise 4.10 tells us that $\tilde{\gamma}$ must be smooth as well. WLOG, we can assume that $\tilde{\gamma}(0) = 0$ and $\tilde{\gamma}(1) = \left(m^1, \cdots, m^n\right)$ for integers $m^1, \cdots, m^n$ (if these are not integers, then $\gamma = \varepsilon^n \circ \tilde{\gamma}$ will not be closed).

\vskip 0.5cm
Going forward, the idea will be to decompose the integral over $\gamma$ into integrals over paths along the $\gamma_j$'s. Define $\alpha_i : [0, 1] \rightarrow \mathbb{T}^n$ as the line segment from $\left(m^1, \cdots, m^{i-1}, 0, \cdots, 0\right)$ to $\left(m^1, \cdots, m^{i}, 0, \cdots, 0\right)$ and let $\alpha : [0, 1] \rightarrow \mathbb{T}^n$ be the concatenation of the $\alpha_i$'s such that $\alpha(0) = 0$ and $\alpha(1) = \left(m^1, \cdots, m^n\right)$. Then, 

\begin{align*}
  \int_{\gamma} \omega &= \int_{{\varepsilon^n \circ \tilde{\gamma} }} \omega \\
  &= \int_{ \tilde{\gamma} } \left(\varepsilon^n\right)^* \omega \\
  &= \int_{\alpha} \left(\varepsilon^n\right)^* \omega \text{  (Prop 11.42; Since $ \tilde{\gamma}, \alpha$ start and end at the same points)} \\
  &= \int_{{\alpha_1}} \left(\varepsilon^n\right)^* \omega + \cdots + \int_{{\alpha_n}} \left(\varepsilon^n\right)^* \omega \\
  &= \int_{\varepsilon^n \circ {\alpha_1}} \omega + \cdots + \int_{\varepsilon^n \circ {\alpha_n}} \omega \\
  &= \int_{\varepsilon^n \circ {\gamma_1}} \omega + \cdots + \int_{\varepsilon^n \circ {\gamma_n}} \omega \text{ (Again due to Prop 11.42)}\\
  &= 0
\end{align*}

\vskip 0.5cm
This shows that $\omega$ is a conservative covector field on a smooth manifold with or without boundary. Thus, by Proposition 11.42 it must be exact. 




\vskip 0.5cm
\hrule 
\vskip 0.5cm




%%%%%%%%%%%%%%%%%%%%%%%%%%%%%%%%%%%%%%%%%%%%%%%%%%%%%%%%%%%%%%%%%
\textbf{Q12-6.} \begin{enumerate}[label=(\alph*)]
  \item Let $\alpha$ be a covariant $k-$tensor on a finite dimensional real vector space $V$. Show that $\mathrm{Sym}\alpha$ is the unique symmetric $k-$tensor satisfying 
  \[  \left(\mathrm{Sym}\alpha\right)\left(v, \cdots, v\right) = \alpha\left(v, \cdots, v\right) \] for all $v \in V$.

  \item Show that the symmetric product is associative: for all symmetric tensors $\alpha, \beta, \gamma$,
  \[ \left(\alpha \beta\right) \gamma = \alpha \left(\gamma \beta\right)  \]

  \item Let $\omega^1, \cdots, \omega^k$ be covectors on a finite-dimensional vector space. Show that their symmetric product satisfies 
  \[  \omega^1, \cdots, \omega^k = \frac{1}{k!}\sum_{\sigma \in S_k} \omega^{\sigma(1)} \otimes \cdots \otimes \omega^{\sigma(k)} \]
\end{enumerate} 
%%%%%%%%%%%%%%%%%%%%%%%%%%%%%%%%%%%%%%%%%%%%%%%%%%%%%%%%%%%%%%%%%

\vskip 0.5cm
\textbf{Proof:}

\begin{enumerate}[label=(\alph*)]
  \item Suppose there exists some other covariant $k-$tensor $\left(\mathrm{Sym}\right)' \in \Sigma_{k}(V^*)$ such that 
  \[ \left(\mathrm{Sym} \; \alpha\right)'(v, \cdots, v) = (v, \cdots, v) \] Define $\beta \equiv \left(\mathrm{Sym}\right) - \left(\mathrm{Sym}\right)' \in \Sigma_{k}(V^*)$. Now, fix $v, w_1 \in V$ and for $\epsilon > 0$ let $\gamma_v :(-\epsilon, \epsilon) \rightarrow V$ be the map
  \[ t \mapsto v + tw \]

  Then, for all $t_0 \in (-\epsilon, \epsilon)$ we have $\beta(\gamma_v(t_0), \cdots, \gamma_v(t_0)) = 0$ so 
  \[ \restr{\frac{d}{dt}}{t_0} \beta(\gamma_v(t_0), \cdots, \gamma_v(t_0)) = 0  \]

  So, in particular, at $t = 0$,

\begin{align*}
  \restr{\frac{d}{dt}}{t = 0} \beta(\gamma_v(t_0), \cdots, \gamma_v(t_0)) = \beta(w_1, v, \cdots, v) + \beta(v, w_1, \cdots, v) + \cdots + \beta(v, \cdots, v, w_1) = 0
\end{align*}

But since $\beta$ is symmetric, this means 
\begin{align*}
  &k\cdot\beta(w_1, v, \cdots, v) = 0 \\
  \implies&\beta(w_1, v, \cdots, v) = 0 \\
  \implies&(\mathrm{Sym} \alpha)(w_1, v, \cdots, v) = (\mathrm{Sym}' \alpha)(w_1, v, \cdots, v)
\end{align*}

We can bascially the same arguement for the remaining $k-1$ entries so we finally obtain 
\[ (\mathrm{Sym} \alpha)(w_1, w_2, \cdots, w_k) = (\mathrm{Sym}' \alpha)(w_1, w_2, \cdots, w_k) \]

Thus, $\mathrm{Sym}\;\alpha$ is unique.

\vskip 0.5cm
\item Suppose we have symmetric covariant tensors $\alpha, \beta, \gamma$ of rank $k, l, m$ on a vector space $V$. 

Now, $\beta \gamma = \mathrm{Sym}(\beta \otimes \gamma)$ is an $(l+m)-$rank covector on $V$, so $\alpha (\beta \gamma) = \mathrm{Sym}\left(\alpha \otimes \mathrm{Sym}(\beta \otimes \gamma)\right)$ is a rank $(k+l+m)$ covariant tensor on $V$.

For any $v \in V$, 

\begin{align*}
  \left(\alpha \beta\right) \gamma &= \left( \mathrm{Sym} \left(\mathrm{Sym}(\alpha \otimes \beta) \otimes \gamma \right) \right)(v, \cdots, v) \\ 
  &= \left(\mathrm{Sym}(\alpha \otimes \beta)\right) (v, \cdots, v) \gamma \left(v, \cdots, v\right) \\
  &= \alpha (v, \cdots, v) \beta(v \cdots, v) \gamma(v, \cdots, v) \\
  &= \alpha(v, \cdots, v) \left(\mathrm{Sym}(\beta \otimes \gamma)\right) (v, \cdots, v) \\
  &= \left( \mathrm{Sym} \alpha \otimes \left( \mathrm{Sym} (\beta \otimes \gamma)\right)  \right) (v, \cdots, v) \\
  &= \alpha (\beta \gamma)
\end{align*}

Therefore, the symmeteric product is associative.
\[ \alpha \left( \beta \gamma \right) = \left(\alpha \beta \right) \gamma = \alpha \beta \gamma \]

\vsize 0.5cm
\item Now let $\omega^1, \cdots, \omega^k$ be covectors on a finte-dimensional vector space.

\vskip 0.25cm
We know from proposition 12.15(b) that 
\[ \omega^1 \omega^2 = \frac{1}{2}\left(\omega^1 \otimes \omega^2 + \omega^2 \otimes \omega^1\right) = \frac{1}{2!} \sum_{\sigma \in S_2} \omega^{\sigma(1)} \otimes \cdots \omega^{\sigma(k)} \]

So the base case $k = 2$ is satisfied.

Now, let's assume that the inductive hypothesis holds for $k-1$ i.e.

\[ \omega^1 \cdots \omega^{k-1} = \frac{1}{(k-1)!} \sum_{\sigma \in S_{k-1}}  \omega^1 \otimes \cdots \otimes \omega^{k-1}   \]

Then, 
\begin{align*}
  \left( \omega^1 \cdots \omega^{k-1} \right) \omega^k &= \frac{1}{2} \left[  \left( \omega^1 \cdots \omega^{k-1} \right) \otimes \omega^{k} + \omega^k \otimes  \left( \omega^1 \cdots \omega^{k-1} \right) \right] \\\
  &= \frac{1}{2} \left[  \frac{1}{(k-1)!} \sum_{\sigma \in S_{k-1}}  \omega^1 \otimes \cdots \otimes \omega^{k-1}\otimes \omega^{k} + \omega^k \otimes  \frac{1}{(k-1)!} \sum_{\sigma \in S_{k-1}}  \omega^1 \otimes \cdots \otimes \omega^{k-1} \right] \\
\end{align*}

Now, by the associativity of the symmetric product shown in (b), we have 
\[ (\omega^1, \cdots, \omega^{k-1}) \omega^k = \omega^1 (\omega^2 \cdots, \omega^{k-1} \omega^k) = \left(\omega^1, \cdots, \omega^{i-1}\right) \omega^i \left(\omega^{i+1}, \cdots, \omega^{k}\right)  \]

There are $k$ ways to choose $i$, and all of these terms are equal. So, dividing by $k$, we find that the $\omega^i$'s can be permuted in $k!$ ways. Thus, 

\begin{align*}
  \omega^1, \cdots, \omega^k &= \frac{1}{k} \sum_{i = 1}^{k} \left(\omega^1, \cdots, \omega^{i-1}\right) \omega^i \left(\omega^{i+1}, \cdots, \omega^{k}\right) \\
  &= \frac{1}{k!} \sum_{\sigma \in S_k} \omega^{\sigma(1)} \otimes \omega^{\sigma(k)}
\end{align*}
 
\end{enumerate}


\vskip 0.5cm
\hrule 
\vskip 0.5cm



%%%%%%%%%%%%%%%%%%%%%%%%%%%%%%%%%%%%%%%%%%%%%%%%%%%%%%%%%%%%%%%%%
\textbf{Q12-11.} Suppose $M$ is a smooth manifold, $A$ is a smooth covariant tensor field on $M$, and $V, W \in \mathfrak{X}(M)$. Show that 
\[ \mathcal{L}_V \mathcal{L}_W A - \mathcal{L}_W \mathcal{L}_V A = \mathcal{L}P_{[V, W]} A \]
%%%%%%%%%%%%%%%%%%%%%%%%%%%%%%%%%%%%%%%%%%%%%%%%%%%%%%%%%%%%%%%%%

\vskip 0.5cm
\textbf{Proof:}


\vskip 0.5cm
\hrule 
\vskip 0.5cm




%%%%%%%%%%%%%%%%%%%%%%%%%%%%%%%%%%%%%%%%%%%%%%%%%%%%%%%%%%%%%%%%%
\textbf{Q13-2.} 
Suppose $F$ is a smooth vector bundle over a smooth manifold $M$ with or without boundary, and $V \subseteq E$ is an open subset with the property that for each $p \in M$, the intersection of $v$ with the fiber $E_p$ is convex and non-empty. By a "section of $V$", we mean a (local or global section of $E$ whose image lies in $V$.)
\begin{enumerate}[label=(\alph*)]
  \item Show that there exists a smooth global section of $V$.
  \item Suppose $\sigma : A \rightarrow V$ is a smooth section of $V$ defined on a closed subset $A \subseteq M$. Show that there exists a smooth global section $\tilde{\sigma}$ of $V$ whose restriction to $A$ is equal to $\sigma$. Show that if $V$ contains the image of the zero section of $E$, then $\tilde{\sigma}$ can be chosen to be supported in any predetermined neighborhood of $A$.
\end{enumerate}
%%%%%%%%%%%%%%%%%%%%%%%%%%%%%%%%%%%%%%%%%%%%%%%%%%%%%%%%%%%%%%%%%

\vskip 0.5cm
\textbf{Proof:}


\vskip 0.5cm
\hrule 
\vskip 0.5cm




%%%%%%%%%%%%%%%%%%%%%%%%%%%%%%%%%%%%%%%%%%%%%%%%%%%%%%%%%%%%%%%%%
\textbf{Q13-13.} Let $(M, g)$ be a Riemannian manifold. A smooth vector field $V$ on $M$ is called a \textbf{Killing vector field for $g$} if the flow of $V$ acts by isometries of $g$.
\begin{enumerate}[label=(\alph*)]
  \item Show that the set of all Killing vector fields on $M$ constitutes a Lie subalgebra of $\mathfrak{X}(M)$.
  \item Show that a smooth vector field $V$ on $M$ is a Killing vector field if and only if it satisfies the follownig equation in each smooth local coordinate chart:
  \[ V^k \frac{\partial g_{ij}}{\partial x^k} + g_{jk} \frac{\partial V^k}{\partial x^i} + g_{ik} \frac{\partial V^k}{\partial x^j} = 0 \]
\end{enumerate}
%%%%%%%%%%%%%%%%%%%%%%%%%%%%%%%%%%%%%%%%%%%%%%%%%%%%%%%%%%%%%%%%%

\vskip 0.5cm
\textbf{Proof:}


\vskip 0.5cm
\hrule 
\vskip 0.5cm


% %%%%%%%%%%%%%%%%%%%%%%%%%%%%%%%%%%%%%%%%%%%%%%%%%%%%%%%%%%%%%%%%%
% \textbf{Q13-.} 
% %%%%%%%%%%%%%%%%%%%%%%%%%%%%%%%%%%%%%%%%%%%%%%%%%%%%%%%%%%%%%%%%%

% \vskip 0.5cm
% \textbf{Proof:}


% \vskip 0.5cm
% \hrule 
% \vskip 0.5cm




% %%%%%%%%%%%%%%%%%%%%%%%%%%%%%%%%%%%%%%%%%%%%%%%%%%%%%%%%%%%%%%%%%
% \textbf{Q13-.} 
% %%%%%%%%%%%%%%%%%%%%%%%%%%%%%%%%%%%%%%%%%%%%%%%%%%%%%%%%%%%%%%%%%

% \vskip 0.5cm
% \textbf{Proof:}


% \vskip 0.5cm
% \hrule 
% \vskip 0.5cm


\end{document}
