\documentclass{article}

% Language setting
% Replace `english' with e.g. `spanish' to change the document language
\usepackage[english]{babel}

% Set page size and margins
% Replace `letterpaper' with`a4paper' for UK/EU standard size
\usepackage[letterpaper,top=2cm,bottom=2cm,left=3cm,right=3cm,marginparwidth=1.75cm]{geometry}

% Useful packages
\usepackage{amsmath}
\usepackage{amssymb}
\usepackage{graphicx}
\usepackage[colorlinks=true, allcolors=blue]{hyperref}

\usepackage{hyperref}
\hypersetup{
    colorlinks=true,
    linkcolor=blue,
    filecolor=magenta,      
    urlcolor=cyan,
    pdftitle={214 Lecture 15},
    pdfpagemode=FullScreen,
    }

\urlstyle{same}

\usepackage{tikz-cd}

%%%%%%%%%%% Box pacakges and definitions %%%%%%%%%%%%%%
\usepackage[most]{tcolorbox}
\usepackage{xcolor}
\usepackage{dashrule}

% Define the colors
\definecolor{boxheader}{RGB}{0, 51, 102}  % Dark blue
\definecolor{boxfill}{RGB}{173, 216, 230}  % Light blue

% Define the tcolorbox environment
\newtcolorbox{mathdefinitionbox}[2][]{%
    colback=boxfill,   % Background color
    colframe=boxheader, % Border color
    fonttitle=\bfseries, % Bold title
    coltitle=white,     % Title text color
    title={#2},         % Title text
    enhanced,           % Enable advanced features
    attach boxed title to top left={yshift=-\tcboxedtitleheight/2}, % Center title
    boxrule=0.5mm,      % Border width
    sharp corners,      % Sharp corners for the box
    #1                  % Additional options
}
%%%%%%%%%%%%%%%%%%%%%%%%%

\newtcolorbox{dottedbox}[1][]{%
    colback=white,    % Background color
    colframe=white,    % Border color (to be overridden by dashrule)
    sharp corners,     % Sharp corners for the box
    boxrule=0pt,       % No actual border, as it will be drawn with dashrule
    boxsep=5pt,        % Padding inside the box
    enhanced,          % Enable advanced features
    overlay={\draw[dashed, thin, black, dash pattern=on \pgflinewidth off \pgflinewidth, line cap=rect] (frame.south west) rectangle (frame.north east);}, % Dotted line
    #1                 % Additional options
}
\usepackage{biblatex}
\addbibresource{sample.bib}


%%%%%%%%%%% New Commands %%%%%%%%%%%%%%
\newcommand*{\T}{\mathcal T}
\newcommand*{\cl}{\text cl}


\newcommand{\ket}[1]{|#1 \rangle}
\newcommand{\bra}[1]{\langle #1|}
\newcommand{\inner}[2]{\langle #1 | #2 \rangle}
\newcommand{\R}{\mathbb{R}}
\newcommand{\C}{\mathbb{C}}
\newcommand{\V}{\mathbb{V}}
\newcommand{\A}{\mathcal{A}}
\newcommand{\halfplane}{\mathbb{H}}
\newcommand{\Hilbert}{\mathcal{H}}
\newcommand{\oper}{\hat{\Omega}}
\newcommand{\lam}{\hat{\Lambda}}
\newcommand{\qedsymbol}{\hfill\blacksquare}

\newcommand{\bigslant}[2]{{\raisebox{.2em}{$#1$}\left/\raisebox{-.2em}{$#2$}\right.}}
\newcommand{\restr}[2]{{% we make the whole thing an ordinary symbol
  \left.\kern-\nulldelimiterspace % automatically resize the bar with \right
  #1 % the function
  \vphantom{\big|} % pretend it's a little taller at normal size
  \right|_{#2} % this is the delimiter
  }}
%%%%%%%%%%%%%%%%%%%%%%%%%%%%%%%%%%%%%%%


\tcbset{theostyle/.style={
    enhanced,
    sharp corners,
    attach boxed title to top left={
      xshift=-1mm,
      yshift=-4mm,
      yshifttext=-1mm
    },
    top=1.5ex,
    colback=white,
    colframe=blue!75!black,
    fonttitle=\bfseries,
    boxed title style={
      sharp corners,
    size=small,
    colback=blue!75!black,
    colframe=blue!75!black,
  } 
}}

\newtcbtheorem[number within=section]{Theorem}{Theorem}{%
  theostyle
}{thm}

\newtcbtheorem[number within=section]{Definition}{Definition}{%
  theostyle
}{def}



\title{Math 214 Notes}
\author{Keshav Balwant Deoskar}

\begin{document}
\maketitle

% \vskip 0.5cm
These are notes taken from lectures on Differential Topology delivered by Eric C. Chen for UC Berekley's Math 214 class in the Sprng 2024 semester. Any errors that may have crept in are solely my fault.
% \pagebreak 

\tableofcontents

\pagebreak

\section{March 5 - }
There was no class on Feb 29 (midterm instead), so this is the first lecture after Sard's Theorem. Midterms will (hopefully) be graded by mid-to-late next week.

\vskip 1cm
\subsection*{Recap}
\begin{itemize}
  \item Last week, we proved \textbf{\emph{Sard's Theorem}}, which told us that the set of critical values of a smooth map between manifolds has measure zero.
  \item We applied Sard's Theorem to show the \emph{\textbf{Whitney Embedding Theorem}}, which tells us that any smooth manifold of dimension $n$ can be embedded in $\R^{n+1}$.
\end{itemize}

Today, we'll see some more applications of Sard's Theorem.

\vskip 0.5cm
\subsection{Whitney Approximation Theorem for functions}

\begin{dottedbox}
  \emph{\textbf{Theorem:}} Consider a sontinuous function $F : M \rightarrow \R^k$ such that $\restr{F}{A}$ is smooth fora closed subset $A \subseteq M$. Then, there is a continuous function $\delta : M \rightarrow \R_+$such that there exists smooth $\tilde{F} : M \rightarrow \R^k$ such that 
  \[ \left| \tilde{F}(x) -F(x) \right| < \delta(x) \]
  for all $x \in M$ and 
  \[ \restr{\tilde{F}}{A} = \restr{F}{A} \]
\end{dottedbox}

\textbf{\emph{Proof:}} Consider smooth mfd $M$ and closed subset $A \subseteq M$ and function $F: M \rightarrow \R^k$ wihch is smoot hon $A$. Using the extension lemma, we can obtain a smooth map $F_0 : M \rightarrow \R^k$ such that $\restr{F_0}{A} = \restr{F}{A}$ (using partitions of unity.)

\vskip 0.25cm
This function $F_0$ agres with $F$ on $A$ but may differe on the rest of $M$, so we will rectify this. Let $U_0= \{ y \in M : \left| F_0(y) - F(y) \right| < \delta(y) \} \supset A$and for any $x \not\in A$ take a neighborhood of $x$, $U_x \subseteq M \setminus A$, such that $\left| F(x) - F(y) \right| < \delta(y)$ for all $y \in U_x$.

\vskip 0.25cm
So, $M = U_0 \cup U_{x \in M \setminus A} U_x$ (open cover). Then, take the partition of unity subordinate to the open cover $\{U_0, \text{all }U_x\}$ to be $\{\psi_0, \{\psi_x\}_{x \in M \setminus A}\}$.

Set $\tilde{F}(y) = \psi_0(y)F_0(y) + \sum_{x \in M \setminus A} \psi_x(y) \cdot F(x) $. This is smooth [write why]. Compare this with $F(y) = \psi_0(y)F(y) + \sum_{x \in M \setminus A} \psi_x(y) \cdot F(x)$. We have 

\vskip 0.25cm
\begin{align*}
  |\tilde{F}(y) - F(y) | &\leq \psi_0(y)|F_0(y)- F(y) | + \sum_{x \in M \setminus A} \psi_x(y) \cdot |F(x)-F(y)| \\
  &\leq \delta(y)
\end{align*}

\vskip 0.5cm
\begin{dottedbox}
  \emph{\textbf{Cor:}} For smooth mfd $M$ with continuous [Write from image]
\end{dottedbox}

\vskip 0.25cm
Now that we've proved it for functions, let's prove a similar statement for maps between smooth manifolds.

\vskip 1cm
\subsection{Whiteney approximation theorem for maps between manifolds, $F : N^n \rightarrow M^m$}

\vskip 0.25cm
We have a bit of an issue, which is we can't generally take linear combinations of oints in $M^m$. To get around this, the plan is to 
\begin{itemize}
  \item View $M \subseteq \R^k$ as an embedded manifold (whitney embedding)
  \item take linear combinations and smooth out the function in $\R^k$
  \item project back to $M$
\end{itemize}

To understand how to project back to $M$, we need to understand the \emph{\textbf{Normal Bundle}}. For embedded manifold $M^m \subseteq \R^k$, take the normal space at $x \in M$ to be 
\[ N_x M = \{ v \in \R^k : v \perp T_xM \} = \underbrace{(T_x M)^{\perp} }_{dim = k-m}\subseteq \R^k \]

Then, the \emph{\textbf{Normal Bundle}} is 
\[ NM = \{ (x, v) \in \R^k \times \R^k : x \in M, v \in N_x M \} \cong \coprod_{x \in M} N_x M \]

Notice that there is a copy of $m$ contained in the normal bundle as 
\[ M \cong M_0 = \{(x, 0):x \in M\} \]

As with the Tangent bundle, the Normal bundle is also a smooth manifold.

\vskip 0.5cm
\begin{dottedbox}
  \textbf{\emph{Lemma:}} Consider $M_0 \subseteq_{submfd} NM \subseteq_{submfd} \R^{2k}$ and ma $E : NM \rightarrow \R^{k}$ defined by 
  \[ (x, v) \mapsto x + v  \]
  is smooth and satisfies $E(M_0) = M$.
\end{dottedbox}

\vskip 0.5cm
With this in mind, we can construct tubular neighborhoods.

\vskip 1cm
\subsection{Tubular Neighborhoods}

\begin{mathdefinitionbox}{}
  \emph{\textbf{Def:}} A neighborhood $M \subseteq U \subseteq_{open} \R^k$ is a tubular neighborhodd of $M$ is there exists an open neighborhood $M_0 \subseteq V \subseteq NM$ such that $\restr{E}{V} : V \rightarrow U$ is a diffeomorphism.
\end{mathdefinitionbox}

\vskip 0.5cm
[Insert image]


\vskip 0.5cm
We will use this in the proof of the whitney approx theorem when trying to project back to $M$. The idea will be to map from $\R^k$ to a tubular neighborhood of $M$ using $E$ and then retract along the "white lines" in the image above to make the image exactly $M$.

\vskip 0.5cm
So, now, what we need to do is prove that every $M$ does indeed have a tubular neighbohood -- allowing the above procedure to work for arbitrary smooth manifold.

\vskip 0.5cm
\begin{dottedbox}
  \emph{\textbf{Theorem:}} Every $M \subseteq_{submfd} \R^k$ has a tubular neighborhood.
\end{dottedbox}

\emph{\textbf{Proof Sketch:}} 
\begin{itemize}
  \item Check that $E$ is a diffeomorphism near any $x \in M$. (Note Image$(dE_{(x,0)}) \supset T_xM + N_xM$ by inverse function thm - expand on this).
  \item To do this, for any $x \in M$, choose $V_x$ such that $\restr{E}{V_x}$ is a local diff. Then, shrink $V_x$ so $\restr{E}{V_x}$ is injective.
\end{itemize}

\vskip 1cm
We have all the tools we need now; let's begin.

\vskip 0.5cm
\begin{dottedbox}
  \textbf{\emph{Theorem: (Whitney extension for $F : M \rightarrow N$)}} 

  If $F : N \rightarrow M$ is continuous and $\restr{F}{A}$ is smooht for $A \subseteq_{closed} N$ then $F$ is homotopic relative $A$ to a smooth map $\tilde{F} : N \rightarrow M$.

  \vskip 0.5cm
  Recall that $F$ being homotopic to $\tilde{F}$ relative to $A$ means there exists continuous map $H : N \times [0,1] \rightarrow M$ such that 
  \begin{itemize}
    \item $H(x, 0) = F$
    \item $H(x, 1) = \tilde{F}$
    \item For $x \in A$, $H(x, t) = F(x)$ regardless of $t$
  \end{itemize}
\end{dottedbox}


\vskip 0.5cm
\emph{\textbf{Proof:}} By the Whiteney embedding theorem, we assume $M \subseteq \R^k$. From the theorem we proved earlier, we know $M$ has a tubular neighbohood $U$ with $M \subseteq U \subseteq \R^k$ and smooth retraction $r : U \rightarrow M$

\vskip 0.5cm
For any $x \in M$, let $\delta(x) = \sup\{ z \leq 1 : B_z(x) \subseteq U \}$

[write the rest from image; paying attention in calss]

\vskip 0.5cm
Now, we get to another application of Sard's Theorem, which is \emph{\textbf{Transversality}}. In a very vague sense, 

\subsection{Transversality}

\begin{mathdefinitionbox}{Ver 1.}
  Given two submanifolds, $S^k, \tilde{S}^{\tilde{k}} \subseteq M^n$ we say that the submanifolds intersect \emph{trasnversely} if 
  \[ T_{p} S + T_{p}\tilde{S} = T_p M \]
  for all $p \in S \cap \tilde{S}$.
\end{mathdefinitionbox}
\vskip 0.5cm

\begin{mathdefinitionbox}{Ver 2.}
  Given a submanifold $S^k \subseteq N^n$, a smooth map $F : N \rightarrow M$ is said to be \emph{transverse to $S^k$} if, for all $x \in F^{-1}(S)$, we have 
  \[   \] 
\end{mathdefinitionbox}

\vskip 0.5cm
\begin{dottedbox}
  \emph{\textbf{Remark:}} In Ver 1, if we take $F : S \hookrightarrow M$ (the inclusion), then $F$ is transverse in the sense of Ver 2, so ver 1 is really just a special case.
\end{dottedbox}

\vskip 0.5cm
\begin{dottedbox}
  \emph{\textbf{Theorem:}} 
  \begin{enumerate}
    \item In ver 1, if $S, \tilde{S}$ are transverse, then $S \cap \tilde{S} \subseteq M$ is a submanifold with 
    \item \[ \text{codim}_M  \left(S \cap \tilde{S}\right) = \text{codim}_M S + \text{codim}_M \tilde{S} \]
    
    \item In ver 2, if $F$ is transverse to $S$ then $F^{-1}(S) \subseteq_{submfd} N$ is a submanifold with $\text{codim}_MN F^{-1}(S) = \text{codim}_M S$
  \end{enumerate}

  Compare this to Regular Level Set theorem; turns out that RLST is a special case of this version.
\end{dottedbox}

\vskip 0.5cm
\begin{dottedbox}
  \emph{Remark:} It suffices to prove (2) because we can translate from ver 2 to ver 1.
\end{dottedbox}

\vskip 0.5cm
\textbf{\emph{Proof:}}  For any point on the submanifold $S$, we have a local defining function $\phi$



\end{document}


