\documentclass{article}

% Language setting
% Replace `english' with e.g. `spanish' to change the document language
\usepackage[english]{babel}

% Set page size and margins
% Replace `letterpaper' with`a4paper' for UK/EU standard size
\usepackage[letterpaper,top=2cm,bottom=2cm,left=3cm,right=3cm,marginparwidth=1.75cm]{geometry}

% Useful packages
\usepackage{amsmath}
\usepackage{amssymb}
\usepackage{graphicx}
\usepackage[colorlinks=true, allcolors=blue]{hyperref}

\usepackage{hyperref}
\hypersetup{
    colorlinks=true,
    linkcolor=blue,
    filecolor=magenta,      
    urlcolor=cyan,
    pdftitle={214 Lecture 16},
    pdfpagemode=FullScreen,
    }

\urlstyle{same}

\usepackage{tikz-cd}

%%%%%%%%%%% Box pacakges and definitions %%%%%%%%%%%%%%
\usepackage[most]{tcolorbox}
\usepackage{xcolor}
\usepackage{dashrule}

% Define the colors
\definecolor{boxheader}{RGB}{0, 51, 102}  % Dark blue
\definecolor{boxfill}{RGB}{173, 216, 230}  % Light blue

% Define the tcolorbox environment
\newtcolorbox{mathdefinitionbox}[2][]{%
    colback=boxfill,   % Background color
    colframe=boxheader, % Border color
    fonttitle=\bfseries, % Bold title
    coltitle=white,     % Title text color
    title={#2},         % Title text
    enhanced,           % Enable advanced features
    attach boxed title to top left={yshift=-\tcboxedtitleheight/2}, % Center title
    boxrule=0.5mm,      % Border width
    sharp corners,      % Sharp corners for the box
    #1                  % Additional options
}
%%%%%%%%%%%%%%%%%%%%%%%%%

\newtcolorbox{dottedbox}[1][]{%
    colback=white,    % Background color
    colframe=white,    % Border color (to be overridden by dashrule)
    sharp corners,     % Sharp corners for the box
    boxrule=0pt,       % No actual border, as it will be drawn with dashrule
    boxsep=5pt,        % Padding inside the box
    enhanced,          % Enable advanced features
    overlay={\draw[dashed, thin, black, dash pattern=on \pgflinewidth off \pgflinewidth, line cap=rect] (frame.south west) rectangle (frame.north east);}, % Dotted line
    #1                 % Additional options
}
\usepackage{biblatex}
\addbibresource{sample.bib}


%%%%%%%%%%% New Commands %%%%%%%%%%%%%%
\newcommand*{\T}{\mathcal T}
\newcommand*{\cl}{\text cl}


\newcommand{\ket}[1]{|#1 \rangle}
\newcommand{\bra}[1]{\langle #1|}
\newcommand{\inner}[2]{\langle #1 | #2 \rangle}
\newcommand{\R}{\mathbb{R}}
\newcommand{\C}{\mathbb{C}}
\newcommand{\V}{\mathbb{V}}
\newcommand{\A}{\mathcal{A}}
\newcommand{\halfplane}{\mathbb{H}}
\newcommand{\Hilbert}{\mathcal{H}}
\newcommand{\oper}{\hat{\Omega}}
\newcommand{\lam}{\hat{\Lambda}}
\newcommand{\qedsymbol}{\hfill\blacksquare}

\newcommand{\bigslant}[2]{{\raisebox{.2em}{$#1$}\left/\raisebox{-.2em}{$#2$}\right.}}
\newcommand{\restr}[2]{{% we make the whole thing an ordinary symbol
  \left.\kern-\nulldelimiterspace % automatically resize the bar with \right
  #1 % the function
  \vphantom{\big|} % pretend it's a little taller at normal size
  \right|_{#2} % this is the delimiter
  }}
%%%%%%%%%%%%%%%%%%%%%%%%%%%%%%%%%%%%%%%


\tcbset{theostyle/.style={
    enhanced,
    sharp corners,
    attach boxed title to top left={
      xshift=-1mm,
      yshift=-4mm,
      yshifttext=-1mm
    },
    top=1.5ex,
    colback=white,
    colframe=blue!75!black,
    fonttitle=\bfseries,
    boxed title style={
      sharp corners,
    size=small,
    colback=blue!75!black,
    colframe=blue!75!black,
  } 
}}

\newtcbtheorem[number within=section]{Theorem}{Theorem}{%
  theostyle
}{thm}

\newtcbtheorem[number within=section]{Definition}{Definition}{%
  theostyle
}{def}



\title{Math 214 Notes}
\author{Keshav Balwant Deoskar}

\begin{document}
\maketitle

% \vskip 0.5cm
These are notes taken from lectures on Differential Topology delivered by Eric C. Chen for UC Berekley's Math 214 class in the Sprng 2024 semester. Any errors that may have crept in are solely my fault.
% \pagebreak 

\tableofcontents

\pagebreak

\section{March 7 - }
\subsection*{Recap}
\begin{itemize}
  \item Last time, we finished proving the Whitney Extension theorems and started studying Transversality.
  \item Write more
  \item The second version of the Transverality Theorem gave us a generalization for the Regular Level Set Theorem.
\end{itemize}

\vskip 0.5cm
Today, we'll explore Transversality more. We'll see that if we hae non transverse intersections, we can often perturb our objects gently to make the intersection transverse.


\begin{dottedbox}
  \emph{\textbf{Theorem: (Parametric Transversality Theorem)}}

  Let $F : N \times S \rightarrow M$ be a smooth map viewed as $F_s= F(\; ,s)$ transverse to msooth submanifold $X \subseteq M$, then $F_S$ os trammsverse tp $X$ for almost every $s \in S$.
\end{dottedbox}

\vskip 0.5cm
\emph{\textbf{Proof:}} Write from image.


\vskip 1cm
Next, we have an application of this theorem which tells us that every smooth map can be deformed to have transverse intersection.

\vskip 0.5cm
\begin{dottedbox}
  \emph{\textbf{Theorem: (Transversality Homotopy Theorem)}} Write from image.
\end{dottedbox}

\vskip 0.5cm
\emph{\textbf{Proof:}} Write from image.

\vskip 0.5cm
This marks the end of our section on Sard's Theorem. Some closing remraks:
\begin{enumerate}
  \item Write from picture. Might not require Sard's Theorem.
  \item Poincare Duality connects tbe (n-1) homology group with the first cohomology group, and then there's something whcih connects that to the set of homotopy classes of maps $M \rightarrow S^1$.
  \item More from image.
\end{enumerate}

\vskip 1cm
\subsection{Onto Lie Groups! Chapter 7 begins.}

\begin{mathdefinitionbox}{Lie Group}
  \vskip 0.25cm
  A \emph{\textbf{Lie Group}} is a smooth manifold $G$ with group structure such that 
  \begin{align*}
    m : G \times G \rightarrow G, \;\;(g,h) \mapsto gh \\
    i : G \rightarrow G, \;\; \mapsto g^{-1}
  \end{align*}
  are smooth.
\end{mathdefinitionbox}

\vskip 0.5cm
\emph{Examples:}
\begin{itemize}
  \item Write from image
\end{itemize}

\vskip 0.5cm
\subsection*{Left/Right translations}

\begin{mathdefinitionbox}{}
  Fix $g \in G$. Then, left and right translation refer to the smooth maps
  \begin{align*}
    write \; from \; image
  \end{align*}
\end{mathdefinitionbox}

\begin{dottedbox}
  \emph{Remark:} 
  \begin{itemize}
    \item $L_{g_1} \circ R_{g_2} = R_{g_2} \circ L_{g_1}$
  \end{itemize}
\end{dottedbox}

\end{document}


