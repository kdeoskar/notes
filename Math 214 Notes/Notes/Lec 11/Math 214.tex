\documentclass{article}

% Language setting
% Replace `english' with e.g. `spanish' to change the document language
\usepackage[english]{babel}

% Set page size and margins
% Replace `letterpaper' with`a4paper' for UK/EU standard size
\usepackage[letterpaper,top=2cm,bottom=2cm,left=3cm,right=3cm,marginparwidth=1.75cm]{geometry}

% Useful packages
\usepackage{amsmath}
\usepackage{amssymb}
\usepackage{graphicx}
\usepackage[colorlinks=true, allcolors=blue]{hyperref}

\usepackage{hyperref}
\hypersetup{
    colorlinks=true,
    linkcolor=blue,
    filecolor=magenta,      
    urlcolor=cyan,
    pdftitle={214 Lecture 11},
    pdfpagemode=FullScreen,
    }

\urlstyle{same}

\usepackage{tikz-cd}

%%%%%%%%%%% Box pacakges and definitions %%%%%%%%%%%%%%
\usepackage[most]{tcolorbox}
\usepackage{xcolor}
\usepackage{dashrule}

% Define the colors
\definecolor{boxheader}{RGB}{0, 51, 102}  % Dark blue
\definecolor{boxfill}{RGB}{173, 216, 230}  % Light blue

% Define the tcolorbox environment
\newtcolorbox{mathdefinitionbox}[2][]{%
    colback=boxfill,   % Background color
    colframe=boxheader, % Border color
    fonttitle=\bfseries, % Bold title
    coltitle=white,     % Title text color
    title={#2},         % Title text
    enhanced,           % Enable advanced features
    attach boxed title to top left={yshift=-\tcboxedtitleheight/2}, % Center title
    boxrule=0.5mm,      % Border width
    sharp corners,      % Sharp corners for the box
    #1                  % Additional options
}
%%%%%%%%%%%%%%%%%%%%%%%%%

\newtcolorbox{dottedbox}[1][]{%
    colback=white,    % Background color
    colframe=white,    % Border color (to be overridden by dashrule)
    sharp corners,     % Sharp corners for the box
    boxrule=0pt,       % No actual border, as it will be drawn with dashrule
    boxsep=5pt,        % Padding inside the box
    enhanced,          % Enable advanced features
    overlay={\draw[dashed, thin, black, dash pattern=on \pgflinewidth off \pgflinewidth, line cap=rect] (frame.south west) rectangle (frame.north east);}, % Dotted line
    #1                 % Additional options
}
\usepackage{biblatex}
\addbibresource{sample.bib}


%%%%%%%%%%% New Commands %%%%%%%%%%%%%%
\newcommand*{\T}{\mathcal T}
\newcommand*{\cl}{\text cl}


\newcommand{\ket}[1]{|#1 \rangle}
\newcommand{\bra}[1]{\langle #1|}
\newcommand{\inner}[2]{\langle #1 | #2 \rangle}
\newcommand{\R}{\mathbb{R}}
\newcommand{\C}{\mathbb{C}}
\newcommand{\V}{\mathbb{V}}
\newcommand{\A}{\mathcal{A}}
\newcommand{\halfplane}{\mathbb{H}}
\newcommand{\Hilbert}{\mathcal{H}}
\newcommand{\oper}{\hat{\Omega}}
\newcommand{\lam}{\hat{\Lambda}}
\newcommand{\qedsymbol}{\hfill\blacksquare}

\newcommand{\bigslant}[2]{{\raisebox{.2em}{$#1$}\left/\raisebox{-.2em}{$#2$}\right.}}
\newcommand{\restr}[2]{{% we make the whole thing an ordinary symbol
  \left.\kern-\nulldelimiterspace % automatically resize the bar with \right
  #1 % the function
  \vphantom{\big|} % pretend it's a little taller at normal size
  \right|_{#2} % this is the delimiter
  }}
%%%%%%%%%%%%%%%%%%%%%%%%%%%%%%%%%%%%%%%


\tcbset{theostyle/.style={
    enhanced,
    sharp corners,
    attach boxed title to top left={
      xshift=-1mm,
      yshift=-4mm,
      yshifttext=-1mm
    },
    top=1.5ex,
    colback=white,
    colframe=blue!75!black,
    fonttitle=\bfseries,
    boxed title style={
      sharp corners,
    size=small,
    colback=blue!75!black,
    colframe=blue!75!black,
  } 
}}

\newtcbtheorem[number within=section]{Theorem}{Theorem}{%
  theostyle
}{thm}

\newtcbtheorem[number within=section]{Definition}{Definition}{%
  theostyle
}{def}



\title{Math 214 Notes}
\author{Keshav Balwant Deoskar}

\begin{document}
\maketitle

% \vskip 0.5cm
These are notes taken from lectures on Differential Topology delivered by Eric C. Chen for UC Berekley's Math 214 class in the Sprng 2024 semester. Any errors that may have crept in are solely my fault.
% \pagebreak 

\tableofcontents

\pagebreak

\section{February 20 - }

\vskip 1cm
\subsection*{Recap}
Some important results we've built in the last few lectures are:
\begin{itemize}
  \item We built the incredibly handy \emph{Rank Theorem}, which essentially tells us that the coordinate representations of constant rank maps between manifolds are canonical in a neighborhood of each point i.e. a map $F : M^m \rightarrow N^n$ of constant rank $r$ can be expressed, in some small neighborhood of any $x = (x^1, \dots, x^m)$, as 
  \[ F(x^1, \dots, x^r, \dots, x^m) = (x^1, \dots, x^r)  \]
  
  \item Last time, we applied this to $k-$slices. We argued that if we have $\mathbb{S}^k \hookrightarrow M^m$ and if locally $\mathbb{S}^k$ is a $k-$slice [finish later]
  \item Then, we spoke about Level Sets. Recall that if we have a smooth map $F : M^m \rightarrow N^n$, the inverse image of a point $q \in N$ i.e. $F^{-1}(q) \subseteq M$ is an embedded submanifold, given that it satisfies certain properties.
\end{itemize}

\vskip 0.5cm
\subsection{Constant Rank Level Set Theorem:}

\begin{dottedbox}
  \emph{\textbf{Theorem:}} If $F: M^m \rightarrow N^n$ is smooth and of constant rank $r$, then for any $q \in N$, the level set $F^{-1}(q) \subset M$ is a proper submanifold of $M$ with dimension $(m-r)$.
\end{dottedbox}

\vskip 0.5cm
\underline{\emph{Proof:}} We want to employ the rank theorem.

\vskip 0.5cm
Let $S = F^{-1}(q)$. Applying the rank theorem, we obtain charts $(U, \psi)$ on $M$ and $(V, \psi)$ on $N$ such that $\psi \circ F \circ \phi^{-1}$ has the coordinate representation 
\[ (x^1, \dots, x^r, x^{r+1}, \dots, x^{m}) \mapsto (x^1, \dots, x^r, 0, \dots, 0)  \]

So, if $\psi(q) = (c^1, \dots, c^r, 0, \dots 0)$, then the preimage in local coordinates on $M$ has the form $\{(c^1, \dots, c^r, x^{r+1}, x^{m}) :  x^{r+1}, x^{m} \in \R \} \cap \phi(U)$ such that 

But this is just a $k-$slice. So, around each point, we have a local k-slice. And we saw last time that a local $k$-slice is an embedded submanifold of $M$. Further, $F^{-1}(q)$ is closed in $M$ so it is a proper embedding.

\vskip 0.5cm
\begin{dottedbox}
  \emph{\textbf{Corollary:}} If $F: M \rightarrow N$ is a submersion, then $F^{-1}(q)$ is an $(m-n)$-dimensional submanifold.
\end{dottedbox}

\vskip 0.5cm
\subsection*{Regular Level sets}

\begin{mathdefinitionbox}{}
  Given a smooth map $F : M^m \rightarrow N^n$
  \begin{itemize}
    \item a point $p \in M$ is \emph{\textbf{regular}} if $dF_p$ is surjective at that point.
    \item Otherwse, the point $p$ is called a \emph{\textbf{critical point}}.
    \item A point $q \in N$ is a \emph{\textbf{regular value}} if all points in $F^{-1}(q)$ are regular.
    \item Otherwise, $q$ is called a \emph{\textbf{critical value}}.
  \end{itemize}
\end{mathdefinitionbox}

\vskip 0.5cm
\underline{Example:} For a smooth function $F : R \rightarrow R$, $x \in \R$ is a critcal point if and only if $f'(x) = 0$.

\vskip 0.5cm
\begin{center}
  \includegraphics*[scale=0.05]{critical points example.png}
\end{center}

\vskip 0.5cm
Just visually speaking, it looks like there are much fewer critical values than critical points. We'll say more about this in the next chapter when we deal with \emph{Sard's Theorem}.

\vskip 0.5cm
\begin{dottedbox}
  \underline{Remark (Proposition 4.1 in LeeSM):} 
  
  The set of regular points in $M$ is open, and the restriction \[ \restr{F}{\{\text{ {Reg. pts.} }\}} \{\text{Reg. pts.}\} \rightarrow N \] is a submersion.

  \vskip 0.25cm
  \emph{Why is this set open?}
  See Example 1.28 in LeeSM.

  \vskip 0.25cm
  \emph{Why is it a submersion?} Because at a regular point $p$, $dF_p$ is surjective i.e. $\mathrm{rank}(dF_p )= n$. This means the $m \times n$ matrix representation of $dF_p$ has an $n \times n$ submatrix with non-zero determinant. The determinant of this submatrix remains non-zero in a small neighborhood of $p$ due to continuity. 
\end{dottedbox}

\vskip 0.5cm
\begin{dottedbox}
  \emph{\textbf{Theorem (Regular Level Set Theorem:)}} If $F : M^m \rightarrow N^n$ is a smooth map between manifolds and $q \in N$ is a regular value, then its inverse image $F^{-1}(q) \subseteq M$ is a smooth $(m-n)-$dimensional submanifold of $M$.
\end{dottedbox}

\vskip 0.5cm
\emph{\textbf{Proof:}}  Let $U = \{p \in M \;:\; rank(dF_p) = dim N\} \subseteq M$. i.e. the set of points where $F$ has full rank. This is an open subset of $M$ by Proposition 4.1, and if $q \in N$ is a regular value then $F^{-1}(q) \subseteq U$.

\vskip 0.25cm
Then $\restr{F}{U}$ is a submersion and by the Constant Rank Theorem for Level Sets, $F^{-1}(q) \hookrightarrow U$ is a smooth embedding, and $F^{-1}(q)$ is an $(m-n)$ dimensional submanifold of $U$. Finally, $U \hookrightarrow M$ is also a smooth embedding, so the composition $\restr{F}{U} \hookrightarrow U \hookrightarrow M$ is also a smooth embedding. It follows that $F^{-1}(q)$ is an embedded submanifold of $M$.

\vskip 0.5cm
\underline{Examples:} 
\begin{itemize}
  \item Consider $F : \R^2 \rightarrow \R^2$ defined as 
  \[ (x, y) \mapsto x^2 - y^2 \]
  
  Then, 
  \[ dF_{(x, y)} = \begin{bmatrix}
    2x & -2y
  \end{bmatrix} \]
  
  which has rank $1$ unless $x = y = 0$. i.e. the set of regular values if $\R^2 \setminus \{0\} \subseteq_{open} \R^2$.
  
  So, $F^{-1}(c)$ is an embedded $(2-1) = 1$ dimensional submanifold of $\R^2$.

  \begin{center}
    \includegraphics*[scale=0.10]{submfd example.png}
  \end{center}

  \vskip 0.5cm
  \item $\mathbb{S}^n \subseteq \R^{n+1}$ can be thought of as a level set of 
  \[  F : \R^{n+1} \rightarrow \R^{1}, \;\;\; (x^1, \dots, x^{n+1}) \mapsto (x^1)^2 + \cdots + (x^{n+1})^2 \]

  So, $\mathbb{S}^n = F^{-1}(1)$. If we can check that $1 \in \R$ is a regular value of $F$ then our results above tell us that $\mathbb{S}^n = F^{-1}(1)$ is an embedded submanifold of $\R^2$.

  \vskip 0.5cm
  To do so, we calculate the differential of $F$ and make sure it is of constant rank at any pre-image of $1$.

  \vskip 0.5cm
  The differential of $F$ is 
  \[ df_{(x, y)} = \begin{bmatrix}
    2x^1 & \cdots & 2x^{n+1}
  \end{bmatrix}  \]
  and this map has constant rank except at $(0, \dots, 0) \not\in \mathbb{S}^n$. Thus, the Regular Level Set Theorem (RLST) tells us that 
  \[ \mathbb{S}^n \subseteq \mathbb{R}^{n+1} \] is an $(n+1-1) = n$ dimensional submanifold.

  \vskip 0.5cm
  \item (Converse not necessarily true) Consider the function $F : \R^2 \rightarrow \R$ which acts as 
  \[ (x, y) \mapsto \left( (x^1)^2  + (x^2)^2 - 1 \right)^2 \]

  Then, the differential is 
  \[ dF_{(x, y)} = \begin{bmatrix}
    4x^1\left( (x^1)^2 + (x^2)^2 - 1 \right) & 4x^2\left( (x^1)^2 + (x^2)^2 - 1 \right)
  \end{bmatrix} \]

  Then, all points of $\mathbb{S}^1$ are critical, but $\mathbb{S}^1$ is still an embedded submanifold of $\R^2$. So, the converse of RLST does not necessarily hold.

  \vskip 0.5cm
  \item Consider the "height function" on a torus $F : \mathbb{T}^2 \xrightarrow{\text{"height func."}} \R^2$
  \begin{center}
    \includegraphics*[scale=0.10]{torus_height.png}
  \end{center}

  The pre-images of the critical values of this function are the figure-eight shaped sets shown in the figure.
\end{itemize}

\vskip 0.5cm
Above, what we did was find the conditions where a level set of a function is a submanifold. Now, given a manifold, how do we \emph{define} a function such that the submanifold forms one of its level sets?

\vskip 1cm
\subsection{Defining Functions}


\begin{mathdefinitionbox}{}
  Given a submanifold $S^k \subseteq M^m$ of $M$, 
  \begin{itemize}
    \item a smooth map $F : M^m \rightarrow N^n$ such that $S = F^{-1}(q)$ for some regular value $q \in N$ is a \emph{\textbf{defining function for $S$}} 
    \item if $p \in S$ and $p \in U \subseteq_{open} M$, then a smooth map $F : U \rightarrow N$ is a \emph{\textbf{local defining function}} for $S$ at $p$ if $S \cap U = F^{-1}(q)$ for some regular value $q \in N$.
  \end{itemize}
\end{mathdefinitionbox}

\vskip 0.5cm
\begin{dottedbox}
  \underline{\emph{Theorem:}} Given a subset $S \subseteq M^m$, $S$ is a $k-$dimensional submanifold if and only if there exsit local defining functions $F : U \rightarrow \R^{m-k}$ at every point $\in S$.

  \vskip 0.5cm
  (Proof follows from k-slicing and constant rank level set theorem)
\end{dottedbox}

\vskip 1cm
\subsection{Tangent Space to a Submanifold}
[Write from image]

\vskip 0.5cm
In a local k-slice chart for $M$, $\left( U, \left( x^1, \dots, x^m \right) \right) $ we have 
\[ S \cap U = \{x^{+1} = c^{k+1}, \dots, x^m = c^{m}\} \]

[Write from image]

\vskip 0.5cm
If $F : U \subseteq M \rightarrow N$ is a local defining function for $S$, i.e. $U \cap S = F^{-1}(q)$ for a regular value $q$, then the extrnisic tangent space to $S$ is 
\[ T_{p}^{\text{extrnisic}} S = \ker dF_p \]

\vskip 1cm
\subsection*{Example: The group of orthogonal matrices}

\vskip 0.5cm
The orthogonal group is 
\[ O(n) = \{ A \in \R^{n \times n} \;:\; A^{T} A = \text{Id}_{n \times n} \} \subseteq \R^{n \times n} = \R^{{n^2}} \]

Defining the map $F : \R^{n \times n} \rightarrow \R^{n \times N}_{\text{symm.}} \left(\cong \R^{n(n+1)/2}\right)$ defined as 
\[ A \mapsto A^T A \]

\begin{dottedbox}
  \emph{Note:} $\left(A^T A\right)^T = A^T \left(A^T\right)^T = A^T A$
\end{dottedbox}

\vskip 0.5cm
Then, the orthogonal group is just the pre-image of the identity i.e. $O(n) = R^{-1}\left(\text{Id}_{n \times n}\right)$. We want to check that $\text{Id}_{n \times n}$ is a regular value of $F$.

\vskip 0.5cm
Let $A \in O(n)$, and compute the differential which maps tangent vectors as
\begin{align*}
  dF_{A} : T_A \R^{n \times n} &\rightarrow T_{F(A)} \R^{n \times n}_{\text{Symm.}} \\
  B &\mapsto \restr{\frac{d}{dt}}{t = 0} F(A + tB)
\end{align*}

Now, 
\begin{align*}
  \restr{\frac{d}{dt}}{t = 0} F(A + tB) &= \restr{\frac{d}{dt}}{t = 0} (A + tB)^T (A + tb) \\
  &= \restr{\frac{d}{dt}}{t = 0} A^T A + t\left( B^T A + A^T B \right) + t^2 B^T B \\
  &= B^T A + A^T B
\end{align*}

We need to check that $dF_A$ is surjective, which is equivalent to checking that the kernal satisfies the Rank Nullity Theorem.

\begin{align*}
  \dim \ker dF_A &=\dim \R^{n \times n} - \dim \R^{n \times n}_{\text{symm.}} \\
  &= n^2 - \frac{n(n+1)}{2} \\
  &= \frac{n(n-1)}{2}
\end{align*}

Notice that this is exactly the dimension of the space of $n \times n$ anti-symmetric matrices which maes sense as 

\begin{align*}
  \ker (dF_A) &= \{  B \in \R^{n \times n} \;:\; B^T A = -A^T B \} \\
  &= \{  B \in \R^{n \times n} \;:\; \left(A^T B\right)^T = -A^T B \} \\
  &= \{ B \in \R^{n \times n} \;:\; B = A C, \text{ for some antsymmetric }C \}
\end{align*}

\vskip 0.5cm
[Fill in from images]

\vskip 0.5cm
Also, a very often used tangent space is 
\[ T_{{\text{Id}_{n \times n}}} \]

$O(n)$ is a lie group and $\R^{n \times n}_{\text{antisymm.}}$ is its Lie Algebra.



\end{document}

