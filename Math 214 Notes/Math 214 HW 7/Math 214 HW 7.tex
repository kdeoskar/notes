\documentclass{article}

% Language setting
% Replace `english' with e.g. `spanish' to change the document language
\usepackage[english]{babel}

% Set page size and margins
% Replace `letterpaper' with`a4paper' for UK/EU standard size
\usepackage[letterpaper,top=2cm,bottom=2cm,left=3cm,right=3cm,marginparwidth=1.75cm]{geometry}

% Useful packages
\usepackage{amsmath}
\usepackage{amssymb}
\usepackage{mathtools}
\usepackage{graphicx}
\usepackage{enumitem}
\usepackage[colorlinks=true, allcolors=blue]{hyperref}

\usepackage{hyperref}
\hypersetup{
    colorlinks=true,
    linkcolor=blue,
    filecolor=magenta,      
    urlcolor=cyan,
    pdftitle={Math 214 HW 7},
    pdfpagemode=FullScreen,
    }

\urlstyle{same}

\usepackage{tikz-cd}

%%%%%%%%%%% Box pacakges and definitions %%%%%%%%%%%%%%
\usepackage[most]{tcolorbox}
\usepackage{xcolor}

% Define the colors
\definecolor{boxheader}{RGB}{0, 51, 102}  % Dark blue
\definecolor{boxfill}{RGB}{173, 216, 230}  % Light blue

% Define the tcolorbox environment
\newtcolorbox{mathdefinitionbox}[2][]{%
    colback=boxfill,   % Background color
    colframe=boxheader, % Border color
    fonttitle=\bfseries, % Bold title
    coltitle=white,     % Title text color
    title={#2},         % Title text
    enhanced,           % Enable advanced features
    attach boxed title to top left={yshift=-\tcboxedtitleheight/2}, % Center title
    boxrule=0.5mm,      % Border width
    sharp corners,      % Sharp corners for the box
    #1                  % Additional options
}
%%%%%%%%%%%%%%%%%%%%%%%%%

\newtcolorbox{dottedbox}[1][]{%
    colback=white,    % Background color
    colframe=white,    % Border color (to be overridden by dashrule)
    sharp corners,     % Sharp corners for the box
    boxrule=0pt,       % No actual border, as it will be drawn with dashrule
    boxsep=5pt,        % Padding inside the box
    enhanced,          % Enable advanced features
    overlay={\draw[dashed, thin, black, dash pattern=on \pgflinewidth off \pgflinewidth, line cap=rect] (frame.south west) rectangle (frame.north east);}, % Dotted line
    #1                 % Additional options
}

\usepackage{biblatex}
\addbibresource{sample.bib}


%%%%%%%%%%% New Commands %%%%%%%%%%%%%%
\newcommand*{\T}{\mathcal T}
\newcommand*{\cl}{\text cl}
\newcommand{\bP}{\mathbb{P}}
\newcommand{\bS}{\mathbb{S}}


\newcommand{\ket}[1]{|#1 \rangle}
\newcommand{\bra}[1]{\langle #1|}
\newcommand{\inner}[2]{\langle #1 | #2 \rangle}
\newcommand{\R}{\mathbb{R}}
\newcommand{\C}{\mathbb{C}}
\newcommand{\A}{\mathbb{A}}
\newcommand{\sphere}{\mathbb{S}}
\newcommand{\V}{\mathbb{V}}
\newcommand{\Hilbert}{\mathcal{H}}
\newcommand{\oper}{\hat{\Omega}}
\newcommand{\lam}{\hat{\Lambda}}
\newcommand{\defeq}{\vcentcolon=}

\newcommand{\bigslant}[2]{{\raisebox{.2em}{$#1$}\left/\raisebox{-.2em}{$#2$}\right.}}
\newcommand{\restr}[2]{{% we make the whole thing an ordinary symbol
  \left.\kern-\nulldelimiterspace % automatically resize the bar with \right
  #1 % the function
  \vphantom{\big|} % pretend it's a little taller at normal size
  \right|_{#2} % this is the delimiter
  }}
%%%%%%%%%%%%%%%%%%%%%%%%%%%%%%%%%%%%%%%


\tcbset{theostyle/.style={
    enhanced,
    sharp corners,
    attach boxed title to top left={
      xshift=-1mm,
      yshift=-4mm,
      yshifttext=-1mm
    },
    top=1.5ex,
    colback=white,
    colframe=blue!75!black,
    fonttitle=\bfseries,
    boxed title style={
      sharp corners,
    size=small,
    colback=blue!75!black,
    colframe=blue!75!black,
  } 
}}

\newtcbtheorem[number within=section]{Theorem}{Theorem}{%
  theostyle
}{thm}

\newtcbtheorem[number within=section]{Definition}{Definition}{%
  theostyle
}{def}



\title{Math 214 Homework 7}
\author{Keshav Balwant Deoskar}

\begin{document}
\maketitle


%%%%%%%%%%%%%%%%%%%%%%%%%%%%%%%%%%%%%%%%%%%%%%%%%%%%%%%%%%%%%%%%%
\textbf{Q7-2.} Let $G$ be a Lie Group.
\begin{enumerate}[label=(\alph*)]
  \item Let $m : G \times G \rightarrow G$ denote the multiplication map. Using proposition 3.14 to identify $T_{(e,e)} \left(G \times G\right)$ with $T_{e}G \oplus T_e G$, show that the differential $dm_{(e,e)} : T_e G \oplus T_e G \rightarrow T_e G$ is given by 
  \[ dm_{(e,e)}\left(X, Y\right) = X + Y \]

  \item Let $i : G \rightarrow G$ denote the invversion map. Show that $di_e : T_e G \rightarrow T_e G$ is given by $di_e(X) = -X $.
\end{enumerate}
%%%%%%%%%%%%%%%%%%%%%%%%%%%%%%%%%%%%%%%%%%%%%%%%%%%%%%%%%%%%%%%%%

\vskip 0.5cm
\textbf{Proof:}

\begin{enumerate}[label=(\alph*)]
  \item 
\end{enumerate}

\vskip 0.5cm
\hrule 
\vskip 0.5cm


%%%%%%%%%%%%%%%%%%%%%%%%%%%%%%%%%%%%%%%%%%%%%%%%%%%%%%%%%%%%%%%%%
\textbf{Q7-4.} Let $\mathrm{det} : GL(n, \R) \rightarrow \R$ denote the determinant function. Use Corollary 3.25 to compute the differential of det, as follows.
\begin{enumerate}[label=(\alph*)]
  \item For any $A \in M(n, \R)$, show that 
  \[ \restr{\frac{d}{dt}}{t = 0} \mathrm{det}(I_n + tA) = \mathrm{tr}A \], where $\mathrm{tr}\left(A^i_j\right) = \sum_{i} A^i_i$ is the trace of $A$.
  
  \item For $X \in GL(n, \R)$ and $B \in T_X GL(n, \R) \cong M(n,\R)$, show that 
  \[ d\left(\mathrm{det}\right)_X (B) = \left( \mathrm{det} X \right) \mathrm{tr} \left(X^{-1} B\right) \].
\end{enumerate}
%%%%%%%%%%%%%%%%%%%%%%%%%%%%%%%%%%%%%%%%%%%%%%%%%%%%%%%%%%%%%%%%%

\vskip 0.5cm
\textbf{Proof:}


\vskip 0.5cm
\hrule 
\vskip 0.5cm


%%%%%%%%%%%%%%%%%%%%%%%%%%%%%%%%%%%%%%%%%%%%%%%%%%%%%%%%%%%%%%%%%
\textbf{Q7-6.} Suppose $G$ is a Lie Group and $U$ is any neighborhood of the identity. Show that there exists a neighobrhood $V$ of the identity such that $V \subseteq U$ and $g h^{-1} \in U$ whenever $g, h \in V$.
%%%%%%%%%%%%%%%%%%%%%%%%%%%%%%%%%%%%%%%%%%%%%%%%%%%%%%%%%%%%%%%%%

\vskip 0.5cm
\textbf{Proof:}


\vskip 0.5cm
\hrule 
\vskip 0.5cm


%%%%%%%%%%%%%%%%%%%%%%%%%%%%%%%%%%%%%%%%%%%%%%%%%%%%%%%%%%%%%%%%%
\textbf{Q7-11.} Repeat Problem 7-9 for $GL(n+1, \C)$ and $\mathbb{CP}^n$.
%%%%%%%%%%%%%%%%%%%%%%%%%%%%%%%%%%%%%%%%%%%%%%%%%%%%%%%%%%%%%%%%%

\vskip 0.5cm
\textbf{Proof:}


\vskip 0.5cm
\hrule 
\vskip 0.5cm


%%%%%%%%%%%%%%%%%%%%%%%%%%%%%%%%%%%%%%%%%%%%%%%%%%%%%%%%%%%%%%%%%
\textbf{Q7-22.} 
% Let $\mathbb{H} = \C \times \C$ and define a bilinear product $\mathbb{H} \times \mathbb{H} \rightarrow \mathbb{H}$ by 
% \[ (a,b)(c,d) = (ac - d\overline{b}, \overline{a}d + cb) \]
% for $a,b,c,d \in \C$.
\begin{enumerate}[label=(\alph*)]
  \item Show that quaternionic multiplication is associative but not commutative.
  \item Show that $(pq)^* = q^* p^*$ for all $p, \in \mathbb{H}$
  \item Show that $\langle p, q \rangle = \frac{1}{2}\left(p^* q + q^* p\right)$ is an iner product on $\mathbb{H}$, whose associated norm satisfies $|pq| = |p||q|$.
  \item Show that every nonzero quaternion has a two-sided multiplicative inverse given by $p^{-1} = |p|^{-2} p^*$.
  \item Show that the se t$\mathbb{H}^*$ of nonzero quaternions is a Lie group under quaternionic multiplication.
\end{enumerate}

%%%%%%%%%%%%%%%%%%%%%%%%%%%%%%%%%%%%%%%%%%%%%%%%%%%%%%%%%%%%%%%%%

\vskip 0.5cm
\textbf{Proof:}


\vskip 0.5cm
\hrule 
\vskip 0.5cm


%%%%%%%%%%%%%%%%%%%%%%%%%%%%%%%%%%%%%%%%%%%%%%%%%%%%%%%%%%%%%%%%%
\textbf{Q7-23.} Let $\mathbb{H}^*$ be the Lie Group of nonzero quaternions and let $\mathcal{S} \subseteq \mathbb{H}^*$ be the set of unit quaternions. Show that $\mathcal{S}$ is a properly embedded Lie subgroup of $\mathbb{H}^*$, isomorphic to $SU(2)$. 
%%%%%%%%%%%%%%%%%%%%%%%%%%%%%%%%%%%%%%%%%%%%%%%%%%%%%%%%%%%%%%%%%

\vskip 0.5cm
\textbf{Proof:}


\vskip 0.5cm
\hrule 
\vskip 0.5cm






% %%%%%%%%%%%%%%%%%%%%%%%%%%%%%%%%%%%%%%%%%%%%%%%%%%%%%%%%%%%%%%%%%
% \textbf{Q7-.} 
% %%%%%%%%%%%%%%%%%%%%%%%%%%%%%%%%%%%%%%%%%%%%%%%%%%%%%%%%%%%%%%%%%

% \vskip 0.5cm
% \textbf{Proof:}


% \vskip 0.5cm
% \hrule 
% \vskip 0.5cm


\end{document}
