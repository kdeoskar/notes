\documentclass{article}

% Language setting
% Replace `english' with e.g. `spanish' to change the document language
\usepackage[english]{babel}

% Set page size and margins
% Replace `letterpaper' with`a4paper' for UK/EU standard size
\usepackage[letterpaper,top=2cm,bottom=2cm,left=3cm,right=3cm,marginparwidth=1.75cm]{geometry}

% Useful packages
\usepackage{amsmath}
\usepackage{amssymb}
\usepackage{mathtools}
\usepackage{graphicx}
\usepackage{enumitem}
\usepackage[colorlinks=true, allcolors=blue]{hyperref}

\usepackage{hyperref}
\hypersetup{
    colorlinks=true,
    linkcolor=blue,
    filecolor=magenta,      
    urlcolor=cyan,
    pdftitle={Math 214 HW 5},
    pdfpagemode=FullScreen,
    }

\urlstyle{same}

\usepackage{tikz-cd}

%%%%%%%%%%% Box pacakges and definitions %%%%%%%%%%%%%%
\usepackage[most]{tcolorbox}
\usepackage{xcolor}

% Define the colors
\definecolor{boxheader}{RGB}{0, 51, 102}  % Dark blue
\definecolor{boxfill}{RGB}{173, 216, 230}  % Light blue

% Define the tcolorbox environment
\newtcolorbox{mathdefinitionbox}[2][]{%
    colback=boxfill,   % Background color
    colframe=boxheader, % Border color
    fonttitle=\bfseries, % Bold title
    coltitle=white,     % Title text color
    title={#2},         % Title text
    enhanced,           % Enable advanced features
    attach boxed title to top left={yshift=-\tcboxedtitleheight/2}, % Center title
    boxrule=0.5mm,      % Border width
    sharp corners,      % Sharp corners for the box
    #1                  % Additional options
}
%%%%%%%%%%%%%%%%%%%%%%%%%

\newtcolorbox{dottedbox}[1][]{%
    colback=white,    % Background color
    colframe=white,    % Border color (to be overridden by dashrule)
    sharp corners,     % Sharp corners for the box
    boxrule=0pt,       % No actual border, as it will be drawn with dashrule
    boxsep=5pt,        % Padding inside the box
    enhanced,          % Enable advanced features
    overlay={\draw[dashed, thin, black, dash pattern=on \pgflinewidth off \pgflinewidth, line cap=rect] (frame.south west) rectangle (frame.north east);}, % Dotted line
    #1                 % Additional options
}

\usepackage{biblatex}
\addbibresource{sample.bib}


%%%%%%%%%%% New Commands %%%%%%%%%%%%%%
\newcommand*{\T}{\mathcal T}
\newcommand*{\cl}{\text cl}
\newcommand{\bP}{\mathbb{P}}
\newcommand{\bS}{\mathbb{S}}


\newcommand{\ket}[1]{|#1 \rangle}
\newcommand{\bra}[1]{\langle #1|}
\newcommand{\inner}[2]{\langle #1 | #2 \rangle}
\newcommand{\R}{\mathbb{R}}
\newcommand{\C}{\mathbb{C}}
\newcommand{\A}{\mathbb{A}}
\newcommand{\sphere}{\mathbb{S}}
\newcommand{\V}{\mathbb{V}}
\newcommand{\Hilbert}{\mathcal{H}}
\newcommand{\oper}{\hat{\Omega}}
\newcommand{\lam}{\hat{\Lambda}}
\newcommand{\defeq}{\vcentcolon=}

\newcommand{\bigslant}[2]{{\raisebox{.2em}{$#1$}\left/\raisebox{-.2em}{$#2$}\right.}}
\newcommand{\restr}[2]{{% we make the whole thing an ordinary symbol
  \left.\kern-\nulldelimiterspace % automatically resize the bar with \right
  #1 % the function
  \vphantom{\big|} % pretend it's a little taller at normal size
  \right|_{#2} % this is the delimiter
  }}
%%%%%%%%%%%%%%%%%%%%%%%%%%%%%%%%%%%%%%%


\tcbset{theostyle/.style={
    enhanced,
    sharp corners,
    attach boxed title to top left={
      xshift=-1mm,
      yshift=-4mm,
      yshifttext=-1mm
    },
    top=1.5ex,
    colback=white,
    colframe=blue!75!black,
    fonttitle=\bfseries,
    boxed title style={
      sharp corners,
    size=small,
    colback=blue!75!black,
    colframe=blue!75!black,
  } 
}}

\newtcbtheorem[number within=section]{Theorem}{Theorem}{%
  theostyle
}{thm}

\newtcbtheorem[number within=section]{Definition}{Definition}{%
  theostyle
}{def}



\title{Math 214 Homework 5}
\author{Keshav Balwant Deoskar}

\begin{document}
\maketitle



%%%%%%%%%%%%%%%%%%%%%%%%%%%%%%%%%%%%%%%%%%%%%%%%%%%%%%%%%%%%%%%%%
\textbf{Q4-5.} Let $\mathbb{CP}^n$ denote the $n-$dimensional complex projective space. 
\begin{enumerate}[label=(\alph*)]
  \item Show that the quotient map $\pi : \C^{n+1} \setminus \{0\} \rightarrow \mathbb{CP}^n$ is a surjective smooth submersion.
  \item Show that $\mathbb{CP}^n$ is diffeomorphic to $\mathbb{S}^n$.
\end{enumerate}
%%%%%%%%%%%%%%%%%%%%%%%%%%%%%%%%%%%%%%%%%%%%%%%%%%%%%%%%%%%%%%%%%

\vskip 0.5cm
\textbf{Proof:}

\begin{enumerate}[label=(\alph*)]
  \item To show that $\pi$ is smooth, let's write it in terms of coordinates. Let $\tilde{U}_k = \{ \left(z^1, \dots, z^{n+1}\right) \in \C^{n+1} \setminus \{0\}, z^k \neq 0 \}$ and let $U_k = \pi(\tilde{U}_k)$. Let $z = (z^1, \dots, z^{n+1}) \in \tilde{U}_k$. 
  
  \vskip 0.25cm
  Then, $\{\left(U_k, \mathrm{id}\right)\}_{k = 1, \dots, n}$ forms a collection of smooth chart which covers $\mathbb{CP}^n$ and $\{ \left( \tilde{U}_k, \phi_k \right) \}_{k=1,\dots,n+1}$ form an open cover of $\C^{n+1} \setminus \{0\}$ where $\phi_i : U_i \rightarrow \C^{n} \cong \R^{2n}$ is defined as the map
  \[ [z^1:\cdots:z^{n+1}] \mapsto \left( \frac{z^{1}}{z^{i}}, \cdots, \frac{z^{i-1}}{z^{i}}, \frac{z^{i+1}}{z^{i}}, \cdots \frac{z^{n+1}}{z^{i}}\right)  \]

  Then, the coordinate representation of $\pi$ on each of these charts is given by 
  \begin{align*}
    \restr{\left(\phi_{k} \circ \pi \circ \mathrm{id}\right)}{\mathrm{id}\left(U_k \cap \pi^{-1}(\tilde{U}_k)\right)}(z^1, \cdots, z^{n+1}) &= \phi_{k} \circ \pi(z^1, \cdots, z^{n+1}) \\
    &= \phi_k \left( [z^1:\cdots:z^{n+1}] \right) \\
    &=\left( \frac{z^{1}}{z^{k}}, \cdots, \frac{z^{k-1}}{z^{k}}, \frac{z^{k+1}}{z^{k}}, \cdots \frac{z^{n+1}}{z^{k}}\right)
  \end{align*}
  
  which is smooth since $z^k \neq 0$ on the domain. From the above, $\pi$ is smooth. Also, quotient maps are surjective by definition. Now, let's show $\pi$ is a submersion. Let's denote $z^j =  z^j + i y^j$. Then, 
  \[ \frac{z^j}{z^k} = \frac{x^j x^k + y^j y^k}{(x^k)^2 + (y^k)^2} + i \frac{x^k y^j - x^j y^k}{(x^k)^2 + (y^k)^2}  \]

  In the coordinates of $U_i$, the differential $d\pi$ can be represented with the matrix 
  \[ \begin{bmatrix}
    D & 0 & \cdots & 0 \\
    0 & D & \cdots & 0 \\
    \vdots & \vdots & \ddots & 0 \\
    0 & 0 & \cdots & D
  \end{bmatrix} \]

  where 
  \[ D = \begin{pmatrix}
    \frac{x^k}{(x^k)^2 + (y^k)^2} & \frac{-y^k}{(x^k)^2 + (y^k)^2} \\
    \frac{y^k}{(x^k)^2 + (y^k)^2} & \frac{x^k}{(x^k)^2 + (y^k)^2}
  \end{pmatrix} \]

  Now, $det(D) = 1$ so $det(A) = 1$. This tells us that $d\pi$ has full rank and therefore $\pi$ is a submersion. This proves part (a).

  \vskip 0.5cm
  \item To show $\mathbb{CP}^{1} \cong_{diff} \mathbb{S}^2$, we define the following map $F : \mathbb{S}^2 \rightarrow \mathbb{CP}^{1}$
  \[  F(x, y, z) = \begin{cases}
    [ 1, \frac{x}{1-z} + i\frac{y}{1-z}], \text{ if }(x, y, z) \in \mathbb{S}^2 \setminus \{N\} \\
    [\frac{x}{1+z} - i\frac{y}{1+z}, 1], \text{ if }(x, y, z) \in \mathbb{S}^2 \setminus \{S\} \\
  \end{cases}  \]

  \vskip 0.5cm
  Now, we note that $\restr{F}{\mathbb{S}^2 \setminus \{N\}} = \phi_2^{-1} \circ i \circ \sigma$ where $i$ is the identification of $\C^1 \cong \R^2$, and $\sigma$ is the stereographic projection from the north. Each of these is a diffeomorphism, thus so is $\restr{F}{\mathbb{S}^2 \setminus \{N\}}$ (on its image). 
  
  \vskip 0.25cm
  Similarly, $\restr{F}{\mathbb{S}^2 \setminus \{S\}}= \phi_1^{-1} \circ \tau \circ i \circ \tilde{\sigma}$ where $\tilde{\sigma}$ is the stereographic projection from the south and $\tau$ is complex conjugation. These are all diffeomorphisms, so $\restr{F}{\mathbb{S}^2 \setminus \{S\}}$ is a diffeomorphism onto its image.  

  \vskip 0.25cm
  Finally, we note that $U_1 \cup U_2$ cover $\mathbb{CP}^1$. So,
  % $\left( \mathbb{S}^2 \setminus \{N\},  \restr{F}{\mathbb{S}^2 \setminus \{N\}} \right)$, $\left( \mathbb{S}^2 \setminus \{S\}, \restr{F}{\mathbb{S}^2 \setminus \{S\}} \right)$
  $\mathbb{CP}^1 \cong_{diff} \mathbb{S}^1$. 
\end{enumerate}


\vskip 0.5cm
\hrule 
\vskip 0.5cm



%%%%%%%%%%%%%%%%%%%%%%%%%%%%%%%%%%%%%%%%%%%%%%%%%%%%%%%%%%%%%%%%%
\textbf{Q4-6.} Let $M$ be a nonempty smooth compact manifold. Show that there is no smooth submersion $F : M \rightarrow \R^{k}$ for any $k > 0$. 
%%%%%%%%%%%%%%%%%%%%%%%%%%%%%%%%%%%%%%%%%%%%%%%%%%%%%%%%%%%%%%%%%

\vskip 0.5cm
\textbf{Proof:}
From LeeSM Proposition 4.28, We know that if $\pi : M \rightarrow N$ is a smooth submersion between smooth manifolds then $\pi$ is an open map. Now, consider $M$ to be a non-empty smooth compact manifold and let $N = \R^k$. $M \subseteq M$ is open when viewed as a subset of itself. However, $F(M)$ is a compact subset of $\R^k$ since $F$ is a smooth map, and compact subsets of euclidean space are not open. Thus, we have a contradiction.

\vskip 0.5cm
\hrule 
\vskip 0.5cm



%%%%%%%%%%%%%%%%%%%%%%%%%%%%%%%%%%%%%%%%%%%%%%%%%%%%%%%%%%%%%%%%%
\textbf{Q4-7.} Suppose $M$ and $N$ are smooth manifolds, and $\pi : M \rightarrow N$ is an injective smooth submersion. Show that there is no other smooth manifold structure on $N$ that satisfies the conclusion of Theorem 4.29. 
%%%%%%%%%%%%%%%%%%%%%%%%%%%%%%%%%%%%%%%%%%%%%%%%%%%%%%%%%%%%%%%%%

\vskip 0.5cm
\textbf{Proof:}

From Theorem 4.28, we know that surjective smooth submersions are quotient maps. Then, from the uniqueness of the quotient topology, we know there is no other smooth manifold structure on $N$ such that the conclusion of Theorem 4.29 holds.

\vskip 0.5cm
\hrule 
\vskip 0.5cm


%%%%%%%%%%%%%%%%%%%%%%%%%%%%%%%%%%%%%%%%%%%%%%%%%%%%%%%%%%%%%%%%%
\textbf{Q4-8.} Let $\pi : \R^2 \rightarrow \R$ be defined by $\pi(x, y) = xy$. Show that $\pi$ is surjective and smooth, and that for each smooth manifold $P$, a map $F : \R \rightarrow P$ is smooth if and only if $F \circ \pi$ is smooth; but $\pi$ is not a smooth submersion.
%%%%%%%%%%%%%%%%%%%%%%%%%%%%%%%%%%%%%%%%%%%%%%%%%%%%%%%%%%%%%%%%%

\vskip 0.5cm
\textbf{Proof:}


For any $t \in \R$, we can simply choose $x = t, y = 1$. Then, $\pi(x, y) = \pi(t, 1) = t$, so the map is surjective. The map is also smooth since the partial derivatives with respect to $x^1, x^2 = x, y$ are smooth 
\[ \frac{\partial f}{\partial x} = y \;\;\;\;\;\;\; \frac{\partial f}{\partial y} = x \] 

However, $\pi$ is not a smooth submersion since the differential of $\pi$ 
\[ d\pi_{(0,0)} = \restr{\begin{pmatrix}
  x \\
  y
\end{pmatrix}}{(0,0)} = \mathbf{0} \]

has rank zero at the origin, whereas it has rank 1 everywhere else on $\R^2$. So, $\pi$ is not a constant rank map.

\vskip 0.5cm
\hrule 
\vskip 0.5cm




%%%%%%%%%%%%%%%%%%%%%%%%%%%%%%%%%%%%%%%%%%%%%%%%%%%%%%%%%%%%%%%%%
\textbf{Q4-9.} Let $M$ be a connected smooth manifold, and let $\pi : E \rightarrow M$ be a topological covering map. Complete the proof of proposition 4.40 by showing that there is only one smooth structure on $E$ such that $\pi$ is a smooth covering map.  
%%%%%%%%%%%%%%%%%%%%%%%%%%%%%%%%%%%%%%%%%%%%%%%%%%%%%%%%%%%%%%%%%

\vskip 0.5cm
\textbf{Proof:}

\begin{dottedbox}
  \emph{\textbf{Theorem 4.40:}} Suppose $M$ is a connected smooth $n-$manifold and $\pi : E \rightarrow M$ is a \emph{topological} covering map. Then $E$ is a topological $(n-1)$ manifold and there exsits a unique smooth structure on $E$ such that $\pi$ is a smooth covering map.
\end{dottedbox}

\vskip 0.5cm
The book proves that $E$ is a topological $(n-1)$ manifold and that there exists a smooth structure on it such that $\pi$ is a smooth covering map. Now, let's suppose $\tilde{E}$ is the same set but with a different smooth strucuture on it, such that $\tilde{\pi} : \tilde{E} \rightarrow M$ is smooth. To show that the two smooth structures on $E$ must be the same, let's prove that $\mathrm{id} : E \rightarrow \tilde{E}$ is a diffeomorphism.

\[\begin{tikzcd}
	E && {\tilde{E}} \\
	& M
	\arrow["{\mathrm{id}}", shift left, from=1-1, to=1-3]
	\arrow["{\tilde{\pi}}", from=1-3, to=2-2]
	\arrow["\pi"', from=1-1, to=2-2]
\end{tikzcd}\]

% \vskip 0.5cm
% Every point in $E$ is in the pre-image of some evenly covered $V \subseteq S$. Let $U$ be the component of $\pi^{-1}(V)$ which contains $p$. So, $\pi(U) \subseteq M$ is also evenly covered because $\pi$ maps $U$ onto it diffeomorphically.

% \vskip 0.5cm
% Now, from LeeSM Exercise 4.39, we know that a set being even covered in the topological sense is the same as it being covered in the smooth sense. i.e. if we know that $\pi$ is a smooth covering map and we show that $\pi$ maps a subset $A \subseteq E$ homeomorphically onto its image in $M$, then we gain \emph{for free} the knowledge that $\pi$ maps $A$ onto its image \emph{diffeomorphically}.

\vskip 0.5cm
Every point $x \in E$ lies in the pre-image (under $\pi$) of some evenly covered subset $V \subseteq M$ which is the domain of a chart $\phi : V \rightarrow \R^m$. 

\vskip 0.25cm
Then, let $U$ be an open neighborhood of $x$ on which $\pi$ restricts to a homeomorphism from $U$ to $V$
\[ \restr{\pi}{U} : U \rightarrow V  \]

\vskip 0.25cm
Then, it follows that $\phi \circ \left(\restr{\pi}{U}\right) : U \rightarrow \R^m$ is a smooth map with respect to both atlases $\mathcal{T}_1, \mathcal{T}_2$ on $E, \tilde{E}$. Doing this for all points $x \in E$, we have a cover of $E$ in both atlases. Thus, the two atlases are the same.

\vskip 0.5cm


\vskip 0.5cm
\hrule 
\vskip 0.5cm




%%%%%%%%%%%%%%%%%%%%%%%%%%%%%%%%%%%%%%%%%%%%%%%%%%%%%%%%%%%%%%%%%
\textbf{Q5-4.} Show that the image of the curve $\beta : (-\pi, \pi) \rightarrow \R^2$ of Example 4.19 is not an embedded submanifold of $\R^2$. 
%%%%%%%%%%%%%%%%%%%%%%%%%%%%%%%%%%%%%%%%%%%%%%%%%%%%%%%%%%%%%%%%%

\vskip 0.5cm
\textbf{Proof:}


If we denote the image of the curve as $S$ and let $U$ be a small open neighborhood in $\R^2$ centered around the origin, then $S \cap U$ is open in $S$ with the subspace topology. However, for small enough $S$, the set $\left(S \cap U\right) \setminus \{\mathbf{0}\}$ (open set in $S$ with just the origin deleted, so still open in $S$) has four connected components whereas any open ball in $\R^n$ after deleting a point has either two connected components (in the $n = 1$ case) or one connected component ($n \neq 1$). Thus, it is impossible for this open subset of $S$ to be homeomorphic to any open set in $\R^n$. So, the image of the curve cannot be an embedded submanifold of $\R^2$.

\vskip 0.5cm
\hrule 
\vskip 0.5cm




%%%%%%%%%%%%%%%%%%%%%%%%%%%%%%%%%%%%%%%%%%%%%%%%%%%%%%%%%%%%%%%%%
\textbf{Q5-6.} Suppose $M \subseteq \R^n$ is an embedded $m-$dimensional submanifold, and let $UM \subseteq T \R^n$ be the set of all \emph{unit} tangent vectors to $M$:
\[ UM = \{ (x, v) \in T \R^n \; : \; x\in M, v \in T_{x} M, \; \left| v \right| = 1 \} \]

This is called the \emph{\textbf{Unit Tangent Bundle of $M$.}} Prove that $UM$ is an embedded  $(2n-1)-$dimensional submanifold of $T \R^n \approx \R^n \times \R^{n}$.
%%%%%%%%%%%%%%%%%%%%%%%%%%%%%%%%%%%%%%%%%%%%%%%%%%%%%%%%%%%%%%%%%

\vskip 0.5cm
\textbf{Proof:}

Consider the map $\Phi : TM \rightarrow \R$ defined, for $x \in M$ and $v \in T_x M$, as 
\[ (x, v) \mapsto |v|^2 = (v^1)^2 + \cdots + (v^n)^2 \]

Then, $UM = \Phi^{-1}(1)$ and $\Phi$ is a smooth map of constant rank $( = 1)$. The differential of $\Phi$ is never singular because $v \neq 0$ so $\dim \ker \Phi = 0$ and so the Rank Nullity Theorem tells us $\dim \R = \dim \mathrm{Im}\Phi = 1$. 

Then $UM$ forms a regular level set of $\Phi$, so by Corollary 5.14 in LeeSM, it is an embedded submanifold whose codimension is equal to $1$. Thus, its dimension is $\dim T\R^n - 1 = 2n - 1$.

\vskip 0.5cm
\hrule 
\vskip 0.5cm




%%%%%%%%%%%%%%%%%%%%%%%%%%%%%%%%%%%%%%%%%%%%%%%%%%%%%%%%%%%%%%%%%
\textbf{Q5-7.} Let $F : \R^2 \rightarrow \R$ be defined as $F(x,y) = x^3 + xy + y^3$. Which level sets of $F$ are embedded submanifolds of $\R^2$? For each level set, prove either that it is or that it is not an embedded submanifold. 
%%%%%%%%%%%%%%%%%%%%%%%%%%%%%%%%%%%%%%%%%%%%%%%%%%%%%%%%%%%%%%%%%

\vskip 0.5cm
\textbf{Proof:}

The differential of $F(x,y)$ is given by 
\[ dF_{(x,y)} = \begin{bmatrix}
  3x^2 + y & x + 3y^2
\end{bmatrix} \]

This differential is non-singular at all points in $\R^2$ other than $(x,y) = (0,0)$ and $(x,y) = \left(-\frac{1}{3}, -\frac{1}{3}\right)$.

% \vskip 0.5cm
% We use the following theorem:
% \begin{dottedbox}
%   \emph{\textbf{Regular Level Set Theorem (LeeSM Thm 5.14):}} Every regular level set of a smooth map between smooth manifolds is a properly embedded submanifold whose codimension is equal to the dimension of the codimension.
% \end{dottedbox} 

\vskip 0.5cm
\begin{itemize}
  \item $F(0, 0) = 0$
  \item $F\left(-\frac{1}{3}, -\frac{1}{3}\right) = \frac{1}{27}$
\end{itemize}

\vskip 0.5cm
So, for any $c \in \R \setminus \{0, \frac{1}{27}\}$, the level set $F^{-1}(c)$ is an embedded submanifold by the Regular Level Set Theorem.

\vskip 0.5cm
Now, when it comes to $F^{-1}(0)$, the level set is the singleton $\{(0,0)\} \subseteq \R^2$. This is a $0-$dimensional submanifold because the inclusion of a point into $\R^2$ is smooth. The same argument holds for the level set $F^{-1}(1/27) = \{\left(-1/3, -1/3\right)\}$



\vskip 0.5cm
\hrule 
\vskip 0.5cm






% %%%%%%%%%%%%%%%%%%%%%%%%%%%%%%%%%%%%%%%%%%%%%%%%%%%%%%%%%%%%%%%%%
% \textbf{Q5-.} 
% %%%%%%%%%%%%%%%%%%%%%%%%%%%%%%%%%%%%%%%%%%%%%%%%%%%%%%%%%%%%%%%%%

% \vskip 0.5cm
% \textbf{Proof:}


% \vskip 0.5cm
% \hrule 
% \vskip 0.5cm


\end{document}
