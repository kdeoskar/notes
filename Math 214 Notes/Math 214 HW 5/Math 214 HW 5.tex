\documentclass{article}

% Language setting
% Replace `english' with e.g. `spanish' to change the document language
\usepackage[english]{babel}

% Set page size and margins
% Replace `letterpaper' with`a4paper' for UK/EU standard size
\usepackage[letterpaper,top=2cm,bottom=2cm,left=3cm,right=3cm,marginparwidth=1.75cm]{geometry}

% Useful packages
\usepackage{amsmath}
\usepackage{amssymb}
\usepackage{mathtools}
\usepackage{graphicx}
\usepackage{enumitem}
\usepackage[colorlinks=true, allcolors=blue]{hyperref}

\usepackage{hyperref}
\hypersetup{
    colorlinks=true,
    linkcolor=blue,
    filecolor=magenta,      
    urlcolor=cyan,
    pdftitle={Math 214 HW 5},
    pdfpagemode=FullScreen,
    }

\urlstyle{same}

\usepackage{tikz-cd}

%%%%%%%%%%% Box pacakges and definitions %%%%%%%%%%%%%%
\usepackage[most]{tcolorbox}
\usepackage{xcolor}

% Define the colors
\definecolor{boxheader}{RGB}{0, 51, 102}  % Dark blue
\definecolor{boxfill}{RGB}{173, 216, 230}  % Light blue

% Define the tcolorbox environment
\newtcolorbox{mathdefinitionbox}[2][]{%
    colback=boxfill,   % Background color
    colframe=boxheader, % Border color
    fonttitle=\bfseries, % Bold title
    coltitle=white,     % Title text color
    title={#2},         % Title text
    enhanced,           % Enable advanced features
    attach boxed title to top left={yshift=-\tcboxedtitleheight/2}, % Center title
    boxrule=0.5mm,      % Border width
    sharp corners,      % Sharp corners for the box
    #1                  % Additional options
}
%%%%%%%%%%%%%%%%%%%%%%%%%

\newtcolorbox{dottedbox}[1][]{%
    colback=white,    % Background color
    colframe=white,    % Border color (to be overridden by dashrule)
    sharp corners,     % Sharp corners for the box
    boxrule=0pt,       % No actual border, as it will be drawn with dashrule
    boxsep=5pt,        % Padding inside the box
    enhanced,          % Enable advanced features
    overlay={\draw[dashed, thin, black, dash pattern=on \pgflinewidth off \pgflinewidth, line cap=rect] (frame.south west) rectangle (frame.north east);}, % Dotted line
    #1                 % Additional options
}

\usepackage{biblatex}
\addbibresource{sample.bib}


%%%%%%%%%%% New Commands %%%%%%%%%%%%%%
\newcommand*{\T}{\mathcal T}
\newcommand*{\cl}{\text cl}
\newcommand{\bP}{\mathbb{P}}
\newcommand{\bS}{\mathbb{S}}


\newcommand{\ket}[1]{|#1 \rangle}
\newcommand{\bra}[1]{\langle #1|}
\newcommand{\inner}[2]{\langle #1 | #2 \rangle}
\newcommand{\R}{\mathbb{R}}
\newcommand{\C}{\mathbb{C}}
\newcommand{\A}{\mathbb{A}}
\newcommand{\sphere}{\mathbb{S}}
\newcommand{\V}{\mathbb{V}}
\newcommand{\Hilbert}{\mathcal{H}}
\newcommand{\oper}{\hat{\Omega}}
\newcommand{\lam}{\hat{\Lambda}}
\newcommand{\defeq}{\vcentcolon=}

\newcommand{\bigslant}[2]{{\raisebox{.2em}{$#1$}\left/\raisebox{-.2em}{$#2$}\right.}}
\newcommand{\restr}[2]{{% we make the whole thing an ordinary symbol
  \left.\kern-\nulldelimiterspace % automatically resize the bar with \right
  #1 % the function
  \vphantom{\big|} % pretend it's a little taller at normal size
  \right|_{#2} % this is the delimiter
  }}
%%%%%%%%%%%%%%%%%%%%%%%%%%%%%%%%%%%%%%%


\tcbset{theostyle/.style={
    enhanced,
    sharp corners,
    attach boxed title to top left={
      xshift=-1mm,
      yshift=-4mm,
      yshifttext=-1mm
    },
    top=1.5ex,
    colback=white,
    colframe=blue!75!black,
    fonttitle=\bfseries,
    boxed title style={
      sharp corners,
    size=small,
    colback=blue!75!black,
    colframe=blue!75!black,
  } 
}}

\newtcbtheorem[number within=section]{Theorem}{Theorem}{%
  theostyle
}{thm}

\newtcbtheorem[number within=section]{Definition}{Definition}{%
  theostyle
}{def}



\title{Math 214 Homework 5}
\author{Keshav Balwant Deoskar}

\begin{document}
\maketitle



%%%%%%%%%%%%%%%%%%%%%%%%%%%%%%%%%%%%%%%%%%%%%%%%%%%%%%%%%%%%%%%%%
\textbf{Q4-5.} Let $\mathbb{CP}^n$ denote the $n-$dimensional complex projective space. 
\begin{enumerate}[label=(\alph*)]
  \item Show that the quotient map $\pi : \C^{n+1} \setminus \{0\} \rightarrow \mathbb{CP}^n$ is a surjective smooth submersion.
  \item Show that $\mathbb{CP}^n$ is diffeomorphic to $\mathbb{S}^n$.
\end{enumerate}
%%%%%%%%%%%%%%%%%%%%%%%%%%%%%%%%%%%%%%%%%%%%%%%%%%%%%%%%%%%%%%%%%

\vskip 0.5cm
\textbf{Proof:}

\vskip 0.5cm
\hrule 
\vskip 0.5cm



%%%%%%%%%%%%%%%%%%%%%%%%%%%%%%%%%%%%%%%%%%%%%%%%%%%%%%%%%%%%%%%%%
\textbf{Q4-6.} Let $M$ be a nonempty smooth compact manifold. Show that there is no smooth submersion $F : M \rightarrow \R^{k}$ for any $k > 0$. 
%%%%%%%%%%%%%%%%%%%%%%%%%%%%%%%%%%%%%%%%%%%%%%%%%%%%%%%%%%%%%%%%%

\vskip 0.5cm
\textbf{Proof:}
From LeeSM Proposition 4.28, We know that if $\pi : M \rightarrow N$ is a smooth submersion between smooth manifolds then $\pi$ is an open map. Now, consider $M$ to be a non-empty smooth compact manifold and let $N = \R^k$. $M \subseteq M$ is open when viewed as a subset of itself. However, $F(M)$ is a compact subset of $\R^k$ since $F$ is a smooth map, and compact subsets of euclidean space are not open. Thus, we have a contradiction.

\vskip 0.5cm
\hrule 
\vskip 0.5cm



%%%%%%%%%%%%%%%%%%%%%%%%%%%%%%%%%%%%%%%%%%%%%%%%%%%%%%%%%%%%%%%%%
\textbf{Q4-7.} Suppose $M$ and $N$ are smooth manifolds, and $\pi : M \rightarrow N$ is an injective smooth submersion. Show that there is no other smooth manifold structure on $N$ that satisfies the conclusion of Theorem 4.29. 
%%%%%%%%%%%%%%%%%%%%%%%%%%%%%%%%%%%%%%%%%%%%%%%%%%%%%%%%%%%%%%%%%

\vskip 0.5cm
\textbf{Proof:}

From Theorem 4.28, we know that surjective smooth submersions are quotient maps. Then, from the uniqueness of the quotient topology, we know there is no other smooth manifold structure on $N$ such that the conclusion of Theorem 4.29 holds.

\vskip 0.5cm
\hrule 
\vskip 0.5cm


%%%%%%%%%%%%%%%%%%%%%%%%%%%%%%%%%%%%%%%%%%%%%%%%%%%%%%%%%%%%%%%%%
\textbf{Q4-8.} Let $\pi : \R^2 \rightarrow \R$ be defined by $\pi(x, y) = xy$. Show that $\pi$ is surjective and smooth, and that for each smooth manifold $P$, a map $F : \R \rightarrow P$ is smooth if and only if $F \circ \pi$ is smooth; but $\pi$ is not a smooth submersion.
%%%%%%%%%%%%%%%%%%%%%%%%%%%%%%%%%%%%%%%%%%%%%%%%%%%%%%%%%%%%%%%%%

\vskip 0.5cm
\textbf{Proof:}


For any $t \in \R$, we can simply choose $x = t, y = 1$. Then, $\pi(x, y) = \pi(t, 1) = t$, so the map is surjective. The map is also smooth since the partial derivatives with respect to $x^1, x^2 = x, y$ are smooth 
\[ \frac{\partial f}{\partial x} = y \;\;\;\;\;\;\; \frac{\partial f}{\partial y} = x \] 

However, $\pi$ is not a smooth submersion since the differential of $\pi$ 
\[ d\pi_{(0,0)} = \restr{\begin{pmatrix}
  x \\
  y
\end{pmatrix}}{(0,0)} = \mathbf{0} \]

has rank zero at the origin, whereas it has rank 1 everywhere else on $\R^2$. So, $\pi$ is not a constant rank map.

\vskip 0.5cm
\hrule 
\vskip 0.5cm




%%%%%%%%%%%%%%%%%%%%%%%%%%%%%%%%%%%%%%%%%%%%%%%%%%%%%%%%%%%%%%%%%
\textbf{Q4-9.} Let $M$ be a connected smooth manifold, and let $\pi : E \rightarrow M$ be a topological covering map. Complete the proof of proposition 4.40 by showing that there is only one smooth structure on $E$ such that $\pi$ is a smooth covering map.  
%%%%%%%%%%%%%%%%%%%%%%%%%%%%%%%%%%%%%%%%%%%%%%%%%%%%%%%%%%%%%%%%%

\vskip 0.5cm
\textbf{Proof:}

\begin{dottedbox}
  \emph{\textbf{Theorem 4.40:}} Suppose $M$ is a connected smooth $n-$manifold and $\pi : E \rightarrow M$ is a \emph{topological} covering map. Then $E$ is a topological $(n-1)$ manifold and there exsits a unique smooth structure on $E$ such that $\pi$ is a smooth covering map.
\end{dottedbox}

\vskip 0.5cm
The book proves that $E$ is a topological $(n-1)$ manifold and that there exists a smooth structure on it such that $\pi$ is a smooth covering map. Now, let's suppose $\tilde{E}$ is the same set but with a different smooth strucuture on it, such that $\tilde{\pi} : \tilde{E} \rightarrow M$ is smooth. To show that the two smooth structures on $E$ must be the same, let's prove that $\mathrm{id} : E \rightarrow \tilde{E}$ is a diffeomorphism.

\vskip 0.5cm
Every point in $E$ is in the pre-image of some evenly covered $V \subseteq S$.

\vskip 0.5cm
\hrule 
\vskip 0.5cm




%%%%%%%%%%%%%%%%%%%%%%%%%%%%%%%%%%%%%%%%%%%%%%%%%%%%%%%%%%%%%%%%%
\textbf{Q5-4.} Show that the image of the curve $\beta : (-\pi, \pi) \rightarrow \R^2$ of Example 4.19 is not an embedded submanifold of $\R^2$. 
%%%%%%%%%%%%%%%%%%%%%%%%%%%%%%%%%%%%%%%%%%%%%%%%%%%%%%%%%%%%%%%%%

\vskip 0.5cm
\textbf{Proof:}


\vskip 0.5cm
\hrule 
\vskip 0.5cm




%%%%%%%%%%%%%%%%%%%%%%%%%%%%%%%%%%%%%%%%%%%%%%%%%%%%%%%%%%%%%%%%%
\textbf{Q5-6.} Suppose $M \subseteq \R^n$ is an embedded $m-$dimensional submanifold, and let $UM \subseteq T \R^n$ be the set of all \emph{unit} tangent vectors to $M$:
\[ UM = \{ (x, v) \in T \R^n \; : \; x\in M, v \in T_{x} M, \; \left| v \right| = 1 \} \]

This is called the \emph{\textbf{Unit Tangent Bundle of $M$.}} Prove that $UM$ is an embedded  $(2n-1)-$dimensional submanifold of $T \R^n \approx \R^n \times \R^{n}$.
%%%%%%%%%%%%%%%%%%%%%%%%%%%%%%%%%%%%%%%%%%%%%%%%%%%%%%%%%%%%%%%%%

\vskip 0.5cm
\textbf{Proof:}


\vskip 0.5cm
\hrule 
\vskip 0.5cm




%%%%%%%%%%%%%%%%%%%%%%%%%%%%%%%%%%%%%%%%%%%%%%%%%%%%%%%%%%%%%%%%%
\textbf{Q5-7.} Let $F : \R^2 \rightarrow \R$ be defined as $F(x,y) = x^3 + xy + y^3$. Which level sets of $F$ are embedded submanifolds of $\R^2$? For each level set, prove either that it is or that it is not an embedded submanifold. 
%%%%%%%%%%%%%%%%%%%%%%%%%%%%%%%%%%%%%%%%%%%%%%%%%%%%%%%%%%%%%%%%%

\vskip 0.5cm
\textbf{Proof:}


\vskip 0.5cm
\hrule 
\vskip 0.5cm






% %%%%%%%%%%%%%%%%%%%%%%%%%%%%%%%%%%%%%%%%%%%%%%%%%%%%%%%%%%%%%%%%%
% \textbf{Q5-.} 
% %%%%%%%%%%%%%%%%%%%%%%%%%%%%%%%%%%%%%%%%%%%%%%%%%%%%%%%%%%%%%%%%%

% \vskip 0.5cm
% \textbf{Proof:}


% \vskip 0.5cm
% \hrule 
% \vskip 0.5cm


\end{document}
