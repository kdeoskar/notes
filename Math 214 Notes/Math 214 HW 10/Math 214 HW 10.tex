\documentclass{article}

% Language setting
% Replace `english' with e.g. `spanish' to change the document language
\usepackage[english]{babel}

% Set page size and margins
% Replace `letterpaper' with`a4paper' for UK/EU standard size
\usepackage[letterpaper,top=2cm,bottom=2cm,left=3cm,right=3cm,marginparwidth=1.75cm]{geometry}

% Useful packages
\usepackage{amsmath}
\usepackage{amssymb}
\usepackage{mathtools}
\usepackage{graphicx}
\usepackage{enumitem}
\usepackage{bbm}
\usepackage[colorlinks=true, allcolors=blue]{hyperref}

\usepackage{hyperref}
\hypersetup{
    colorlinks=true,
    linkcolor=blue,
    filecolor=magenta,      
    urlcolor=cyan,
    pdftitle={Math 214 HW 10},
    pdfpagemode=FullScreen,
    }

\urlstyle{same}

\usepackage{tikz-cd}

%%%%%%%%%%% Box pacakges and definitions %%%%%%%%%%%%%%
\usepackage[most]{tcolorbox}
\usepackage{xcolor}

% Define the colors
\definecolor{boxheader}{RGB}{0, 51, 102}  % Dark blue
\definecolor{boxfill}{RGB}{173, 216, 230}  % Light blue

% Define the tcolorbox environment
\newtcolorbox{mathdefinitionbox}[2][]{%
    colback=boxfill,   % Background color
    colframe=boxheader, % Border color
    fonttitle=\bfseries, % Bold title
    coltitle=white,     % Title text color
    title={#2},         % Title text
    enhanced,           % Enable advanced features
    attach boxed title to top left={yshift=-\tcboxedtitleheight/2}, % Center title
    boxrule=0.5mm,      % Border width
    sharp corners,      % Sharp corners for the box
    #1                  % Additional options
}
%%%%%%%%%%%%%%%%%%%%%%%%%

\newtcolorbox{dottedbox}[1][]{%
    colback=white,    % Background color
    colframe=white,    % Border color (to be overridden by dashrule)
    sharp corners,     % Sharp corners for the box
    boxrule=0pt,       % No actual border, as it will be drawn with dashrule
    boxsep=5pt,        % Padding inside the box
    enhanced,          % Enable advanced features
    overlay={\draw[dashed, thin, black, dash pattern=on \pgflinewidth off \pgflinewidth, line cap=rect] (frame.south west) rectangle (frame.north east);}, % Dotted line
    #1                 % Additional options
}

\usepackage{biblatex}
\addbibresource{sample.bib}


%%%%%%%%%%% New Commands %%%%%%%%%%%%%%
\newcommand*{\T}{\mathcal T}
\newcommand*{\cl}{\text cl}
\newcommand{\bP}{\mathbb{P}}
\newcommand{\bS}{\mathbb{S}}


\newcommand{\ket}[1]{|#1 \rangle}
\newcommand{\bra}[1]{\langle #1|}
\newcommand{\inner}[2]{\langle #1 | #2 \rangle}
\newcommand{\R}{\mathbb{R}}
\newcommand{\C}{\mathbb{C}}
\newcommand{\A}{\mathbb{A}}
\newcommand{\sphere}{\mathbb{S}}
\newcommand{\V}{\mathbb{V}}
\newcommand{\Hilbert}{\mathcal{H}}
\newcommand{\oper}{\hat{\Omega}}
\newcommand{\lam}{\hat{\Lambda}}
\newcommand{\defeq}{\vcentcolon=}

\newcommand{\bigslant}[2]{{\raisebox{.2em}{$#1$}\left/\raisebox{-.2em}{$#2$}\right.}}
\newcommand{\restr}[2]{{% we make the whole thing an ordinary symbol
  \left.\kern-\nulldelimiterspace % automatically resize the bar with \right
  #1 % the function
  \vphantom{\big|} % pretend it's a little taller at normal size
  \right|_{#2} % this is the delimiter
  }}
%%%%%%%%%%%%%%%%%%%%%%%%%%%%%%%%%%%%%%%


\tcbset{theostyle/.style={
    enhanced,
    sharp corners,
    attach boxed title to top left={
      xshift=-1mm,
      yshift=-4mm,
      yshifttext=-1mm
    },
    top=1.5ex,
    colback=white,
    colframe=blue!75!black,
    fonttitle=\bfseries,
    boxed title style={
      sharp corners,
    size=small,
    colback=blue!75!black,
    colframe=blue!75!black,
  } 
}}

\newtcbtheorem[number within=section]{Theorem}{Theorem}{%
  theostyle
}{thm}

\newtcbtheorem[number within=section]{Definition}{Definition}{%
  theostyle
}{def}



\title{Math 214 Homework 10}
\author{Keshav Balwant Deoskar}

\begin{document}
\maketitle



%%%%%%%%%%%%%%%%%%%%%%%%%%%%%%%%%%%%%%%%%%%%%%%%%%%%%%%%%%%%%%%%%
\textbf{Q10-15.} Let $V$ be a finite(n)-dimensional real vector space, and let $G_k(V)$ be the Grassmannian of $k-$dimensional subspaces of $V$. Let $T$ be the subset $G_k(V) \times V$ defined by 
\[ T = \{(S, v) \in G_k(V) \times V \;:\; v \in S \} \] Show that $T$ is a smooth rank-$k$ subbundle of the product bundle $G_k(V) \times V \rightarrow G_k(V)$, and is thus a smooth rank-$k$ vector bundle over $G_k(V)$.
%%%%%%%%%%%%%%%%%%%%%%%%%%%%%%%%%%%%%%%%%%%%%%%%%%%%%%%%%%%%%%%%%

\vskip 0.5cm
\textbf{Proof:}

We have a product bundle $G_k(V) \times V \rightarrow G_k(V)$ and want to show that $T = \{(S, v) \in G_k(V) \times V \;:\; v \in S \}$ is a subbundle.

\vskip 0.5cm
Notice that for each $k-$dimensional subspace of $x \in G_k(V)$, we have a linear subspace \[ T_x = \{(x, v) \;:\; x \in G_k(V), v \in x \} \subseteq G_k(V) \times V \]

\vskip 0.5cm
Lemma 10.32 tells us that $T = \bigcup_{x \in G_k(V)} T_x$ is a subbundle of $G_k(V) \times V$ if and only if each point of $G_k(V)$ has a neighborhood $U$ on which there exist smooth sections $\sigma_1, \cdots, \sigma_{\mathrm{dim}(G_k(V))} : U \rightarrow G_k(V) \times V$ with the property that $\sigma_1, \cdots, \sigma_{k(n-k)} $ form a basis for $T_x$ at each $x \in U$.

\vskip 0.5cm
Recall that we can cover $G_k(V)$ with charts that look like $\phi_I : \mathrm{GL}(V_I, V_J) \rightarrow \mathrm{Gr}_k(V)$ defined by 
\[ \phi_I(L) = \mathrm{graph}(L) = \{ v + L(v) \;:\; v \in V_I \} \subseteq V \]
where $I \subseteq \{1, \cdots, n\}$,  $J = \{1, \cdots, n\} \setminus I$, and $V_I = \mathrm{span}(e_i)_{i \in I}$. Also, note that $\mathrm{dim} \mathrm{GL}(V_I, V_J) = k(n-k)$.


\vskip 0.5cm
Each $x \in G_k(V)$ is a $k-$dimensional linear space whose elements are $n-$dimensional vectors. So, for an open set $U$ containing $x = \mathrm{span}\{ v_1, \cdots, v_k \} \in G_k(V)$ we can define the section $\sigma_i : U \rightarrow G_k(V) \times V$ by
\[ x \mapsto (x, x^i) \] where $x^i$ is the $i^{\text{th}}$ coordinate function in the coordinate $\phi_I(L)$ for $1 \leq i \leq k(n-k)$. Then, certainly, the sections $\sigma_1, \cdots, \sigma_{k(n-k)}$ form a basis for $T_x$. So, Lemma 10.32 gives us the desired result.

\vskip 0.5cm
\hrule 
\vskip 0.5cm


%%%%%%%%%%%%%%%%%%%%%%%%%%%%%%%%%%%%%%%%%%%%%%%%%%%%%%%%%%%%%%%%%
\textbf{Q10-17.} Suppose $M \subseteq \R^n$ is an immersed submanifold. Prove that the ambient tangent bundle $\restr{T\R^n}{M}$ is isomorphic to the Whitney sum $TM \oplus NM$, where $NM \rightarrow M$ is the normal bundle.
%%%%%%%%%%%%%%%%%%%%%%%%%%%%%%%%%%%%%%%%%%%%%%%%%%%%%%%%%%%%%%%%%

\vskip 0.5cm
\textbf{Proof:}

Given an immersed submanifold $M \subseteq \R^n$, we define the ambient tangent bundle $\restr{T\R^n}{M}$ to be the set \[ \restr{T\R^n}{M} = \bigcup_{p \in M} E_p = \bigcup_{p \in M} T_p\R^n  \] with the projection $\pi_M : \restr{T\R^n}{M} \rightarrow M $ obtained by restricting $\pi$.

\vskip 0.5cm
Recall that the normal space to a manifold $M$ at a point $x \in M$ is the $(n-m)$ dimensional subspace $N_x M \subseteq T_x \R^n$ consisting of all vectors orthogonal to $T_x M$ and the normal bundle to manifold $M$ is the subset of $T\R^n \approx \R^n \times \R^n$ defined as  
\[ NM = \{ (x, v) \in \R^n \times \R^n \;:\; x \in M, v \in N_x M \} \]

\vskip 0.5cm
To show that $\restr{T\R^n}{M} \cong TM \oplus NM$, let's construct an isomorphism between the two. For any $p \in M$,since $N_p M$ and $T_p M$ are orthogonal compliments in $T_p \R^n$ we have a natural isomorphism $N_p M \oplus T_p M \cong T_p \R^n$. Then, we have 
\[ \bigcup_{p \in M} N_p M \oplus T_pM \cong \bigcup_{p \in M} T_p \R^n \]

We want to show that these local isomorphisms can be "combined" to get isomorphism between the whitney sum bundle and the ambient bundle.

\vskip 0.5cm
To do so, note that the map $\phi : TM \oplus NM \rightarrow \restr{T\R^n}{M}$ defined by $\phi(p, (v,w)) = (p, \phi_p(v,w))$ is smooth since the local isomorphisms $\phi_p$ are linear and bundle structures are smooth. Also, $\phi$ restricts to a linear isomorphism on each fiber, since $\phi_p$ is a linear isomorphism for each $p \in M$.

\vskip 0.5cm
To show that $\phi$ is an isomorphism, we need to construct the inverse. Define a map $\psi : \restr{T\R^n}{M} \rightarrow TM \oplus NM$ by $\psi(p, x) = (p, (x_p, y_p))$ where $x_p$ is the orthogonal projection of $x$ onto $T_p M $ and $y_p = x - x_p$ is the orthogonal projection of $x$ onto $N_x M$. Note that $\psi$ is smooth, since the orthogonal prjections are smooth linear maps. 

\vskip 0.5cm
Both maps are one-to-one and onto, thus we have that $\phi$ is an isomorphism. So,

\[ \boxed{\restr{T\R^n}{M} \cong TM \oplus NM} \]
 

\vskip 0.5cm
\hrule 
\vskip 0.5cm


%%%%%%%%%%%%%%%%%%%%%%%%%%%%%%%%%%%%%%%%%%%%%%%%%%%%%%%%%%%%%%%%%
\textbf{Q11-5.} For any smooth manifold $M$, show that $T^*M$ is a trivial vector bundle if and only if $TM$ is trivial. 
%%%%%%%%%%%%%%%%%%%%%%%%%%%%%%%%%%%%%%%%%%%%%%%%%%%%%%%%%%%%%%%%%

\vskip 0.5cm
\textbf{Proof:}


Suppose $TM$ is trivial. Then, $M$ is parallelizable and admits a global frame $\left(E_1, \cdots, E_n \right)$ which serves as a basis of $T_p M$, for all $p \in M$. Take some open cover $\{U_{\alpha}\}_{\alpha \in A}$ of $M$. Then, $\restr{\left(E_1, \cdots, E_n\right)}{U_{\alpha}}$ is a smooth local frame for all $\alpha \in A$. 

\vskip 0.5cm
Then, as in Example 11.13 in LeeSM, every smooth local frame has a dual smooth local coframe $\restr{\left(\epsilon^1, \cdots, \epsilon^n\right)}{U_{\alpha}}$ such that $\restr{\epsilon^j}{U_{\alpha}} \restr{E_i}{U_{\alpha}} = \delta^j_i$. THen, by the gluing lemma for smooth manifolds applied to $\epsilon^j : M \rightarrow T^*M$, we can extend to a global section of the tangent bundle to obtaina global coframe for $T^* M$. Thus $T^*M$ is also trivial. 

\vskip 0.5cm
The converse follows exactly the same procedure.


\vskip 0.5cm
\hrule 
\vskip 0.5cm



%%%%%%%%%%%%%%%%%%%%%%%%%%%%%%%%%%%%%%%%%%%%%%%%%%%%%%%%%%%%%%%%%
\textbf{Q11-6.} Suppose $M$ is a smooth $n-$manifold, $p \in M$, and $y^1, \cdots, y^k$ are smooth real-valued functions defined on a neighborhood of $p$ in $M$. Prove the following statements.
\begin{enumerate}[label=(\alph*)]
  \item If $k = n$ and $\left( \restr{dy^1}{p}, \cdots, \restr{dy^n}{p} \right)$ is a basis for $T_p^*M$, then $\left(y^1, \cdots, y^n\right)$ are smooth coordinates for $M$ in some neighborhood of $p$. 
  \item If $\left(\restr{dy^1}{p}, \cdots, \restr{dy^k}{p}\right)$ is a linearly independent $k-$tuple of covectors and $k < n$, then there are smooth functions $y^{k+1}, \cdots, y^n$ such that $\left(y^1, \cdots, y^{n}\right)$ are smooth coordinates for $M$ in a neighborhood of $p$.
  \item If $\left(\restr{dy^1}{p}, \cdots, \restr{dy^k}{p}\right)$ span $T_p^* M$, there are indices $i_1, \cdots, i_n$ such that $\left(y^{i_1}, \cdots, y^{i_n}\right)$ are smooth coordinates for $M$ in a neighborhood of $p$.
\end{enumerate}
%%%%%%%%%%%%%%%%%%%%%%%%%%%%%%%%%%%%%%%%%%%%%%%%%%%%%%%%%%%%%%%%%

\vskip 0.5cm
\textbf{Proof:}

\begin{enumerate}[label=(\alph*)]
  \item Let $\phi = \left(y^1, \cdots, y^n\right) : U \rightarrow M$ for some open subset $p \in U \subseteq_{open} M$. To show that $y^1, \cdots, y^n$ form smooth coordinates for $M$ in some neighborhood of $p$ we need to show that $\phi$ is a local diffeomorphism.
  
  
  This can be achieved by showing that $d\phi$ is invertible (since then the Inverse Function Theorem will imply that $\phi$ is a local diffeomorphism). 


  We know that $d\phi_p : T_{p} U \cong T_{p} M \rightarrow T_{\phi(p)} \R^n$ is a map between tangent spaces of equal dimension. So, to show bijection, it suffices to show $d\phi$ is injective.


  Since $\left(\restr{dy^1}{p}, \cdots, \restr{dy^n}{p}\right)$ forms a smooth coframe for the cotangent bundle, we know there must be a smooth frame $\left( \partial / \partial y^1, \cdots, \partial / \partial y^n \right)$ for the tangent bundle dual to the coframe. 

  Now, if $\left(x^1, \cdots, x^n\right)$ are the coordinate functions on $\R^n$, the coordinate representation of $\phi$ has components $\hat{\phi}^j = x^j \circ \phi = y^j$. Then, consider any $v = v^i \frac{\partial}{\partial x^i} \in T_pM, v \neq 0$. Then, 
  \begin{align*}
    d\phi_p(v) &= d\phi_p \left(v^i(p) \frac{\partial}{\partial y^i} \right) \\
    &= \frac{\partial \hat{\phi}^j}{\partial y^i}(\hat{p}) \frac{\partial}{\partial x^j} \\
    &= \underbrace{\frac{\partial y^j}{\partial y^i}}_{\delta^j_i}(\hat{p}) \frac{\partial}{\partial x^j} \\
    &=v^j \frac{\partial}{\partial x^j}
  \end{align*}
  which is non-zero since at least one of the components $v^j$ is non-zero. This shows that $d\phi$ is injective.

  Therefore, $d\phi$ is a bijection i.e. it is invertible, so by the Inverse Function Theorem, $\phi$ is a local diffeomorphism meaning $\left(y^1, \cdots, y^n\right)$ form coordinate functions on some open neighborhood of $p$.

  \vskip 0.5cm
  \begin{dottedbox}
    \emph{Lemma for (b):} Let $M$ be a smooth manifold and $p \in M$ with $\lambda \in T_p^*M$. Then, there exists a neighborhood $U$ of $p$ and smooth function $y^j : M \rightarrow \R$ such that $\restr{dy}{p} = \lambda_p$.

    \vskip 0.5cm
    \emph{Proof:} Let $\left(U, (x^i)\right)$ be a smooth chart with $p \in U$. Let $\restr{\frac{\partial}{\partial x^i}}{p}$ be the standard basis for $T_p M$, and $\restr{dx^i}{p}$ be the dual basis i.e. basis for $T_p^* M$. Then, we can write 
    \[ \lambda = \lambda_i \restr{dx^i}{p} \] for scalars $\lambda_i$. Define $y = \lambda^i x_i$ where $\lambda^i = \lambda_i$. This is smooth since its just a linear combination of the coordinate functions. Then, indeed, we find 
    \begin{align*}
      \restr{dy}{p} &= d \left( \lambda^i x_i \right) \\
      &= \lambda_i dx^i \\
      &= \lambda
    \end{align*}
  \end{dottedbox}


  \vskip 0.5cm
  \item $T_p^* M$ is an $n-$dimensional vector space, so we can choose $\omega^{k+1}, \cdots, \omega^n \in T_p^*M$ such that $\left(dy^1, \cdots, dy^k, \omega^{k+1}, \cdots, \omega^{n}\right)$ form a basis for $T_p^*M$. Then, we know from the Lemma above that there exists an open neighborhood $U$ of $p$ and smooth coordinate maps $y_{k+1}, \cdots, y_n$ such that $d\left(y_{k+1}\right)|_{p} = \omega^{k+1}, \cdots, d\left(y_{n}\right)|_{p} = \omega^{n}$. Then, by part (a), $y_1, \cdots, y_n$ form smooth coordinates for $M$ in a neighborhood of $p$.
  
  \vskip 0.5cm
  \item Assuming $k > n$, the fact that $\left(\restr{dy^1}{p}, \cdots, \restr{dy^k}{p}\right)$ span $T_p^* M$ means that there is some subset of $n$ of these vectors which also span $T_p^*M$ i.e. there exist indices $i_1, \cdots, i_n$ such that $\left(\restr{dy^{i_1}}{p}, \cdots, \restr{dy^{i_n}}{p}\right)$ form a basis for $T_p^* M$. Then, once again, by part (a) we get the desired result.
\end{enumerate}


\vskip 0.5cm
\hrule 
\vskip 0.5cm


%%%%%%%%%%%%%%%%%%%%%%%%%%%%%%%%%%%%%%%%%%%%%%%%%%%%%%%%%%%%%%%%%
\textbf{Q11-7.} In the following problems, $M$ and $N$ are smooth manifolds, $F : M \rightarrow N$ is a smooth map, and $\omega \in \mathfrak{X}^*(N)$. Compute $F^* \omega$ in each case.
\begin{enumerate}[label=(\alph*)]
  \item $M = N = \R^2$; \\ $F(s, t) = \left(st, e^t\right)$; \\$\omega = xdy - ydx$
  \item $M = \R^2, N = \R^3$; \\$F(\theta, \phi) = \left( \left(\cos\phi + 2\right)\cos\theta, \left(\cos\phi + 2\right)\sin\theta, \sin\phi \right)$; \\$\omega =z^2 dx$
  \item $M = \{(s, t) \in \R^2 \;:\;s^2 + t^2 < 1 \}, N = \R^3 \setminus \{0\}$; \\$F(s,t) = \left( s, t, \sqrt{1-s^2-t^2} \right)$; \\$\omega = \left(1-x^2-y^2\right)dz$
\end{enumerate} 
%%%%%%%%%%%%%%%%%%%%%%%%%%%%%%%%%%%%%%%%%%%%%%%%%%%%%%%%%%%%%%%%%

\vskip 0.5cm
\textbf{Solutions:}

Using Proposition 11.26, we can compute the pullback of a covector field $\omega \in \mathfrak{X}^*(N)$ under the action of smooth map $F : M \rightarrow N$ using the formula
\[ F^* \omega = \left(\omega_j \circ F\right) d \left(y^j \circ F\right) \]

\begin{enumerate}[label=(\alph*)]
  \item We have 
  \begin{align*}
    F^* \omega &= \left(x \circ F\right) d\left(y \circ F\right) + \left(y \circ F\right)d\left(x \circ F\right) \\
    &= st \cdot d\left( e^t \right) + e^t d\left( st \right) \\
    &= st \cdot te^t + st \cdot e^t \\
  \implies F^*\omega &= st\cdot (e^{t} + 1)
  \end{align*}

  \vskip 0.5cm
  \item \begin{align*}
    F^*\omega &= \left( z^2 \circ F \right) d\left(x \circ F\right) \\
    &= \left( \sin^2\phi \right) \cdot d \left( \left( \cos\phi  + 2\right) \cos\theta \right) \\
    &= \sin^2\phi \cdot \left[ -\left(\cos\phi + 2\right)\sin\theta - \sin\phi \cos\theta \right] \\
    &=-\sin^2\phi \cdot \left[ \left(\cos\phi + 2\right)\sin\theta + \sin\phi \cos\theta \right] \\
  \end{align*}

  \vskip 0.5cm
  \item \begin{align*}
    F^* \omega &= \left((1-x^2-y^2) \circ F\right) d\left( z \circ F \right) \\
    &= \left(1-s^2-t^2\right) \cdot d\left(\sqrt{1-s^2-t^2}\right) \\
    &= \left(1-s^2-t^2\right) \cdot \left[ \frac{-s -t}{\sqrt{1-s^2-t^2}} \right] \\
    &= -(s+t)\sqrt{1-s^2-t^2}
  \end{align*}
\end{enumerate}


\vskip 0.5cm
\hrule 
\vskip 0.5cm


%%%%%%%%%%%%%%%%%%%%%%%%%%%%%%%%%%%%%%%%%%%%%%%%%%%%%%%%%%%%%%%%%
\textbf{Q11-11.} Let $M$ be a smooth manifold, and $C \subseteq M$ be an embedded submanifold. Let $f \in C^{\infty}(M)$, and suppose $p \in C$ is a point at which $f$ attains a local maximum or minmium value among points in $C$. Given a smooth local defining function $\Phi : U \rightarrow \R^k$ for $C$ on a neighborhood $U$ of $p$ in $M$, show that there are real numbers $\lambda_1, \cdots, \lambda_k$ (called \emph{\textbf{Lagrange Multipliers}}) such that 
\[ df_p = \lambda_1 \restr{d\Phi^1}{p} + \cdots + \lambda_1 \restr{d\Phi^k}{p} \] 
%%%%%%%%%%%%%%%%%%%%%%%%%%%%%%%%%%%%%%%%%%%%%%%%%%%%%%%%%%%%%%%%%

\vskip 0.5cm
\textbf{Proof:}

% For each point $p \in C$, we have a neighborhood $U \subseteq M$ and a smooth local defining map $\Phi : U \rightarrow \R^k$.

% \vskip 0.5cm
% Since $\Phi$ is a map from $U$ to $\R^{\mathrm{dim}M - \mathrm{dim}C}$, $C$ has codimension $k$ and dimension $(n-k)$. Now, since $C$ is an embedded submanifold it satisfies the local $(n-k)$ slice condition i.e. there is a chart $(V, (x^i) )$ of $M$ such that any $p \in C \cap V$ has coordinates of the form 
% \[ \left(x^1, \cdots, x^{n-k}, c^{n-k+1}, \cdots, c^{n}\right) \]

% WLOG we can assume that the last $n-k+1$ coordinates vanish on $C \cap V$.

\vskip 0.25cm
The smooth function $f : M \rightarrow \R$ attains a local maximum or minmium on $C \subseteq_{\text{embed.}} M$. So, for $p \in C$ and $v \in T_p C$, we have $df_p(v) = 0$. Now, $\Phi$ is a local defining function for $C$, so its differential has full rank $k$ at every $p \in C$. So the component functions of the differential, $d\Phi^i$ for $i = 1,\cdots,k$, are linearly independent.

\vskip 0.25cm
The component functions $d\Phi^i$ form a basis for a $k-$dimensional subspace of the cotangent space $T_p^*M$. In fact, this subspace is exactly the annihilator of $T_p C$ i.e. the set of all covectors that vanish on $T_p C$. 

\begin{dottedbox}
  This can be seen as 
  \begin{itemize}
    \item For any $v \in T_p C$ we have $d\Phi^i(v) = 0$ because $f$ attains a local extremum on $C$. 
  \end{itemize}
\end{dottedbox}

Now, any covector that vanishes on $T_p C$ must be a linear combination of the basis elements, so 
\[ df_p = \lambda_1 \restr{d\Phi^1}{p} + \cdots + \lambda_k \restr{d\Phi^k}{p} \] for scalars $\lambda_1, \cdots, \lambda_k$.

\vskip 0.5cm
\hrule 
\vskip 0.5cm



% %%%%%%%%%%%%%%%%%%%%%%%%%%%%%%%%%%%%%%%%%%%%%%%%%%%%%%%%%%%%%%%%%
% \textbf{Q11-.} 
% %%%%%%%%%%%%%%%%%%%%%%%%%%%%%%%%%%%%%%%%%%%%%%%%%%%%%%%%%%%%%%%%%

% \vskip 0.5cm
% \textbf{Proof:}


% \vskip 0.5cm
% \hrule 
% \vskip 0.5cm


\end{document}
