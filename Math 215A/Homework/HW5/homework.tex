\documentclass[11pt]{article}

% basic packages
\usepackage[margin=1in]{geometry}
\usepackage[pdftex]{graphicx}
\usepackage{amsmath,amssymb,amsthm}
\usepackage{custom}
\usepackage{lipsum}

\usepackage{xcolor}
\usepackage{tikz-cd}

\usepackage[most]{tcolorbox}
\usepackage{xcolor}
\usepackage{mdframed}

% page formatting
\usepackage{fancyhdr}
\pagestyle{fancy}

\renewcommand{\sectionmark}[1]{\markright{\textsf{\arabic{section}. #1}}}
\renewcommand{\subsectionmark}[1]{}
\lhead{\textbf{\thepage} \ \ \nouppercase{\rightmark}}
\chead{}
\rhead{}
\lfoot{}
\cfoot{}
\rfoot{}
\setlength{\headheight}{14pt}

\linespread{1.03} % give a little extra room
\setlength{\parindent}{0.2in} % reduce paragraph indent a bit
\setcounter{secnumdepth}{2} % no numbered subsubsections
\setcounter{tocdepth}{2} % no subsubsections in ToC


%%%%%%%%%%%%%%%%%%%%%%%%%%%%%%%%%%%%%%%%%%%%%%%%%%%%%%%%%%%%%%%%%
% CUSTOM BOXES AND STUFF
\newtcolorbox{redbox}{colback=red!5!white,colframe=red!75!black, breakable}
\newtcolorbox{bluebox}{colback=blue!5!white,colframe=blue!75!black,breakable}

\newtcolorbox{dottedbox}[1][]{%
    colback=white,    % Background color
    colframe=white,    % Border color (to be overridden by dashrule)
    sharp corners,     % Sharp corners for the box
    boxrule=0pt,       % No actual border, as it will be drawn with dashrule
    boxsep=5pt,        % Padding inside the box
    enhanced,          % Enable advanced features
    breakable,         % Enables it to span multiple pages
    overlay={\draw[dashed, thin, black, dash pattern=on \pgflinewidth off \pgflinewidth, line cap=rect] (frame.south west) rectangle (frame.north east);}, % Dotted line
    #1                 % Additional options
}

% Define the colors
\definecolor{boxheader}{RGB}{0, 51, 102}  % Dark blue
\definecolor{boxfill}{RGB}{173, 216, 230}  % Light blue


% Define the tcolorbox environment
\newtcolorbox{mathdefinitionbox}[2][]{%
    colback=boxfill,   % Background color
    colframe=boxheader, % Border color
    fonttitle=\bfseries, % Bold title
    coltitle=white,     % Title text color
    title={#2},         % Title text
    enhanced,           % Enable advanced features
    breakable,
    attach boxed title to top left={yshift=-\tcboxedtitleheight/2}, % Center title
    boxrule=0.5mm,      % Border width
    sharp corners,      % Sharp corners for the box
    #1                  % Additional options
}
%%%%%%%%%%%%%%%%%%%%%%%%%


\definecolor{lightblue}{RGB}{173,216,230} % Light blue color
\definecolor{darkblue}{RGB}{0,0,139} % Dark blue color

% Define the custom proof environment
\newtcolorbox{ex}[2][Example]{
  colback=red!5!white, % Light blue background
  colframe=red!75!black, % Darker blue border
  coltitle=white, % Title color
  fonttitle=\bfseries, % Title font style
  title={{#2}},
  arc=1mm, % Rounded corners with 4mm radius,
  boxrule=0.5mm,
  left=2mm, right=2mm, top=2mm, bottom=2mm, % Padding inside the box
  breakable, % Allow box to be broken across pages
  before=\vspace{10pt}, % Padding above the box
  after=\vspace{10pt}, % Padding below the box
  before upper={\parindent15pt} % Ensure indentation
}

% Define the custom proof environment
\newtcolorbox{defn}[2][Definition]{
  colback=green!5!white, % Light blue background
  colframe=green!75!black, % Darker blue border
  coltitle=white, % Title color
  fonttitle=\bfseries, % Title font style
  title={{#2}},
  arc=1mm, % Rounded corners with 4mm radius,
  boxrule=0.5mm,
  left=2mm, right=2mm, top=2mm, bottom=2mm, % Padding inside the box
  breakable, % Allow box to be broken across pages
  before=\vspace{10pt}, % Padding above the box
  after=\vspace{10pt}, % Padding below the box
  before upper={\parindent15pt} % Ensure indentation
}


%%%%%%%%%%%%%%%%%%%%%%%%%%%%%%%%%%%%%%%%%%%%%%%%%%%%%%%%%%%%%%%%%


\begin{document}

% make title page
\thispagestyle{empty}
\bigskip \
\vspace{0.1cm}

\begin{center}
{\fontsize{22}{22} \selectfont Professor: Alexander Givental}
\vskip 16pt
{\fontsize{30}{30} \selectfont \bf \sffamily Math 215A: Algebraic Topology}
\vskip 24pt
{\fontsize{14}{14} \selectfont \rmfamily Homework 4} 
\vskip 6pt
{\fontsize{14}{14} \selectfont \ttfamily kdeoskar@berkeley.edu} 
\vskip 24pt
\end{center}

% {\parindent0pt \baselineskip=15.5pt \lipsum[1-4]} 

% make table of contents
% \newpage

\begin{bluebox}
  \textbf{Question 1: Prove that all fibers of a Hurewicz Fibration with a path-connected base are homotopy equivalent to each other.} 
\end{bluebox}

\vskip 0.5cm
\textbf{\underline{Solution:}}
\\
\\
Let's recall the definition. A triple $(E, B, p)$ is a Hurewicz fibration if  $p \text{ : } E \rightarrow B$ is surjective, and the triple has the Covering Homotopy Property for any topological space $X$ i.e. for mappings
\begin{enumerate}
  \item $f \text{ : } X \times [0, 1] \rightarrow B, ~~~F_0 \text{ : } X \cong X \times \{0\} \rightarrow E $
  \item with $$ \restr{f}{X \times \{0\}} = p \circ F_0 $$
\end{enumerate} there exists a homotopy 
$$ F \text{ : } X \times [0, 1] \rightarrow E $$ such that 
\begin{enumerate}
  \item $ \restr{F}{X \times \{0\}} = F_0 $ and 
  \item $p \circ F = f$
\end{enumerate} Suppose the base space $B$ is path-connected. Consider points $b, b' \in B$. Denote the inclusion of $p^{-1}(b)$ into $E$ as $g$, and $h$ be a continuous map from $p^{-1}(b) \times I$ to $B$.  \[\begin{tikzcd}
	{p^{-1}(b)} & E \\
	{p^{-1}(b) \times I} & B
	\arrow["g", from=1-1, to=1-2]
	\arrow["{i_0}"', from=1-1, to=2-1]
	\arrow["p", from=1-2, to=2-2]
	\arrow["h"', from=2-1, to=2-2]
\end{tikzcd}\] Since $B$ is path-connected there must exist a path $\gamma(t) \text{ : } I \rightarrow X$ between $b$ and $b'$. Notice that $g \circ p$ sends every point in $p^{-1}(b)$ to $b \in B$ i.e. $h(x, 0) = b$ for any $x \in p^{-1}(b)$
\\
\\
Let's choose $h$ to be
\begin{align*}
  h \text{ : } p^{-1}(b) \times I &\rightarrow B \\
              (x, t) &\mapsto \gamma(t)
\end{align*} where $\gamma(t)$ is a path connecting $b$ and $b'$. Then, since $(E, B, p)$ is a Hurewicz fibration, the CHP gives us $\tilde{H} \text{ : } p^{-1}(b) \times I \rightarrow E$ which defines a homotopy between $p^{-1}(b)$ and $p^{-1}(b')$ (the continuous maps in either direction are obtained by varying the parameter $t$).




% \[\begin{tikzcd}
% 	{p^{-1}(b)} & E \\
% 	{p^{-1}(b) \times I} & B
% 	\arrow["g", from=1-1, to=1-2]
% 	\arrow["{i_0}"', from=1-1, to=2-1]
% 	\arrow["p", from=1-2, to=2-2]
% 	\arrow["{\tilde{H}}"{description}, dashed, from=2-1, to=1-2]
% 	\arrow["h"', from=2-1, to=2-2]
% \end{tikzcd}\] Now, $\tilde{H}$ defines a homotopy from $$



% Consider a Hurewicz fibration $(E, B, p)$ with path-connected base $B$. That is, for any $x_0, x_1 \in B$ there exists a path from $x_0$ to $x_1$.
% \\
% \\
% Consider any topological space $X$ and let $f \text{ : } X \rightarrow p^{-1}(x_0)$ bw a continuous map, $s \text{ : } I \rightarrow B$ be a path connecting $x_0$ with $x_1$.
% \\
% \\
% Let's define $\tilde{\varphi} \text{ : } X \rightarrow E$ to be the composition of $f$ with the inclusion of $p^{-1}(x_0)$ into $E$, and define $\Phi \text{ : } X \times I \rightarrow B $ by $\v$



\vskip 0.5cm
\hrule
\pagebreak


\begin{bluebox}
  \textbf{Question 2: Prove that if the fiber $F$ of a Serre Fibration $f \text{ : } E \to B$ is contractible in $E$, then $\pi_n(B)$ is the direct sum of $\pi_n(E)$ and $\pi_{n-1}(F)$.} 
\end{bluebox}

\vskip 0.5cm
\textbf{\underline{Solution:}}
\\
\\
This follows directly from the following two results: 

\begin{redbox}
  \textsc{Exercise 14, Section 8.8:} If $A$ is contractible within $X$, then
  $$ \pi_n(X, A) \cong \pi_n(X) \oplus \pi_{n-1}(A) $$
\end{redbox}

\begin{redbox}
  \textsc{Lemma from Section 9.8:} Let $(E, B, p)$ be a Serre fibration, let $e_0 \in E$ be an arbitrary point, let $b_0 = p(e_0)$, and let $F = p^{-1}(b_0)$. Then the map $$ p_* \text{ : } \pi_n(E, F, e_0) = \pi_n(B, b_0) $$ is an isomorphism for all $n$.
\end{redbox} 

\vskip 0.5cm
\textbf{Proof of Exercise 14:}
\\
Suppose $A$ is contractible in $X$. Let $j$ denote the identity map $X \rightarrow X$ regarded as a map $(X, x_0) \rightarrow (X, A)$ and the induced homomorphism is $j_* \text{ : } \pi_n(X, x_0) \rightarrow \pi_n(X, A, x_0)$ $i \text{ : } A \rightarrow X$ is the inclusion map, $\partial$ denotes the connecting homomorphism.
\\
\\
Then, we have the homotopy sequence of the pair: 
\begin{align*}
  &\cdots \xrightarrow{\partial} \pi_n(A, x_0) \xrightarrow{i_*} \pi_n(X, x_0) \xrightarrow{j_*} \pi_n(X, A, x_0) \xrightarrow{\partial} \pi_{n-1}(A, x_0) \\
  &\cdots \xrightarrow{i_*} \pi_1(X, x_0) \xrightarrow{j_*} \pi_1(X, A, x_0) \xrightarrow{\partial} \pi_{0}(A, x_0) \xrightarrow{i_*} \pi_0(X, x_0) \\
\end{align*}
which we know is exact.
\\
\\
Consider this part of the sequence: 
$$ \cdots \rightarrow \pi_n(A, x_0) \xrightarrow{i_*} \pi_n(X, x_0) \xrightarrow{j_*} \pi_n(X, A, x_0) \xrightarrow{\partial} \pi_{n-1}(A, x_0) \rightarrow \cdots  $$ Recall that we say $A$ is contractible in $X$ if the inclusion $i \text{ : } A \rightarrow X$ is homotopic to a constant map $A \rightarrow X$. So, clearly, $i_* \text{ : } \pi_n(A) \rightarrow \pi_n(X)$ is a zero homomorphism.
\\
\\
Now, $j_*$ is injective and $\partial$ is surjective. Thus, we have $\pi_n(X, A, x_0) \cong \pi_n(X) \oplus \pi_{n-1}(A)$.


\vskip 0.5cm
\hrule
\pagebreak


\begin{bluebox}
  \textbf{Question 3: Compute the third homotopy groups of the Unitary groups $U(n)$ for all $n$.} 
\end{bluebox}

\vskip 0.5cm
\textbf{\underline{Solution:}}
\\
\\
Let's first find the 3rd homotopy groups for $U(1)$ and $U(2)$ using the fibration provided by the determinant map $\mathrm{det} \text{ : } U(2) \rightarrow U(1)$. Under $\mathrm{det}$, each $c \in U(1)$ has fiber $SU(2) \cong \sph^2$ (intuitively, because we can rotate by any amount without impacting the determinant)
\\
\\
The fibration induces the homotopy sequence:
$$ \pi_4(U(1)) \rightarrow \pi_3(\sph^3) \rightarrow \pi_3(U(2)) \rightarrow \pi_3(U(1)) $$
Note that since $U(1)$ is homeomorphic to $\sph^1$, we have $\pi_3(U(1)) \cong \pi_3(\sph^1) = 0$ and also $\pi_4(U(1)) \cong \pi_4(\sph^1) = 0$ so our sequence is 
$$ 0 \rightarrow \pi_3(\sph^3) \rightarrow \pi_3(U(2)) \rightarrow 0 $$ Thus, $\pi_3(U(2)) \cong \pi_3(\sph^3) = \mathbb{Z}$.
\\
\\
Now for larger $n$, let's use the fact that $$ \sph^{2n-1} = U(n)/U(n-1) $$ and so we have fiber bundle structure $U(n-1) \xhookrightarrow{} U(n) \rightarrow \sph^{2n-1} $ which induces the sequence 
$$  \pi_{k+1}(\sph^{2n-1}) \rightarrow \pi_k(U(n)) \rightarrow  \pi_k(U(n-1)) \rightarrow \pi_{k}(\sph^{2n-1}) $$ so, for $k < 2n$ we have $$ \pi_k(U(n)) = \pi_k(U(n-1)) $$

Thus, $\pi_3(U(n)) = \mathbb{Z}$ for $n > 1$.

\vskip 0.5cm
\hrule
\pagebreak















% \begin{bluebox}
%   \textbf{Question 1:} 
% \end{bluebox}

% \vskip 0.5cm
% \textbf{\underline{Solution:}}
% \\
% \\
% text
% \vskip 0.5cm
% \hrule
% \pagebreak



% %%%%%%%%%%%%%%%%%%%%%%%%%%%%%%%%%%%%%%%%%%%%%%
% \newpage
% % \section{References}
% %%%%%%%%%%%%%%%%%%%%%%%%%%%%%%%%%%%%%%%%%%%%%%
% \vskip 0.5cm
% \bibliographystyle{plain} % We choose the "plain" reference style
% \bibliography{citation}




\end{document}










