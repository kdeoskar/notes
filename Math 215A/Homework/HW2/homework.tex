\documentclass[11pt]{article}

% basic packages
\usepackage[margin=1in]{geometry}
\usepackage[pdftex]{graphicx}
\usepackage{amsmath,amssymb,amsthm}
\usepackage{custom}
\usepackage{lipsum}

\usepackage{xcolor}
\usepackage{tikz}

\usepackage[most]{tcolorbox}
\usepackage{xcolor}
\usepackage{mdframed}

% page formatting
\usepackage{fancyhdr}
\pagestyle{fancy}

\renewcommand{\sectionmark}[1]{\markright{\textsf{\arabic{section}. #1}}}
\renewcommand{\subsectionmark}[1]{}
\lhead{\textbf{\thepage} \ \ \nouppercase{\rightmark}}
\chead{}
\rhead{}
\lfoot{}
\cfoot{}
\rfoot{}
\setlength{\headheight}{14pt}

\linespread{1.03} % give a little extra room
\setlength{\parindent}{0.2in} % reduce paragraph indent a bit
\setcounter{secnumdepth}{2} % no numbered subsubsections
\setcounter{tocdepth}{2} % no subsubsections in ToC


%%%%%%%%%%%%%%%%%%%%%%%%%%%%%%%%%%%%%%%%%%%%%%%%%%%%%%%%%%%%%%%%%
% CUSTOM BOXES AND STUFF
\newtcolorbox{redbox}{colback=red!5!white,colframe=red!75!black, breakable}
\newtcolorbox{bluebox}{colback=blue!5!white,colframe=blue!75!black,breakable}

\newtcolorbox{dottedbox}[1][]{%
    colback=white,    % Background color
    colframe=white,    % Border color (to be overridden by dashrule)
    sharp corners,     % Sharp corners for the box
    boxrule=0pt,       % No actual border, as it will be drawn with dashrule
    boxsep=5pt,        % Padding inside the box
    enhanced,          % Enable advanced features
    breakable,         % Enables it to span multiple pages
    overlay={\draw[dashed, thin, black, dash pattern=on \pgflinewidth off \pgflinewidth, line cap=rect] (frame.south west) rectangle (frame.north east);}, % Dotted line
    #1                 % Additional options
}

% Define the colors
\definecolor{boxheader}{RGB}{0, 51, 102}  % Dark blue
\definecolor{boxfill}{RGB}{173, 216, 230}  % Light blue


% Define the tcolorbox environment
\newtcolorbox{mathdefinitionbox}[2][]{%
    colback=boxfill,   % Background color
    colframe=boxheader, % Border color
    fonttitle=\bfseries, % Bold title
    coltitle=white,     % Title text color
    title={#2},         % Title text
    enhanced,           % Enable advanced features
    breakable,
    attach boxed title to top left={yshift=-\tcboxedtitleheight/2}, % Center title
    boxrule=0.5mm,      % Border width
    sharp corners,      % Sharp corners for the box
    #1                  % Additional options
}
%%%%%%%%%%%%%%%%%%%%%%%%%


\definecolor{lightblue}{RGB}{173,216,230} % Light blue color
\definecolor{darkblue}{RGB}{0,0,139} % Dark blue color

% Define the custom proof environment
\newtcolorbox{ex}[2][Example]{
  colback=red!5!white, % Light blue background
  colframe=red!75!black, % Darker blue border
  coltitle=white, % Title color
  fonttitle=\bfseries, % Title font style
  title={{#2}},
  arc=1mm, % Rounded corners with 4mm radius,
  boxrule=0.5mm,
  left=2mm, right=2mm, top=2mm, bottom=2mm, % Padding inside the box
  breakable, % Allow box to be broken across pages
  before=\vspace{10pt}, % Padding above the box
  after=\vspace{10pt}, % Padding below the box
  before upper={\parindent15pt} % Ensure indentation
}

% Define the custom proof environment
\newtcolorbox{defn}[2][Definition]{
  colback=green!5!white, % Light blue background
  colframe=green!75!black, % Darker blue border
  coltitle=white, % Title color
  fonttitle=\bfseries, % Title font style
  title={{#2}},
  arc=1mm, % Rounded corners with 4mm radius,
  boxrule=0.5mm,
  left=2mm, right=2mm, top=2mm, bottom=2mm, % Padding inside the box
  breakable, % Allow box to be broken across pages
  before=\vspace{10pt}, % Padding above the box
  after=\vspace{10pt}, % Padding below the box
  before upper={\parindent15pt} % Ensure indentation
}


%%%%%%%%%%%%%%%%%%%%%%%%%%%%%%%%%%%%%%%%%%%%%%%%%%%%%%%%%%%%%%%%%


\begin{document}

% make title page
\thispagestyle{empty}
\bigskip \
\vspace{0.1cm}

\begin{center}
{\fontsize{22}{22} \selectfont Professor: Alexander Givental}
\vskip 16pt
{\fontsize{30}{30} \selectfont \bf \sffamily Math 215A: Algebraic Topology}
\vskip 24pt
{\fontsize{14}{14} \selectfont \rmfamily Homework 2} 
\vskip 6pt
{\fontsize{14}{14} \selectfont \ttfamily kdeoskar@berkeley.edu} 
\vskip 24pt
\end{center}

% {\parindent0pt \baselineskip=15.5pt \lipsum[1-4]} 

% make table of contents
% \newpage

\begin{bluebox}
  \textbf{Question 1:} For the manifold of \textit{complete} flags $V^1 \subset V^2 \subset \C^3$ in (say, complex) $3-$space describe explicitly which flags belong to which Bruhat cell.
\end{bluebox}

\vskip 0.5cm
\textbf{\underline{Solution:}}
\\
\\
The set $\C F(n; 1,2,\cdots,n-1)$ is the set of complete flags i.e. $$ \C F(n;1,2,\cdots,n-1) = \left\{ \text{chains } \left(V_1 \subset V_2 \cdots \subset V_n\right) \;|\; \mathrm{dim}_{\C}\left(V_i\right) = i \right\} $$
\\
In the Bruhat Cell Decomposition of a flag manifold, the Bruhat cells are characterized by the dimensions of intersections $d_{ij} = \mathrm{dim}\left( V_i \cup \C^j \right)$.
\\
\\
In our case, we're considering the manifold of complete flags $\C F(3;1,2)$ so we have a total of $6$ Bruhat cells, corresponding to the six permutations of the sequence $(1,2,3)$:
\begin{align*}
  &(1,2,3), \;\;\; (3,1,2), \;\;\; (2,3,1) \\
  &(1,3,2), \;\;\; (2,1,3), \;\;\; (3,2,1) \\
\end{align*} The dimension of the cell $e[m_1, \cdots, m_n]$ is equal to the number of pairs $(i,j)$ for which $i < j, m_i > m_j$
\\
So, 
\begin{align*}
  &e[1,2,3] \text{ has dimension } 0 \\
  &e[1,3,2] \text{ and } e[2,1,3] \text{ have dimension } 1 \\
  &e[3,1,2] \text{ and } e[2,3,1] \text{ have dimension } 2 \\
  &e[3,2,1] \text{ has dimension } 3 \\
\end{align*} The flags in the cells are specified as below ($V_3 = \C^3$ in all cases below) :
\\
\\
\begin{enumerate}[label=(\alph*)]
  \item The flag contained in the 0-dimensional cell $e[1,2,3]$ is $V_1 = \C^1, V_2 = \C^2, V_3 = \C^3$. 
  \\
  \item The two flags in the 1-dimensional cells are 
  \begin{itemize}
    \item $V_1 = \C^1$ and $V_2 = $ any 2-d complex plane containing $\C^1$ other than the standard copy of $\C^2$, $V_3 = \C^3$
    \item $V_2 = \C^2$ and $V_1 = $ any 1-d complex line contained in $\C^2$ other than the standard copy of $\C^1$ 
  \end{itemize}
  \vskip 0.5cm
  \item The two flags in the 2-dimensional cells are:
  \begin{itemize}
    \item $V_1$ being any line in $\C^2$ and $V_2$ being any 2d plane containing $V_1$ other than $\C^2$
    \item $V^2 \neq \C^2$ and $V_1$ being any line contained in $V^2$ which is not $\C^1$
  \end{itemize}
  \vskip 0.5cm
  \item The one 3-dimensional flag is $V_1 = $ any 1-d complex line other than $\C^1$, $V_2 = $ any 2-d complex plane other than $\C^2$
\end{enumerate}

\vskip 0.5cm
\hrule
\pagebreak





\begin{bluebox}
  \textbf{Question 2:} 
  Let $c_k$ be the number of $k-$dimensional (in complex units) Bruhat cells in the manifold of complete flags in an $n-$dimensional complex space. Show that the generating function $c_0 + c_1 q + c_2 q^2 + \cdots$ of this sequence is equal to the "$q-$factorial" : the product of $\frac{(1-q^k)}{(1-q)}$ over $k=1,\cdots, n$ and check this for your answer in $(a)$.
\end{bluebox}

\vskip 0.5cm
\textbf{\underline{Solution:}}
(Collaborated with Finn Fraser Grathwol for this question)
\\
\\
Let's consider a finite field. The complete $n-$th flag manifold over finite field $\mathbb{F}_q = \{1, \cdots, q\}$ is $\mathbb{F} F(n; 1,\cdots,n-1) = \bigslant{\mathrm{GL}_n(\mathbb{F})}{B}$ where $B$ is the subgroup of $\mathrm{GL}_n(\mathbb{F})$ formed by upper triangular matrices.
\\
\\
$\mathbb{F}_q^n$ contains $q^n$ vectors, with $q^n - 1$ of them being non-zero. Now, consider some 1-d subspace $V_1 \subseteq \mathbb{F}_q^n$. The $q-1$ non-zero scalar multiples of the $q^n - 1$ nonzero vectors span the same subspaces, so there are $$ \frac{q^n-1}{q-1} $$ vectors that could intersect $V_1$. Similarly for a 2d subspace $V_2$, since one dimension is already fixed, there are $\frac{q^{n-1} - 1}{q-1}$ choices we can make, and so on for $V_i$ until we hit $V_n$ for which the number of choices is $\frac{q-1}{q-1} = 1$.
\\
\\
Thus, we find that the number of complete flags is 
\begin{align*}
  \Pi_{k = 1}^{n} \frac{q^k - 1}{q - 1} 
\end{align*}
On the other hand, each Bruhat cell of dimension $k$ is parametrized by $k-$points (from a $k-$dimensional affine space). Thus, if we denote the number of cells as $c_k$ then we have the result.
\\
\\
This matches up with (a) wherein we have $f_3 = \left(q^3 - 1\right)\left(q^2 - 1\right)\left(q^2 - 1\right) / \left(q - 1\right)^3 = 1 + 2q + 2q^2 + 2q^3$.
\vskip 0.5cm
\hrule
\pagebreak



\begin{bluebox}
  \textbf{Question 3:} Prove that $\sph^{\infty}$ is contractible.
\end{bluebox}

\vskip 0.5cm
\textbf{\underline{Solution:}}
\\
\\
% Let's do this in two steps. Let's show
% \begin{enumerate}
%   \item For every $n \in \mathbb{N}$, $\sph^n$ is contractible in $\sph^{n+1}$.
%   \item A CW Cell $$ X = \bigcup_{i} X_i $$ with $X_0 \subset X_1 \subset \cdots X_i \subset \cdots$ is contractible if and only if $X_i$ is contractible in $X_{i+1}$.
% \end{enumerate} 
Let's denote the natural "equitorial" inclusion $x \mapsto (x, 0)$ as $\iota \text{ : } \sph^n \xhookrightarrow{} \sph^{n+1}$. Consider the following map: 
\begin{align*}
  F \text{ : } \sph^n \times I &\rightarrow \sph^{n+1} \\
  (x, t) &\mapsto \left( \sqrt{1-t^2}x, t \right)
\end{align*} This is a continuous map since its components are continuous, and we notice that $F(x, 0) = (x, 0) = \iota(x)$ and $F(x, 1) = (0, 1)$. Thus, $F(x, t)$ is a homotopy between $\iota$ and the contant map $ \sph^n \ni x \mapsto (0, 1) \in \mathbb{S}^{n+1}$. Thus, for every $n \in \mathbb{N}$, $\sph^n$ is contractible in $\sph^{n+1}$.
\\
\\
This particular homotopy can be visualized as dragging $\sph^{n}$ from the equator to the north pole of $\sph^{n+1}$, but an equivalent homotopy would be to imagine one point $x_0$ on the inclusion of $\sph^n$ to be fixed and to drag the rest of $\sph^n$ over the surface of $\sph^{n+1}$, passing the nole pole, and collecting into the fixed point $x_0$.
\\
\\
Now, to make $\sph^{\infty}$ contractible we can extend homotopies between the finite-dimensional spheres to $\sph^{\infty}$ using Borsuk's theorem.
\\
\\
For time-interval $[0, 1/2)$ contract $\sph^{1}$ to a point $x_0 \in \sph^2$, and then extend the homotopy $F_1 \text{ : } \sph^1 \times I \rightarrow \sph^2$ to a homotopy from $S^{\infty} \times I$ to $\sph^2$, for time-interval $[1/2, 3/4)$ contract $\sph^2$ to a point $x_0 \in \sph^3$ (imagining this to be the same fixed point mentioned earlier) and similarly extend it to a homotopy on $\sph^{\infty}$ using Borsuk's Theorem.
\\
\\
Doing this for all $\sph^{n}$, and taking the composition of all the homotopies we get a  map $\sph^{\infty} \times [0, 1) \rightarrow \sph^{\infty}$ (we never actually hit $t = 1$ in the description above since we keep going for infinitely many $n$), which we can extend to a map $\sph^{\infty} \times [0, 1] \rightarrow \sph^{\infty}$ such that the map $\sph^{\infty} \times 0 \rightarrow \sph^{\infty}$ is just the identity and the map $\sph^{1} \times 1 \rightarrow \sph^{\infty}$ is the constant map to $x_0$.
\\
\\
Thus, we have $\sph^{\infty}$ is contractible to a point.
\vskip 0.5cm
\hrule
\pagebreak








% \begin{bluebox}
%   \textbf{Question 1:} 
% \end{bluebox}

% \vskip 0.5cm
% \textbf{\underline{Solution:}}
% \\
% \\
% text
% \vskip 0.5cm
% \hrule
% \pagebreak



% %%%%%%%%%%%%%%%%%%%%%%%%%%%%%%%%%%%%%%%%%%%%%%
% \newpage
% % \section{References}
% %%%%%%%%%%%%%%%%%%%%%%%%%%%%%%%%%%%%%%%%%%%%%%
% \vskip 0.5cm
% \bibliographystyle{plain} % We choose the "plain" reference style
% \bibliography{citation}




\end{document}










