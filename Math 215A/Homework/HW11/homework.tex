\documentclass[11pt]{article}

% basic packages
\usepackage[margin=1in]{geometry}
\usepackage[pdftex]{graphicx}
\usepackage{amsmath,amssymb,amsthm}
\usepackage{custom}
\usepackage{lipsum}

\usepackage{xcolor}
\usepackage{tikz-cd}

\usepackage[most]{tcolorbox}
\usepackage{xcolor}
\usepackage{mdframed}

% page formatting
\usepackage{fancyhdr}
\pagestyle{fancy}

\renewcommand{\sectionmark}[1]{\markright{\textsf{\arabic{section}. #1}}}
\renewcommand{\subsectionmark}[1]{}
\lhead{\textbf{\thepage} \ \ \nouppercase{\rightmark}}
\chead{}
\rhead{}
\lfoot{}
\cfoot{}
\rfoot{}
\setlength{\headheight}{14pt}

\linespread{1.03} % give a little extra room
\setlength{\parindent}{0.2in} % reduce paragraph indent a bit
\setcounter{secnumdepth}{2} % no numbered subsubsections
\setcounter{tocdepth}{2} % no subsubsections in ToC


%%%%%%%%%%%%%%%%%%%%%%%%%%%%%%%%%%%%%%%%%%%%%%%%%%%%%%%%%%%%%%%%%
% CUSTOM BOXES AND STUFF
\newtcolorbox{redbox}{colback=red!5!white,colframe=red!75!black, breakable}
\newtcolorbox{bluebox}{colback=blue!5!white,colframe=blue!75!black,breakable}

\newtcolorbox{dottedbox}[1][]{%
    colback=white,    % Background color
    colframe=white,    % Border color (to be overridden by dashrule)
    sharp corners,     % Sharp corners for the box
    boxrule=0pt,       % No actual border, as it will be drawn with dashrule
    boxsep=5pt,        % Padding inside the box
    enhanced,          % Enable advanced features
    breakable,         % Enables it to span multiple pages
    overlay={\draw[dashed, thin, black, dash pattern=on \pgflinewidth off \pgflinewidth, line cap=rect] (frame.south west) rectangle (frame.north east);}, % Dotted line
    #1                 % Additional options
}

% Define the colors
\definecolor{boxheader}{RGB}{0, 51, 102}  % Dark blue
\definecolor{boxfill}{RGB}{173, 216, 230}  % Light blue


% Define the tcolorbox environment
\newtcolorbox{mathdefinitionbox}[2][]{%
    colback=boxfill,   % Background color
    colframe=boxheader, % Border color
    fonttitle=\bfseries, % Bold title
    coltitle=white,     % Title text color
    title={#2},         % Title text
    enhanced,           % Enable advanced features
    breakable,
    attach boxed title to top left={yshift=-\tcboxedtitleheight/2}, % Center title
    boxrule=0.5mm,      % Border width
    sharp corners,      % Sharp corners for the box
    #1                  % Additional options
}
%%%%%%%%%%%%%%%%%%%%%%%%%


\definecolor{lightblue}{RGB}{173,216,230} % Light blue color
\definecolor{darkblue}{RGB}{0,0,139} % Dark blue color

% Define the custom proof environment
\newtcolorbox{ex}[2][Example]{
  colback=red!5!white, % Light blue background
  colframe=red!75!black, % Darker blue border
  coltitle=white, % Title color
  fonttitle=\bfseries, % Title font style
  title={{#2}},
  arc=1mm, % Rounded corners with 4mm radius,
  boxrule=0.5mm,
  left=2mm, right=2mm, top=2mm, bottom=2mm, % Padding inside the box
  breakable, % Allow box to be broken across pages
  before=\vspace{10pt}, % Padding above the box
  after=\vspace{10pt}, % Padding below the box
  before upper={\parindent15pt} % Ensure indentation
}

% Define the custom proof environment
\newtcolorbox{defn}[2][Definition]{
  colback=green!5!white, % Light blue background
  colframe=green!75!black, % Darker blue border
  coltitle=white, % Title color
  fonttitle=\bfseries, % Title font style
  title={{#2}},
  arc=1mm, % Rounded corners with 4mm radius,
  boxrule=0.5mm,
  left=2mm, right=2mm, top=2mm, bottom=2mm, % Padding inside the box
  breakable, % Allow box to be broken across pages
  before=\vspace{10pt}, % Padding above the box
  after=\vspace{10pt}, % Padding below the box
  before upper={\parindent15pt} % Ensure indentation
}


%%%%%%%%%%%%%%%%%%%%%%%%%%%%%%%%%%%%%%%%%%%%%%%%%%%%%%%%%%%%%%%%%


\begin{document}

% make title page
\thispagestyle{empty}
\bigskip \
\vspace{0.1cm}

\begin{center}
{\fontsize{22}{22} \selectfont Professor: Alexander Givental}
\vskip 16pt
{\fontsize{30}{30} \selectfont \bf \sffamily Math 215A: Algebraic Topology}
\vskip 24pt
{\fontsize{14}{14} \selectfont \rmfamily Homework 11} 
\vskip 6pt
{\fontsize{14}{14} \selectfont \ttfamily kdeoskar@berkeley.edu} 
\vskip 24pt
\end{center}

% {\parindent0pt \baselineskip=15.5pt \lipsum[1-4]} 

% make table of contents
% \newpage



\begin{bluebox}
  \textbf{Question 1:} Prove the \textbf{Morse Lemma:} \textit{A smooth function in a neighborhood of a non-degenerate critical point with zero critical value is locally diffeomorphic to the quadratic form defined by its second differential. }
  % \\
  % \\
  % Namely, let $f_t$ be the linear interpolation between the function $f_1$ and the quadratic form $f_0$
\end{bluebox}

\vskip 0.5cm
\textbf{\underline{Solution:}}
\\
\\
The approach: 
\begin{enumerate}
  \item Let $f_T$ be a linear interpolation between the function $f$ and the quadratic form $f_0$.
  \item Apply the homotopy method: 
  \begin{itemize}
    \item Look for a family of local diffeomorphisms $g_t$ such that $f_t(g_t(x)) = f_0(x)$. Differentiate this relation in $t$ in order to obtain an infinitesimal version of the equation, which would require finding a time-dependent family of vector fields $v_t$ from which $g_t$ can be recovered using the uniqueness and existence theorem for solutions of ordinary differential equations. 
    \item To solve that infinitesimal equation for $v_t$, the following Hadamard's lemma can be useful: In $\R^n$ with coordinates $y_1,...,y_n$, a smooth function vanishing at the origin can be written as $y_1G_1(y)+...+y_nG_n(y)$ where $G_i$ are some smooth functions. If you use it, prove it too (of find its proof somewhere).
  \end{itemize}
\end{enumerate}
\vskip 0.5cm
\hrule
\pagebreak




\begin{bluebox}
  \textbf{Question 2:} In $\mathbb{CP}^2 \times \mathbb{CP}^2$, consider the hypersurface defined $F$ given by $x_1y_1 + x_2y_2 + x_3y_3 = 0$ where $(x_1, x_2, x_3)$ and $(y_1, y_2, y_3)$ are homogeneous coordinates on the left and right projective planes respectively. Identify $F$ with the manifold of complete flags in $\C^3$, find the Kernel of the homomorphism $$ \mathbb{Z}[u,v]/(u^3, v^3) $$ where $u, v$ are the generators in the cohomology algebras of the left and right projective planes Poincar\'e-dual to projective lines therein, and show the isomorphism is surjective.
\end{bluebox}

\vskip 0.5cm
\textbf{\underline{Solution:}}
\\
\\
text
\vskip 0.5cm
\hrule
\pagebreak




\begin{bluebox}
  \textbf{Question 3:} Use Intersection Theory to prove the classical \textbf{Borsuk-Ulam Theorem:} \textit{$n-$odd continuous functions on $\sph^n$ have a common zero.}
\end{bluebox}

\vskip 0.5cm
\textbf{\underline{Solution:}}
\\
\\
text
\vskip 0.5cm
\hrule
\pagebreak










% \begin{bluebox}
%   \textbf{Question 1:} 
% \end{bluebox}

% \vskip 0.5cm
% \textbf{\underline{Solution:}}
% \\
% \\
% text
% \vskip 0.5cm
% \hrule
% \pagebreak



% %%%%%%%%%%%%%%%%%%%%%%%%%%%%%%%%%%%%%%%%%%%%%%
% \newpage
% % \section{References}
% %%%%%%%%%%%%%%%%%%%%%%%%%%%%%%%%%%%%%%%%%%%%%%
% \vskip 0.5cm
% \bibliographystyle{plain} % We choose the "plain" reference style
% \bibliography{citation}




\end{document}










