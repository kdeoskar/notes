\documentclass[11pt]{article}

% basic packages
\usepackage[margin=1in]{geometry}
\usepackage[pdftex]{graphicx}
\usepackage{amsmath,amssymb,amsthm}
\usepackage{custom}
\usepackage{lipsum}

\usepackage{xcolor}
\usepackage{tikz-cd}

\usepackage[most]{tcolorbox}
\usepackage{xcolor}
\usepackage{mdframed}

% page formatting
\usepackage{fancyhdr}
\pagestyle{fancy}

\renewcommand{\sectionmark}[1]{\markright{\textsf{\arabic{section}. #1}}}
\renewcommand{\subsectionmark}[1]{}
\lhead{\textbf{\thepage} \ \ \nouppercase{\rightmark}}
\chead{}
\rhead{}
\lfoot{}
\cfoot{}
\rfoot{}
\setlength{\headheight}{14pt}

\linespread{1.03} % give a little extra room
\setlength{\parindent}{0.2in} % reduce paragraph indent a bit
\setcounter{secnumdepth}{2} % no numbered subsubsections
\setcounter{tocdepth}{2} % no subsubsections in ToC


%%%%%%%%%%%%%%%%%%%%%%%%%%%%%%%%%%%%%%%%%%%%%%%%%%%%%%%%%%%%%%%%%
% CUSTOM BOXES AND STUFF
\newtcolorbox{redbox}{colback=red!5!white,colframe=red!75!black, breakable}
\newtcolorbox{bluebox}{colback=blue!5!white,colframe=blue!75!black,breakable}

\newtcolorbox{dottedbox}[1][]{%
    colback=white,    % Background color
    colframe=white,    % Border color (to be overridden by dashrule)
    sharp corners,     % Sharp corners for the box
    boxrule=0pt,       % No actual border, as it will be drawn with dashrule
    boxsep=5pt,        % Padding inside the box
    enhanced,          % Enable advanced features
    breakable,         % Enables it to span multiple pages
    overlay={\draw[dashed, thin, black, dash pattern=on \pgflinewidth off \pgflinewidth, line cap=rect] (frame.south west) rectangle (frame.north east);}, % Dotted line
    #1                 % Additional options
}

% Define the colors
\definecolor{boxheader}{RGB}{0, 51, 102}  % Dark blue
\definecolor{boxfill}{RGB}{173, 216, 230}  % Light blue


% Define the tcolorbox environment
\newtcolorbox{mathdefinitionbox}[2][]{%
    colback=boxfill,   % Background color
    colframe=boxheader, % Border color
    fonttitle=\bfseries, % Bold title
    coltitle=white,     % Title text color
    title={#2},         % Title text
    enhanced,           % Enable advanced features
    breakable,
    attach boxed title to top left={yshift=-\tcboxedtitleheight/2}, % Center title
    boxrule=0.5mm,      % Border width
    sharp corners,      % Sharp corners for the box
    #1                  % Additional options
}
%%%%%%%%%%%%%%%%%%%%%%%%%


\definecolor{lightblue}{RGB}{173,216,230} % Light blue color
\definecolor{darkblue}{RGB}{0,0,139} % Dark blue color

% Define the custom proof environment
\newtcolorbox{ex}[2][Example]{
  colback=red!5!white, % Light blue background
  colframe=red!75!black, % Darker blue border
  coltitle=white, % Title color
  fonttitle=\bfseries, % Title font style
  title={{#2}},
  arc=1mm, % Rounded corners with 4mm radius,
  boxrule=0.5mm,
  left=2mm, right=2mm, top=2mm, bottom=2mm, % Padding inside the box
  breakable, % Allow box to be broken across pages
  before=\vspace{10pt}, % Padding above the box
  after=\vspace{10pt}, % Padding below the box
  before upper={\parindent15pt} % Ensure indentation
}

% Define the custom proof environment
\newtcolorbox{defn}[2][Definition]{
  colback=green!5!white, % Light blue background
  colframe=green!75!black, % Darker blue border
  coltitle=white, % Title color
  fonttitle=\bfseries, % Title font style
  title={{#2}},
  arc=1mm, % Rounded corners with 4mm radius,
  boxrule=0.5mm,
  left=2mm, right=2mm, top=2mm, bottom=2mm, % Padding inside the box
  breakable, % Allow box to be broken across pages
  before=\vspace{10pt}, % Padding above the box
  after=\vspace{10pt}, % Padding below the box
  before upper={\parindent15pt} % Ensure indentation
}


%%%%%%%%%%%%%%%%%%%%%%%%%%%%%%%%%%%%%%%%%%%%%%%%%%%%%%%%%%%%%%%%%


\begin{document}

% make title page
\thispagestyle{empty}
\bigskip \
\vspace{0.1cm}

\begin{center}
{\fontsize{22}{22} \selectfont Professor: Alexander Givental}
\vskip 16pt
{\fontsize{30}{30} \selectfont \bf \sffamily Math 215A: Algebraic Topology}
\vskip 24pt
{\fontsize{14}{14} \selectfont \rmfamily Homework 6} 
\vskip 6pt
{\fontsize{14}{14} \selectfont \ttfamily kdeoskar@berkeley.edu} 
\vskip 24pt
\end{center}

% {\parindent0pt \baselineskip=15.5pt \lipsum[1-4]} 

% make table of contents
% \newpage


\begin{bluebox}
  \textbf{Question 1:} Compute $\pi_2(\sph^1 \vee \sph^2)$ and the action of $\pi_1(\sph^1 \vee \sph^2)$ on it.
\end{bluebox}

\vskip 0.5cm
\textbf{\underline{Solution:}} (Inspired by \href{https://math.stackexchange.com/questions/1675337/calculate-pi-2s2-vee-s1}{this stackexchange post}.)
\\
\\
To compute $\pi_2(\sph^1 \vee \sph^2)$ we'll use the following lemma:

\begin{redbox}
  \begin{lemma}
    For the universal cover $\widetilde{U} \rightarrow U$ of a CW Complex $U$, we have $$ \pi_n(\widetilde{X}) \cong \pi_n(X) $$ for all $n \in \mathbb{N}$.
  \end{lemma}
\end{redbox}
\vskip 0.25cm
\begin{proof}
  
\end{proof} \\
\\
Going back to the computation, we have $$ \pi_2(\sph^1 \vee \sph^2) \cong \pi_2(\widetilde{ \sph^1 \vee \sph^2 }) $$ where $\widetilde{\sph^1 \vee \sph^2}$ is the universal covering of $\sph^1 \vee \sph^2$, visualized as:
\begin{center}
  Include picture
\end{center} Then, contracting each of the segments between the consecutive integers, we have $$ \widetilde{\sph^1 \vee \sph^2} \cong {\bigvee}_{k \in \mathbb{Z}} \sph^2_{k} $$ where each $\sph^2_k$ is a copy of $\sph^2$, labelled by the integer $k$. 
\\
\\
So, we have 
\begin{align*}
  \pi_2\left(\widetilde{\sph^1 \vee \sph^2}\right) &\cong \pi_2 \left( {\bigvee}_{k \in \mathbb{Z}} \sph^2_{k} \right) 
\end{align*} What is this space? \begin{note}
  {Fill this in.}
\end{note}
\\
\\
Thus,
\[ \boxed{ \pi_2(\sph^1 \vee \sph^2) \cong \bigoplus_{k \in \mathbb{Z}} \mathbb{Z} } \] \\
\\
\underline{Action of $\pi_1$:} Generally, $\pi_1(X)$ acts on $\pi_n(X)$ ($n \geq 1$) by "prepending" a loop i.e. move along a circle before an $n-$spheriod.
\\
\\
The inclusion of $\sph^2 \hookrightarrow X \text{:}= \sph^1 \vee \sph^2$ gives us an element $\alpha \in \pi_2(X)$, which generates a cyclic subgroup of $\pi_2(X)$. 
\\
\\
However, notice that if we consider a loop $\gamma$ that goes around the $\sph^1$ factor once in $\sph^1 \vee \sph^2$ then first moving along $\gamma$ brings us back to the basepoint in $\sph^1 \vee \sph^2$ so following it up with some $\alpha \in \pi_2(X)$ is just another element of 2-spheroid i.e. $\alpha \cdot \gamma \in \pi_2(X)$. This $\gamma \cdot \alpha$ also generates a cyclic subgroup of $\pi_2(X)$. Continuing on with this patter we can see that $\gamma^n \circ \alpha \in \pi_2(X)$ for every $n \in \mathbb{N}$ and each of these generate (disjoint) cyclic subgroups.

\vskip 0.5cm
\hrule
\pagebreak



\begin{bluebox}
  \textbf{Question 2:} Compute the 2nd Homotopy Groups of Grassmannians $G(n, k)$ when $k, n-k > 1$
\end{bluebox}

\vskip 0.5cm
\textbf{\underline{Solution:}}
\\
\\
Let $G(n, k)$ denote the set of $k-$dimensionl subspaces of $\R^n$. We know that $$ G(n,k)\cong O(n) / ( O(n) \times O(n-k) ) $$
so we have a (serre) fibration $$ O(n-k) \hookrightarrow O(n) \rightarrow G(n,k) $$ which induces the exact sequence $$
\pi_2\left( O(n) \right) \rightarrow \pi_2(G(n,k)) \rightarrow \pi_1 \left(O(n) \times O(n-k)\right) = \pi_1(O(n)) \times \pi_1(O(n-k)) \rightarrow \pi_1(O(n))
$$ 
\\
\\
Now, to actually calculate $\pi_2(G(n,k))$ for the different $n, k$ values we'll need to use the following results (common in the literature) which can be obtained using fibrations as well:
\\
\\
\begin{enumerate}[label=(\alph*)]
  \item $$\pi_1(O(N)) = \begin{cases}
    \mathbb{Z}_2,~n = 1 \\
    \mathbb{Z},~n = 2 \\
    \mathbb{Z}_2,~n \geq 3 \\
  \end{cases}$$

  \item $$
  \pi_2(O(2)) = 0
  $$

  \item $$
  \pi_2(O(3)) = 0
  $$

  \item When it comes to the homotopy groups of $O(N)$ for $N \in \mathbb{Z}$ we have, for $n \geq k + 2$, by \textbf{Bott Periodicity}:
  $$ 
  \pi_k(O(N)) \cong \pi_k(SO(N)) = \begin{cases}
    0,~~k = 2,4,5,6~\text{(mod 8)} \\
    \mathbb{Z}_2,~~k = 0,1~\text{(mod 8)} \\
    \mathbb{Z},~~k = 7~\text{(mod 8)}
  \end{cases}
  $$ 
\end{enumerate}

Now, in our question, we have the following cases (we're considering $n, n-k > 1$)

\begin{enumerate}
  \item \underline{$n, k$ such that $n, (n-k) \geq 3$ $n - k > 2$ and:}
  \begin{align*}
    &\pi_2(O(n)) \rightarrow \pi_2(G(n, k)) \rightarrow \pi_1(O(n)) \times \pi_1(O(n-k)) \rightarrow \pi_1(O(n)) \\
    % \text{i.e. } & 0 \rightarrow \pi_2(G(n, k)) \rightarrow \pi_1()
  \end{align*}
\end{enumerate}



\vskip 0.5cm
\hrule
\pagebreak



\begin{bluebox}
  \textbf{Question 3:} Let $X$ be a $K(G, n)$ and $Y$ a cellular $K(H, n)$. Show that the map of $Y$ to $X$ inducing a given group homomorphism $\phi \text{ : } H \rightarrow G$ exists, and is unique up to homotopy.
\end{bluebox}

\vskip 0.5cm
\textbf{\underline{Solution:}}
\\
\\
We know there exists group homomorphism $\phi \text{ : } H \rightarrow G$ and recall that an Eilenberg-Maclane Space $X = K(F, n)$ is one for which
\begin{align*}
  \pi_k(X) &= \begin{cases}
    F,~ k = n \\
    0,~ k \neq n
  \end{cases}
\end{align*}
Now, we have $X = K(G, n)$, $Y = K(H, n)$, and $Y$ is known to be a CW complex. We can assume that $Y$ is a CW Complex obtained from $\mathrm{sk_n} Y \text{:}= \bigwedge_{\alpha} \sph^n_{\alpha}$ where each $\alpha$ corresponds to a generator of $H$ and attaching all cells (attach $(n+1)-$cells according to the relations in $H$ and \begin{note}{Fill in some more.}\end{note})
\\
\\
We can define $f_n \text{ : } \mathrm{sk}_n Y \rightarrow X $ by mapping each $S_{\alpha}^{n}$ to a corresponding spheroid $f_n(\alpha) \in \pi_n(X, x_0)$.
\\
\\
Also, the attaching maps $\partial D^{n+1} \rightarrow \mathrm{sk}_n Y$ for $(n+1)-$dimensional cells represent the identity element in $H$ (so its image under $f_n$ is trivial in $G$
+), which we can use to extend $f_n$ to $f_{n+1} \text{ : } \mathrm{sk}_{n+1} Y \rightarrow X$. 
\\
\\
We can do this inductively to extend $f_{n+1}$ to $f_{n+k} \text{ : } \mathrm{sk}_{n+k} X \rightarrow Y$ with $k > 1$.
\\
\\
We want to show that any two maps $f, g \text{ : } X \rightarrow Y$ which induce a given homomorphism $\phi \text{ : } H \rightarrow G$ are the same up to homotopy. To do so, let's use cell induction.
\\
\\
Suppose $y_0$ and $x_0$ are the basepoints on $Y$ and $X$ respectively. For the base case, let's assume there exists some path in $X$ between $f(y_0)$ and $g(y_0)$. This path then gives us the homotopy between $\restr{f}{\mathrm{sk}_0  Y} \text{ : } \mathrm{sk}_0 Y \rightarrow X$ and $\restr{g}{\mathrm{sk}_0 Y} \text{ : } \mathrm{sk}_0 Y \rightarrow X$.
\\
\\
For the induction step, suppose we already have a homotopy $h_{k-1} \times [0, 1] \text{ : }
\mathrm{sk}_{k-1} Y \rightarrow X $ between $\restr{f}{\mathrm{sk}_{k-1} Y}$ and $\restr{g}{\mathrm{sk}_{k-1} Y}$ and a $k-$cell $D^k$ of $Y$ with characteristic map $\Phi \text{ : } D^k \rightarrow \mathrm{sk}_k Y$. We want to extend this to a map $D^k \times [0, 1] \rightarrow X$ such that the map agrees with $f \circ \Phi$ on $D^k \times \{0\}$, with $g \circ \Phi$ on $D^k \times \{1\}$, and with $\restr{h \circ \Phi}{partial D^{k}}$ on $\partial D^k \times [0, 1]$.
\\
\\
Note that $(D^k \times \{0\}) \cup (D^k \times \{1\}) \cup (\partial D^k \times [0, 1]) \approx \partial \left(D^k \times [0, 1]\right)$, so all of the conditions listed above together encode a $k-$spheroid $\partial \left( D^k \times [0, 1] \right) \rightarrow X$, and the extension to $D^k \times [0, 1]$ we desired can be define if the spheroid can be contracted in $X$. Since we're working with Eilenberg-Maclane spaces, we have $\pi_k(X) = 0$ for $k \neq n$, so we can make the extension. For the $k = n$ case we can argue that the claim holds because $\restr{f}{\mathrm{sk}_{n} Y}$ and $\restr{g}{\mathrm{sk}_{n} Y}$ represent the same element in $G = \pi_n(X)$ and are thus homotopic.

\vskip 0.5cm
\hrule
\pagebreak













% \begin{bluebox}
%   \textbf{Question 1:} 
% \end{bluebox}

% \vskip 0.5cm
% \textbf{\underline{Solution:}}
% \\
% \\
% text
% \vskip 0.5cm
% \hrule
% \pagebreak



% %%%%%%%%%%%%%%%%%%%%%%%%%%%%%%%%%%%%%%%%%%%%%%
% \newpage
% % \section{References}
% %%%%%%%%%%%%%%%%%%%%%%%%%%%%%%%%%%%%%%%%%%%%%%
% \vskip 0.5cm
% \bibliographystyle{plain} % We choose the "plain" reference style
% \bibliography{citation}




\end{document}










