\documentclass{article}

% Language setting
% Replace `english' with e.g. `spanish' to change the document language
\usepackage[english]{babel}

% Set page size and margins
% Replace `letterpaper' with`a4paper' for UK/EU standard size
\usepackage[letterpaper,top=2cm,bottom=2cm,left=3cm,right=3cm,marginparwidth=1.75cm]{geometry}

% Useful packages
\usepackage{amsmath}
\usepackage{amssymb}
\usepackage{graphicx}
\usepackage{enumitem}
\usepackage[colorlinks=true, allcolors=blue]{hyperref}
\usepackage[most]{tcolorbox}

\usepackage{hyperref}
\hypersetup{
    colorlinks=true,
    linkcolor=blue,
    filecolor=magenta,      
    urlcolor=cyan,
    pdftitle={Overleaf Example},
    pdfpagemode=FullScreen,
    }

\urlstyle{same}

\usepackage{tikz-cd}


%%%%%%%%%%% New Commands %%%%%%%%%%%%%%
\newcommand*{\T}{\mathcal T}
\newcommand*{\cl}{\text cl}
\newcommand{\bigslant}[2]{{\raisebox{.2em}{$#1$}\left/\raisebox{-.2em}{$#2$}\right.}}
\newcommand{\restr}[2]{{% we make the whole thing an ordinary symbol
  \left.\kern-\nulldelimiterspace % automatically resize the bar with \right
  #1 % the function
  \vphantom{\big|} % pretend it's a little taller at normal size
  \right|_{#2} % this is the delimiter
  }}
%%%%%%%%%%%%%%%%%%%%%%%%%%%%%%%%%%%%%%%


\newtcolorbox{dottedbox}[1][]{%
    colback=white,    % Background color
    colframe=white,    % Border color (to be overridden by dashrule)
    sharp corners,     % Sharp corners for the box
    boxrule=0pt,       % No actual border, as it will be drawn with dashrule
    boxsep=5pt,        % Padding inside the box
    enhanced,          % Enable advanced features
    overlay={\draw[dashed, thin, black, dash pattern=on \pgflinewidth off \pgflinewidth, line cap=rect] (frame.south west) rectangle (frame.north east);}, % Dotted line
    #1                 % Additional options
}

\tcbset{theostyle/.style={
    enhanced,
    sharp corners,
    attach boxed title to top left={
      xshift=-1mm,
      yshift=-4mm,
      yshifttext=-1mm
    },
    top=1.5ex,
    colback=white,
    colframe=blue!75!black,
    fonttitle=\bfseries,
    boxed title style={
      sharp corners,
    size=small,
    colback=blue!75!black,
    colframe=blue!75!black,
  } 
}}

\newtcbtheorem[number within=section]{Theorem}{Theorem}{%
  theostyle
}{thm}

\newtcbtheorem[number within=section]{Definition}{Definition}{%
  theostyle
}{def}



\title{Introduction to Smooth Manifolds - Some solutions}
\author{Keshav Balwant Deoskar}

\begin{document}
\maketitle

\begin{abstract}
This is a collection of my personal solutions to some exercises from the book 'Introduction to Smooth Manifolds' (2nd edition) by John M. Lee.

These solutions are being written up and I work through the book, however I do not write up every exercise/problem and leave many to be updated later.

This has been written solely to deepen my own understanding of the material, but please feel free to contact me with corrections, concerns, or comments at kdeoskar@berkeley.edu.
\end{abstract}

% \pagebreak 

\tableofcontents

\pagebreak

%%%%%%%%%%%%%%%%%%%%%%%%%%%%%%%%%%%%%%%%%%%%%%%%%%%%%%%%%%%%%%%%%%
\section{Chapter 1: Smooth Manifolds}
%%%%%%%%%%%%%%%%%%%%%%%%%%%%%%%%%%%%%%%%%%%%%%%%%%%%%%%%%%%%%%%%%%
\vskip 0.5cm

%%%%%%%%%%%%%%%%%%%%%%%%%%%%%%%%%%%%%%%%%%%%%%%%%%%%%%%%%%%%%%%%%%
\subsection{Topological Manifolds}
%%%%%%%%%%%%%%%%%%%%%%%%%%%%%%%%%%%%%%%%%%%%%%%%%%%%%%%%%%%%%%%%%%

\vskip 0.5cm

\textbf{Exercise 1.6:} Show that $\mathbb{RP}^n$ is Hausdorff and second-countable, and is therefore a topological $n$-manifold. 

\vskip 0.5cm
\textbf{Proof:}

\vskip 0.5cm
To show Hausdorffness, consider two distinct points $x, y \in \mathbb{RP}^n, x \neq y$. Then, 

\vskip 0.5cm
To show second-countability, we use the following lemma:
\begin{dottedbox}
    For a topological space $X$ with countable basis $\mathcal{B}$
  \begin{enumerate}[label=(\alph*)]
    \item Any of its subsets $Y \subseteq X$ has a countable basis as well ($\mathcal{B}_Y = \{U : U = B \cap Y, B \in \mathcal{B}\}$)
    \item If $Z = \bigslant{X}{\sim}$ is a quotient space of $X$ by an equivalence relation $\sim$, then 
    \[ \{ [x]_{\sim} : x \in B \in \mathcal{B} \} \]
    will be a countable base for the topology induced on $Z$ 
\end{enumerate}
\vskip 0.5cm

\underline{\textbf{Proof of Lemma:}} Do later.
\end{dottedbox}

\vskip 0.5cm
Applying $(a)$ tells us that $\mathbb{R}^{n+1} \setminus \{0\}$ as a subspace of $\mathbb{R}^n$ has a countable basis. Then, applying $(b)$ with $Z = \mathbb{RP}^n = \bigslant{\mathbb{R}^{n+1} \setminus \{0\}}{\sim}$ where $x \sim y$ if and only if $x = \lambda y$ for some $\lambda \in \mathbb{R}$, tells us that $\mathbb{RP}^n$ too has a countable basis.
 
\vskip 0.5cm
\hrule
\vskip 0.5cm

\textbf{Exercise 1.7:} Show that $\mathbb{RP}^n$ is compact.  

\vskip 0.5cm

\textbf{Proof:}

\vskip 0.5cm
\hrule
\vskip 0.5cm

\textbf{Exercise 1.:} 

\vskip 0.5cm

\textbf{Proof:}

\vskip 0.5cm
\hrule
\vskip 0.5cm


\end{document}

