\documentclass{article}

% Language setting
% Replace `english' with e.g. `spanish' to change the document language
\usepackage[english]{babel}

% Set page size and margins
% Replace `letterpaper' with`a4paper' for UK/EU standard size
\usepackage[letterpaper,top=2cm,bottom=2cm,left=3cm,right=3cm,marginparwidth=1.75cm]{geometry}

% Useful packages
\usepackage{amsmath}
\usepackage{amssymb}
\usepackage{graphicx}
\usepackage{enumitem}
\usepackage[colorlinks=true, allcolors=blue]{hyperref}
\usepackage[most]{tcolorbox}

\usepackage{hyperref}
\hypersetup{
    colorlinks=true,
    linkcolor=blue,
    filecolor=magenta,      
    urlcolor=cyan,
    pdftitle={Overleaf Example},
    pdfpagemode=FullScreen,
    }

\urlstyle{same}

\usepackage{tikz-cd}


%%%%%%%%%%% New Commands %%%%%%%%%%%%%%
\newcommand*{\T}{\mathcal T}
\newcommand*{\cl}{\text cl}
\newcommand{\bigslant}[2]{{\raisebox{.2em}{$#1$}\left/\raisebox{-.2em}{$#2$}\right.}}
\newcommand{\restr}[2]{{% we make the whole thing an ordinary symbol
  \left.\kern-\nulldelimiterspace % automatically resize the bar with \right
  #1 % the function
  \vphantom{\big|} % pretend it's a little taller at normal size
  \right|_{#2} % this is the delimiter
  }}
%%%%%%%%%%%%%%%%%%%%%%%%%%%%%%%%%%%%%%%


\newtcolorbox{dottedbox}[1][]{%
    colback=white,    % Background color
    colframe=white,    % Border color (to be overridden by dashrule)
    sharp corners,     % Sharp corners for the box
    boxrule=0pt,       % No actual border, as it will be drawn with dashrule
    boxsep=5pt,        % Padding inside the box
    enhanced,          % Enable advanced features
    overlay={\draw[dashed, thin, black, dash pattern=on \pgflinewidth off \pgflinewidth, line cap=rect] (frame.south west) rectangle (frame.north east);}, % Dotted line
    #1                 % Additional options
}

\tcbset{theostyle/.style={
    enhanced,
    sharp corners,
    attach boxed title to top left={
      xshift=-1mm,
      yshift=-4mm,
      yshifttext=-1mm
    },
    top=1.5ex,
    colback=white,
    colframe=blue!75!black,
    fonttitle=\bfseries,
    boxed title style={
      sharp corners,
    size=small,
    colback=blue!75!black,
    colframe=blue!75!black,
  } 
}}

\newtcbtheorem[number within=section]{Theorem}{Theorem}{%
  theostyle
}{thm}

\newtcbtheorem[number within=section]{Definition}{Definition}{%
  theostyle
}{def}



\title{Introduction to Smooth Manifolds - Some solutions}
\author{Keshav Balwant Deoskar}

\begin{document}
\maketitle

\begin{abstract}
This is a collection of my personal solutions to some exercises from the book 'Introduction to Smooth Manifolds' (2nd edition) by John M. Lee.

These solutions are being written up and I work through the book, however I do not write up every exercise/problem and leave many to be updated later.

This has been written solely to deepen my own understanding of the material, but please feel free to contact me with corrections, concerns, or comments at kdeoskar@berkeley.edu.
\end{abstract}

% \pagebreak 

\tableofcontents

\pagebreak

%%%%%%%%%%%%%%%%%%%%%%%%%%%%%%%%%%%%%%%%%%%%%%%%%%%%%%%%%%%%%%%%%%
\section{Chapter 1: Smooth Manifolds}
%%%%%%%%%%%%%%%%%%%%%%%%%%%%%%%%%%%%%%%%%%%%%%%%%%%%%%%%%%%%%%%%%%
\vskip 0.5cm

%%%%%%%%%%%%%%%%%%%%%%%%%%%%%%%%%%%%%%%%%%%%%%%%%%%%%%%%%%%%%%%%%%
\subsection{Topological Manifolds}
%%%%%%%%%%%%%%%%%%%%%%%%%%%%%%%%%%%%%%%%%%%%%%%%%%%%%%%%%%%%%%%%%%

\vskip 0.5cm

\textbf{Exercise 1.6:} Show that $\mathbb{RP}^n$ is Hausdorff and second-countable, and is therefore a topological $n$-manifold. 

\vskip 0.5cm
\textbf{Proof:}

\vskip 0.5cm
To show Hausdorffness, consider two distinct points $x, y \in \mathbb{RP}^n, x \neq y$. Then, 

\vskip 0.5cm
To show second-countability, we use the following lemma:
\begin{dottedbox}
    For a topological space $X$ with countable basis $\mathcal{B}$
  \begin{enumerate}[label=(\alph*)]
    \item Any of its subsets $Y \subseteq X$ has a countable basis as well ($\mathcal{B}_Y = \{U : U = B \cap Y, B \in \mathcal{B}\}$)
    \item If $Z = \bigslant{X}{\sim}$ is a quotient space of $X$ by an equivalence relation $\sim$, then 
    \[ \{ [x]_{\sim} : x \in B \in \mathcal{B} \} \]
    will be a countable base for the topology induced on $Z$ 
\end{enumerate}
\vskip 0.5cm

\underline{\textbf{Proof of Lemma:}} Do later.
\end{dottedbox}

\vskip 0.5cm
Applying $(a)$ tells us that $\mathbb{R}^{n+1} \setminus \{0\}$ as a subspace of $\mathbb{R}^n$ has a countable basis. Then, applying $(b)$ with $Z = \mathbb{RP}^n = \bigslant{\mathbb{R}^{n+1} \setminus \{0\}}{\sim}$ where $x \sim y$ if and only if $x = \lambda y$ for some $\lambda \in \mathbb{R}$, tells us that $\mathbb{RP}^n$ too has a countable basis.
 
\vskip 0.5cm
\hrule
\vskip 0.5cm

\textbf{Exercise 1.7:} Show that $\mathbb{RP}^n$ is compact.  

\vskip 0.5cm

\textbf{Proof:}

\vskip 0.5cm
\hrule
\vskip 0.5cm

\textbf{Exercise 1.:} 

\vskip 0.5cm

\textbf{Proof:}

\vskip 0.5cm
\hrule
\vskip 0.5cm

\pagebreak

%%%%%%%%%%%%%%%%%%%%%%%%%%%%%%%%%%%%%%%%%%%%%%%%%%%%%%%%%%%%%%%
\section{Chapter 4: Submersions, Immersions, and Embeddings}
%%%%%%%%%%%%%%%%%%%%%%%%%%%%%%%%%%%%%%%%%%%%%%%%%%%%%%%%%%%%%%%

\vskip 1cm
%%%%%%%%%%%%%%%%%%%%%%%%%%%%%%%%%%%%%%%%%%%%%%%%%%%%%%%%%%%%%%%
\subsection{Maps of Constant Rank}
%%%%%%%%%%%%%%%%%%%%%%%%%%%%%%%%%%%%%%%%%%%%%%%%%%%%%%%%%%%%%%%

\vskip 1cm
%%%%%%%%%%%%%%%%%%%%%%%%%%%%%%%%%%%%%%%%%%%%%%%%%%%%%%%%%%%%%%%
\subsubsection{\emph{Local Diffeomorphisms}}
%%%%%%%%%%%%%%%%%%%%%%%%%%%%%%%%%%%%%%%%%%%%%%%%%%%%%%%%%%%%%%%

\vskip 1cm
%%%%%%%%%%%%%%%%%%%%%%%%%%%%%%%%%%%%%%%%%%%%%%%%%%%%%%%%%%%%%%%
\textbf{Exercise 4.9:} Show that the conclusions of Proposition 4.8 still hold if $N$ is a manifold with boundary but not if $M$ is.
%%%%%%%%%%%%%%%%%%%%%%%%%%%%%%%%%%%%%%%%%%%%%%%%%%%%%%%%%%%%%%%
\vskip 0.5cm

\textbf{Proof (Sketch):}

\begin{itemize}
  \item  It follows from the result of Problem 4-2 that if $p \in M$ such that $dF_p$ is non-singular, then $F(p) \in \mathrm{Int}(N)$, which is an manifold without boundary, so naturally the results still apply. 
  
  \item If, however, $M$ has non-empty boundary, then the Inverse Function Theorem for manifolds breaks down (See result of Problem 4-1).
\end{itemize}

\vskip 0.5cm
\hrule
\vskip 0.5cm


%%%%%%%%%%%%%%%%%%%%%%%%%%%%%%%%%%%%%%%%%%%%%%%%%%%%%%%%%%%%%%%
\textbf{Exercise 4.10:} Suppose $M, N, P$ are manifolds with or without boundary and $F : M \rightarrow N$ is a local diffeomorphism. Prove the following:
\begin{enumerate}[label=(\alph*)]
  \item If $G : P \rightarrow M$ is countinuous, $G$ is smooth iff $F \circ G$ is smooth.
  \item If $F$ is surjective and $G : N \rightarrow P$ is any map, then $G$ is smooth iff $G \circ F$ is smooth.
\end{enumerate} 
%%%%%%%%%%%%%%%%%%%%%%%%%%%%%%%%%%%%%%%%%%%%%%%%%%%%%%%%%%%%%%%
\vskip 0.5cm

\textbf{Proof:}

\begin{enumerate}[label=(\alph*)]
  \item Suppose $G$ is smooth, then at each point $p \in M$, $F \circ G$ is the composition of a two smooth maps (local diffeomorphisms restrict to smooth maps around each point). Conversely if $F \circ G$ is smooth, let $p \in P$ and let $U \subseteq M$ be an open set containing $G(p)$ such that $\restr{F}{U} : U \rightarrow V \subseteq N$ is a local diffeomorphsm. Since $G$ is continuous, $G^{-1}(U) \subseteq P$ is open. So, of course, $\restr{F \circ G}{G^{-1}(U)}$ restricts to a smooth map on $G^{-1}(U)$. Then, $\restr{G}{G^{-1}(U)} : G^{-1}(U) \rightarrow U$ can be written as 
  \[ \restr{G}{G^{-1}(U)} = \restr{\left(F \circ G\right)}{G^{-1}(U)} \circ \restr{F^{-1}}{F(U)} \]
  so $\restr{G}{G^{-1}(U)}$ is a composition of smooth maps, and is thus smooth itself. We have smooth restrictions for each point $p \in P$ which we can glue together using the Gluing Lemma for smooth functions, to find that $G$ is continuous.

  \vskip 0.5cm
  \item Suppose $G$ smooth, then $G \circ F$ is the composition of smooth maps and is thus smooth. Conversely, suppose $G \circ F$ is smooth. $F$ is a surjective local diffeomorphism, so for any point $n \in N$ we can write 
  \[ G = \left( G \circ F \right) \circ F^{-1}   \]
  So, $G$ is the composition of two smooth maps and is itself a smooth map.
\end{enumerate}


\vskip 0.5cm
\hrule
\vskip 0.5cm

%%%%%%%%%%%%%%%%%%%%%%%%%%%%%%%%%%%%%%%%%%%%%%%%%%%%%%%%%%%%%%%
\textbf{Rank Theorem:} 
%%%%%%%%%%%%%%%%%%%%%%%%%%%%%%%%%%%%%%%%%%%%%%%%%%%%%%%%%%%%%%%
\vskip 0.5cm

\textbf{Proof:}
See https://math.colorado.edu/~macz9339/math6230/outline.pdf.


\vskip 0.5cm
\hrule
\vskip 0.5cm

\subsection{Embeddings}

%%%%%%%%%%%%%%%%%%%%%%%%%%%%%%%%%%%%%%%%%%%%%%%%%%%%%%%%%%%%%%%
\textbf{Exercise 4.16:} Show that every composition of smooth embeddings is a smooth embedding. 
%%%%%%%%%%%%%%%%%%%%%%%%%%%%%%%%%%%%%%%%%%%%%%%%%%%%%%%%%%%%%%%
\vskip 0.5cm

\textbf{Proof:}


Consider manifolds without boundaries $M, N, P$ and smooth embeddings $F : M \rightarrow N, G : N \rightarrow P$. Then, $F$ and $G$ are individually homeomorphisms onto their images, thus their composition $G \circ F$ is as well. 

\vskip 0.25cm
Also, since both $F, G$ are smooth immersions, they are both smooth maps whose differentials are injective. So, their composition $G \circ F$ is smooth too, and its differential (given by the chain rule as) at a point $p \in M$
\[ d\left(G \circ F\right)_{p} = dG_{F(p)} \circ dF_{p} \]
is the composition of two injective linear maps, thus it is also injective.

\vskip 0.5cm
\hrule
\vskip 0.5cm

\pagebreak

\section{Chapter 6: Sard's Theorem}


%%%%%%%%%%%%%%%%%%%%%%%%%%%%%%%%%%%%%%%%%%%%%%%%%%%%%%%%%%%%%%%
\textbf{Exercise 6.1:} 
%%%%%%%%%%%%%%%%%%%%%%%%%%%%%%%%%%%%%%%%%%%%%%%%%%%%%%%%%%%%%%%
\vskip 0.5cm

\textbf{Proof:}
[Write formal proof later]


\vskip 0.5cm
\hrule
\vskip 0.5cm



%%%%%%%%%%%%%%%%%%%%%%%%%%%%%%%%%%%%%%%%%%%%%%%%%%%%%%%%%%%%%%%
\textbf{Exercise 6.7:} Let $M$ be a smooth manifold with or without boundary. Show that a countable union of sets of measure zero in $M$ has measure zero.
%%%%%%%%%%%%%%%%%%%%%%%%%%%%%%%%%%%%%%%%%%%%%%%%%%%%%%%%%%%%%%%
\vskip 0.5cm

\textbf{Proof:}
A subset $A \subseteq M$ has measure zero in $M$ if for every smooth chart $(U, \phi)$ for $M$, the subset $\phi(U \cap A)$ has $n$-dimensional measure zero.

\vskip 0.25cm
If we have a countable collection of measure zero sets $\{U_{\alpha}\}$ in $M$, each of them has the above property and for a given chart $(U, \phi)$, we have $\phi \left( U \cap \left(\bigcup_{\alpha} U_{\alpha}\right) \right) = \bigcup_{\alpha} U \cap U_{\alpha}$.

\vskip 0.25cm
This set has $n-$dimensional measure zero since it is a countable union of sets with $n-$dimensional measure zero. [Prove this later maybe]


\vskip 0.5cm
\hrule
\vskip 0.5cm





% %%%%%%%%%%%%%%%%%%%%%%%%%%%%%%%%%%%%%%%%%%%%%%%%%%%%%%%%%%%%%%%
% \textbf{Exercise 6.:} 
% %%%%%%%%%%%%%%%%%%%%%%%%%%%%%%%%%%%%%%%%%%%%%%%%%%%%%%%%%%%%%%%
% \vskip 0.5cm

% \textbf{Proof:}



% \vskip 0.5cm
% \hrule
% \vskip 0.5cm



\end{document}

