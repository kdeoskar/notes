\documentclass{article}

% Language setting
% Replace `english' with e.g. `spanish' to change the document language
\usepackage[english]{babel}

% Set page size and margins
% Replace `letterpaper' with`a4paper' for UK/EU standard size
\usepackage[letterpaper,top=2cm,bottom=2cm,left=3cm,right=3cm,marginparwidth=1.75cm]{geometry}

% Useful packages
\usepackage{amsmath}
\usepackage{amssymb}
\usepackage{graphicx}
\usepackage{tensor}
\usepackage[colorlinks=true, allcolors=blue]{hyperref}

\usepackage{hyperref}
\hypersetup{
    colorlinks=true,
    linkcolor=blue,
    filecolor=magenta,      
    urlcolor=cyan,
    pdftitle={Overleaf Example},
    pdfpagemode=FullScreen,
    }

\urlstyle{same}

\usepackage{tikz-cd}

%%%%%%%%%%% Box pacakges and definitions %%%%%%%%%%%%%%
\usepackage[most]{tcolorbox}
\usepackage{xcolor}

% Define the colors
\definecolor{boxheader}{RGB}{0, 51, 102}  % Dark blue
\definecolor{boxfill}{RGB}{173, 216, 230}  % Light blue

% Define the tcolorbox environment
\newtcolorbox{mathdefinitionbox}[2][]{%
    colback=boxfill,   % Background color
    colframe=boxheader, % Border color
    fonttitle=\bfseries, % Bold title
    coltitle=white,     % Title text color
    title={#2},         % Title text
    enhanced,           % Enable advanced features
    attach boxed title to top left={yshift=-\tcboxedtitleheight/2}, % Center title
    boxrule=0.5mm,      % Border width
    sharp corners,      % Sharp corners for the box
    #1                  % Additional options
}
%%%%%%%%%%%%%%%%%%%%%%%%%

\newtcolorbox{dottedbox}[1][]{%
    colback=white,    % Background color
    colframe=white,    % Border color (to be overridden by dashrule)
    sharp corners,     % Sharp corners for the box
    boxrule=0pt,       % No actual border, as it will be drawn with dashrule
    boxsep=5pt,        % Padding inside the box
    enhanced,          % Enable advanced features
    overlay={\draw[dashed, thin, black, dash pattern=on \pgflinewidth off \pgflinewidth, line cap=rect] (frame.south west) rectangle (frame.north east);}, % Dotted line
    #1                 % Additional options
}

\usepackage{biblatex}
\addbibresource{sample.bib}


%%%%%%%%%%% New Commands %%%%%%%%%%%%%%
\newcommand*{\T}{\mathcal T}
\newcommand*{\cl}{\text cl}


\newcommand{\ket}[1]{|#1 \rangle}
\newcommand{\bra}[1]{\langle #1|}
\newcommand{\inner}[2]{\langle #1 | #2 \rangle}
\newcommand{\R}{\mathbb{R}}
\newcommand{\C}{\mathbb{C}}
\newcommand{\V}{\mathbb{V}}
\newcommand{\Hilbert}{\mathcal{H}}
\newcommand{\oper}{\hat{\Omega}}
\newcommand{\lam}{\hat{\Lambda}}
\newcommand{\ki}{\hat{\mathbf{k}}_i^{\star}}
\newcommand{\kf}{\hat{\mathbf{k}}_f^{\star}}
\newcommand{\M}{\mathcal{M}}
\newcommand{\K}{\mathcal{K}}

\DeclareMathOperator*{\SumInt}{%
\mathchoice%
  {\ooalign{$\displaystyle\sum$\cr\hidewidth$\displaystyle\int$\hidewidth\cr}}
  {\ooalign{\raisebox{.14\height}{\scalebox{.7}{$\textstyle\sum$}}\cr\hidewidth$\textstyle\int$\hidewidth\cr}}
  {\ooalign{\raisebox{.2\height}{\scalebox{.6}{$\scriptstyle\sum$}}\cr$\scriptstyle\int$\cr}}
  {\ooalign{\raisebox{.2\height}{\scalebox{.6}{$\scriptstyle\sum$}}\cr$\scriptstyle\int$\cr}}
}



\newcommand{\bigslant}[2]{{\raisebox{.2em}{$#1$}\left/\raisebox{-.2em}{$#2$}\right.}}
\newcommand{\restr}[2]{{% we make the whole thing an ordinary symbol
  \left.\kern-\nulldelimiterspace % automatically resize the bar with \right
  #1 % the function
  \vphantom{\big|} % pretend it's a little taller at normal size
  \right|_{#2} % this is the delimiter
  }}
%%%%%%%%%%%%%%%%%%%%%%%%%%%%%%%%%%%%%%%


\tcbset{theostyle/.style={
    enhanced,
    sharp corners,
    attach boxed title to top left={
      xshift=-1mm,
      yshift=-4mm,
      yshifttext=-1mm
    },
    top=1.5ex,
    colback=white,
    colframe=blue!75!black,
    fonttitle=\bfseries,
    boxed title style={
      sharp corners,
    size=small,
    colback=blue!75!black,
    colframe=blue!75!black,
  } 
}}

\newtcbtheorem[number within=section]{Theorem}{Theorem}{%
  theostyle
}{thm}

\newtcbtheorem[number within=section]{Definition}{Definition}{%
  theostyle
}{def}


\title{Research Notes}
\author{Keshav Balwant Deoskar}

\begin{document}
\maketitle

\begin{abstract}
These are some random notes written while reading up on exponentially-suppressed effects in finte-volume matrix elements. These are mainly just for my own understanding, but if you've stumbbled upon these notes I hope they prove useful in some capacity.
\end{abstract}

% \pagebreak 

\tableofcontents

\pagebreak

%%%%%%%%%%%%%%%%%%%%%%%%%%%%%%%%%%%%%%%%%%%%%%%%%%%%%%%%%%%%%%%%%%
\section{Finite-volume functions}
%%%%%%%%%%%%%%%%%%%%%%%%%%%%%%%%%%%%%%%%%%%%%%%%%%%%%%%%%%%%%%%%%%

The current objective is to understand Appendix C of \cite{Briceño} and reproduce results on page 18 of the paper. The paper studies $2 \rightarrow 2$ and $2 + \mathcal{J} \rightarrow 2$ transition amplitudes. We are interested in the \textbf{analytic-continuations of the finite-volume functions $F(P, L)$, $G(P, L)$, and $G^{\mu = 0}(P, L)$ below threshold}.

\begin{mathdefinitionbox}{Analytic Continuation Below threshold - What does this mean?}

  ***THIS IS A TEMPORARY EXPLANATION. RE-WRITE THIS.
  \vskip 0.5cm
  
  \begin{itemize}
    \item We are studying interactions between two particles, so the energy possessed by their combined system \emph{must be greater than} their rest-mass energy due to $E^2 = m^2c^4 + p^2c^2$. This minimum energy corresponding to the sum of their rest-masses is the \textbf{(kinematic) threshold} being referred to. 
    
    \vskip 0.5cm
    \item The finite-volume functions are complex-functions with the energy-momentum of the pair being the input-parameter, and we are interested in the anayltic continuation of these functions below the energy/mass threshold described above.
  \end{itemize}
\end{mathdefinitionbox}

\vskip 0.5cm

Note that here $L$ is the finite-box length and $P = (E, \mathbf{P})$ is the total energy-momentum of the two particle system. We can boost to the Center-of-Momentum Frame (CMF), where we define $P^{\star} = (E^{\star}, \mathbf{0})$. We have the relation 
\[ (E^{\star})^2 \equiv s \equiv P_{\mu}P^{\mu} = E^2 - \mathbf{P}^2 \]
where we have generic $P$ and the Mandelstam variable $s$. 

\begin{mathdefinitionbox}{Mandelstam Variables}
  \begin{itemize}
    \item Numerical quantities used to encode energy, momentum, and scattering angle in $2 \rightarrow 2$ interactions. 
    
    \item If the Minkowski metric is chosen to be $diag(+, -, -, -)$ then the three Mandelstam variables are 
    \begin{align*}
      s &= (p_1 + p_2)^2 c^2 = (p_3 + p_4)^2 c^2 \\
      t &= (p_1 - p_3)^2 c^2 = (p_4 - p_2)^2 c^2 \\
      u &= (p_1 - p_4)^2 c^2 = (p_3 - p_2)^2 c^2 \\
    \end{align*}
    where $p_1, p_2$ and $p_1, p_2$ are the four-momenta of the incoming and outgoing particles.

    \item $s$ and $t$ are the squares of CMF Energy and Momentum respectively. (Check this)
    \href{https://en.wikipedia.org/wiki/Mandelstam_variables}{source.}
  \end{itemize}
\end{mathdefinitionbox}

\vskip 0.5cm
Initial and final 3-momentum states in the Center-of-Momentum Frame are denoted as $\mathbf{k_i}^{\star}$ and $\mathbf{k_f}^{\star}$. With this notation we can introduce the partial-wave expansion of the Elastic Scattering Matrix $\mathcal{M}$:

\[ \M (s, \ki, \kf) = 4\pi \sum_{l ,m_l} Y_{l m_l} (\kf) \M(s) Y_{l m_l}^{*}(\ki) \]

Which we can express in terms of the $K-$matrix as
\[ \M(s) = \K(s) \frac{1}{1 - i \rho(s) \K(s)} \]

$\rho(s)$ is the two-body phase space 
\[ \rho(s) = \frac{q^{\star}}{8\pi E^{\star}} = \frac{1}{16\pi}\sqrt{1 - \frac{4m^2}{s}} \]

where $q^{\star}$ is the relative momentum of the two particles in the CMF, $q^{\star} = \sqrt{s/4 - m^2}$ -- the square root introduces a branch cut in the complex $s$ plane.

\subsection{Analytic Continuation of $c_{JM}^{(n)}$}

The functions $F(P, L), G(P, L), G^{\mu = 0}(P, L)$ can be expressed in terms of the function in terms of a class of functions 
\[ c^{(n)}_{JM}(P, L) = \left[ \frac{1}{L^3} \SumInt_k \right] \frac{\omega_k^{\star}}{\omega_{k}} \frac{\sqrt{4\pi} k^{\star} \tensor[^J]{Y}{_{JM}} (\mathbf{\hat{k}}^{\star}) }{(q^{\star^{2}} - k^{\star^{2}} + i\epsilon)^n } \]

The relations we require are given by 
\begin{align*}
  F(P, L) &= \frac{1}{2E^{\star}} c_{00}^(1)(P, L) \\
  G(P, L) &= \frac{1}{4E^{\star}} c_{00}^(2)(P, L) \\
  G^{\mu = 0}(P, L) &= -\frac{E}{4E^{\star^{3}}} c_{00}^(1)(P, L) \\
\end{align*}

\vskip 0.5cm
\begin{dottedbox}
  We will end up using the \underline{\textbf{Poisson Summation Formula}} in the following form:
  \[ \frac{1}{L^3} \sum_{\vec{k}} g(\vec{k}) = \int \frac{d^3 k}{(2\pi)^3} g(\vec{k}) + \sum_{\vec{l} \neq 0} \int \frac{d^3 k}{(2\pi)^3} e^{i L \vec{l} \cdot \vec{k}} g(\vec{k}) \]

  \vskip 0.5cm
  \underline{\textbf{Proof:}}
\end{dottedbox}

\vskip 0.5cm

\begin{itemize}
  \item We want to analytically continue each of these functions below the kinematic threshold i.e. $P < (2m)^2$.
  
  \item At a sub-threshold momentum $P_{\kappa}$, we have 
  \[ m^2 - P_{\kappa}^2/4 = \kappa^2 \] (why?)

  \item We apply the Poisson Summation Formula 
  \[ \boxed{ \frac{1}{L^3} \sum_{\vec{k}} g(\vec{k}) = \int \frac{d^3 k}{(2\pi)^3} g(\vec{k}) + \sum_{\vec{l} \neq 0} \int \frac{d^3 k}{(2\pi)^3} e^{i L \vec{l} \cdot \vec{k}} g(\vec{k}) } \]
  to $c_{JM}^{(n)}$.
  (source: https://arxiv.org/pdf/hep-lat/0507006.pdf)
  
  \item Since we are dealing with sub-threshold momenta, we don't need to worry about the singularity, so the $i \epsilon$ vanishes (check this logic...)
\end{itemize}

Then 
\begin{align*}
  &\left[ \frac{1}{L^3} \sum_{\vec{k}} - \int \frac{d^3 k}{(2\pi)^3} \right] g(\vec{k}) = \sum_{\vec{m} \neq 0} \int \frac{d^3 k}{(2\pi)^3} e^{i L \vec{m} \cdot \vec{k}} g(\vec{k}) \\
  \implies &\left[ \frac{1}{L^3} \sum_{\vec{k}} - \int \frac{d^3 k}{(2\pi)^3} \right] \frac{\omega_k^{\star}}{\omega_{k}} \frac{\sqrt{4\pi} k^{\star} \tensor[^J]{Y}{_{JM}} (\mathbf{\hat{k}}^{\star}) }{(q^{\star^{2}} - k^{\star^{2}})^n } = \sum_{\vec{m} \neq 0} \int \frac{d^3 k}{(2\pi)^3} \cdot \frac{\omega_k^{\star}}{\omega_{k}} \frac{\sqrt{4\pi} k^{\star} \tensor[^J]{Y}{_{JM}} (\mathbf{\hat{k}}^{\star}) }{(q^{\star^{2}} - k^{\star^{2}})^n } \cdot e^{i L \vec{m} \cdot \vec{k}} \\
  \implies & c_{JM}^{(n)}(P_{\kappa}, L) = \sum_{\vec{m} \neq 0} \int \frac{d^3 k}{(2\pi)^3} \cdot \frac{\omega_k^{\star}}{\omega_{k}} \frac{\sqrt{4\pi} k^{\star} \tensor[^J]{Y}{_{JM}} (\mathbf{\hat{k}}^{\star}) }{(q^{\star^{2}} - k^{\star^{2}})^n } \cdot e^{i L \vec{m} \cdot \vec{k}}
\end{align*}

Now, we use the facts that 
\begin{itemize}
  \item $q^{\star} = \sqrt{s/4 - m^2}$
  \item $m^2 - P_{\kappa}^2/4 = \kappa^2$
  \item The Integration measure is Lorentz Invariant i.e. 
  \[ \frac{d^3 \mathbf{k}^{\star} }{\omega_{\mathbf{k}}^{\star}} = \frac{d^3 \mathbf{k} }{\omega_{\mathbf{k}}} \implies d^3 \mathbf{k}^{\star} = d^3 \mathbf{k} \cdot \frac{\omega_{\mathbf{k}}^{\star}}{\omega_{\mathbf{k}}} \]
\end{itemize}

\begin{align*}
  \implies c_{JM}^{(n)}(P_{\kappa}, L) &= \sum_{\vec{m} \neq 0} \int \frac{d^3 \mathbf{k}^{\star}}{(2\pi)^3} \cdot \frac{\sqrt{4\pi} k^{\star} \tensor[^J]{Y}{_{JM}} (\mathbf{\hat{k}}^{\star}) }{(s/4 - m^2 - k^{\star^{2}})^n } \cdot e^{i L \vec{m} \cdot \vec{k}} 
\end{align*}

We're integrating with respect to $d^3 \mathbf{k}^{\star} / (2\pi)^3$ i.e. working in the Center-of-Momentum Frame, where $\mathbf{P}_{\kappa}^{\star} = (E_{\kappa}^{\star}, \mathbf{0})$ and so $E^{\star^2} = P_{\kappa}^2 = s$. Then we have 
\[ s/4 - m^2 - k^{\star^2} = \frac{P_{\kappa}^2}{4} - m^2 - k^{\star^2} = -\kappa^2 - k^{\star^2} \] 

Thus, 
\begin{align*}
  \implies c_{JM}^{(n)}(P_{\kappa}, L) &= \sum_{\vec{m} \neq 0} \int \frac{d^3 \mathbf{k}^{\star}}{(2\pi)^3} \cdot \frac{\sqrt{4\pi} k^{\star} \tensor[^J]{Y}{_{JM}} (\mathbf{\hat{k}}^{\star}) }{(-\kappa^2 - k^{\star^{2}})^n } \cdot e^{i L \vec{m} \cdot \vec{k}} \\ 
\end{align*}

\[ \implies \boxed{c_{JM}^{(n)}(P_{\kappa}, L) = (-1)^n \sum_{\vec{m} \neq 0} \int \frac{d^3 \mathbf{k}^{\star}}{(2\pi)^3} \cdot \frac{\sqrt{4\pi} k^{\star} \tensor[^J]{Y}{_{JM}} (\mathbf{\hat{k}}^{\star}) }{(\kappa^2 + k^{\star^{2}})^n } \cdot e^{i L \vec{m} \cdot \vec{k}} } \]

% \vskip 0.5cm
% \subsection{What are these functions: $F(P, L)$, $G(P, L)$, and $G^{\mu = 0}(P, L)$?}
% **Write later after learning more


\vskip 1cm
\subsection{Volume Effect on Energy and Phase Shift $\delta s$}

\pagebreak
\printbibliography


\end{document}
