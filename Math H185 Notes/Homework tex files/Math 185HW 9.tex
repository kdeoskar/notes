\documentclass{article}

% Language setting
% Replace `english' with e.g. `spanish' to change the document language
\usepackage[english]{babel}

% Set page size and margins
% Replace `letterpaper' with`a4paper' for UK/EU standard size
\usepackage[letterpaper,top=2cm,bottom=2cm,left=3cm,right=3cm,marginparwidth=1.75cm]{geometry}

% Useful packages
\usepackage{amsmath}
\usepackage{amssymb}
\usepackage{graphicx}
\usepackage{enumitem}
\usepackage[colorlinks=true, allcolors=blue]{hyperref}

\usepackage{hyperref}
\hypersetup{
    colorlinks=true,
    linkcolor=blue,
    filecolor=magenta,      
    urlcolor=cyan,
    pdftitle={Math 185 Homework 9},
    pdfpagemode=FullScreen,
    }

\urlstyle{same}

\usepackage{tikz-cd}

%%%%%%%%%%% Box pacakges and definitions %%%%%%%%%%%%%%
\usepackage[most]{tcolorbox}
\usepackage{xcolor}
\usepackage{csquotes}
% Define the colors
\definecolor{boxheader}{RGB}{0, 51, 102}  % Dark blue
\definecolor{boxfill}{RGB}{173, 216, 230}  % Light blue

% Define the tcolorbox environment
\newtcolorbox{mathdefinitionbox}[2][]{%
    colback=boxfill,   % Background color
    colframe=boxheader, % Border color
    fonttitle=\bfseries, % Bold title
    coltitle=white,     % Title text color
    title={#2},         % Title text
    enhanced,           % Enable advanced features
    attach boxed title to top left={yshift=-\tcboxedtitleheight/2}, % Center title
    boxrule=0.5mm,      % Border width
    sharp corners,      % Sharp corners for the box
    #1                  % Additional options
}
%%%%%%%%%%%%%%%%%%%%%%%%%

\usepackage{biblatex}
\addbibresource{sample.bib}


%%%%%%%%%%% New Commands %%%%%%%%%%%%%%
\newcommand*{\T}{\mathcal T}
\newcommand*{\cl}{\text cl}


\newcommand{\ket}[1]{|#1 \rangle}
\newcommand{\bra}[1]{\langle #1|}
\newcommand{\inner}[2]{\langle #1 | #2 \rangle}
\newcommand{\R}{\mathbb{R}}
\newcommand{\C}{\mathbb{C}}
\newcommand{\V}{\mathbb{V}}
\newcommand{\Hilbert}{\mathcal{H}}
\newcommand{\oper}{\hat{\Omega}}
\newcommand{\lam}{\hat{\Lambda}}

\newcommand{\bigslant}[2]{{\raisebox{.2em}{$#1$}\left/\raisebox{-.2em}{$#2$}\right.}}
\newcommand{\restr}[2]{{% we make the whole thing an ordinary symbol
  \left.\kern-\nulldelimiterspace % automatically resize the bar with \right
  #1 % the function
  \vphantom{\big|} % pretend it's a little taller at normal size
  \right|_{#2} % this is the delimiter
  }}
%%%%%%%%%%%%%%%%%%%%%%%%%%%%%%%%%%%%%%%


\newtcolorbox{dottedbox}[1][]{%
    colback=white,    % Background color
    colframe=white,    % Border color (to be overridden by dashrule)
    sharp corners,     % Sharp corners for the box
    boxrule=0pt,       % No actual border, as it will be drawn with dashrule
    boxsep=5pt,        % Padding inside the box
    enhanced,          % Enable advanced features
    overlay={\draw[dashed, thin, black, dash pattern=on \pgflinewidth off \pgflinewidth, line cap=rect] (frame.south west) rectangle (frame.north east);}, % Dotted line
    #1                 % Additional options
}

\tcbset{theostyle/.style={
    enhanced,
    sharp corners,
    attach boxed title to top left={
      xshift=-1mm,
      yshift=-4mm,
      yshifttext=-1mm
    },
    top=1.5ex,
    colback=white,
    colframe=blue!75!black,
    fonttitle=\bfseries,
    boxed title style={
      sharp corners,
    size=small,
    colback=blue!75!black,
    colframe=blue!75!black,
  } 
}}

\newtcbtheorem[number within=section]{Theorem}{Theorem}{%
  theostyle
}{thm}

\newtcbtheorem[number within=section]{Definition}{Definition}{%
  theostyle
}{def}



\title{Math H185 Homework 9}
\author{Keshav Balwant Deoskar}

\begin{document}
\maketitle



%%%%%%%%%%%%%%%%%%%%%%%%%%%%%%%%%%%%%%%%%%%%%%%%%%%%%%%%%%%%%%%%
\begin{mathdefinitionbox}{Question 1}
\vskip 0.5cm
For the following functions, classify the singularity at $\infty$ and calculate the residue at $\infty$.
\begin{enumerate}[label=(\alph*)]
  \item $f(z) = z\sin(1/z)$
  \item $f(z) = e^z$
  \item $f(z) = 3z^4 + z^3 + 4z^2 + z + 5$
\end{enumerate}
\end{mathdefinitionbox}
%%%%%%%%%%%%%%%%%%%%%%%%%%%%%%%%%%%%%%%%%%%%%%%%%%%%%%%%%%%%%%%%%

\vskip 0.5cm
\underline{\textbf{Solution:}}

\begin{enumerate}[label=(\alph*)]
  \item We have $f(z) = z\sin(1/z)$ so 
  \begin{align*}
    F(z) &= f(1/z) \\
    &= \frac{\sin(z)}{z} \\
    &= \frac{1}{z} \left[z - \frac{z^3}{3!} + \frac{z^5}{5!} + \cdots \right] \\
  \end{align*}
  $F(z)$ has a removable singularity at $0$ since it can be extended to the function 
  \[ F(z) = 1 - \frac{z^2}{3!} + \frac{z^2}{5!} + \cdots \] Therefore, $f(z)$ has a removable singularity at $\infty$ and the residue is zero.

  \vskip 0.5cm
  \item We have $f(z) = e^z$ so 
  \[ F(z) = f(1/z) = e^{1/z} \]
  $F(z)$ has an essential singularity at $z = 0$ so $f(z)$ has an essential singularity at $\infty$. To find the residue let's expand $F(z)$ out:
  \begin{align*}
    F(z) &= e^{1/z} \\
    &= 1 + \frac{1}{z} + \frac{1}{2!z^2} + \frac{1}{3!z^3} + \cdots 
  \end{align*}
  This has a residue of $1$ at zero, so $\mathrm{Res}_{z_0 = \infty}f(z) = 1$.

  \vskip 0.5cm
  \item We have $f(z) = 3z^4 + z^3 + 4z^2 + z + 5$ so 
  \begin{align*}
    F(z) &= f(1/z) \\
    &= \frac{3}{z^4} + \frac{1}{z^3} + 4\frac{1}{z^2} + \frac{1}{z} + 5 
  \end{align*}
  which has a pole of order one and residue equal to one at $0$. Therefore, $f(z)$ has a pole of order $1$ at $\infty$ and $\mathrm{Res}_{z_0 = \infty} f(z) = 1$.
\end{enumerate}


\vskip 0.5cm
\hrule 
\vskip 0.5cm



%%%%%%%%%%%%%%%%%%%%%%%%%%%%%%%%%%%%%%%%%%%%%%%%%%%%%%%%%%%%%%%%
\begin{mathdefinitionbox}{Question 2}
\vskip 0.5cm
Let $f(z) = \frac{z(z-2)^3}{\sin(z^2)(z-4)^5}$. Use the argument principle to calculate 
\[ \frac{1}{2\pi i} \int_{\partial B_r(0)} \frac{f'(z)}{f(z)} dz  \] for $r = 1, 3, 5$
\end{mathdefinitionbox}
%%%%%%%%%%%%%%%%%%%%%%%%%%%%%%%%%%%%%%%%%%%%%%%%%%%%%%%%%%%%%%%%%

\vskip 0.5cm
\underline{\textbf{Proof:}}

The Argument Principle tells us that 
\[ \frac{1}{2\pi i }\int_{\partial B_r(0)} \frac{f'(z)}{f(z)} dz = \text{\# of zeroes of $f$ in $B_r(0)$} - \text{\# of poles of $f$ in $B_r(0)$} \]

We have 
\[ f(z) = \frac{z(z-2)^3}{\sin\left(z^2\right)(z-4)^5}  \]

which can be written as 

\begin{align*}
  \frac{z(z-2)^3}{\sin\left(z^2\right)(z-4)^5} &= \frac{z}{\sin(z^2)} \cdot \frac{(z-2)^3}{(z-4)^5} \\
  &= \frac{z}{z^2 - \frac{(z^2)^3}{3!} + \frac{(z^2)^5}{5!} + \cdots} \cdot \frac{(z-2)^3}{(z-4)^5} \\
  &= \frac{1}{z - \frac{z^5}{3!} + \frac{z^9}{5!} + \cdots} \cdot \frac{(z-2)^2}{(z-4)^5} \\
\end{align*}

The zeroes of this function occur at $z = 0$ (multiplicity $1$) $z = 2$ (with multiplicity $3$). The poles of this function occur at $z = 0$ (with multiplicity $2$) and at $z = 4$ (with multiplicity $5$).


\begin{enumerate}[label=(\alph*)]
  \item \underline{$r = 1$:} The only pole or zero lying in this ball are the ones at the origin, so in this region, 
  \[ \frac{1}{2\pi i} \int_{\partial B_1(0)} \frac{f'(z)}{f(z)} dz = 1-2 = -1 \]

  \item \underline{$r = 3$:} In this region we also have the zero at $z = 2$. So, 
  \[ \frac{1}{2\pi i} \int_{\partial B_3(0)} \frac{f'(z)}{f(z)} dz = 4 - 6 = -2 \] 

  \item \underline{$r = 5$:} In this region we have all of the zeroes and poles, so
  \[ \frac{1}{2\pi i} \int_{\partial B_5(0)} \frac{f'(z)}{f(z)} dz = 4 - 21 = -17 \] 
\end{enumerate}


\vskip 0.5cm
\hrule 
\vskip 0.5cm




%%%%%%%%%%%%%%%%%%%%%%%%%%%%%%%%%%%%%%%%%%%%%%%%%%%%%%%%%%%%%%%%
\begin{mathdefinitionbox}{Question 3}
\vskip 0.5cm
Do the following problem from Stein-Shakarchi. Recall that an entire function is a function $f : \C \rightarrow \C$ which is holomorphic at all $z \in C$:

\begin{dottedbox}
  Prove that all entire functions that are also injective take the form $f(z) = az + b$ with $a, b \in \C$ and $a \neq 0$.
\end{dottedbox}

\end{mathdefinitionbox}
%%%%%%%%%%%%%%%%%%%%%%%%%%%%%%%%%%%%%%%%%%%%%%%%%%%%%%%%%%%%%%%%%

\vskip 0.5cm
\underline{\textbf{Proof:}}
Consider an entire and injective function $f$. Then, $f$ is non-constant. Let's define $g(z) = f(1/z)$. 

\vskip 0.5cm
If the singularity of $g(z)$ at $0$ were an essential singularity, then the Casorati-Weierstass theorem would imply that the image $g\left(B_1(0) \setminus \{0\}\right)$ is dense in $\C$. However, $g\left(B_{1/2}(2)\right)$ is an open set by the open mapping theorem and these two maps intersect, which shows that $g(z)$ (and hence $f(z)$) is not injective.

\vskip 0.5cm
Thus, the singularity at $z = 0$ must be a pole, implying that $f(z)$ is a polynomial. Suppose $f(z)$ is a polynomial of degree $m$. Then $f$ has $m$ zeroes, accounting for multiplicity. But if $f$ were to have any number of distinct roots greater than $1$, then it would not be injective since the roots would both be mapped to zero. So, $f$ must have the form $f(z) = c(z-z_0)^m$ for $c, z_0 \in \C$. But for $m \geq 2$ this is \emph{also} not necessarily injective as $f(z_0 + 1) = f(z_0 + e^{2\pi i/m})$. Thus, we must have $m = 1$ meaning $f(z)$ is a linear polynomial i.e. it is of the form 
\[ f(z) = az + b \] for $a, b \in \C, a \neq 0$.

\vskip 0.5cm
\hrule 
\vskip 0.5cm



%%%%%%%%%%%%%%%%%%%%%%%%%%%%%%%%%%%%%%%%%%%%%%%%%%%%%%%%%%%%%%%%
\begin{mathdefinitionbox}{Question 4}
\vskip 0.5cm
Use the Cauchy inequalities or the maximum modulus principle to solve the following problems:

\begin{dottedbox}
  Prove that if $f$ is an entire function that satisfies 
  \[ \sup_{|z| = R} \left| f(z) \right| \leq AR^k + B \] for all $R > 0$,, and for some integer $k \geq 0$ and some constants $A, B > 0$, then $f$ is a polynomial of degree $\leq k$.
\end{dottedbox}

\end{mathdefinitionbox}
%%%%%%%%%%%%%%%%%%%%%%%%%%%%%%%%%%%%%%%%%%%%%%%%%%%%%%%%%%%%%%%%%

\vskip 0.5cm
\underline{\textbf{Proof:}}

We have an entire function $f(z)$ such that 
\[ \sup_{|z| = R} \left| f(z) \right| \leq AR^k + B  \]

\vskip 0.5cm
Since $f$ is entire, it is holomorphic (thus analytic) on all of $\C$ and can be written in terms of a power series expansion about the origin:

\[ f(z) = \sum_{n = 1}^{\infty} \frac{f^{(n)}(0)}{n!} \cdot z^n \]

\vskip 0.5cm
Now, since $f(z)$ is entire, so are each of its derivatives. Let $g(z) = f^{(k+1)}(z)$. Cauchy's inequality on the disc of radius $R$, $D_R$, tells us 

\begin{align*}
  g^{(l)}(0) &\leq \frac{(l + k + 1)!}{R^{l + k + 1}} \sup_{\partial D_R} |f(z)|, \;\;\;\;l = 0, 1, 2, \cdots
\end{align*}

Then, by the assumption,
\begin{align*}
  g^{(l)}(0) &\leq \frac{(l + k + 1)!}{R^{l + k + 1}} \sup_{\partial D_R} AR^k + B
\end{align*}

Now, in the limit as $R \rightarrow \infty$, $\frac{(l + k + 1)!}{R^{l + k + 1}} \sup_{\partial D_R} AR^k + B \rightarrow 0$ so $g^{(l)}(0) = f^{(l + k + 1)}(0)$ for $l = 0, 1, 2, \cdots$ must be equal to zero. i.e. all of the coefficients the power series expansion beyond the $z^k$ term are zero. Therefore, $f(z)$ is a polynomial of degree $\leq k$.


\vskip 0.5cm
\hrule 
\vskip 0.5cm



%%%%%%%%%%%%%%%%%%%%%%%%%%%%%%%%%%%%%%%%%%%%%%%%%%%%%%%%%%%%%%%%
\begin{mathdefinitionbox}{Question 5}
\vskip 0.5cm
Let $w_1, \cdots, w_n$ be the points on the unit circle in the complex plane. Prove that there exists a point $z$ on the unit circle such that the product of the distances from $z$ to the points $w_j$, $1 \leq j \leq n$, is at least $1$. Conclude that there exists a point $w$ on the unit circle such that the product of the distances from $w$ to the points $w_j$, $1 \leq j \leq n$, is exactly equal to 1.
\end{mathdefinitionbox}
%%%%%%%%%%%%%%%%%%%%%%%%%%%%%%%%%%%%%%%%%%%%%%%%%%%%%%%%%%%%%%%%%

\vskip 0.5cm
\underline{\textbf{Proof:}}

Define a function $g(z) : B_1(0) \rightarrow \C$ as 
\[ g(z) = \prod_{j = 1}^{n} (z - w_j) \]

This function is holomorphic on the unit disc. Then, by the Maximum Modulus Principle, $g(z)$ can only attain its maximum (modulus) on the unit circle. But note that 
\[ g(0) = \prod_{j = 1}^{n} w_j \implies \left| g(0) \right| = \prod_{j = 1}^{n} \underbrace{|w_j|}_{=1} = 1 \]

So, the maximum modulus of $g(z)$ must at least be greater than $1$. So, there exists a point $z$ on the unit circle such that $g(z)$ is atleast 1.

Now, if we define $f(z)$ to be 
\[ f(z) = \prod_{j = 1}^{n} |z - w_j| \]

Then, $f(z)$ can be zero (if $z = w_j$) and it is at least $1$ from the argument above. Therefore, by the Intermediate Value Theorem, there must exist some $w$ on the unit circle such that $f(w)$ is exactly $1$.

\vskip 0.5cm
\hrule 
\vskip 0.5cm



%%%%%%%%%%%%%%%%%%%%%%%%%%%%%%%%%%%%%%%%%%%%%%%%%%%%%%%%%%%%%%%%
\begin{mathdefinitionbox}{Question 6}

\vskip 0.5cm
Suppose $f$ and $g$ are holomorphic in a region containing the disc $|z| \leq 1$. Suppose that $f$ has a simple zero at $z = 0$ and vanishes nowhere else in $|z| \leq 1$. Let 
\[ f_{\epsilon}(z) = f(z) + \epsilon g(z) \] 

Show that if $\epsilon$ is sufficiently small, then 
\begin{enumerate}[label=(\alph*)]
  \item We know that $f$ has a unique zero at $z = 0$. Now, $f_{\epsilon}(z)$ has a unique zero in $|z| \leq 1$, and 
  \item if $z_{\epsilon}$ is this zero, the mapping $\epsilon \mapsto z_{\epsilon}$ is continuous.
\end{enumerate}

\end{mathdefinitionbox}
%%%%%%%%%%%%%%%%%%%%%%%%%%%%%%%%%%%%%%%%%%%%%%%%%%%%%%%%%%%%%%%%%

\vskip 0.5cm
\underline{\textbf{Proof:}}

\begin{enumerate}[label=(\alph*)]
  \item The functions $f, g$ are holomorphic on a region $\Omega$ containing the unit disc. We can assume WLOG that $\Omega$ is an open set (otherwise we could simply replace it with a suitable open set.) 
  
  \vskip 0.5cm
  We have $f_{\epsilon}(z) - f(z) = \epsilon g(z)$. For sufficiently small $\epsilon$, we have
  \[ \left| f_{\epsilon}(z) - f(z) \right| = |\epsilon g(z)| < f(z) \] on $\partial \mathbb{D}$. $\epsilon g(z)$ is still holomorphic on $\Omega$, so Rouche's theorem applies. It tells us that $f(z)$ and $f(z) + \epsilon g(z)$ have the same number of zeros in the unit circle. Therefore, $f_{\epsilon}(z)$ has a unique zero in the unit disc.

  \vskip 0.5cm
  % \item The mapping $\epsilon \xrightarrow{h} z_{\epsilon}$ is said to be continuous if for any $\tau > 0$ there exists a $\delta > 0$ such that $|\epsilon_2 - \epsilon_1| < \delta$ implies $|z_{\epsilon_{2}} - z_{\epsilon_{2}} | < \tau$.
  
  \item We have $f_{\epsilon}(z) = f(z) + \epsilon g(z)$ as in part (a). Now, choose some $\epsilon_0 < \epsilon$ and the associated function $f_{{\epsilon_0}} = f(z) + \epsilon_0 g(z)$ whose root is denoted $z_{{\epsilon_0}}$. 
  
  \vskip 0.25cm
  Since $f$ has a unique zero at $z = 0$, we can choose $r > 0$ such that $f$ such that $\overline{B}_r(z_{{\epsilon_0}}) \subseteq \mathbb{D}$ and $f$ is non-vanishing on $\partial \overline{B}_r(z_{{\epsilon_0}})$. Then, we can appropriately choose $\delta > 0$ so that 
  \[  \min_{z \in \partial \overline{B}_r(z_{{\epsilon_0}})} |f(z)| > \delta \max_{z \in \partial \overline{B}_r(z_{{\epsilon_0}})} |g(z)|  \] 
  
  Then, applying Rouche's Theorem again, we see that $f_{\epsilon}$ has only $1$ one zero inside $B_r(z_{{\epsilon_0}})$. But we saw earlier that $f_{\epsilon}$ has a unique zero. Thus, it must be the case that $z_{\epsilon} \in {B}_r(z_{{\epsilon_0}})$. i.e. $|z_{\epsilon} - z_{{\epsilon_0}} | < r$. The above is exactly the statement that $\epsilon \mapsto z_{\epsilon}$ is continuous at $\epsilon_0$.
\end{enumerate}

\vskip 0.5cm
\hrule 
\vskip 0.5cm


%%%%%%%%%%%%%%%%%%%%%%%%%%%%%%%%%%%%%%%%%%%%%%%%%%%%%%%%%%%%%%%%
\begin{mathdefinitionbox}{Question 7}
\vskip 0.5cm
Let $f$ be non-constant and holomorphic in an open set containing the closed unit disc.
\begin{enumerate}[label=(\alph*)]
  \item Show that if $|f(z)| = 1$ whenever $|z| = 1$, then the image of $f$ contains the unit disc. 
  \item If $|f(z)| \geq 1$ whenever $|z| = 1$ and there exists point $z_0 \in \mathbb{D}$ such that $|f(z_0)| < 1$, then the image of $f$ contains the unit disc.
\end{enumerate}
\end{mathdefinitionbox}
%%%%%%%%%%%%%%%%%%%%%%%%%%%%%%%%%%%%%%%%%%%%%%%%%%%%%%%%%%%%%%%%%

\vskip 0.5cm
\underline{\textbf{Proof:}}

\begin{enumerate}[label=(\alph*)]
  \item The goal is to show that $f$ takes on every value in $\mathbb{D}$ i.e. $f(z) - w_0 = 0$ has a root for every $w_0 \in \mathbb{D}$. 
  
  \vskip 0.5cm
  Since $f$ is holomorphic on an open set containing the unit circle with $|f(z)| = 1$ for $|z| = 1$ and $|w_0| < 1$, so is $f-w_0$. Thus, Rouche's Theorem applies, meaning $f$ and $f-w_0$ have the same number of zeros inside the unit circle for any $w_0 \in \mathbb{D}$.

  \vskip 0.5cm
  So, if we show that $f - 0 = f$ has a root on $\mathbb{D}$, the rest will follow. Suppose for contradiction  that $f$ has no roots on $\mathbb{D}$. Then $1/f : \mathbb{D} \rightarrow \C$ is holomorphic. So, by the Maximum Modulus principle, $\left| 1/f \right| \leq 1$ or equivalently $|f(z)| \geq 1$ on $\mathbb{D}$.

  \vskip 0.5cm
  However, we know that for any $z \in \partial \mathbb{D}$, $|f(z)| = 1$. If $U$ is a small neighborhood of $z \in \partial \mathbb{D}$ contained in the domain of $f$ (the open set $U$), then $f(U)$ is an open neighborhood of $f(z)$ whic contradicts $1 \leq |f(z)|$ for $z \in \mathbb{D}$ as there must be some values of $z$ for which $f(z) > 1$ in order for $f(U)$ to be open.

  \vskip 0.5cm
  This is a contradiction. Therefore, there must exist a root for $f(z)$ that lies in $\mathbb{D}$. The desired result follows from our discussion above.

  \vskip 0.5cm
  \item The exact same reasoning works for part (b).
\end{enumerate}

\vskip 0.5cm
\hrule 
\vskip 0.5cm


%%%%%%%%%%%%%%%%%%%%%%%%%%%%%%%%%%%%%%%%%%%%%%%%%%%%%%%%%%%%%%%%
\begin{mathdefinitionbox}{Question 8}
\vskip 0.5cm
Suppose $f$ is a non-vanishing continuous function on $\overline{B}(0)$ that is holomorphic in $B_1(0)$. Prove that if $|f(z)| = 1$ whenever $|z| = 1$ then $f$ is constant.
\end{mathdefinitionbox}
%%%%%%%%%%%%%%%%%%%%%%%%%%%%%%%%%%%%%%%%%%%%%%%%%%%%%%%%%%%%%%%%%

\vskip 0.5cm
\underline{\textbf{Proof:}}

We have $f$ holomorphic on $B_1(0)$ and $|f(z)| = 1$ on $\partial B_1(0)$. By the maximum modulus principle, $f(z)$ cannot attain a maximum modulus on $\mathrm{int}\left(B_1(0)\right) = B_1(0)$, so it must be the case that $|f(z)| \leq 1$.

\vskip 0.5cm
However, since $f(z)$ is non-vanishing on $\overline{B}_1(0) = \mathbb{D}$, the function $\frac{1}{f(z)}$ satisfies all of the above conditions too, meaning that $\left| \frac{1}{f(z)} \right| \leq 1$ or equivalently that $|f(z)| \geq 1$.

\vskip 0.5cm
Therefore, it must be the case that $|f(z)| = 1$ on all of $\mathbb{D}$ i.e. it is constant.


\vskip 0.5cm
\hrule 
\vskip 0.5cm




%%%%%%%%%%%%%%%%%%%%%%%%%%%%%%%%%%%%%%%%%%%%%%%%%%%%%%%%%%%%%%%%%
% \begin{mathdefinitionbox}{Question }
% \vskip 0.5cm

% \end{mathdefinitionbox}
% %%%%%%%%%%%%%%%%%%%%%%%%%%%%%%%%%%%%%%%%%%%%%%%%%%%%%%%%%%%%%%%%%

% \vskip 0.5cm
% \underline{\textbf{Proof:}}

% \vskip 0.5cm
% \hrule 
% \vskip 0.5cm



\end{document}
