\documentclass{article}

% Language setting
% Replace `english' with e.g. `spanish' to change the document language
\usepackage[english]{babel}

% Set page size and margins
% Replace `letterpaper' with`a4paper' for UK/EU standard size
\usepackage[letterpaper,top=2cm,bottom=2cm,left=3cm,right=3cm,marginparwidth=1.75cm]{geometry}

% Useful packages
\usepackage{amsmath}
\usepackage{amssymb}
\usepackage{graphicx}
\usepackage{enumitem}
\usepackage[colorlinks=true, allcolors=blue]{hyperref}

\usepackage{hyperref}
\hypersetup{
    colorlinks=true,
    linkcolor=blue,
    filecolor=magenta,      
    urlcolor=cyan,
    pdftitle={Math 185 Homework 2},
    pdfpagemode=FullScreen,
    }

\urlstyle{same}

\usepackage{tikz-cd}

%%%%%%%%%%% Box pacakges and definitions %%%%%%%%%%%%%%
\usepackage[most]{tcolorbox}
\usepackage{xcolor}

% Define the colors
\definecolor{boxheader}{RGB}{0, 51, 102}  % Dark blue
\definecolor{boxfill}{RGB}{173, 216, 230}  % Light blue

% Define the tcolorbox environment
\newtcolorbox{mathdefinitionbox}[2][]{%
    colback=boxfill,   % Background color
    colframe=boxheader, % Border color
    fonttitle=\bfseries, % Bold title
    coltitle=white,     % Title text color
    title={#2},         % Title text
    enhanced,           % Enable advanced features
    attach boxed title to top left={yshift=-\tcboxedtitleheight/2}, % Center title
    boxrule=0.5mm,      % Border width
    sharp corners,      % Sharp corners for the box
    #1                  % Additional options
}
%%%%%%%%%%%%%%%%%%%%%%%%%

\usepackage{biblatex}
\addbibresource{sample.bib}


%%%%%%%%%%% New Commands %%%%%%%%%%%%%%
\newcommand*{\T}{\mathcal T}
\newcommand*{\cl}{\text cl}


\newcommand{\ket}[1]{|#1 \rangle}
\newcommand{\bra}[1]{\langle #1|}
\newcommand{\inner}[2]{\langle #1 | #2 \rangle}
\newcommand{\R}{\mathbb{R}}
\newcommand{\C}{\mathbb{C}}
\newcommand{\V}{\mathbb{V}}
\newcommand{\Hilbert}{\mathcal{H}}
\newcommand{\oper}{\hat{\Omega}}
\newcommand{\lam}{\hat{\Lambda}}

\newcommand{\bigslant}[2]{{\raisebox{.2em}{$#1$}\left/\raisebox{-.2em}{$#2$}\right.}}
\newcommand{\restr}[2]{{% we make the whole thing an ordinary symbol
  \left.\kern-\nulldelimiterspace % automatically resize the bar with \right
  #1 % the function
  \vphantom{\big|} % pretend it's a little taller at normal size
  \right|_{#2} % this is the delimiter
  }}
%%%%%%%%%%%%%%%%%%%%%%%%%%%%%%%%%%%%%%%


\newtcolorbox{dottedbox}[1][]{%
    colback=white,    % Background color
    colframe=white,    % Border color (to be overridden by dashrule)
    sharp corners,     % Sharp corners for the box
    boxrule=0pt,       % No actual border, as it will be drawn with dashrule
    boxsep=5pt,        % Padding inside the box
    enhanced,          % Enable advanced features
    overlay={\draw[dashed, thin, black, dash pattern=on \pgflinewidth off \pgflinewidth, line cap=rect] (frame.south west) rectangle (frame.north east);}, % Dotted line
    #1                 % Additional options
}

\tcbset{theostyle/.style={
    enhanced,
    sharp corners,
    attach boxed title to top left={
      xshift=-1mm,
      yshift=-4mm,
      yshifttext=-1mm
    },
    top=1.5ex,
    colback=white,
    colframe=blue!75!black,
    fonttitle=\bfseries,
    boxed title style={
      sharp corners,
    size=small,
    colback=blue!75!black,
    colframe=blue!75!black,
  } 
}}

\newtcbtheorem[number within=section]{Theorem}{Theorem}{%
  theostyle
}{thm}

\newtcbtheorem[number within=section]{Definition}{Definition}{%
  theostyle
}{def}



\title{Math H185 Homework 2}
\author{Keshav Balwant Deoskar}

\begin{document}
\maketitle

% \vskip 0.5cm
% \pagebreak 

%%%%%%%%%%%%%%%%%%%%%%%%%%%%%%%%%%%%%%%%%%%%%%%%%%%%%%%%%%%%%%%%%
\begin{mathdefinitionbox}{Question 1}
\vskip 0.5cm
State the subsets of $\C$ on which the following functions are holomorphic, and calculate their derivatives on their domains of holomorphicity:
  \begin{enumerate}[label=(\alph*)]
    \item $f(z) = e^{z^2 + 3z + 4}$
    \item $f(z) = \frac{1}{e^{z}}$
    \item $f(z) = \frac{1}{z^2 - 3z + 2}$
  \end{enumerate}
\end{mathdefinitionbox}
%%%%%%%%%%%%%%%%%%%%%%%%%%%%%%%%%%%%%%%%%%%%%%%%%%%%%%%%%%%%%%%%%

\vskip 0.5cm
\underline{\textbf{Solution:}} 

A function $f(z)$ is holomorphic at $z_0$ if $f$ is $C^1$ and the Cauchy-Riemann equations hold at $z_0$.

\begin{enumerate}[label=(\alph*)]
  \item The function $f(z) = e^{z^2 + 3z + 4}$ is a composition of holomorphic functions as $f(z) = (g \circ h)(z)$ where $h(z) = z^2 + 3z + 4$ and $g(z) = e^{z}$.

  \vskip 0.5cm
  Then, using the chain-rule, 
  \begin{align*}
    f'(z) &= g'(h(z)) \cdot h'(z) \\
    &= e^{z^2 + 3z + 4} \cdot (2z + 3)
  \end{align*}
  \[ \implies \boxed{ f'(z) = (2z + 3)e^{z^2 + 3z + 4} } \]

  This function is holomorphic on all of $\C$.

  \vskip 0.5cm
  \item We can write $f(z) = \frac{1}{e^z}$ as being $f(z) = (g \circ h)(z)$ where $g(z) = \frac{1}{z}$ and $h(z) = e^{z}$.
  

  \vskip 0.5cm
  Using the chain-rule, the derivative in the region of holomorphicity is: 
  \begin{align*}
    f'(z) &= g'(h(z)) \cdot h'(z) \\
    &= \left(\frac{-1}{(e^z)^2}\right) \cdot e^{z} \\
    &= \frac{-1}{e^z}
  \end{align*}
  \[ \implies \boxed{ f'(z) = \frac{-1}{e^z} } \]
  
  This function is also holomorphic on all of $\C$.

  \vskip 0.5cm
  \item Once again, we can express
  \[ f(z) = \frac{1}{z^2 - 3z + 2} \]
  as a composition of holomorphic functions $f(z) = (g \circ h)(z)$ where $g(z) = \frac{1}{z}$ and $h(z) = z^2 - 3z + 2$.

  % \vskip 0.25cm
  % The polynomial $h$ is holomorphic over all of $\C$ while $g$ is holomorphic over $\C \setminus \{0\}$, so the function $h$ is also holomorphic over all of $\C \setminus \{0\}$.

  \vskip 0.25cm
  Its derivative in the region where it is holomorphic is 
  \begin{align*}
    f'(z) &= g'(h(z)) \cdot h'(z) \\
    &= \frac{1}{(z^2 - 3z + 2)^2} \cdot (2z + 3) \\
    &= \frac{2z + 3}{z^2 - 3z + 2}
  \end{align*}
  
  The function is holomorphic as long as the denominator in $f'(z)$ is non-zero: 
  \begin{align*}
    z^2 - 3z + 2 &= (z-1)(z-2)
  \end{align*}
  Thus, the function is holomorphic for $\C \setminus \{1, 2\}$

\end{enumerate}

\vskip 0.5cm
\hrule 
\vskip 0.5cm

%%%%%%%%%%%%%%%%%%%%%%%%%%%%%%%%%%%%%%%%%%%%%%%%%%%%%%%%%%%%%%%%%
\begin{mathdefinitionbox}{Question 3}
\vskip 0.5cm
Using Euler's Theorem, prove the angle addition formulas
\[ \sin(\theta_1 + \theta_2) = \sin(\theta_1)\cos(\theta_2)  +\cos(\theta_1)\sin(\theta_2) \]
and 
\[ \cos(\theta_1 + \theta_2) = \cos(\theta_1)\cos(\theta_2)  - \sin(\theta_1)\sin(\theta_2)   \]
\end{mathdefinitionbox}
%%%%%%%%%%%%%%%%%%%%%%%%%%%%%%%%%%%%%%%%%%%%%%%%%%%%%%%%%%%%%%%%%
  
\vskip 0.5cm
\underline{\textbf{Proof:}}

Euler's theorem tells us that, for $t \in \R$,
\[ \boxed{ e^{it}= \cos(t) + i\sin(t) } \]


\vskip 0.5cm
Then, we have
\[ \boxed{e^{i(\theta_1 + \theta_2)} = \cos(\theta_1 + \theta_2) + i\sin(\theta_1 + \theta_2) } \;\;\;\; (1)\]

\vskip 0.5cm
But we can also write $e^{i(\theta_1 + \theta_2)}$ as 
\begin{align*}
  e^{i(\theta_1 + \theta_2)} &= e^{i\theta_1} \cdot e^{i\theta_2} \\
  &= \left[ \cos(\theta_1) + i\sin(\theta_1) \right] \cdot \left[ \cos(\theta_2) + i\sin(\theta_2) \right] \\
  &= \cos(\theta_1)\cos(\theta_2) + i\cos(\theta_1)\sin(\theta_2) + i\sin(\theta_1) \cos(\theta_2) + i^2 \sin(\theta_1)\sin(\theta_2) \\
  &= \left[ \cos(\theta_1)\cos(\theta_2) - \sin(\theta_1)\sin(\theta_2) \right] + i \left[ \cos(theta_1)\sin(\theta_2) + \sin(\theta_1)\cos(\theta_2) \right]
\end{align*}
\[ \boxed{e^{i(\theta_1 + \theta_2)} = \left[ \cos(\theta_1)\cos(\theta_2) - \sin(\theta_1)\sin(\theta_2) \right] + i \left[ \cos(\theta_1)\sin(\theta_2) + \sin(\theta_1)\cos(\theta_2) \right]} \;\;\;\; (2) \]

\vskip 0.5cm
So, we must have 
\[ \cos(\theta_1 + \theta_2) + i\sin(\theta_1 + \theta_2) = \left[ \cos(\theta_1)\cos(\theta_2) - \sin(\theta_1)\sin(\theta_2) \right] + i \left[ \cos(\theta_1)\sin(\theta_2) + \sin(\theta_1)\cos(\theta_2) \right] \;\;\;\; (3) \]

\vskip 0.5cm
And, taking the real and imaginary parts of equation (3), we get 
\begin{align*}
  \cos(\theta_1 + \theta_2) &= \cos(\theta_1)\cos(\theta_2) - \sin(\theta_1)\sin(\theta_2)  \\
  \sin(\theta_1 + \theta_2) &= \cos(\theta_1)\sin(\theta_2) + \sin(\theta_1)\cos(\theta_2) 
\end{align*}

which are exactly the angle addition formulas.

\vskip 0.5cm
\hrule 
\vskip 0.5cm

%%%%%%%%%%%%%%%%%%%%%%%%%%%%%%%%%%%%%%%%%%%%%%%%%%%%%%%%%%%%%%%%%
\begin{mathdefinitionbox}{Question 3}
\vskip 0.5cm
Show that there is an everywhere holomorphic function $f:\C \rightarrow \C$ such that $f(z) = \sin(z)/z $ for $z\neq0$. What is $f(0)$? Write down $f'(z)$ as a power series.
\end{mathdefinitionbox}
%%%%%%%%%%%%%%%%%%%%%%%%%%%%%%%%%%%%%%%%%%%%%%%%%%%%%%%%%%%%%%%%%

\vskip 0.5cm
\underline{\textbf{Proof:}}

\vskip 0.5cm
% The function $\sin(z)/z$ is holomorphic on $\C \setminus \{0\}$ and fails at the origin because it is not defined there. We can continue function analytically to the entire complex plane by defining 
% \[ f(z) = \begin{cases}
%   \sin(z)/z, \;\;\;  z \neq 0 \\
%   1, \;\;\;\;\;\;\;\;\;\;\;\;\;\; z = 0
% \end{cases} \]

% where we've set the functions value at the origin to be $1$ because 
% \[ \lim_{z \rightarrow 0} \frac{\sin(z)}{z} = 1\]

% This makes the function holomorphic at the origin since 
% \begin{align*}
%   \lim_{z \rightarrow 0} \frac{f(z) - f(0)}{z - 0} &= 
% \end{align*}

The function $\sin(z)/z$ is undefined, however we notice that $\sin(z)$ can be expressed as a power series:
\[ \sin(z) = \sum_{k = 0}^{\infty} \frac{(-1)^k}{(2k+1)!}z^{2k+1} \]

So, we can analytically continue $\sin(z)/z$ as 
\[ \frac{\sin(z)}{z} =  \frac{1}{z} \sum_{k = 0}^{\infty} \frac{(-1)^k}{(2k+1)!}z^{2k+1} = \sum_{k = 0}^{\infty} \frac{(-1)^k}{(2k+1)!}z^{2k} \] 

which has the value $f(0) = 1$ at the origin.

\vskip 0.5cm
This series expansion has infinite radius of convergence, and we proved in class that if $f(z)$ can be written as a power series of convergence radius $R$, then $f'(z)$ equals the $f(z)$ series differentiated term-by-term, and has the same radius of convergence. 

\vskip 0.5cm
Thus, $f(z) = \sum_{k = 0}^{\infty} \frac{(-1)^k}{(2k+1)!}z^{2k}$ is everywhere holomorphic.

\vskip 0.5cm
\hrule 
\vskip 0.5cm

%%%%%%%%%%%%%%%%%%%%%%%%%%%%%%%%%%%%%%%%%%%%%%%%%%%%%%%%%%%%%%%%%
\begin{mathdefinitionbox}{Question 4}
\vskip 0.5cm
Determine the radius of convergence of the power series 
\[ \sum_{z \geq 0} a_n z^n  \]
in the following cases:
\begin{enumerate}[label=(\alph*)]
  \item $a_n = n$
  \item $a_n = \frac{1}{n^2}$
  \item $a_n = \frac{n^2 - 5}{4^n + 3n}$
\end{enumerate}
\end{mathdefinitionbox}
%%%%%%%%%%%%%%%%%%%%%%%%%%%%%%%%%%%%%%%%%%%%%%%%%%%%%%%%%%%%%%%%%

\vskip 0.5cm
\underline{\textbf{Proof:}}

\vskip 0.5cm
We showed in class that for a power series of the form
\[ \sum_{z \geq 0} a_n z^n  \]
the radius of convergence is given by 
\[ \limsup_{n \rightarrow \infty} \lvert a_n \rvert^{1/n} \]

\begin{enumerate}[label=(\alph*)]
  \item For $a_n = n$, 
  \begin{align*}
    R &= \left(\limsup_{n \rightarrow \infty} \lvert a_n \rvert^{1/n}\right)^{-1} \\
    &= \left(\limsup_{n \rightarrow \infty} \lvert n \rvert^{1/n}\right)^{-1} \\
    &= 1
  \end{align*}
  as $\lim_{n \rightarrow \infty} \sqrt[n]{n} = 1$ is a well known limit from real analysis.

  \vskip 0.5cm
  \item For $a_n = 1/n^2$, 
  \begin{align*}
    R &= \left(\limsup_{n \rightarrow \infty} \lvert a_n \rvert^{1/n} \right)^{-1} \\
    &= \left(\limsup_{n \rightarrow \infty} \big\lvert \frac{1}{n^2} \big\rvert^{1/n}\right)^{-1} \\
    &= \frac{1}{\left(\frac{1}{\limsup_{n \rightarrow \infty} \sqrt[n]{n}}\right)^2} \\
    &= \frac{1}{1^2} \\
    &= 1
  \end{align*}
  
  \vskip 0.5cm
  \item For $a_n = \frac{n^2 - 5}{4^n + 3n}$, the radius of convergence is given by:
  \begin{align*}
    R &= \left(\limsup_{n \rightarrow \infty} \lvert a_n \rvert^{1/n} \right)^{-1} \\
    &= \left( \limsup_{n \rightarrow \infty} \big\lvert \frac{n^2 - 5}{4^n + 3n}\big\rvert^{1/n} \right)^{-1}
  \end{align*}

  We use the following theorem and its corollary from Real Analysis:
  \begin{dottedbox}
    \underline{Theorem 12.3 (Ross):} \\
    Let $(s_n)$ be any sequence of non-zero real numbers. Then we have
    \[ \liminf \big\lvert \frac{s_{n+1}}{s_n} \big\rvert \leq \liminf \lvert s_n \rvert^{1/n} \leq \limsup \lvert s_n \rvert^{1/n} \leq \limsup \big\lvert \frac{s_{n+1}}{s_n} \big\rvert \]

    Note: If $\lim_{n \rightarrow \infty} \big\lvert \frac{s_{n+1}}{s_n } \big\rvert$ exists and equals $L$, then all four quantities in the expresson above are equal to $L$.

    \vskip 0.5cm
    \underline{Corollary:} \\
    If $\lim_{n \rightarrow \infty} \big\lvert \frac{s_{n+1}}{s_n } \big\rvert$ exists and equals $L$, then  $\lim_{n \rightarrow \infty} \big\lvert s_n \big\rvert^{1/n}$ exists and equals $L$.
  \end{dottedbox}
   
  Now, 
  \begin{align*}
    \lim_{n \rightarrow \infty} \left| \frac{a_{n+1}}{a_n} \right| &= \lim_{n \rightarrow \infty} \left| \left( \frac{(n+1)^2 - 5}{4^{n+1} + 3(n+1)} \right) \cdot \left( \frac{4^n + 3n}{n^2 - 5} \right) \right| \\
    &= \lim_{n \rightarrow \infty} \left| \frac{4^n}{4^{n+1}} \right| \text{   (Since the exponential dominates)} \\
    &= \frac{1}{4}
  \end{align*}

  Since the limit exists, we have that 
  \[ \limsup_{n \rightarrow \infty} |a_n|^{1/n} =\frac{1}{4} \]

  so the radius of convergence comes out to be $4$.
\end{enumerate}

\vskip 0.5cm
\hrule 
\vskip 0.5cm

%%%%%%%%%%%%%%%%%%%%%%%%%%%%%%%%%%%%%%%%%%%%%%%%%%%%%%%%%%%%%%%%%
\begin{mathdefinitionbox}{Question 5}
\vskip 0.5cm
Prove the following statements:
\begin{enumerate}[label=(\alph*)]
  \item The power series $\sum_n nz^n$ does not converge at any point of the unit circle.
  \item The power series $\sum_n \frac{z^n}{n^2}$ converges at every point of the unit circle.
  \item 
\end{enumerate}
\end{mathdefinitionbox}
%%%%%%%%%%%%%%%%%%%%%%%%%%%%%%%%%%%%%%%%%%%%%%%%%%%%%%%%%%%%%%%%%

\vskip 0.5cm
\underline{\textbf{Proof:}}

\vskip 0.5cm
\begin{enumerate}[label=(\alph*)]
  % \item The radius of covergence for the power series $\sum_n nz^n$ is given by 
  % \begin{align*}
  %   R &= \left( \limsup_{n \rightarrow \infty} \left| a_n \right|^{1/n} \right)^{-1} \\
  %   &= \left( \limsup_{n \rightarrow \infty} \left| n \right|^{1/n} \right)^{-1}
  % \end{align*}

  \item If $\left| z \right| = 1$, we have $nz^n \rightarrow \infty$ so the series does not converge for any point on the unit circle.
  \item If $\left| z \right| = 1$, we have $\left| \frac{z}{n^2} \right| = \frac{\left| z \right|^n}{\left| n^2 \right|} = \frac{1}{n^2}$, and we know from real analysis that $\sum_{n} \frac{1}{n^2}$ converges. So, the series $\sum_{n} \left|\frac{z^n}{n^2} \right|$ converges, which allows us to conclude that $\sum_n \frac{z^ n}{n^2}$ converges absolutely.
\end{enumerate}

\vskip 0.5cm
\hrule 
\vskip 0.5cm

%%%%%%%%%%%%%%%%%%%%%%%%%%%%%%%%%%%%%%%%%%%%%%%%%%%%%%%%%%%%%%%%
\begin{mathdefinitionbox}{Question 6}
\vskip 0.5cm
Calculate the derivative matrix 
\[ \begin{pmatrix}
  \frac{\partial u }{\partial x} & \frac{\partial u }{\partial y} \\
  \frac{\partial v }{\partial x} & \frac{\partial v }{\partial y}
\end{pmatrix} \]
for the function $f : \C \rightarrow \C, f(z) = |z|^2$, and using the Cauchy-Riemann equations, find all points $z_0 \in \C$ at which it is holomorphic.
\end{mathdefinitionbox}
%%%%%%%%%%%%%%%%%%%%%%%%%%%%%%%%%%%%%%%%%%%%%%%%%%%%%%%%%%%%%%%%%

\vskip 0.5cm
\underline{\textbf{Proof:}}

\vskip 0.5cm
For $z = x+ iy$ we can write the function $f(z) = |z|^2$ as 
\begin{align*}
  f(z) &= |z|^2 \\
  &= z \cdot z^{*} \\
  &= (x + iy) \cdot (x - iy) \\
  &= x^2 + y^2
\end{align*}

\vskip 0.5cm
So, $\text{Re}(f(z)) = u(x, y) = x^2 + y^2$ and $\text{Im}(f(z)) = v(x, y) = 0$.

\vskip 0.5cm
Thus, 
\[ \frac{\partial u}{\partial x} = 2x, \;\;\;\; \frac{\partial u}{\partial y} = 2y \]
and 
\[ \frac{\partial v}{\partial x} = 0, \;\;\;\; \frac{\partial v}{\partial y} = 0 \]

\vskip 0.5cm
Thus, the derivative matrix is 
\[  \begin{pmatrix}
  2x & 2y \\
  0 & 0
\end{pmatrix}  \]

\vskip 0.5cm
Now, we know that if a function $f(z)$ is $C^1$ and the Cauchy-Riemann equations hold at $z_0$, then $f(z)$ is holomorphic at $z_0$. We're already found each of the first derivatives, and observed that they're continuous real functions. So, $f(z) = |z|^2$ is certainly $C^1$.

\vskip 0.25cm
Now at a point $z_0 = x_0 +i y_0 \in \C$, in order for the CR equations to hold, we need 

\begin{align*}
  \frac{\partial u}{\partial x}(z_0) &= \frac{\partial v}{\partial y}(z_0) \implies 2x_0 = 0 \\
  &\text{and}  \\
  \frac{\partial u}{\partial y}(z_0) &= -\frac{\partial v}{\partial x}(z_0)  \implies 2y_0 = 0
\end{align*}

So, the only point where $f(z) = |z|^2$ satisfies the Cauchy-Riemann equations is $z = \mathbf{0}$. However, $\{\mathbf{0}\} \subseteq \C$ is not an open subset. So, there are no open subsets $\Omega \subseteq_{open} \C$ where $f(z) = |z|^2$ is $C^1$ and satisfies the Cauchy-Riemann equations. 

\vskip 0.5cm
Therefore, the function $f(z) = |z|^2$ is \emph{\textbf{holomorphic nowhere}}.


\vskip 0.5cm
\hrule 
\vskip 0.5cm

%%%%%%%%%%%%%%%%%%%%%%%%%%%%%%%%%%%%%%%%%%%%%%%%%%%%%%%%%%%%%%%%
\begin{mathdefinitionbox}{Question 7}
\vskip 0.5cm
For a piece-wise smooth parametrized curve $\gamma : [a, b] \rightarrow \C$, let $\gamma^{-} : [a, b] \rightarrow \C$ be the curve with the reverse orientation, defined by $\gamma^{-}(t) = \gamma(a + b - t)$. Show that 
\begin{enumerate}[label=(\alph*)]
  \item $\int_{\gamma} f(z) dz =  - \int_{\gamma^{-}} f(z) dz$
  \item Let $z_0 \in \C$ and $r$ be a real positive number with $r < |z_0|$. For every $n \in \mathbb{Z}$, find (with proof) the value of 
  \[ \int_{\partial B_{r}(z_0)} z^n dz \]
\end{enumerate} 
\end{mathdefinitionbox}
%%%%%%%%%%%%%%%%%%%%%%%%%%%%%%%%%%%%%%%%%%%%%%%%%%%%%%%%%%%%%%%%%

\vskip 0.5cm
\underline{\textbf{Proof:}}

\begin{enumerate}[label=(\alph*)]
  \item The integral of $f(z)$ along the curve $\gamma$ is defined to be
  \[ \int_{\gamma} f(z)dz = \int_{a}^{b} f(\gamma(t)) \gamma'(t) dt \]
  So, the integral over $\gamma^{-}$ is 
  \begin{align*}
    \int_{\gamma^{-}} f(z) dz &= \int_{a}^{b} f(\gamma^{-}(t))(\gamma^{-})'(t) dt \\
    &= \int_{a}^{b} f(\gamma(a + b - t)) \gamma'(a + b - t) dt
  \end{align*}
  Introduce the substitution $u = a + b - t$. Then, $\gamma'(u)du = (-\gamma'(t))(-dt) = \gamma'(t) dt$,  and 
  \[   \begin{cases}
  t = a \implies u = b \\
  t = b \implies u = a
  \end{cases} \]

  Then, 
  \begin{align*}
    \int_{\gamma^{-}} f(z)dz &= \int_{b}^{a} f(\gamma(u)) \gamma'(u) (du) \\
    &= -\int_{a}^{b} f(\gamma(u)) \gamma'(u) (du) \\
    &= -\int_{a}^{b} f(\gamma(t)) \gamma'(t) (dt) \;\;\;\text{ (Since $u$ and $t$ are just dummy variables)} \\
    &= -\int_{\gamma} f(\gamma(t)) \gamma'(t) dt
  \end{align*}

  Thus, reversing the orientation of the path introduces a negative sign!
  \[ \boxed{ \int_{\gamma} f(z)dz = -\int_{\gamma^{-}} f(z)dz} \]

  \vskip 0.5cm
  \item We can parametrize the circle of radius $r < |z_0|$ around the point $z_0$ i.e. $\partial B_{r}(z_0)$ as 
  \[ \gamma : [0, 2\pi] \rightarrow \C,\;\;\; \gamma(t) = z_0 + re^{it} \]

  Now, we want to integrate $f(z) = z^n$ over this curve.
  \begin{align*}
    \int_{\gamma} z^n dz &= \int_{0}^{2\pi} f(\gamma(t)) \; \gamma'(t) dt \\
    &= \int_{0}^{2\pi} (z_0 + re^{it})^n \cdot (ire^{it}) dt \\
  \end{align*}
  % Using the Binomial Theorem 
  % \[ (x + y)^n = \sum_{k = 0}^{n} \binom{n}{k} x^n y^{n-k} \]
  % we can write

  The easiest way to do so is to use the \emph{\textbf{Fundamental Theorem of Contour Integrals}}:
  \begin{dottedbox}
    \underline{Theorem:} If a continuous function $f$ has primitive $F$ on open set $\Omega$ and $\gamma : [a, b] \rightarrow \C$ is a piece-wise smooth continuous curve in $\Omega$ with $\gamma(a) = w_1, \gamma(b) = w_2$ then
    \[ \int_{\gamma} f(z)dz = F(w_2) - F(w_1)  \]
  \end{dottedbox}
  
  \vskip 0.5cm
  \underline{Proof:} If $\gamma$ is smooth, the proof is a simple application of the Chain rule and Fundamental Theorem of Calculus (over the real numbers):
  \begin{align*}
    \int_{\gamma} f(z) dz &= \int_{w_1}^{w_2} f(\gamma(t)) \gamma'(t) dt \\
    &= \int_{a}^{b} F'(\gamma(t)) \gamma'(t) dt \\
    &= \int_{a}^{b} \left( \frac{d}{dt} F(\gamma(t)) \right) dt \\
    &= F(\gamma(b)) - F(\gamma(a)) \\
    &= F(w_2) - F(w_1)
  \end{align*} 

  \vskip 0.5cm
  If $\gamma$ is only piece-wise smooth, with points $a = a_0 < a_1 < \dots < a_{n} = b$ such that it's smooth on each $[a_k, a_{k+1}]$ for $0 \leq k \leq n-1$, then we can argue exactly as before each interval and then obtain the entire integral from a telescoping series:

  \begin{align*}
    \int_{\gamma} f(z)dz &= \int_{a}^{b} f(\gamma(t)) \cdot \gamma'(t) dt \\
    &= \sum_{k = 0}^{n - 1} \int_{a_k}^{a_{k+1}} f(\gamma(t)) \cdot \gamma'(t) dt \\
    &= \sum_{k = 0}^{n - 1} F(\gamma(a_{k+1})) - F(\gamma(a_k)) \\
    &= F(\gamma(a_{n})) - F(\gamma(a_0)) \\
    &= F(\gamma(b)) - F(\gamma(a)) \\
    &= F(w_2) - F(w_1)
  \end{align*}
  
  \begin{dottedbox}
    \underline{Corollary:} If $\gamma$ is a path such that $\gamma(b) = \gamma(a)$, then 
    \[ \int_{\gamma} f(z)dz = F(\gamma(b)) - F(\gamma(a)) = 0 \]
  \end{dottedbox}    

  \vskip 0.5cm
  Now returning to the integral
  \begin{align*}
    \int_{\gamma} z^n dz  
  \end{align*}

  \underline{If $n \neq -1$}, then $z^n$ has a primitive (namely $z^{n+1}/(n+1)$) and the path we're integrating is such that the stard and end points are the same. Thus, by the Corollary to the Fundamental Theorem of Contour Integrals, we have 
  \[ \int_{\gamma} z^n dz = 0, \;\;\;\;n \neq -1  \]
  
  \vskip 0.5cm
  \underline{If $n = -1$}, using the curve parametrization $\gamma(t) = z_0 + re^{it}$ where $t \in [0, 2\pi]$ we have 
  
  \begin{align*}
    \int_{\gamma} \frac{1}{z} dz &= \int_{0}^{2\pi} \frac{1}{\gamma(t)} \cdot \gamma'(t) dt \\
    &= \int_{0}^{2\pi} \frac{1}{z_0 + re^{it}} \cdot (ire^{it}) dt \\
    &= \frac{i r}{z_0} \int_{0}^{2\pi} \frac{e^{it}}{1 + \frac{r}{z_0} e^{it}} dt
  \end{align*}

  Note that because our integral doesn't contain the origin, we have $r < |z_0| \implies |\frac{r}{z_0}e^{it}| < 1$. This allows us to expand the $1/(1 + \frac{r}{z_0}e^{it})$ in the integrand as a power series

  \begin{align*}
    \frac{ir}{z_0} \int _{0}^{2\pi} \frac{e^{it}}{1 + \frac{r}{z_0}e^{it}} dt &= \frac{ir}{z_0}\int _{0}^{2\pi} e^{it} \sum_{k = 0}^{\infty} \left( \frac{-r}{z_0} e^{it} \right)^k dt \\
    &= -i \sum_{k = 0}^{\infty} \left( \frac{-r}{z_0} \right)^{k+1} \int_{0}^{2\pi} e^{i(k+1)t} dt \\
    &= -i \sum_{k = 0}^{\infty} \left( \frac{-r}{z_0} \right)^{k+1} \cdot \frac{e^{i(k+1)t}}{i(k+1)} \Big\vert_{0}^{2\pi} \\
    &= 0 \text{  Since $e^{i0} = e^{i2\pi}$}
  \end{align*}
  where we were able to switch the integral and sum in the second equality because the sum converges absolutely as the sum of abolute values
  \begin{align*}
    \sum_{k = 0}^{\infty} \big\lvert -\frac{r}{z_0} e{it} \big\rvert^k &= \sum_{k = 0}^{\infty} 1 \cdot \underbrace{\big\lvert \frac{r}{z_0} \big\rvert^k}_{<1} \\
    &= \frac{1}{1 - |\frac{r}{z_0}|} \;\;\text{ (Geometric series of real numbers)}
  \end{align*}

  \begin{dottedbox}
    In conclusion, we found that 
    \[ \int_{\partial B_r(z_0)} z^n dz = 0 \]
    for $r < |z_0|$ and all $n \in \mathbb{Z}$.
  \end{dottedbox}
\end{enumerate}

\vskip 0.5cm
\hrule 
\vskip 0.5cm




% %%%%%%%%%%%%%%%%%%%%%%%%%%%%%%%%%%%%%%%%%%%%%%%%%%%%%%%%%%%%%%%%%
% \begin{mathdefinitionbox}{Question }
% \vskip 0.5cm

% \end{mathdefinitionbox}
% %%%%%%%%%%%%%%%%%%%%%%%%%%%%%%%%%%%%%%%%%%%%%%%%%%%%%%%%%%%%%%%%%

% \vskip 0.5cm
% \underline{\textbf{Proof:}}

% \vskip 0.5cm
% \hrule 
% \vskip 0.5cm

\end{document}
