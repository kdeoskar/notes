\documentclass{article}

% Language setting
% Replace `english' with e.g. `spanish' to change the document language
\usepackage[english]{babel}

% Set page size and margins
% Replace `letterpaper' with`a4paper' for UK/EU standard size
\usepackage[letterpaper,top=2cm,bottom=2cm,left=3cm,right=3cm,marginparwidth=1.75cm]{geometry}

% Useful packages
\usepackage{amsmath}
\usepackage{amssymb}
\usepackage{graphicx}
\usepackage{enumitem}
\usepackage[colorlinks=true, allcolors=blue]{hyperref}

\usepackage{hyperref}
\hypersetup{
    colorlinks=true,
    linkcolor=blue,
    filecolor=magenta,      
    urlcolor=cyan,
    pdftitle={Math 185 Homework 10},
    pdfpagemode=FullScreen,
    }

\urlstyle{same}

\usepackage{tikz-cd}

%%%%%%%%%%% Box pacakges and definitions %%%%%%%%%%%%%%
\usepackage[most]{tcolorbox}
\usepackage{xcolor}
\usepackage{csquotes}
% Define the colors
\definecolor{boxheader}{RGB}{0, 51, 102}  % Dark blue
\definecolor{boxfill}{RGB}{173, 216, 230}  % Light blue

% Define the tcolorbox environment
\newtcolorbox{mathdefinitionbox}[2][]{%
    colback=boxfill,   % Background color
    colframe=boxheader, % Border color
    fonttitle=\bfseries, % Bold title
    coltitle=white,     % Title text color
    title={#2},         % Title text
    enhanced,           % Enable advanced features
    attach boxed title to top left={yshift=-\tcboxedtitleheight/2}, % Center title
    boxrule=0.5mm,      % Border width
    sharp corners,      % Sharp corners for the box
    #1                  % Additional options
}
%%%%%%%%%%%%%%%%%%%%%%%%%

\usepackage{biblatex}
\addbibresource{sample.bib}


%%%%%%%%%%% New Commands %%%%%%%%%%%%%%
\newcommand*{\T}{\mathcal T}
\newcommand*{\cl}{\text cl}


\newcommand{\ket}[1]{|#1 \rangle}
\newcommand{\bra}[1]{\langle #1|}
\newcommand{\inner}[2]{\langle #1 | #2 \rangle}
\newcommand{\R}{\mathbb{R}}
\newcommand{\C}{\mathbb{C}}
\newcommand{\V}{\mathbb{V}}
\newcommand{\Hilbert}{\mathcal{H}}
\newcommand{\oper}{\hat{\Omega}}
\newcommand{\lam}{\hat{\Lambda}}

\newcommand{\bigslant}[2]{{\raisebox{.2em}{$#1$}\left/\raisebox{-.2em}{$#2$}\right.}}
\newcommand{\restr}[2]{{% we make the whole thing an ordinary symbol
  \left.\kern-\nulldelimiterspace % automatically resize the bar with \right
  #1 % the function
  \vphantom{\big|} % pretend it's a little taller at normal size
  \right|_{#2} % this is the delimiter
  }}
%%%%%%%%%%%%%%%%%%%%%%%%%%%%%%%%%%%%%%%


\newtcolorbox{dottedbox}[1][]{%
    colback=white,    % Background color
    colframe=white,    % Border color (to be overridden by dashrule)
    sharp corners,     % Sharp corners for the box
    boxrule=0pt,       % No actual border, as it will be drawn with dashrule
    boxsep=5pt,        % Padding inside the box
    enhanced,          % Enable advanced features
    overlay={\draw[dashed, thin, black, dash pattern=on \pgflinewidth off \pgflinewidth, line cap=rect] (frame.south west) rectangle (frame.north east);}, % Dotted line
    #1                 % Additional options
}

\tcbset{theostyle/.style={
    enhanced,
    sharp corners,
    attach boxed title to top left={
      xshift=-1mm,
      yshift=-4mm,
      yshifttext=-1mm
    },
    top=1.5ex,
    colback=white,
    colframe=blue!75!black,
    fonttitle=\bfseries,
    boxed title style={
      sharp corners,
    size=small,
    colback=blue!75!black,
    colframe=blue!75!black,
  } 
}}

\newtcbtheorem[number within=section]{Theorem}{Theorem}{%
  theostyle
}{thm}

\newtcbtheorem[number within=section]{Definition}{Definition}{%
  theostyle
}{def}



\title{Math H185 Homework 10}
\author{Keshav Balwant Deoskar}

\begin{document}
\maketitle


%%%%%%%%%%%%%%%%%%%%%%%%%%%%%%%%%%%%%%%%%%%%%%%%%%%%%%%%%%%%%%%%
\begin{mathdefinitionbox}{Question 1}
\vskip 0.5cm
Show that if a Mobius transformation $g \in \mathrm{PSL}_2(\C)$ sends $\hat{\R}$ to $\hat{\R}$, then $g$ is an element of $\mathrm{PGL}_2(\R)$. Exhibit a Mobius transformation which sends $\hat{\R}$ to $\hat{\R}$ but is not given by an element of $\mathrm{PSL}_2(\R)$.
\end{mathdefinitionbox}
%%%%%%%%%%%%%%%%%%%%%%%%%%%%%%%%%%%%%%%%%%%%%%%%%%%%%%%%%%%%%%%%%

\vskip 0.5cm
\underline{\textbf{Proof:}}

Consider a  Mobius transformation $g \in \mathrm{PSL}_2(\C)$ which has corresponding matrices $M, -M \in \mathrm{SL}_2(\R)$ described by 

\[ M = \begin{pmatrix}
  a & b \\
  c & d
\end{pmatrix} \]

Assume $M$ has determinant $1$ (so $-M$ has det = $-1$). This gives us the constraint $ad - bc = 1$. We also know that $g$ sends $\hat{\R}$ to itself, so a complex number $z = x + 0i$ for any $x \in \hat{\R}$ gets sent to another complex number $g \cdot z = y + 0i$, $y \in \hat{\R}$.

\begin{align*}
  &\begin{pmatrix}
    a & b \\
    c & d
  \end{pmatrix} \begin{pmatrix}
    x \\ 0
  \end{pmatrix} = \begin{pmatrix}
    y \\ 0
  \end{pmatrix} \\
  \implies&\begin{cases}
    ax = y \\
    cx = 0
  \end{cases}
\end{align*}

So, we have an additional constraint: $c = 0$. Which means $ad = 1 \implies a = 1/d$. We still have no constraints on 

[Come back to this one]


\vskip 0.5cm
\hrule 
\vskip 0.5cm


%%%%%%%%%%%%%%%%%%%%%%%%%%%%%%%%%%%%%%%%%%%%%%%%%%%%%%%%%%%%%%%%
\begin{mathdefinitionbox}{Question 2}
\vskip 0.5cm
Let $U, V$ be open subsets of $\C$. Suppose $f : U \rightarrow V$ is holomorphic and injective. Show that $f'(z) \neq 0$ for all $z \in U$. 
\end{mathdefinitionbox}
%%%%%%%%%%%%%%%%%%%%%%%%%%%%%%%%%%%%%%%%%%%%%%%%%%%%%%%%%%%%%%%%%

\vskip 0.5cm
\underline{\textbf{Proof:}}

Suppose for contradiction there exists $z_0 \in U$ such that $f'(z_0) = 0$. Then, WLOG, we can assume the power series expansion of $f$ about $z_0$ has the form 

\begin{align*}
  f(z) &= a_k(z-z_0)^k + a_{k+1}(z-z_0)^{k+1} \cdots \\
  &= \left(z-z_0\right)^k \left( a_k + a_{k+1} (z - z_0) + \cdots \right)
\end{align*}
wth $a_k = f^{(k)}(z_0), k \geq 2$.

Let $g(z) = a_k + a_{k+1}(z-z_0) + \cdots$. Note that $g(z_0) \neq 0$. Hence, there exists $\delta > 0$ so that in $D_{\delta}(z_0)$ we have a branch $g^{1/k}(z)$ of the $k-$th-root of $g(z)$. Therefore, $f(z) = \left(h(z)\right)^k$, where $h$ is a holomorphic function defined by 
\[ h(z) = (z-z_0) g^{1/k}z \] and hence $h(z_0) = 0, h;(z_0) = g^{1/k}(z_0) \neq 0$. Since, on a small disk, $h$ is injective and $z^k$ is also one-to-one, $f$ is $k-$to-one in a neighborhood of $z_0$. This contradicts the assumption that $f(z)$ is injective.


\vskip 0.5cm
\hrule 
\vskip 0.5cm


%%%%%%%%%%%%%%%%%%%%%%%%%%%%%%%%%%%%%%%%%%%%%%%%%%%%%%%%%%%%%%%%
\begin{mathdefinitionbox}{Question 3}
\vskip 0.5cm
Find a biholomorphism between the upper half-disk $\mathbb{D} \cap \mathbb{H}$ and the unit disk $\mathbb{D}$.
\end{mathdefinitionbox}
%%%%%%%%%%%%%%%%%%%%%%%%%%%%%%%%%%%%%%%%%%%%%%%%%%%%%%%%%%%%%%%%%

\vskip 0.5cm
\underline{\textbf{Solution:}}

First, the map 
\[f_1 :  z \mapsto \frac{1+z}{1-z} \] takes the upper half of the unit disk to the first quadrant. 

Then, 
\[ f_2 : z \mapsto z^2  \] takes the first quadrant to the upper half plane.

Finally, 
\[ f_3 : z \mapsto \frac{z-i}{z+i} \] maps the upper half-plane to the unit disk. 

Altogether, the map we need is 
\begin{align*}
  f(z) &= (f_3 \circ f_2 \circ f_1) (z) = \frac{\left(\frac{1+z}{1-z}\right)^2 - i}{\left(\frac{1+z}{1-z}\right)^2 + i} \\
  &= -i \cdot \frac{z^2 + 2iz + 1}{z^2 - 2iz + 1}
\end{align*}

\vskip 0.5cm
\hrule 
\vskip 0.5cm



%%%%%%%%%%%%%%%%%%%%%%%%%%%%%%%%%%%%%%%%%%%%%%%%%%%%%%%%%%%%%%%%
\begin{mathdefinitionbox}{Question 4}
\vskip 0.5cm
Find a biholomorphism between the half-strip \[ U = \{ z = x + iy \in \C \;:\; 0 < y < \pi, x > 0 \}   \] and the strip \[ V = \{ z  = x + iy \in \C \;:\; 0 < y < 2\pi \}  \]
\end{mathdefinitionbox}
%%%%%%%%%%%%%%%%%%%%%%%%%%%%%%%%%%%%%%%%%%%%%%%%%%%%%%%%%%%%%%%%%

\vskip 0.5cm
\underline{\textbf{Solution:}}

The map 

\vskip 0.5cm
\hrule 
\vskip 0.5cm



%%%%%%%%%%%%%%%%%%%%%%%%%%%%%%%%%%%%%%%%%%%%%%%%%%%%%%%%%%%%%%%%
\begin{mathdefinitionbox}{Question 5}
\vskip 0.5cm
Find a biholomorphism between the half-strip \[ \{ z = x + iy \in \C \;:\; 0 < y < 2\pi \}   \] and the upper half-plane.
\end{mathdefinitionbox}
%%%%%%%%%%%%%%%%%%%%%%%%%%%%%%%%%%%%%%%%%%%%%%%%%%%%%%%%%%%%%%%%%

\vskip 0.5cm
\underline{\textbf{Solution:}}

% We can define a biholomorphism between $U = \{ z = x + iy \in \C \;:\; 0 < y < 2\pi \} $ and $\mathbb{H}$ as $\psi : U \rightarrow \mathbb{H}$

% \begin{align*}
%   (x + iy) \mapsto (x + i(xy))
% \end{align*}

% The map just scales the imaginary part of the complex numbers to be arbitrary rather than constrained between $0$ and $2\pi$.

The map $z \mapsto 2\log(z)$ with the negative imaginary axis deleted takes the upper half-plane to the half-strip. The inverse map is $\omega \mapsto e^{2\omega}$.


\vskip 0.5cm
\hrule 
\vskip 0.5cm



%%%%%%%%%%%%%%%%%%%%%%%%%%%%%%%%%%%%%%%%%%%%%%%%%%%%%%%%%%%%%%%%
\begin{mathdefinitionbox}{Question 6}
\vskip 0.5cm
Does there exist a holomorphic surjection from the unit disk to $\C$? Either write down an explicit formula for one, or prove that none exists.
\end{mathdefinitionbox}
%%%%%%%%%%%%%%%%%%%%%%%%%%%%%%%%%%%%%%%%%%%%%%%%%%%%%%%%%%%%%%%%%

\vskip 0.5cm
\underline{\textbf{Solution:}}

Yes, we can construct a holomorphic surjection from the unit disk to $\C$ via a composition of multiple maps. 

\vskip 0.5cm
We have the following chain of maps:
\[ \mathbb{D} \xrightarrow[]{F} \mathbb{H} \xrightarrow[]{g} \left(\mathbb{H} - i\right)  \xrightarrow[]{h} \C \]

\vskip 0.5cm
\begin{itemize}
  \item First, we have the conformal map $F : \mathbb{D} \rightarrow \mathbb{H}$ given by \[ F(z) = \frac{i-z}{i+z} \] which was discussed in class (and proven to be a conformal map). 
  \item Then, we can shift the entire half-plane down by one unit via the map $g : z \mapsto z - i$. Denote the image of this map as 
  \[ \mathbb{H} - i = \{ z \in \C : \mathrm{Im}(z) > -1 \} \]
  \item Finally we have the square-map $h : z \mapsto z^2$. This map doubles the argument of $z$ i.e. it sends $re^{i\theta} \mapsto r^2 e^{i(2\theta)}$ so the image is the entire complex plane. 
\end{itemize}

\vskip 0.5cm
Explicitly, the formula for the map is 
\[ z \mapsto \left(F(z) - 1\right)^2 \]

\vskip 0.5cm
\hrule 
\vskip 0.5cm



%%%%%%%%%%%%%%%%%%%%%%%%%%%%%%%%%%%%%%%%%%%%%%%%%%%%%%%%%%%%%%%%
\begin{mathdefinitionbox}{Question 7}
\vskip 0.5cm
Let $f : \mathbb{H} \rightarrow \C$ be a holomorphic function satisfying $|f(z)| \leq 1$ and $f(i) = 0$. Prove that \[ |f(z)| \leq \left| \frac{z-i}{z+i}  \right| \] for all $z \in \mathbb{H}$.
\end{mathdefinitionbox}
%%%%%%%%%%%%%%%%%%%%%%%%%%%%%%%%%%%%%%%%%%%%%%%%%%%%%%%%%%%%%%%%%

\vskip 0.5cm
\underline{\textbf{Proof:}}

Since $|f(z)| \leq 1$ we can essentially think of $f$ as a function between $\mathbb{H}$ and $\overline{\mathbb{D}}$. 

\vsize 0.5cm
Let $T : \mathbb{D} \rightarrow \mathbb{H}$ be the map defined by \[ T(z) = i \frac{1+z}{1-z} \] 



Since $z \in \mathbb{D}$ this map is holomorphic. Then, we see that $f \circ T : \mathbb{D} \rightarrow \overline{\mathbb{D}}$ is holomorphic with $f(T(0)) = f(i) = 0$. By the Schwartz Lemma, we have $|f(T(z))| \leq |z|$. 


Also, we have 
\[ T^{-1}(z) = i\frac{z-1}{z+1} \]


Using this fact, we see that 

\begin{align*}
  |f(z)| = |\left(f \circ T \circ T^{-1}\right) (z)| &= \left| \left(f \circ T\right) \left(i \frac{z-1}{z+1}\right) \right| \\
  &\leq \left| \left(i  \frac{z-1}{z+1}\right) \right| \\
  &=\left| \frac{z-1}{z+1}\right| \\
\end{align*}

Thus, 
\[ \boxed{ |f(z)| \leq \left| \frac{z-1}{z+1} \right| } \]


\vskip 0.5cm
\hrule 
\vskip 0.5cm



%%%%%%%%%%%%%%%%%%%%%%%%%%%%%%%%%%%%%%%%%%%%%%%%%%%%%%%%%%%%%%%%
\begin{mathdefinitionbox}{Question 8}
\vskip 0.5cm
A complex number $\omega \in \mathbb{D}$ is a \emph{fixed point} of a map $f : \mathbb{D} \rightarrow \mathbb{D}$ if $f(w) = w$.
\begin{enumerate}[label=(\alph*)]
  \item Prove that if $f : \mathbb{D} \rightarrow \mathbb{D}$ is analytic and has two distinct fixed points, then $f$ is the identity map.
  \item Must every holomorphic function $f : \mathbb{D} \rightarrow \mathbb{D}$ have a fixed point?
\end{enumerate}
\end{mathdefinitionbox}
%%%%%%%%%%%%%%%%%%%%%%%%%%%%%%%%%%%%%%%%%%%%%%%%%%%%%%%%%%%%%%%%%

\vskip 0.5cm
\underline{\textbf{Proof:}}

\begin{enumerate}[label=(\alph*)]
  \item The function $f : \mathbb{D} \rightarrow \mathbb{D}$ is analytic (and thus also holomorphic) and there exist two fixed points $w_1, w_2 \in \mathbb{D}$ of $f$. 
  
  We know by the Schwartz Lemma that since $f(z)$ has a fixed point, it must be a rotation i.e. $f(z) = e^{i\theta} z$ for some $\theta \in \R$. But recall that there are \emph{two} distinct fixed points i.e. $w_1 \neq w_2$. i.e. at least one of these fixed points must be non-trival (both can't be the origin else we'd have $w_1 = w_2$). Let $w_1$ be the (guaranteed) non-trivial fixed point. Then, $f(z_1) = e^{i\theta} z_1$ for non-zero $z_1$ means we must have $\theta = 0$.

  \vskip 0.5cm
  \item No. For a counterexample, consider the function 
  \[ f(z) = \frac{z+1}{2} \] 

  This function essentially moves every point along the real axis a little bit, but the factor of $1/2$ ensures that $f(z) \in \mathbb{D}$ given $z \in \mathbb{D}$.
\end{enumerate}

\vskip 0.5cm
\hrule 
\vskip 0.5cm



%%%%%%%%%%%%%%%%%%%%%%%%%%%%%%%%%%%%%%%%%%%%%%%%%%%%%%%%%%%%%%%%
\begin{mathdefinitionbox}{Question 9}
\vskip 0.5cm
Prove that all conformal mappings from the upper half-plane $\mathbb{H}$ to the unit disk $\mathbb{D}$ take the form \[ z \mapsto e^{i\theta} \frac{z - \beta}{z + \beta}, \;\;\;\; \theta \in \R, \beta \in \mathbb{H} \] 
\end{mathdefinitionbox}
%%%%%%%%%%%%%%%%%%%%%%%%%%%%%%%%%%%%%%%%%%%%%%%%%%%%%%%%%%%%%%%%%

\vskip 0.5cm
\underline{\textbf{Proof:}}

Let $f : \mathbb{H} \rightarrow \mathbb{D}$ be a generic conformal map from the upper half plane to the unit disk. Let $G : \mathbb{D} \rightarrow \mathbb{H}$ be the conformal map defined as \[ G(w) = i\frac{1-w}{1+w} \]

This map has inverse \[ G^{-1}(z) = \frac{i-z}{i+z} \]


Then, $\left(f \circ G\right) : \mathbb{D} \rightarrow \mathbb{D}$ is an automorphism on the unit disk. We know the automorphisms on the unit disk are of the form 
\[ F(w) = e^{i\theta}  \frac{\alpha - z}{1 - \overline{\alpha} z} \] where $\alpha \in \mathbb{D}$, $\theta \in \R$. 



Thus, for $w \in \mathbb{D}$ and some appropriate value $\alpha$, we have

\begin{align*}
  &\left(f \circ G\right)(w) = F(w) \\
  \implies& f\left( i\frac{1-w}{1+w} \right) = e^{i\theta}  \frac{\alpha - w}{1 - \overline{\alpha} w}\\
\end{align*}

So, solving for $f(z)$ we get 

\begin{align*}
  &f(z) = e^{i\theta}\frac{\alpha - G^{-1}(z)}{1 - \overline{\alpha} G^{-1}(z)} \\
  \implies&f(z) = e^{i\theta} \frac{\alpha - \left(\frac{i-z}{i+z} \right) }{1 - \overline{\alpha} \left(\frac{i-z}{i+z} \right)} \\
  \implies&f(z) = e^{i\theta} \underbrace{\left( \frac{1-\alpha}{1+\overline{\alpha}} \right)}_{\text{modulus} = 1} \frac{z - \beta}{z + \beta}  \\
  \implies&f(z) = e^{i\theta'} \frac{z - \beta}{z + \beta}  \\
\end{align*}

where $\beta = i \frac{1-\alpha}{1+\alpha} \in \mathbb{H}$. 


\vskip 0.5cm
\hrule 
\vskip 0.5cm




%%%%%%%%%%%%%%%%%%%%%%%%%%%%%%%%%%%%%%%%%%%%%%%%%%%%%%%%%%%%%%%%%
% \begin{mathdefinitionbox}{Question }
% \vskip 0.5cm

% \end{mathdefinitionbox}
% %%%%%%%%%%%%%%%%%%%%%%%%%%%%%%%%%%%%%%%%%%%%%%%%%%%%%%%%%%%%%%%%%

% \vskip 0.5cm
% \underline{\textbf{Proof:}}

% \vskip 0.5cm
% \hrule 
% \vskip 0.5cm



\end{document}
