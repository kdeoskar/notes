\documentclass{article}

% Language setting
% Replace `english' with e.g. `spanish' to change the document language
\usepackage[english]{babel}

% Set page size and margins
% Replace `letterpaper' with`a4paper' for UK/EU standard size
\usepackage[letterpaper,top=2cm,bottom=2cm,left=3cm,right=3cm,marginparwidth=1.75cm]{geometry}

% Useful packages
\usepackage{amsmath}
\usepackage{amssymb}
\usepackage{graphicx}
\usepackage{enumitem}
\usepackage[colorlinks=true, allcolors=blue]{hyperref}

\usepackage{hyperref}
\hypersetup{
    colorlinks=true,
    linkcolor=blue,
    filecolor=magenta,      
    urlcolor=cyan,
    pdftitle={Math 185 Homework 5},
    pdfpagemode=FullScreen,
    }

\urlstyle{same}

\usepackage{tikz-cd}

%%%%%%%%%%% Box pacakges and definitions %%%%%%%%%%%%%%
\usepackage[most]{tcolorbox}
\usepackage{xcolor}
\usepackage{csquotes}
% Define the colors
\definecolor{boxheader}{RGB}{0, 51, 102}  % Dark blue
\definecolor{boxfill}{RGB}{173, 216, 230}  % Light blue

% Define the tcolorbox environment
\newtcolorbox{mathdefinitionbox}[2][]{%
    colback=boxfill,   % Background color
    colframe=boxheader, % Border color
    fonttitle=\bfseries, % Bold title
    coltitle=white,     % Title text color
    title={#2},         % Title text
    enhanced,           % Enable advanced features
    attach boxed title to top left={yshift=-\tcboxedtitleheight/2}, % Center title
    boxrule=0.5mm,      % Border width
    sharp corners,      % Sharp corners for the box
    #1                  % Additional options
}
%%%%%%%%%%%%%%%%%%%%%%%%%

\usepackage{biblatex}
\addbibresource{sample.bib}


%%%%%%%%%%% New Commands %%%%%%%%%%%%%%
\newcommand*{\T}{\mathcal T}
\newcommand*{\cl}{\text cl}


\newcommand{\ket}[1]{|#1 \rangle}
\newcommand{\bra}[1]{\langle #1|}
\newcommand{\inner}[2]{\langle #1 | #2 \rangle}
\newcommand{\R}{\mathbb{R}}
\newcommand{\C}{\mathbb{C}}
\newcommand{\V}{\mathbb{V}}
\newcommand{\Hilbert}{\mathcal{H}}
\newcommand{\oper}{\hat{\Omega}}
\newcommand{\lam}{\hat{\Lambda}}

\newcommand{\bigslant}[2]{{\raisebox{.2em}{$#1$}\left/\raisebox{-.2em}{$#2$}\right.}}
\newcommand{\restr}[2]{{% we make the whole thing an ordinary symbol
  \left.\kern-\nulldelimiterspace % automatically resize the bar with \right
  #1 % the function
  \vphantom{\big|} % pretend it's a little taller at normal size
  \right|_{#2} % this is the delimiter
  }}
%%%%%%%%%%%%%%%%%%%%%%%%%%%%%%%%%%%%%%%


\newtcolorbox{dottedbox}[1][]{%
    colback=white,    % Background color
    colframe=white,    % Border color (to be overridden by dashrule)
    sharp corners,     % Sharp corners for the box
    boxrule=0pt,       % No actual border, as it will be drawn with dashrule
    boxsep=5pt,        % Padding inside the box
    enhanced,          % Enable advanced features
    overlay={\draw[dashed, thin, black, dash pattern=on \pgflinewidth off \pgflinewidth, line cap=rect] (frame.south west) rectangle (frame.north east);}, % Dotted line
    #1                 % Additional options
}

\tcbset{theostyle/.style={
    enhanced,
    sharp corners,
    attach boxed title to top left={
      xshift=-1mm,
      yshift=-4mm,
      yshifttext=-1mm
    },
    top=1.5ex,
    colback=white,
    colframe=blue!75!black,
    fonttitle=\bfseries,
    boxed title style={
      sharp corners,
    size=small,
    colback=blue!75!black,
    colframe=blue!75!black,
  } 
}}

\newtcbtheorem[number within=section]{Theorem}{Theorem}{%
  theostyle
}{thm}

\newtcbtheorem[number within=section]{Definition}{Definition}{%
  theostyle
}{def}



\title{Math H185 Homework 5}
\author{Keshav Balwant Deoskar}

\begin{document}
\maketitle


%%%%%%%%%%%%%%%%%%%%%%%%%%%%%%%%%%%%%%%%%%%%%%%%%%%%%%%%%%%%%%%%%
\begin{mathdefinitionbox}{Question 1}
\vskip 0.5cm
Are the following subsets of $\C$ simply connected? Answer "yes" or "no".
\begin{enumerate}[label=(\alph*)]
  \item $\R$
  \item $\C \setminus \R$
  \item $\{ z \in \C : \text{Im}(z) \geq 0 \}$
  \item $\C \setminus B_r(z_0)$ where $z_0 \in \C$ and $r \in \R_{\geq 0}$
  \item $\C \setminus \{ z \in \C : \text{Re}(z) \leq 0 \}$
\end{enumerate}
\end{mathdefinitionbox}
%%%%%%%%%%%%%%%%%%%%%%%%%%%%%%%%%%%%%%%%%%%%%%%%%%%%%%%%%%%%%%%%%

\vskip 0.5cm
\underline{\textbf{Proof:}}


\begin{enumerate}[label=(\alph*)]
  \item Yes  
  \item No
  \item Yes 
  \item No 
  \item Yes
\end{enumerate}

\vskip 0.5cm
\hrule 
\vskip 0.5cm


%%%%%%%%%%%%%%%%%%%%%%%%%%%%%%%%%%%%%%%%%%%%%%%%%%%%%%%%%%%%%%%%%
\begin{mathdefinitionbox}{Question 2}
\vskip 0.5cm
Let $U = C \setminus i\R_{\leq 0}$. Let $\log_U$ be the logarithm function on $U$. Express the following complex numbers in the form $x + iy$ where $x, \in \R$.
\begin{enumerate}[label=(\alph*)]
  \item $\log_{U} (1 + \sqrt{3}i)$
  \item $\log_{U} (-e)$
  \item $\log_{U} (1 - i)$
  \item $\log_{U} (1 - \sqrt{3}i)$
\end{enumerate}
\end{mathdefinitionbox}
%%%%%%%%%%%%%%%%%%%%%%%%%%%%%%%%%%%%%%%%%%%%%%%%%%%%%%%%%%%%%%%%%

\vskip 0.5cm
\underline{\textbf{Proof:}}

The logarithm with the branch cut along the negative imaginary axis is defined as \[ \log(z) = \log(r) + i\theta \] where $z = re^{i\theta}$ with $\theta \in \left(-\pi/2, +3\pi/2\right)$.

\vskip 0.5cm
\begin{enumerate}[label=(\alph*)]
  \item Let's begin by expressing $z = 1 + \sqrt{3}i$ in polar form:
  \begin{align*}
    |1 + \sqrt{3}| &= \sqrt{1^2 + \left(\sqrt{3}\right)^2} \\
    &= \sqrt{1 + 3} = \sqrt{4} = 2 \\
    \implies& r = 2
  \end{align*}
  and 
  \begin{align*}
    \theta &= \arctan\left(\frac{\sqrt{3}}{1}\right) \\
    \implies& \theta = \frac{\pi}{3}
  \end{align*}
  Thus,
  \[ \boxed{ \log_U\left(1 + \sqrt{3}i\right) = \log(2) + i\frac{\pi}{3}} \]

  \vskip 0.5cm
  \item Now, $z = -e$ can be expressed in polar form as $z = e \cdot e^{i \pi}$, so 
  \begin{align*}
    \log_U \left(-e\right) &= \log(e) + i\pi
  \end{align*}
  \[ \boxed{ \log_U \left(-e\right) =  1 + i\pi} \]

  \vskip 0.5cm
  \item In polar form, $1 - i = \sqrt{2} e^{i \cdot \left(-\pi/4\right)}$ so 
  \[ \boxed{ \log_U(1 - i) = \log\left(\sqrt{2}\right) - i \frac{\pi}{4} } \]

  \vskip 0.5cm
  \item In polar form, $1 - \sqrt{3}i = 2e^{i \cdot -pi/3}$, so 
  \[ \boxed{\log_U\left(1 - \sqrt{3}i\right) = \log(2) - i\frac{\pi}{3}} \]
\end{enumerate}

\vskip 0.5cm
\hrule 
\vskip 0.5cm


%%%%%%%%%%%%%%%%%%%%%%%%%%%%%%%%%%%%%%%%%%%%%%%%%%%%%%%%%%%%%%%%
\begin{mathdefinitionbox}{Question 3}
\vskip 0.5cm
Give an example of a simply connected open subset $U \subset \C$ and $z_1, z_2 \in U$ such that $z_1 z_2 \in U$ but 
\[ \log_U\left(z_1 z_2\right) \neq \log_U \left(z_1\right) + \log_U \left(z_2\right) \]
\end{mathdefinitionbox}
%%%%%%%%%%%%%%%%%%%%%%%%%%%%%%%%%%%%%%%%%%%%%%%%%%%%%%%%%%%%%%%%%

\vskip 0.5cm
\underline{\textbf{Proof:}}

Consider the open subset $U \equiv \C \setminus \R_{\leq 0}$ and consider $z_1 = 2e^{i \pi/2}$, $z_2 = 2e^{i 3\pi/4}$. Then, 
\[ z_1 z_2 = 4 e^{i \left(\pi/2 + 3\pi/4 \right)} = 4 e^{i \left(5\pi/4 \right)} = 4e^{-i3\pi/4} \]

where we must re-express the argument $\theta$ because the logarithm on $U \C \setminus \R_{\leq 0}$ is defined as 
\[ \log_U(z) = \log(r) + i\theta \]
where $z = r + i\theta$ and $\left| \theta \right| < \pi$

and 
\begin{align*}
  \log_U(z_1z_2) &= \log_U(4e^{i \cdot \left(-3\pi/4\right)}) \\
  &= \log(2) - i\frac{3\pi}{4}
\end{align*}

whereas 
\begin{align*}
  \log_U\left(z_1\right) + \log_U\left(z_1\right) &= \log_U\left(2e^{0\pi/2}\right) + \log_U\left(2e^{i3\pi/4}\right) \\
  &= \left( \log(2) + i\frac{\pi}{2} \right) + \left( \log(2) + i\frac{3\pi}{4}\right) \\
  &= 2 \cdot \log(2) + i \left( \frac{\pi}{2} + \frac{3\pi}{4}\right) \\
  &= \log(4) + i \frac{5\pi}{4}
\end{align*}

So we notice that when $z_1z_2$ crosses the branch cut, as in our case, we get 
\[ \log_U\left(z_1 z_2\right) \neq \log_U \left(z_1\right) + \log_U \left(z_2\right) \]


\vskip 0.5cm
\hrule 
\vskip 0.5cm


%%%%%%%%%%%%%%%%%%%%%%%%%%%%%%%%%%%%%%%%%%%%%%%%%%%%%%%%%%%%%%%%
\begin{mathdefinitionbox}{Question 4}
\vskip 0.5cm
Let $f = \frac{1}{z^2 - z}$, viewed as a function on $U \equiv \C \setminus \{0, 1, -1\}$. For the following curves, evaluate the quantity 
\[  \int_{{\gamma_1}} f(z) dz -  \int_{{\gamma_0}} f(z) dz  \]
\begin{enumerate}[label=(\alph*)]
  \item Draw image later
  \item Draw image later
  \item Draw image later
\end{enumerate}

\end{mathdefinitionbox}
%%%%%%%%%%%%%%%%%%%%%%%%%%%%%%%%%%%%%%%%%%%%%%%%%%%%%%%%%%%%%%%%%

\vskip 0.5cm
\underline{\textbf{Proof:}}

\begin{enumerate}[label=(\alph*)]
  \item We want to evaluate 
  \[  \int_{{\gamma_1}} f(z) dz - \int_{{\gamma_0}} f(z) dz = \int_{{\gamma_1}} f(z) dz + \int_{{-\gamma_0}} f(z) dz  \]
  
  Integrating over $\gamma_1$ and then integrating over $-\gamma_0$ amounts to integrating over a closed loop. So, 
  \begin{align*}
    \int_{{\gamma_1}} f(z) dz - \int_{{\gamma_0}} f(z) dz &= \int_{\gamma} f(z) dz \\
    &= \int_{\gamma} \frac{1}{z(z-1)} dz \\
    &= \int_{\gamma} \frac{g(z)}{z - 0} dz
  \end{align*}
  where $g(z) = 1/(z-1)$ and $\gamma$ is a closed loop centered at the origin. Then, applying the Cauchy Integral formula, 
  \begin{align*}
    \int_{\gamma} \frac{g(z)}{z - 0} dz &= 2\pi i \cdot g(0) \\
    &= 2\pi i \cdot (-1) \\
    &= -2\pi i
  \end{align*}
  So, 
  \[ \boxed{ \int_{{\gamma_1}} f(z) dz - \int_{{\gamma_0}} f(z) dz = -2\pi i  } \]

  \vskip 0.5cm
  \item Carrying out the same procedure, we have 
  \begin{align*}
    \int_{{\gamma_1}} f(z) dz - \int_{{\gamma_0}} f(z) dz = \int_{\gamma} \frac{g(z)}{z-0} + \frac{h(z)}{z-1} dz 
  \end{align*}
  where $g(z) = \frac{1}{z-1}$, $h(z) = \frac{1}{z-0}$, and $\gamma$ is the closed curve enclosing $0$ and $1$ formed by traversing $\gamma_1$ and then $-\gamma_0$.
  
  \vskip 0.25cm
  Applying Cauchy's Integral Formula, 
  \begin{align*}
    \int_{{\gamma_1}} f(z) dz - \int_{{\gamma_0}} f(z) dz &= 2\pi i g(0) + 2\pi i h(1) \\
    &= 2\pi i \left[ -1 + 1 \right] \\
    &= 0
  \end{align*}

  \vskip 0.5cm
  \item This time, the contour $\gamma_1 + \left( - \gamma_0\right)$ is a closed contour enclosing $-1$, So
  \begin{align*}
    \int_{{\gamma_1}} f(z) dz - \int_{{\gamma_0}} f(z) dz &= \int_{\gamma} \frac{1}{z^2 - z} dz \\
    &= \int_{\gamma} \frac{1}{z^2 - z} \cdot \frac{z^2 + z}{z^2 + z} dz \\
    &= \int_{\gamma} \frac{z^2 + z}{z^4 - z^2} dz \\
    &= \int_{\gamma} \frac{z^2 + z}{z^2 \left(z^2 - 1\right)} dz \\
    &= \int_{\gamma} \frac{j(z)}{z^2 - 1} dz
  \end{align*}
  where $j(z) = \frac{z^2 + z}{z} = 1 + \frac{1}{z}$.

  \vskip 0.5cm
  Then, applying Cauchy's Integral Formula, 
  \begin{align*}
    \int_{{\gamma_1}} f(z) dz - \int_{{\gamma_0}} f(z) dz &= 2\pi i j(-1) \\
    &= 2\pi i \cdot \left(1-1\right) \\
    &= 0
  \end{align*}
\end{enumerate}

\vskip 0.5cm
\hrule 
\vskip 0.5cm



%%%%%%%%%%%%%%%%%%%%%%%%%%%%%%%%%%%%%%%%%%%%%%%%%%%%%%%%%%%%%%%%
\begin{mathdefinitionbox}{Question 5}
\vskip 0.5cm
Assume the result that if $\gamma_0, \gamma_1$ are homotopic curves in a subset $U \subseteq \C$ and $f$ is a holomorphic function on $U$, then 
\[  \int_{{\gamma_0}} f(z)dz =  \int_{{\gamma_1}} f(z)dz \]

Using this, show that if $f$ is any holomorphic function on a simply connected open subset $U \subseteq \C$, then $f$ has a primitive on $U$.
\end{mathdefinitionbox}
%%%%%%%%%%%%%%%%%%%%%%%%%%%%%%%%%%%%%%%%%%%%%%%%%%%%%%%%%%%%%%%%%

\vskip 0.5cm
\underline{\textbf{Proof:}}

Choose any $z_0 \in U$, and define the function for any $z \in U$ as $F(z) = \int_{\gamma} f(w) dw$ where $\gamma$ is a path between $z$ and $z_0$. This map is well defined because on a simply connected open subset, any two curves $\gamma_0, \gamma_1$ between $z_0, z$ will be homotopic and so by our assumption we have 
\[ \int_{{\gamma_0}} f(w) dw = \int_{{\gamma_1}} f(w)dw  \]

\vskip 0.5cm
We claim that $F(z)$ is the primitive of $f(z)$ on $U$. Let's now prove this.

\vskip 0.5cm
For small enough $h \in \C$, we will have $\overline{B_h(z_0)} \in U$ i.e. $z, z+h \in \C$. Then,
\begin{align*}
  F(z+h) - F(z) &= \int_{z_0}^{z} f(w) dw - \int_{z_0}^{z+h} f(w) dw \\
  &= \int_{z}^{z+h} f(w) dw \\
  \implies \frac{F(z+h) - F(z)}{h} &= \frac{1}{h}\int_{z}^{z+h} f(w) dw  \\
  \implies \lim_{h \rightarrow \infty} \frac{F(z+h) - F(z)}{h} &= \lim_{h \rightarrow 0} \frac{1}{h} \int_{z}^{z+h} f(w) dw
\end{align*}

We can calculate the limit by integrating along the straight line connecting $z$ and $z+h$ which we can parametrize as 
\[ \gamma(t) = z + th \] for $t \in [0, 1]$
\begin{align*}
  \lim_{h \rightarrow 0} \frac{1}{h} \int_{z}^{z+h} f(w) dw &=  \lim_{h \rightarrow 0} \frac{1}{h} \int_{0}^{1} f(z + th) \cdot \left(h\right) dt\\
  &= \lim_{h \rightarrow 0} \int_{0}^{1} f(z + th) dt \\
  &= f(z)
\end{align*}

Therefore, 
\[  \boxed{ \lim_{h \rightarrow \infty} \frac{F(z+h) - F(z)}{h}  = f(z) }  \]

\vskip 0.5cm
So, $F(z)$ is indeed the primitive of $f(z)$!

\vskip 0.5cm
\hrule 
\vskip 0.5cm




%%%%%%%%%%%%%%%%%%%%%%%%%%%%%%%%%%%%%%%%%%%%%%%%%%%%%%%%%%%%%%%%
\begin{mathdefinitionbox}{Question 6}
\vskip 0.5cm
Let $U \subseteq \C$ be an open subset and $f : U \rightarrow \C$ be a function such that $\int_T f(z) dz = 0$ for every (parametrized) triangle $T \subseteq U$. Prove that $f$ is holomorphic on all of $U$.
\end{mathdefinitionbox}
%%%%%%%%%%%%%%%%%%%%%%%%%%%%%%%%%%%%%%%%%%%%%%%%%%%%%%%%%%%%%%%%%

\vskip 0.5cm
\underline{\textbf{Proof:}}


Consder a point $a \in U$. Since $U$ is an open set, there is some $r > 0$ such that $B_r(a) \subseteq U$. The restriction of $f$ to this open ball, $\restr{f}{B_r(a)}$, is continuous and  satisfies the property that 
\[ \int_{\gamma} f(z) dz = 0 \]
for all triangular contours contained in $B_r(a)$.

\vskip 0.5cm
Let's define $F : B_r(a) \rightarrow \C$ as 
\[ F(z) = \int_{[a,z]} f(z) dz \]

where $[a, z]$ is the line segment from $a$ to $z$ in $\C$. This function is well defined because $B_r(a)$ is simply-connected.

\vskip 0.5cm
Now,
\begin{align*}
  F'(z) &=\lim_{h\to0}\frac{F(z+h)-F(z)}h\\
  &=\lim_{h\to0}\frac{\int_{[a,z+h]}f(w)\,\mathrm{d}w-\int_{[a,z]}f(w)\,\mathrm{d}w}h\\
  &=\lim_{h\to0}\frac{\overbrace{\int_{[a,z+h]}f(w)\,\mathrm{d}w+\int_{[z+h,z]}f(w)\,\mathrm{d}w+\int_{[z,a]}f(w)\,\mathrm{d}w}^{=\int_\gamma f(w)\,\mathrm{d}w=0}+\int_{[z,z+h]}f(w)\,\mathrm{d}w}h\\
  &=\lim_{h\to0}\frac{\int_{[z,z+h]}f(w)\,\mathrm{d}w}h\\
  &=\lim_{h\to0}\frac1h\int_0^1f(z+th)\cdot h\,\mathrm{d}t\\
  &=\lim_{h\to0}\int_0^1f(z+th)\,\mathrm{d}t\\
  &=\int_0^1\lim_{h\to0}f(z+th)\,\mathrm{d}t\\
  &=\int_0^1f(z)\,\mathrm{d}t\qquad(f\text{ continuous})\\
  &=f(z)
\end{align*}

\vskip 0.5cm
So, $F(z)$ is holomorphic on $B_r(a)$ with derivative $f(z)$. But we know that holomorphic functions are infinitely differentiable -- meaning $f(z)$ is also holomorphic on $B_r(a)$.

\vskip 0.5cm
Since $a \in U$ was chosen arbtrarily, the above argument holds for all points in $U$ so we have arrived at the desired result.


\vskip 0.5cm
\hrule 
\vskip 0.5cm





%%%%%%%%%%%%%%%%%%%%%%%%%%%%%%%%%%%%%%%%%%%%%%%%%%%%%%%%%%%%%%%%%
% \begin{mathdefinitionbox}{Question }
% \vskip 0.5cm

% \end{mathdefinitionbox}
% %%%%%%%%%%%%%%%%%%%%%%%%%%%%%%%%%%%%%%%%%%%%%%%%%%%%%%%%%%%%%%%%%

% \vskip 0.5cm
% \underline{\textbf{Proof:}}

% \vskip 0.5cm
% \hrule 
% \vskip 0.5cm




%%%%%%%%%%%%%%%%%%%%%%%%%%%%%%%%%%%%%%%%%%%%%%%%%%%%%%%%%%%%%%%%%
% \begin{mathdefinitionbox}{Question }
% \vskip 0.5cm

% \end{mathdefinitionbox}
% %%%%%%%%%%%%%%%%%%%%%%%%%%%%%%%%%%%%%%%%%%%%%%%%%%%%%%%%%%%%%%%%%

% \vskip 0.5cm
% \underline{\textbf{Proof:}}

% \vskip 0.5cm
% \hrule 
% \vskip 0.5cm



\end{document}
