\documentclass{article}

% Language setting
% Replace `english' with e.g. `spanish' to change the document language
\usepackage[english]{babel}

% Set page size and margins
% Replace `letterpaper' with`a4paper' for UK/EU standard size
\usepackage[letterpaper,top=2cm,bottom=2cm,left=3cm,right=3cm,marginparwidth=1.75cm]{geometry}

% Useful packages
\usepackage{amsmath}
\usepackage{amssymb}
\usepackage{graphicx}
\usepackage{enumitem}
\usepackage[colorlinks=true, allcolors=blue]{hyperref}
\usepackage{mathtools}

\usepackage{hyperref}
\hypersetup{
    colorlinks=true,
    linkcolor=blue,
    filecolor=magenta,      
    urlcolor=cyan,
    pdftitle={Math H185 Lecture (not sure)},
    pdfpagemode=FullScreen,
    }

\urlstyle{same}

\usepackage{tikz-cd}

%%%%%%%%%%% Box pacakges and definitions %%%%%%%%%%%%%%
\usepackage[most]{tcolorbox}
\usepackage{xcolor}

% Define the colors
\definecolor{boxheader}{RGB}{0, 51, 102}  % Dark blue
\definecolor{boxfill}{RGB}{173, 216, 230}  % Light blue

% Define the tcolorbox environment
\newtcolorbox{mathdefinitionbox}[2][]{%
    colback=boxfill,   % Background color
    colframe=boxheader, % Border color
    fonttitle=\bfseries, % Bold title
    coltitle=white,     % Title text color
    title={#2},         % Title text
    enhanced,           % Enable advanced features
    attach boxed title to top left={yshift=-\tcboxedtitleheight/2}, % Center title
    boxrule=0.5mm,      % Border width
    sharp corners,      % Sharp corners for the box
    #1                  % Additional options
}
%%%%%%%%%%%%%%%%%%%%%%%%%

\newtcolorbox{dottedbox}[1][]{%
    colback=white,    % Background color
    colframe=white,    % Border color (to be overridden by dashrule)
    sharp corners,     % Sharp corners for the box
    boxrule=0pt,       % No actual border, as it will be drawn with dashrule
    boxsep=5pt,        % Padding inside the box
    enhanced,          % Enable advanced features
    overlay={\draw[dashed, thin, black, dash pattern=on \pgflinewidth off \pgflinewidth, line cap=rect] (frame.south west) rectangle (frame.north east);}, % Dotted line
    #1                 % Additional options
}

\usepackage{biblatex}
\addbibresource{sample.bib}


%%%%%%%%%%% New Commands %%%%%%%%%%%%%%
\newcommand*{\T}{\mathcal T}
\newcommand*{\cl}{\text cl}


\newcommand{\ket}[1]{|#1 \rangle}
\newcommand{\bra}[1]{\langle #1|}
\newcommand{\inner}[2]{\langle #1 | #2 \rangle}
\newcommand{\R}{\mathbb{R}}
\newcommand{\C}{\mathbb{C}}
\newcommand{\V}{\mathbb{V}}
\newcommand{\Hilbert}{\mathcal{H}}
\newcommand{\oper}{\hat{\Omega}}
\newcommand{\lam}{\hat{\Lambda}}
\newcommand{\defn}{\underline{\textbf{Def: }}}
\newcommand{\defeq}{\vcentcolon=}


\newcommand{\bigslant}[2]{{\raisebox{.2em}{$#1$}\left/\raisebox{-.2em}{$#2$}\right.}}
\newcommand{\restr}[2]{{% we make the whole thing an ordinary symbol
  \left.\kern-\nulldelimiterspace % automatically resize the bar with \right
  #1 % the function
  \vphantom{\big|} % pretend it's a little taller at normal size
  \right|_{#2} % this is the delimiter
  }}
%%%%%%%%%%%%%%%%%%%%%%%%%%%%%%%%%%%%%%%


\tcbset{theostyle/.style={
    enhanced,
    sharp corners,
    attach boxed title to top left={
      xshift=-1mm,
      yshift=-4mm,
      yshifttext=-1mm
    },
    top=1.5ex,
    colback=white,
    colframe=blue!75!black,
    fonttitle=\bfseries,
    boxed title style={
      sharp corners,
    size=small,
    colback=blue!75!black,
    colframe=blue!75!black,
  } 
}}

\newtcbtheorem[number within=section]{Theorem}{Theorem}{%
  theostyle
}{thm}

\newtcbtheorem[number within=section]{Definition}{Definition}{%
  theostyle
}{def}



\title{Math H185 Lecture (not sure)}
\author{Keshav Balwant Deoskar}

\begin{document}
\maketitle

% \vskip 0.5cm
These are notes taken from lectures on Complex Analysis delivered by Professor Tony Feng for UC Berekley's Math H185 class in the Sprng 2024 semester.
% \pagebreak 

\tableofcontents

\pagebreak

%%%%%%%%%%%%%%%%%%%%%%%%%%%%%%%%%%%%%%%%%%%%%%%%%%%%%%%%%%%%%%%%%%
\section{April 8 - The Agument Principle}
%%%%%%%%%%%%%%%%%%%%%%%%%%%%%%%%%%%%%%%%%%%%%%%%%%%%%%%%%%%%%%%%%%

To start, let's recall a lemma from the homework.

\begin{dottedbox}
  \emph{\textbf{Lemma:}} Let $f$ be meromorphic at $z_0$ with zero of order $z \in \mathbb{Z}$. Then, $f'/f$ has a simple pole at $z_0$ if $n \neq 0$ or a removable pole at $z_0$ if $n = 0$ with residue $n$.
\end{dottedbox}

\emph{\textbf{Proof: (outline)}}
By the structure theorem, 
\[ f(z)  (z-z_0)^n h(z) \]  where $h(z)$ is holomorphic and non-vanishing at $z_0$. Then, 
\begin{align*}
  \frac{f'}{f} &= \frac{n(z-z_0)^{n-1}h(z) + h'(z)(z-z_0)^n}{(z-z_0)^n h(z)} \\
  &= \frac{n}{z-z_0} + \underbrace{\frac{h'(z)}{h(z)}}_{\text{holomorphic}}
\end{align*}

\vskip 0.5cm
The Argument Principle arises from combining this observation with the Residue theorem.

\vskip 0.5cm
\begin{dottedbox}
  \emph{\textbf{Theorem: (The Arguement Principle)}} Let $f$ be meromorphic on a neighborhood $\overline{U}$. Then,
  \[ \frac{1}{2\pi i} \int_{\partial U} \frac{f'(z)}{f(z)} dz = \begin{matrix}
    &\text{\# of zeroes of $f$ in $U$} \\
    &-\text{\# of poles of $f$ in $U$} \\
  \end{matrix} \] 
  where the zeroes and poles are counted \emph{with} multiplicity.
\end{dottedbox}

\emph{\textbf{Proof:}} Using the Residue Theorem, 
\begin{align*}
  \frac{1}{2\pi i} \int_{\partial U} \frac{f'(z)}{f(z)}dz &= \sum_{{z_j}} \mathrm{Res}_{{z_j}} \left(\frac{f'(z)}{f(z)}\right)
\end{align*}

where the $z_j$'s are zeros/poles of $f$. By the lemma above, $\mathrm{Res}_{{z_j}} \left(\frac{f'(z)}{f(z)}\right)$ is just the order of each zero (or "minus" the order if it is a pole). So, we have arrived at the desired result.

\vskip 1cm
\subsection{Some Consequences the Argument Principle}

\begin{dottedbox}
  \emph{\textbf{Rouche's Theorem:}} Suppose $f, g$ are holomorphic on a neighborhood of $\overline{U} \subseteq \C$. Assume that $|f(z)| > |g(z)|$ for all $z \in \partial U$. Then, the number of zeroes of $f$ in $U$ equals the number of zeroes of $f + g$ in $U$.
\end{dottedbox}

\vskip 0.5cm
\emph{\textbf{Proof:}}

Let $f_{\lambda} = f + \lambda g$ for $\lambda \in [0, 1]$.
\begin{itemize}
  \item $f_{\lambda}$ is holomorphic around $\overline{U}$.
  \item $f_0 = f_1$, $f_1 = f + g$.
\end{itemize}

By the argument prnciple, 
\begin{align*}
  \text{\# of zeroes of } f_{\lambda} &= \frac{1}{2\pi i} \int_{\partial U} \frac{f'_{\lambda}(z)}{f_{\lambda}(z)} dz 
\end{align*}

The RHS is continuous in $\lambda$ (why?) and is never zero on the boundary because $|f| > |g|$ on $\partial U$. The RHS is a continuous function, whereas the LHS is just an integer. So, if the RHS to be a continuous function $[0, 1] \rightarrow \mathbb{Z}$ then it must be a constant function.

\vskip 1cm
\begin{dottedbox}
  \emph{\textbf{Cor:}} A polynomial of the form 
  \[ P(z) = z^n + a_{n-1} z^{n-1} + \cdots + a_0, \;\;\; n > 0 \] has $n$ roots (with multiplicity). 
\end{dottedbox}

\emph{\textbf{Proof:}}
Let $f(z) = z^n$ and $g(z) = a_{n-1} z^{n-1} + \cdots + a_0$. Then, $P(z) = f(z) + g(z)$. Let's try to apply Rouche's Theorem. In order to do so let's choose $R > 0$ such that for all $r \geq R$, 
\begin{align*}
  & r^n > |a_{n-1}| r^{n-1} + \cdots + a_0 \\
  \implies & |f| > |g| \text{ on } \partial B_r(0) \forall r \geq R
\end{align*}

Then, Rouche's Theorem tells us that $n = \text{\# of zeros(f) in } B_r(0) = \text{ \# of zeros(P) in } B_r(0)$.

\vskip 0.5cm
Another consequence of Rouche's Theorem is the Open Mapping Theorem.

\begin{mathdefinitionbox}{Open maps}
\vskip 0.25cm
  \emph{\textbf{Definition:}} A map $f : U \rightarrow V$ between topological spaces $U, V$ is said to be \emph{\textbf{open}} if $f(\text{Open set in }U)$ is open in $V$.
\end{mathdefinitionbox}

\vskip 0.5cm
\underline{Ex:} $V = \C$ and $f = $ constant is an open map. 

\underline{Ex:} [Insert image from lecture.] Idea is that any map that sends an open interval to a single point is \emph{not} open.

\vskip 0.5cm
\begin{dottedbox}
  \emph{\textbf{Theorem:}} Let $f : U \rightarrow \C$ be nonconstant and holomorphic. If $U$ is connected, then $f$ is open.
\end{dottedbox}

\emph{\textbf{Proof:}} Let $z_0 \in U$. It suffices to show that $\forall$ sufficiently small $\epsilon > 0$, $f\left(B_{\epsilon}(z_0)\right)$ is open. Moreover it suffices to show that $f\left(B_{\epsilon}(z_0)\right)$ contains $B_r(w_0)$ for some $r > 0$ where $w_0 = f(z_0)$.

\vskip 0.5cm
[Insert image from lecture]

\vskip 0.5cm
i.e. we want: $\forall w \in B_r(w_0), \;\;\exists z \in B_{\epsilon}(z_0)$ such that $f(z) = w$, i.e. $f(z) - w = 0$. 
\[ \text{"Aha! A zero counting problem" - Tony.} \]

Now, since $f(z_0) = w_0 \neq w$ we can pick $\epsilon$ small enough so that $f(z) \neq w_0$ on $\partial B_{\epsilon} (z_0)$.
\begin{align*}
  \implies& |f(z) - w_0| > \delta > 0\;\;\forall z \in \partial B_{\epsilon}(z_0) \\
  \implies& |f(z) - w| > \frac{\delta}{2} > 0\;\;\forall w \in \partial B_{\delta/2}(w_0)
\end{align*}

Take $r = \delta/2$ Then, 
\begin{align*}
  \implies& \left|f_w(z_0)\right| > 0\;\;\forall z \in \partial B_{\epsilon}(z_0) \\
  \implies& \frac{f_w'(z)}{f_w(z)} \text{ is continuous in } z \in \partial B_{\epsilon}(z_0) \text{ i.e. } B_r(w_0) \\
  \text{\# zeroes of } f_w &= \frac{1}{2\pi i} \int_{\partial B_{\epsilon}(z_0)} \frac{f'_w(z)}{f_w(z)} dz \text{ is continuous in } w
\end{align*}
Then, by the same reasoning as Rouche's Theorem, both sides are constant $>$ 0.

\end{document}
