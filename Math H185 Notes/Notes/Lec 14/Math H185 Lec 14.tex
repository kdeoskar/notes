\documentclass{article}

% Language setting
% Replace `english' with e.g. `spanish' to change the document language
\usepackage[english]{babel}

% Set page size and margins
% Replace `letterpaper' with`a4paper' for UK/EU standard size
\usepackage[letterpaper,top=2cm,bottom=2cm,left=3cm,right=3cm,marginparwidth=1.75cm]{geometry}

% Useful packages
\usepackage{amsmath}
\usepackage{amssymb}
\usepackage{graphicx}
\usepackage{enumitem}
\usepackage[colorlinks=true, allcolors=blue]{hyperref}
\usepackage{mathtools}

\usepackage{hyperref}
\hypersetup{
    colorlinks=true,
    linkcolor=blue,
    filecolor=magenta,      
    urlcolor=cyan,
    pdftitle={Math H185 Lecture 14},
    pdfpagemode=FullScreen,
    }

\urlstyle{same}

\usepackage{tikz-cd}

%%%%%%%%%%% Box pacakges and definitions %%%%%%%%%%%%%%
\usepackage[most]{tcolorbox}
\usepackage{xcolor}

% Define the colors
\definecolor{boxheader}{RGB}{0, 51, 102}  % Dark blue
\definecolor{boxfill}{RGB}{173, 216, 230}  % Light blue

% Define the tcolorbox environment
\newtcolorbox{mathdefinitionbox}[2][]{%
    colback=boxfill,   % Background color
    colframe=boxheader, % Border color
    fonttitle=\bfseries, % Bold title
    coltitle=white,     % Title text color
    title={#2},         % Title text
    enhanced,           % Enable advanced features
    attach boxed title to top left={yshift=-\tcboxedtitleheight/2}, % Center title
    boxrule=0.5mm,      % Border width
    sharp corners,      % Sharp corners for the box
    #1                  % Additional options
}
%%%%%%%%%%%%%%%%%%%%%%%%%

\newtcolorbox{dottedbox}[1][]{%
    colback=white,    % Background color
    colframe=white,    % Border color (to be overridden by dashrule)
    sharp corners,     % Sharp corners for the box
    boxrule=0pt,       % No actual border, as it will be drawn with dashrule
    boxsep=5pt,        % Padding inside the box
    enhanced,          % Enable advanced features
    overlay={\draw[dashed, thin, black, dash pattern=on \pgflinewidth off \pgflinewidth, line cap=rect] (frame.south west) rectangle (frame.north east);}, % Dotted line
    #1                 % Additional options
}

\usepackage{biblatex}
\addbibresource{sample.bib}


%%%%%%%%%%% New Commands %%%%%%%%%%%%%%
\newcommand*{\T}{\mathcal T}
\newcommand*{\cl}{\text cl}


\newcommand{\ket}[1]{|#1 \rangle}
\newcommand{\bra}[1]{\langle #1|}
\newcommand{\inner}[2]{\langle #1 | #2 \rangle}
\newcommand{\R}{\mathbb{R}}
\newcommand{\C}{\mathbb{C}}
\newcommand{\V}{\mathbb{V}}
\newcommand{\Hilbert}{\mathcal{H}}
\newcommand{\oper}{\hat{\Omega}}
\newcommand{\lam}{\hat{\Lambda}}
\newcommand{\defn}{\underline{\textbf{Def: }}}
\newcommand{\defeq}{\vcentcolon=}


\newcommand{\bigslant}[2]{{\raisebox{.2em}{$#1$}\left/\raisebox{-.2em}{$#2$}\right.}}
\newcommand{\restr}[2]{{% we make the whole thing an ordinary symbol
  \left.\kern-\nulldelimiterspace % automatically resize the bar with \right
  #1 % the function
  \vphantom{\big|} % pretend it's a little taller at normal size
  \right|_{#2} % this is the delimiter
  }}
%%%%%%%%%%%%%%%%%%%%%%%%%%%%%%%%%%%%%%%


\tcbset{theostyle/.style={
    enhanced,
    sharp corners,
    attach boxed title to top left={
      xshift=-1mm,
      yshift=-4mm,
      yshifttext=-1mm
    },
    top=1.5ex,
    colback=white,
    colframe=blue!75!black,
    fonttitle=\bfseries,
    boxed title style={
      sharp corners,
    size=small,
    colback=blue!75!black,
    colframe=blue!75!black,
  } 
}}

\newtcbtheorem[number within=section]{Theorem}{Theorem}{%
  theostyle
}{thm}

\newtcbtheorem[number within=section]{Definition}{Definition}{%
  theostyle
}{def}



\title{Math H185 Lecture 14}
\author{Keshav Balwant Deoskar}

\begin{document}
\maketitle

% \vskip 0.5cm
These are notes taken from lectures on Complex Analysis delivered by Professor Tony Feng for UC Berekley's Math H185 class in the Sprng 2024 semester.
% \pagebreak 

\tableofcontents

\pagebreak

%%%%%%%%%%%%%%%%%%%%%%%%%%%%%%%%%%%%%%%%%%%%%%%%%%%%%%%%%%%%%%%%%%
\section{February 16 - The Logarithm}
%%%%%%%%%%%%%%%%%%%%%%%%%%%%%%%%%%%%%%%%%%%%%%%%%%%%%%%%%%%%%%%%%%

\vskip 0.5cm

Today we'll be discussing the logarithm. Before getting to it though, lte's revisit the exponential function, which we defined pretty early in the course.

\subsection{Exponential Function}


\begin{mathdefinitionbox}{The exponential function}
  \[ \exp(z) =  e^z = 1 + z + \frac{z^2}{2!} + \cdots + \sum_{n = 0}^{\infty} \frac{1}{n!} z^n \]
\end{mathdefinitionbox}

\vskip 0.5cm
The series expansion has infinite radius of convergence, making the exponential a holomorphic function, and further allowing us to obtain the derivative by just differentiating the series term by term.

\vskip 0.5cm
\begin{align*}
  \implies \exp'(z) &=\sum_{n=1}^{\infty} \frac{1}{n!} nz^{n-1}\\
  &= \sum_{n=1}^{\infty} \frac{1}{(n-1)!} z^{n-1} \\
  &= \exp(z)
\end{align*}

\vskip 0.5cm
\begin{dottedbox}
  \emph{\textbf{Note:}} There exists a power series $f$ with radius of convergence $\infty$ such that 
  \[ f'(z) = f(z),\;\;\;\;f(0) = 1 \] 

  This property actually haracterizes the exponential.
\end{dottedbox}

\subsection{Range of the Exponential Function}

[Complete by watching recording]

So, it turns out that every complex number other than the origin is a target of the exponential map.

\vskip 0.5cm
In showing this, we used the property $e^{a+b} = e^a e^b$. Strictly speaking, we must prove this holds for complex numbers. Let's do so quickly.

\vskip 0.5cm
\begin{dottedbox}
  \textbf{\emph{Lemma:}} For $z_1, z_2 \in \C$ we have 
  \[ e^{z_1 + z_2} = e^{z_1} e^{z_2} \] 
\end{dottedbox}

\vskip 0.5cm
\textbf{\emph{Proof:}}
If we fix $z_1$ and view the exponential as a function of $z_2$, we have 
\[ \frac{\partial}{\partial z_2}\left(e^{z_1 + z_2}\right) = e^{z_1 + z_2} \]

and similarly if we fix $z_2$ and view it as a function of $z_1$, we have
\[ \frac{\partial}{\partial z_1}\left(e^{z_1 + z_2}\right) = e^{z_1 + z_2} \]

[Finish this proof later]

\vskip 0.5cm
\begin{dottedbox}
  \emph{Corollary:} $e^z \neq 0$ since for any $e^z$ there exists an inverse $e^{-z}$ such that $e^{z} e^{-z} = 1$.
\end{dottedbox}

\vskip 1cm
\subsection{The Logarithm Function}
We now want to define the logarithm, $\log(z)$, such that 
\begin{itemize}
  \item $e^{\log(z)} = z $
  \item Logarithm is continuous.
\end{itemize}

% But this does not uniquely pin down the function since we could add any multiple of $2\pi$ to the exponent i.e. $e^{\log(z) + 2\pi n}$ without changing anything. This is an issue because then, 

But we have a problem! If we start at $\theta = 0$, rotate around, and eventualy hit $\theta = 2\pi$ which is the same point as $\theta = 0$, we will have a discontinuity in the neighborhood of $\theta = 0$ since the function will be small if we move coounterclockwise but large if we move clockwise.

\vskip 0.5cm
So, let's tackle the issue using the following theorem:

\vskip 0.5cm
\begin{dottedbox}
  \textbf{\emph{Theorem:}} Let $U \subseteq_{open} \C$ be simply connected such that $0 \neq U, 1 \neq U$. Then, there exists a unique function
  \[ \log_{U} : U \rightarrow \C \]
  such that 
  \begin{itemize}
    \item it is holomorphic
    \item $\log_U(1) = 0$
    \item $e^{\log_U(z)} = z$ for all $z \in U$.
  \end{itemize}
\end{dottedbox}

\vskip 0.5cm
\textbf{\emph{Proof:}} Define $\log_U(z)$ to be the primitive of $\frac{1}{z}$ i.e. 
\[ \log_U(z) = \int_{1}^{z} \frac{1}{w} dw \]
This is well define since $U$ is simply connected and certaintly satisfies the condition $\log_U(1) = 0$.

To check the last condition, let's verify that 
\[ \frac{d}{dz} \log_U(z) = \frac{1}{z} \]

[complete after lecture]

\vskip 0.5cm

\vskip 0.5cm
To define the logarithm, we can choose a curve containing the origin and consider its \emph{complement} since the complement is guaranteed to be simply connected. The curve we choose is called a \textbf{\emph{branch cut}}.

\vskip 0.5cm
\subsubsection*{Examples of different branch cuts}

\begin{itemize}
  \item \textbf{\emph{Principal Branch cut:}} $U = \C \setminus \R_{\geq 0}$. Insert picture
  
  \vskip 0.25cm
  This is kind of like the "canonical" branch cut since, even for real numbers, we usually don't bother defining the logarithm for negative numbers.
\end{itemize}

\vskip 0.5cm
\subsubsection*{Midterm Monday March 4}
\begin{itemize}
  \item 10 true/false questions
  \item 3-4 calculations
  \item No proof on the first midterm.
  \item A sample midterm will be released.
\end{itemize}

\end{document}
