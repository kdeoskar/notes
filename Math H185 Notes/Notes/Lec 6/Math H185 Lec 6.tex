\documentclass{article}

% Language setting
% Replace `english' with e.g. `spanish' to change the document language
\usepackage[english]{babel}

% Set page size and margins
% Replace `letterpaper' with`a4paper' for UK/EU standard size
\usepackage[letterpaper,top=2cm,bottom=2cm,left=3cm,right=3cm,marginparwidth=1.75cm]{geometry}

% Useful packages
\usepackage{amsmath}
\usepackage{amssymb}
\usepackage{graphicx}
\usepackage{enumitem}
\usepackage[colorlinks=true, allcolors=blue]{hyperref}
\usepackage{mathtools}

\usepackage{hyperref}
\hypersetup{
    colorlinks=true,
    linkcolor=blue,
    filecolor=magenta,      
    urlcolor=cyan,
    pdftitle={Overleaf Example},
    pdfpagemode=FullScreen,
    }

\urlstyle{same}

\usepackage{tikz-cd}

%%%%%%%%%%% Box pacakges and definitions %%%%%%%%%%%%%%
\usepackage[most]{tcolorbox}
\usepackage{xcolor}

% Define the colors
\definecolor{boxheader}{RGB}{0, 51, 102}  % Dark blue
\definecolor{boxfill}{RGB}{173, 216, 230}  % Light blue

% Define the tcolorbox environment
\newtcolorbox{mathdefinitionbox}[2][]{%
    colback=boxfill,   % Background color
    colframe=boxheader, % Border color
    fonttitle=\bfseries, % Bold title
    coltitle=white,     % Title text color
    title={#2},         % Title text
    enhanced,           % Enable advanced features
    attach boxed title to top left={yshift=-\tcboxedtitleheight/2}, % Center title
    boxrule=0.5mm,      % Border width
    sharp corners,      % Sharp corners for the box
    #1                  % Additional options
}
%%%%%%%%%%%%%%%%%%%%%%%%%

\newtcolorbox{dottedbox}[1][]{%
    colback=white,    % Background color
    colframe=white,    % Border color (to be overridden by dashrule)
    sharp corners,     % Sharp corners for the box
    boxrule=0pt,       % No actual border, as it will be drawn with dashrule
    boxsep=5pt,        % Padding inside the box
    enhanced,          % Enable advanced features
    overlay={\draw[dashed, thin, black, dash pattern=on \pgflinewidth off \pgflinewidth, line cap=rect] (frame.south west) rectangle (frame.north east);}, % Dotted line
    #1                 % Additional options
}

\usepackage{biblatex}
\addbibresource{sample.bib}


%%%%%%%%%%% New Commands %%%%%%%%%%%%%%
\newcommand*{\T}{\mathcal T}
\newcommand*{\cl}{\text cl}


\newcommand{\ket}[1]{|#1 \rangle}
\newcommand{\bra}[1]{\langle #1|}
\newcommand{\inner}[2]{\langle #1 | #2 \rangle}
\newcommand{\R}{\mathbb{R}}
\newcommand{\C}{\mathbb{C}}
\newcommand{\V}{\mathbb{V}}
\newcommand{\Hilbert}{\mathcal{H}}
\newcommand{\oper}{\hat{\Omega}}
\newcommand{\lam}{\hat{\Lambda}}
\newcommand{\defn}{\underline{\textbf{Def: }}}
\newcommand{\defeq}{\vcentcolon=}


\newcommand{\bigslant}[2]{{\raisebox{.2em}{$#1$}\left/\raisebox{-.2em}{$#2$}\right.}}
\newcommand{\restr}[2]{{% we make the whole thing an ordinary symbol
  \left.\kern-\nulldelimiterspace % automatically resize the bar with \right
  #1 % the function
  \vphantom{\big|} % pretend it's a little taller at normal size
  \right|_{#2} % this is the delimiter
  }}
%%%%%%%%%%%%%%%%%%%%%%%%%%%%%%%%%%%%%%%


\tcbset{theostyle/.style={
    enhanced,
    sharp corners,
    attach boxed title to top left={
      xshift=-1mm,
      yshift=-4mm,
      yshifttext=-1mm
    },
    top=1.5ex,
    colback=white,
    colframe=blue!75!black,
    fonttitle=\bfseries,
    boxed title style={
      sharp corners,
    size=small,
    colback=blue!75!black,
    colframe=blue!75!black,
  } 
}}

\newtcbtheorem[number within=section]{Theorem}{Theorem}{%
  theostyle
}{thm}

\newtcbtheorem[number within=section]{Definition}{Definition}{%
  theostyle
}{def}



\title{Math H185 Lecture64}
\author{Keshav Balwant Deoskar}

\begin{document}
\maketitle

% \vskip 0.5cm
These are notes taken from lectures on Complex Analysis delivered by Professor Tony Feng for UC Berekley's Math H185 class in the Sprng 2024 semester.
% \pagebreak 

\tableofcontents

\pagebreak

%%%%%%%%%%%%%%%%%%%%%%%%%%%%%%%%%%%%%%%%%%%%%%%%%%%%%%%%%%%%%%%%%%
\section{January 28 - Integrating over curves}
%%%%%%%%%%%%%%%%%%%%%%%%%%%%%%%%%%%%%%%%%%%%%%%%%%%%%%%%%%%%%%%%%%

\vskip 0.5cm
In calculus, we learned how to integrate over intervals. Today, we'll learn how to integrate complex valued functions over curves in the complex plane.

\vskip 0.5cm
\begin{mathdefinitionbox}{Integral over a curve}
  \vskip 0.5cm
  We define the integral of a function $f(z)$ over a curve $\gamma$
  \[ \int_{\gamma} f(z) dz \defeq \int_{\gamma} \text{Re}(f(z))dz + i \int_{\gamma} \text{Im}(f(z))dz \]
\end{mathdefinitionbox}


\vskip 0.5cm
\subsection{What is a curve?}
\begin{itemize}
  \item A \textbf{\emph{parameterized curve}} is a continuous function $\gamma : \underbrace{[a, b]}_{\subset \R} \rightarrow \C$

  \item We say $\gamma$ is \textbf{\emph{piece-wise smooth}} if there exist finite subdivisions of $[a, b]$ on which $\gamma$ is smooth (in the Math 104 sense i.e. the real and imaginary parts are separately infinitely differentiable).
  
  [include graphs of piece-wise smooth paths]

  \item \underline{Example:} $\gamma(t) = z_0 + re^{i\theta}$ where $z_0 \in \C$, $r \in \R_{\geq 0}$, $t \in [0, 2\pi]$. This path traces out a circle of radius $r$ centered around the point $z_0$.
  
  [include graph]

  \item \underline{Example:} Given a path $\gamma : [a, b] \rightarrow \C$. Let $\gamma^{-} : [a, b] \rightarrow \C$ be $\gamma^{-}(t) = \gamma(a + b - t)$. Then $\gamma^{-}$ is the same curve but traversed in reverse orientation.
\end{itemize}

\vskip 0.5cm
\subsection{So, how do we actually integrate?}
Let $\gamma : [a, b] \rightarrow \C$ be a "nice" curve, where "nice" means piece-wise smooth parameterized. Then,

\vskip 0.5cm
\begin{dottedbox}
  \underline{\textbf{Theorem:}} The integral of $f(z)$ over $\gamma$ is
  \[ \int_{\gamma} f(z) dz = \int_{a}^{b} f(\gamma(t)) \gamma'(t) dt \]
  and these integrals have some nice properties:
  \begin{enumerate}
    \item \underline{Linearity:} Say $\lambda, \mu \in \C$ Then
    \[ \int_{\gamma} \lambda f(z) + \mu g(z) dz = \lambda \in_{\gamma} f(z)dz + \mu \int_{\gamma} g(z)dz \]

    \item \underline{orientation:} 
    \[ \int_{\gamma} f(z) dz = - \int_{\gamma^{-}} f(z) dz \]
  \end{enumerate} 
\end{dottedbox}

\vskip 0.5cm
\underline{\textbf{Examples:}}
\begin{enumerate}[label=(\alph*)]
  \item $\gamma : [a,b] \rightarrow \C$ given by $\gamma(t) = t$:
  \[ \int_{\gamma} f(z)dz = \int_{a}^{b} f(t)dt \]

  \item $\gamma : [a,b] \rightarrow \C$ given by $\gamma(t) = it$:
  \[ \int_{\gamma} f(z)dz = \int f(it) i dt = i \int f(it) dt \]
\end{enumerate}

\vskip 0.5cm
Recall that a \textbf{\emph{primitive}} (i.e. antiderivative) of $f$ is $F$ such that $F'(z) = f(z)$. Then, the fundamental theorem of calculus is

\begin{dottedbox}
  \textbf{\emph{Fundamental Theorem of Calculus:}} For a "nice" curve $\gamma : [a,b] \rightarrow \C$ and function $f$ with primitive $F$,
  \[ \int_{\gamma} f(z)dz = F(\gamma(b)) - F(\gamma(a)) \]
\end{dottedbox}

\vskip 0.5cm
\textbf{\emph{Corollary:}} If $\gamma$ is a \textbf{\emph{closed}} curve i.e. $\gamma(a) = \gamma(b)$ and $f$ has a primitive on (an open neighborhood of) $\gamma$, then 

\[ \int_{\gamma} f(z)dz = 0 \]

\vskip 1cm
\subsection{Most fundamental example: $f(z) = z^n, n \in \mathbb{Z}$}
  
Consider  the function $f(z) = z^n, n \in \mathbb{Z}$ and the curve $\gamma : [0, 2\pi] \rightarrow \C$ where $\gamma(t) = re^{it}$. What is $\int_{\gamma} f(z) dz$?

\begin{dottedbox}
  The integral is 
  \begin{align*}
    \int_{\gamma} f(z) dz &= \int_{0}^{2\pi} re^{int} (ri e^{it})dt \\
    &= r^{n+1}i \int_{0}^{2\pi} e^{i(n+1)t} dt
  \end{align*}

  \vskip 0.5cm
  \underline{If $n \neq -1$:} $e^{i(n+1)t}dt$ has primitive 
  \[ \frac{1}{i(n+1)}e^{i(n+1)t} \]

  So, the integral is 
  \[ \int_{0}^{2\pi} e^{i(n+1)t} dt = 0 \]

  \vskip 0.5cm
  \underline{If $n = -1$:} Then, we have 
  \begin{align*}
    \int_{\gamma}f(z) dz &= i\int_{0}^{2\pi} e^{i(n+1)t} dt \\
    &= \int_{0}^{2\pi} \\
    &= 2\pi i 
  \end{align*}
\end{dottedbox}

\vskip 0.5cm
Precisely why does the primitive not work for $n = -1$? The issue lies with the fact that the primitive of $\frac{1}{z}$ is the \textbf{\emph{logarithm}}.

The complex logarithm isn't defined along a full circle around the origin. We'll revisit this when studying branch cuts soon.

\vskip 0.5cm
\begin{mathdefinitionbox}{Conclusion}
  \vskip 0.5cm
  We find that 
  \[ \int_{\partial B_r(0)} z^n dz = \begin{cases}
    0 \;\;\; n \neq -1 \\
    2\pi i \;\;\; n = 1
  \end{cases} \]
\end{mathdefinitionbox}

\vskip 0.5cm
\textbf{\emph{Note:}} A very interesting observation is that this in \textbf{\emph{independent of $r$}}. This is surprising, and foreshadows some incredible results we'll see soon.

\end{document}
