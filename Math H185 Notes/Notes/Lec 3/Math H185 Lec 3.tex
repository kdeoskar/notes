\documentclass{article}

% Language setting
% Replace `english' with e.g. `spanish' to change the document language
\usepackage[english]{babel}

% Set page size and margins
% Replace `letterpaper' with`a4paper' for UK/EU standard size
\usepackage[letterpaper,top=2cm,bottom=2cm,left=3cm,right=3cm,marginparwidth=1.75cm]{geometry}

% Useful packages
\usepackage{amsmath}
\usepackage{amssymb}
\usepackage{graphicx}
\usepackage[colorlinks=true, allcolors=blue]{hyperref}

\usepackage{hyperref}
\hypersetup{
    colorlinks=true,
    linkcolor=blue,
    filecolor=magenta,      
    urlcolor=cyan,
    pdftitle={Overleaf Example},
    pdfpagemode=FullScreen,
    }

\urlstyle{same}

\usepackage{tikz-cd}

%%%%%%%%%%% Box pacakges and definitions %%%%%%%%%%%%%%
\usepackage[most]{tcolorbox}
\usepackage{xcolor}

% Define the colors
\definecolor{boxheader}{RGB}{0, 51, 102}  % Dark blue
\definecolor{boxfill}{RGB}{173, 216, 230}  % Light blue

% Define the tcolorbox environment
\newtcolorbox{mathdefinitionbox}[2][]{%
    colback=boxfill,   % Background color
    colframe=boxheader, % Border color
    fonttitle=\bfseries, % Bold title
    coltitle=white,     % Title text color
    title={#2},         % Title text
    enhanced,           % Enable advanced features
    attach boxed title to top left={yshift=-\tcboxedtitleheight/2}, % Center title
    boxrule=0.5mm,      % Border width
    sharp corners,      % Sharp corners for the box
    #1                  % Additional options
}
%%%%%%%%%%%%%%%%%%%%%%%%%

\newtcolorbox{dottedbox}[1][]{%
    colback=white,    % Background color
    colframe=white,    % Border color (to be overridden by dashrule)
    sharp corners,     % Sharp corners for the box
    boxrule=0pt,       % No actual border, as it will be drawn with dashrule
    boxsep=5pt,        % Padding inside the box
    enhanced,          % Enable advanced features
    overlay={\draw[dashed, thin, black, dash pattern=on \pgflinewidth off \pgflinewidth, line cap=rect] (frame.south west) rectangle (frame.north east);}, % Dotted line
    #1                 % Additional options
}

\usepackage{biblatex}
\addbibresource{sample.bib}


%%%%%%%%%%% New Commands %%%%%%%%%%%%%%
\newcommand*{\T}{\mathcal T}
\newcommand*{\cl}{\text cl}


\newcommand{\ket}[1]{|#1 \rangle}
\newcommand{\bra}[1]{\langle #1|}
\newcommand{\inner}[2]{\langle #1 | #2 \rangle}
\newcommand{\R}{\mathbb{R}}
\newcommand{\C}{\mathbb{C}}
\newcommand{\V}{\mathbb{V}}
\newcommand{\Hilbert}{\mathcal{H}}
\newcommand{\oper}{\hat{\Omega}}
\newcommand{\lam}{\hat{\Lambda}}
\newcommand{\defn}{\underline{\textbf{Def:}}}


\newcommand{\bigslant}[2]{{\raisebox{.2em}{$#1$}\left/\raisebox{-.2em}{$#2$}\right.}}
\newcommand{\restr}[2]{{% we make the whole thing an ordinary symbol
  \left.\kern-\nulldelimiterspace % automatically resize the bar with \right
  #1 % the function
  \vphantom{\big|} % pretend it's a little taller at normal size
  \right|_{#2} % this is the delimiter
  }}
%%%%%%%%%%%%%%%%%%%%%%%%%%%%%%%%%%%%%%%


\tcbset{theostyle/.style={
    enhanced,
    sharp corners,
    attach boxed title to top left={
      xshift=-1mm,
      yshift=-4mm,
      yshifttext=-1mm
    },
    top=1.5ex,
    colback=white,
    colframe=blue!75!black,
    fonttitle=\bfseries,
    boxed title style={
      sharp corners,
    size=small,
    colback=blue!75!black,
    colframe=blue!75!black,
  } 
}}

\newtcbtheorem[number within=section]{Theorem}{Theorem}{%
  theostyle
}{thm}

\newtcbtheorem[number within=section]{Definition}{Definition}{%
  theostyle
}{def}



\title{Math H185 Lecture 3}
\author{Keshav Balwant Deoskar}

\begin{document}
\maketitle

% \vskip 0.5cm
These are notes taken from lectures on Complex Analysis delivered by Professor Tony Feng for UC Berekley's Math H185 class in the Sprng 2024 semester.
% \pagebreak 

\tableofcontents

\pagebreak

%%%%%%%%%%%%%%%%%%%%%%%%%%%%%%%%%%%%%%%%%%%%%%%%%%%%%%%%%%%%%%%%%%
\section{January 22 - Holomorphic functions}
%%%%%%%%%%%%%%%%%%%%%%%%%%%%%%%%%%%%%%%%%%%%%%%%%%%%%%%%%%%%%%%%%%

\vskip 1cm
\subsection{Sequences and Series}

\vskip 0.5cm
Recall that a sequence of complex numbers $\{z_n \in \C\}$ is said to \emph{converge} to $z \in \C$ if for all $\epsilon > 0$ there exists a natural number $N \geq 1$ such that 
\[ |z_n - z| < \epsilon \] for all $n \geq N$.

Equivalently, 
\[ \lim_{n \rightarrow N} |z_n - z| = 0 \]
\begin{dottedbox}
  In HW1, we show that if $z_n = x_n + iy_n$ and $z = x + iy$ where $x, y, x_n, y_n \in \R$ then
  \[ \lim_{n \rightarrow \infty} |z_n - z| = 0 \iff \begin{cases}
    \lim_{n \rightarrow \infty} |x_n - x| = 0 \\
    \lim_{n \rightarrow \infty} |y_n - y| = 0
  \end{cases} \]
\end{dottedbox}

\vskip 1cm
\subsection{Complex Dfferentiability}
Let $f : U \subseteq_{\text{open}} \C \rightarrow \C$.

\begin{mathdefinitionbox}{Holomorphic Functions}
\vskip 0.5cm
 We say $f$ is Holomorphic at $z_0 \in U$ if \[ \lim_{h \rightarrow 0} \frac{f(z_0 + h) - f(z_0)}{h} = \lim_{z \rightarrow z_0} \frac{f(z) - f(z_0)}{z - z_0} \] exists in $\C$. If so, its value is denoted as $f'(z_0)$. 

\vskip 0.5cm
Note: Keep in mind that $h$ is a complex number.
\end{mathdefinitionbox}

\vskip 0.5cm
\begin{itemize}
  \item   This means that for any sequence $h_n \rightarrow 0$ \underline{or} $\forall \epsilon > 0$ $\exists \delta > 0$ such that 
  \[ |h| < \delta \implies \lvert \frac{f(z_0 + h)-f(z_0)}{h} - f'(z_0) \rvert < \epsilon\]

  \item This is the most important definition of the course.
\end{itemize}

\underline{\textbf{Remark:}} Although $\C$ is the same as $\R^2$ as a metric space, Holomorphicity is much stronger than differentiability of a function $f : \R^2 \rightarrow \R^2$ because the limits along every path to a point are required to be equal. In contrast to this, differentiability in $\R^2$ requires that \emph{a} limit exists along each path, but not that all limits are equal.

\vskip 0.5cm
\underbar{\textbf{Example:}} Consider the function $f(z) = \overline{z}$.

\vskip 0.5cm
We observe that 
\begin{align*}
  \frac{f(z+h)-f(z)}{h} &= \frac{\overline{z+h}-\overline{z}}{h} \\
  &= \frac{\overline{h}}{h}
\end{align*}
Now, if we take the limit as $h \rightarrow 0$ along the real line, we get $\frac{\overline{h}}{h} = \frac{h}{h} = 1$, however if we take the limit along 
the imaginary line we get $\frac{\overline{h}}{h} = \frac{-h}{h} = -1$

\vskip 0.5cm
On the other hand if we consider the counterpart of this function in $\R^2$ as 
$f(x, y) = (x, -y)$, this function is \emph{smooth everywhere}. In contrast to this, the complex function $f(z) = \overline{z}$ is \emph{not holomorphic at \underline{any} $z_0 \in \C$}.

\vskip 0.5cm
We will see that holomorphic functions have strong \underline{rigidity} properties not shared by real differentiable functions. For instance,
\begin{itemize}
  \item If $f, g$ are holomorphic on a connected open set $U \subseteq \C$ and $f = g$ on a line segment in $U$, then in fact they agree at \emph{all} points in $U$: $f(z) = g(z) \forall z \in U$. This is the \textbf{Principle of Analytic Continuation.}
  \item Another example of surprising rigidity is that if $f$ is holomorphic on $U$ i.e. it is one differentiable on $U$, then in fact it is \emph{infinitely} differentiable on $U$.
\end{itemize}

\vskip 0.5cm
\underline{\textbf{Examples:}}
\begin{enumerate}
  \item $f(z) = z^n$ 
  
  \vskip 0.5cm
  \underline{Calculate:} 
  \begin{align*}
    \frac{f(z+h)-f(z)}{h} &= \frac{(z+h)^n - z^n}{h} \\
    &=_{\text{binom thm.}} \left[  \frac{1}{h} \left( z^n + nz^{n-1}h + \cdots + nzh^{n-1} + h^n \right) - z^n  \right] \\
    &= nz^{n-1} + h(\cdots)
  \end{align*}
  So, just like in Real Analysis, we have 
  \[ \boxed{\lim_{h \rightarrow 0} \frac{f(z + h) - f(z)}{h} = nz^{n-1}} \]
\end{enumerate}

\vskip 1cm
\subsection{Stability Properties}
While holomorphicity is different from real dfferentiability, there are a number of properties which are justified by the same $\epsilon-\delta$ proofs as those from $\R^2$ analysis.

\begin{itemize}
  \item If $f,g : U \rightarrow \C$ are holomorphic at $z_0 \in U$ then 
  \begin{itemize}
    \item $(f+g)$ is holomorphic at $z_0$ and 
    \[ (f+g)'(z_0) = f'(z_0) + g'(z_0) \]
  
    \item $fg$ is holomorphic at $z_0$ and
    \[ (fg)'(z_0) = f'g(z_0)g(z_0) + f(z_0)g'(z_0) \]
    
    \vskip 0.5cm
    \item \underline{Chain Rule:} $(f \circ g)$ is holomorphic at $z_0$ and 
    \[ (f \circ g)'(z_0) = f'(g(z_0)) \cdot g'(z_0) \]
    
    \vskip 0.5cm
    \item \underline{Division:} $(f/g)$ is holomorphic at $z_0$ if $g(z_0) \neq 0$ and
    \[ \left( \frac{f}{g} \right) = \frac{f'(z_0)g(z_0) + f(z_0)g(z_0)}{g(z_0)^2} \]
  \end{itemize}
\end{itemize}

\vskip 0.5cm
\begin{dottedbox}
  \underline{Polynomials:} Finite sum of monomials.
  \[ f(z) = a_nz^n + \cdots + a_0 \]
  By linearity (Stability property 1), \emph{all} Polynomials are holomorphic on $C$.
  
  \vskip 1cm
  \underline{Rational Functions:} Ratios of Polynomials.
  \[ h(z) = \frac{f(z)}{g(z)} \]
  By Stability property 3, all rational functions are holomorphic on $\{ z \in \C : g(z) \neq 0 \} \subseteq_{\text{open}} \C$.
\end{dottedbox}

\vskip 0.5cm
\underline{\textbf{Warm-down examples:}} Where are the following functions holomorphic, and what are their derivatives in those regions?

\begin{enumerate}
  \item $f(z) = \frac{1}{z}$
  \item $f(z) = z^2 + 3z + \frac{1}{2}$
  \item $f(z) = \text{Re}(z)$
  \item $f(z) = i \cdot \text{Im}(z)$
  \item $f(z) = \text{Re}(z) + i \cdot \text{Im}(z)$
\end{enumerate}

\vskip 0.5cm
\underline{\textbf{Answers:}}
\begin{enumerate}
  \item Holomorphic on $\C \setminus \{0\}$, and derivative in the region is 
  \[ \frac{-1}{z^2} \]
  
  \item Holomorphic on $\C$, and derivative in the region is 
  \[ 2z + 3 \]

  \item Not holomorphic \emph{anywhere}, snice limit vertically is always zero but limit horizontally will be non-zero. 

  \item Not holomorphic \emph{anywhere}, snice limit horizontally is always zero but limit vertically will be non-zero. 

  \item Holomorphic on $\C$, and derivative in the region is $1$ ($f(z) = z$, so $f'(z) = 1$ at all points).
\end{enumerate}

\end{document}
