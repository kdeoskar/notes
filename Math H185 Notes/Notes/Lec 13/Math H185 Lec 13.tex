\documentclass{article}

% Language setting
% Replace `english' with e.g. `spanish' to change the document language
\usepackage[english]{babel}

% Set page size and margins
% Replace `letterpaper' with`a4paper' for UK/EU standard size
\usepackage[letterpaper,top=2cm,bottom=2cm,left=3cm,right=3cm,marginparwidth=1.75cm]{geometry}

% Useful packages
\usepackage{amsmath}
\usepackage{amssymb}
\usepackage{graphicx}
\usepackage{enumitem}
\usepackage[colorlinks=true, allcolors=blue]{hyperref}
\usepackage{mathtools}

\usepackage{hyperref}
\hypersetup{
    colorlinks=true,
    linkcolor=blue,
    filecolor=magenta,      
    urlcolor=cyan,
    pdftitle={Math H185 Lecture 12},
    pdfpagemode=FullScreen,
    }

\urlstyle{same}

\usepackage{tikz-cd}

%%%%%%%%%%% Box pacakges and definitions %%%%%%%%%%%%%%
\usepackage[most]{tcolorbox}
\usepackage{xcolor}

% Define the colors
\definecolor{boxheader}{RGB}{0, 51, 102}  % Dark blue
\definecolor{boxfill}{RGB}{173, 216, 230}  % Light blue

% Define the tcolorbox environment
\newtcolorbox{mathdefinitionbox}[2][]{%
    colback=boxfill,   % Background color
    colframe=boxheader, % Border color
    fonttitle=\bfseries, % Bold title
    coltitle=white,     % Title text color
    title={#2},         % Title text
    enhanced,           % Enable advanced features
    attach boxed title to top left={yshift=-\tcboxedtitleheight/2}, % Center title
    boxrule=0.5mm,      % Border width
    sharp corners,      % Sharp corners for the box
    #1                  % Additional options
}
%%%%%%%%%%%%%%%%%%%%%%%%%

\newtcolorbox{dottedbox}[1][]{%
    colback=white,    % Background color
    colframe=white,    % Border color (to be overridden by dashrule)
    sharp corners,     % Sharp corners for the box
    boxrule=0pt,       % No actual border, as it will be drawn with dashrule
    boxsep=5pt,        % Padding inside the box
    enhanced,          % Enable advanced features
    overlay={\draw[dashed, thin, black, dash pattern=on \pgflinewidth off \pgflinewidth, line cap=rect] (frame.south west) rectangle (frame.north east);}, % Dotted line
    #1                 % Additional options
}

\usepackage{biblatex}
\addbibresource{sample.bib}


%%%%%%%%%%% New Commands %%%%%%%%%%%%%%
\newcommand*{\T}{\mathcal T}
\newcommand*{\cl}{\text cl}


\newcommand{\ket}[1]{|#1 \rangle}
\newcommand{\bra}[1]{\langle #1|}
\newcommand{\inner}[2]{\langle #1 | #2 \rangle}
\newcommand{\R}{\mathbb{R}}
\newcommand{\C}{\mathbb{C}}
\newcommand{\V}{\mathbb{V}}
\newcommand{\Hilbert}{\mathcal{H}}
\newcommand{\oper}{\hat{\Omega}}
\newcommand{\lam}{\hat{\Lambda}}
\newcommand{\defn}{\underline{\textbf{Def: }}}
\newcommand{\defeq}{\vcentcolon=}


\newcommand{\bigslant}[2]{{\raisebox{.2em}{$#1$}\left/\raisebox{-.2em}{$#2$}\right.}}
\newcommand{\restr}[2]{{% we make the whole thing an ordinary symbol
  \left.\kern-\nulldelimiterspace % automatically resize the bar with \right
  #1 % the function
  \vphantom{\big|} % pretend it's a little taller at normal size
  \right|_{#2} % this is the delimiter
  }}
%%%%%%%%%%%%%%%%%%%%%%%%%%%%%%%%%%%%%%%


\tcbset{theostyle/.style={
    enhanced,
    sharp corners,
    attach boxed title to top left={
      xshift=-1mm,
      yshift=-4mm,
      yshifttext=-1mm
    },
    top=1.5ex,
    colback=white,
    colframe=blue!75!black,
    fonttitle=\bfseries,
    boxed title style={
      sharp corners,
    size=small,
    colback=blue!75!black,
    colframe=blue!75!black,
  } 
}}

\newtcbtheorem[number within=section]{Theorem}{Theorem}{%
  theostyle
}{thm}

\newtcbtheorem[number within=section]{Definition}{Definition}{%
  theostyle
}{def}



\title{Math H185 Lecture 12}
\author{Keshav Balwant Deoskar}

\begin{document}
\maketitle

% \vskip 0.5cm
These are notes taken from lectures on Complex Analysis delivered by Professor Tony Feng for UC Berekley's Math H185 class in the Sprng 2024 semester.
% \pagebreak 

\tableofcontents

\pagebreak

%%%%%%%%%%%%%%%%%%%%%%%%%%%%%%%%%%%%%%%%%%%%%%%%%%%%%%%%%%%%%%%%%%
\section{February 12 - Proof of Cauchy's Theorem (Sketch)}
%%%%%%%%%%%%%%%%%%%%%%%%%%%%%%%%%%%%%%%%%%%%%%%%%%%%%%%%%%%%%%%%%%

\vskip 0.5cm
Recall that a primitive (i.e. an antiderivative) of $f : U \subseteq_{open} \C \rightarrow \C$ is a function $F: U \rightarrow \C$ such that $F'(z) = f(z)$ for all $z \in U$.

\vskip 0.5cm
Now, by the Fundamental Theorem of Calculus, 
\begin{dottedbox}
If 
\begin{itemize}
  \item $f$ has a primitive on an open neighborhood of $\gamma$ and
  \item $\gamma$ is closed 
\end{itemize}
Then 
\[ \int_{\gamma} f(z)dz = 0 \]
\end{dottedbox}

\vskip 0.5cm
This is to be contrasted with the \emph{Cauchy-Goursat Theorem} which states that 
\begin{dottedbox} 
If $f$ is holomorphic on a neighborhood of $U$, then 
\[ \int_{\partial U} f(z) dz = 0 \]
\end{dottedbox}


\vskip 0.5cm
Though the two statements above are very similar, they're not quite the same. The first one requires a primitive locally whereas the second requires it over an entire region.


\vskip 0.5cm
We will \underline{sketch} a proof of Cauchy's Theorem.


\vskip 1cm
\subsection{Sketch}

\begin{itemize}
\vskip 0.5cm
\item \underline{Step 1:} Approximate the path $\gamma$ by polygons.

\vskip 0.25cm
[Draw Image]

\vskip 0.5cm
\item \underline{Step 2:} Subdivide into triangles.

\vskip 0.25cm
[Draw Image]

\vskip 0.25cm
Then the integral over the entire curve is the same as the sum of the integrals over triangles.

We then want to show that 
\[ \int_{\Delta} f = 0\]

where $\Delta$ is a triangle. We can do so by taking the triangle $\Delta$ and carrying out the barycentric subdivision i.e. take the midpoints of all sides and draw lines between them. Then, we get 

\[ \int_{\Delta}  = \int_{\Delta_{1}} + \int_{\Delta_{2}} + \int_{\Delta_{3}} + \int_{\Delta_{4}}   \]

By the Triangle inequality $\left| A + B \right| \leq \left| A \right| + \left|B \right|$, we have 

\begin{align*}
  \implies \left| \int_{\Delta} f(z)dz \right| \leq 4 \sup_{i} \left| \int_{\Delta_{i}} f(z)dz \right|
\end{align*}

Then, we take the biggest $\Delta_i$ and do the same procedure again. Repeatedly subdividing and using the Triangle Inequality, we have 

\begin{align*}
  \implies \left| \int_{\text{original }\Delta} f(z)dz \right| \leq 4^n \sup_{\Delta^{(n)}} \left| \int_{\Delta^{(n)}} f(z)dz \right|
\end{align*}
where $\Delta^{(n)}$ is the triangle obtained after subdividing $n-$times. So, $\Delta^{(1)} \supseteq \Delta^{(2)} \cdots \Delta^{(n)}$.

Now, there exists a limit point $z_0$ for this sequence of shrinking triangles. The limit point lies in the intersection of all the triangles
\[ z_0 \in \bigcap_{n} \Delta^{(n)} \]

\vskip 0.25cm
\underline{Key:} Near a point $z_0$, the function $f$ is well approximated by a linear function $f(z) = f(z_0) + f'(z_0)(z-z_0) + \epsilon(z)$ where $\lim_{z \rightarrow z_0} \frac{\epsilon(z)}{z - z_0} = 0$. We want to use this fact to control how big the integral over $\Delta^{(n)}$ can get.

\begin{align*}
  \int_{\Delta^{(n)}} f &= \underbrace{\int_{\Delta^{(n)}} f(z_0)}_{=0} + \underbrace{\int_{\Delta^{(n)}}  f'(z_0)(z - z_0)}_{=0} + \int_{\Delta^{(n)}}  \epsilon(z) 
\end{align*}
where the first two integrals are zero because constant and linear functions have primitives on $\C$ and thus their integrals over any closed curve is zero by the Fundamental Theorem of Calculus. So, what we really need to control is the \emph{error term}.

\vskip 0.5cm
We have 
\begin{align*}
  4^n \left| \int_{\Delta^{(n)}} \epsilon(z) dz \right| \leq 4^n \left| \int_{\Delta^{(n)}} \left|\epsilon(z)\right| \left|dz\right|  \right| <<< 4^n \int_{\Delta^{(n)}}  \left| z - z_0 \right| \left| dz \right|
\end{align*}

where $\left|dz\right| = \left|z'(t)\right| \left|dt\right|$ and  $A_n <<< B_n$ denotes 
\[ \lim_{n \rightarrow \infty} \frac{A_n}{B_n} = 0 \]

Let's call the greatest possible distance between two points in a triangle as the \emph{diameter} of the triangle, diam$(\Delta^{(n)})$. So, 
\[ \left| z - z_0 \right| \leq \text{diam}(\Delta^{(n)}) = 2^{-n}\text{diam}(\Delta^{(\text{original})}) \]

[Get some more details from lecture recording]

\vskip 0.5cm
\subsection*{Conclusion}

\[ \left| \int_{\Delta} f(z)dz \right| <<< C \]
where $C$ is some fixed constant, and the LHS and RHS are constant sequences (but sequences none-the-less), meaning LHS/RHS $\rightarrow 0$ as $n \rightarrow \infty$ and so LHS $ = 0$.

\end{itemize}

\subsection*{Summary}
\begin{itemize}
  \item Subdivide $\Delta$ into small triangles where $f$ is well approximated by a linear function (because it's holomorphic).
\end{itemize}


\end{document}
