\documentclass{article}

% Language setting
% Replace `english' with e.g. `spanish' to change the document language
\usepackage[english]{babel}

% Set page size and margins
% Replace `letterpaper' with`a4paper' for UK/EU standard size
\usepackage[letterpaper,top=2cm,bottom=2cm,left=3cm,right=3cm,marginparwidth=1.75cm]{geometry}

% Useful packages
\usepackage{amsmath}
\usepackage{amssymb}
\usepackage{graphicx}
\usepackage{enumitem}
\usepackage[colorlinks=true, allcolors=blue]{hyperref}
\usepackage{mathtools}

\usepackage{hyperref}
\hypersetup{
    colorlinks=true,
    linkcolor=blue,
    filecolor=magenta,      
    urlcolor=cyan,
    pdftitle={Math H185 Lecture 34},
    pdfpagemode=FullScreen,
    }

\urlstyle{same}

\usepackage{tikz-cd}

%%%%%%%%%%% Box pacakges and definitions %%%%%%%%%%%%%%
\usepackage[most]{tcolorbox}
\usepackage{xcolor}

% Define the colors
\definecolor{boxheader}{RGB}{0, 51, 102}  % Dark blue
\definecolor{boxfill}{RGB}{173, 216, 230}  % Light blue

% Define the tcolorbox environment
\newtcolorbox{mathdefinitionbox}[2][]{%
    colback=boxfill,   % Background color
    colframe=boxheader, % Border color
    fonttitle=\bfseries, % Bold title
    coltitle=white,     % Title text color
    title={#2},         % Title text
    enhanced,           % Enable advanced features
    attach boxed title to top left={yshift=-\tcboxedtitleheight/2}, % Center title
    boxrule=0.5mm,      % Border width
    sharp corners,      % Sharp corners for the box
    #1                  % Additional options
}
%%%%%%%%%%%%%%%%%%%%%%%%%

\newtcolorbox{dottedbox}[1][]{%
    colback=white,    % Background color
    colframe=white,    % Border color (to be overridden by dashrule)
    sharp corners,     % Sharp corners for the box
    boxrule=0pt,       % No actual border, as it will be drawn with dashrule
    boxsep=5pt,        % Padding inside the box
    enhanced,          % Enable advanced features
    overlay={\draw[dashed, thin, black, dash pattern=on \pgflinewidth off \pgflinewidth, line cap=rect] (frame.south west) rectangle (frame.north east);}, % Dotted line
    #1                 % Additional options
}

\usepackage{biblatex}
\addbibresource{sample.bib}


%%%%%%%%%%% New Commands %%%%%%%%%%%%%%
\newcommand*{\T}{\mathcal T}
\newcommand*{\cl}{\text cl}


\newcommand{\ket}[1]{|#1 \rangle}
\newcommand{\bra}[1]{\langle #1|}
\newcommand{\inner}[2]{\langle #1 | #2 \rangle}
\newcommand{\R}{\mathbb{R}}
\newcommand{\C}{\mathbb{C}}
\newcommand{\V}{\mathbb{V}}
\newcommand{\Hilbert}{\mathcal{H}}
\newcommand{\oper}{\hat{\Omega}}
\newcommand{\lam}{\hat{\Lambda}}
\newcommand{\defn}{\underline{\textbf{Def: }}}
\newcommand{\defeq}{\vcentcolon=}


\newcommand{\bigslant}[2]{{\raisebox{.2em}{$#1$}\left/\raisebox{-.2em}{$#2$}\right.}}
\newcommand{\restr}[2]{{% we make the whole thing an ordinary symbol
  \left.\kern-\nulldelimiterspace % automatically resize the bar with \right
  #1 % the function
  \vphantom{\big|} % pretend it's a little taller at normal size
  \right|_{#2} % this is the delimiter
  }}
%%%%%%%%%%%%%%%%%%%%%%%%%%%%%%%%%%%%%%%


\tcbset{theostyle/.style={
    enhanced,
    sharp corners,
    attach boxed title to top left={
      xshift=-1mm,
      yshift=-4mm,
      yshifttext=-1mm
    },
    top=1.5ex,
    colback=white,
    colframe=blue!75!black,
    fonttitle=\bfseries,
    boxed title style={
      sharp corners,
    size=small,
    colback=blue!75!black,
    colframe=blue!75!black,
  } 
}}

\newtcbtheorem[number within=section]{Theorem}{Theorem}{%
  theostyle
}{thm}

\newtcbtheorem[number within=section]{Definition}{Definition}{%
  theostyle
}{def}



\title{Math H185 Lecture 34}
\author{Keshav Balwant Deoskar}

\begin{document}
\maketitle

\tableofcontents
\pagebreak

\section{Normal Famlies}

\vskip 0.5cm
\begin{mathdefinitionbox}{Normal Families}
\vskip 0.25cm
  A family of functions $\mathcal{F}$ on $U \subseteq \C$ is said to be \emph{\textbf{normal}} if for all sequences $f_1, f_2, f_3, \cdots \in \mathcal{F}$, there exists a convergent subsequence.
\end{mathdefinitionbox}

\vskip 0.5cm
This definition expresses that a family of functions is compact in a sense.

\vskip 0.5cm

\begin{dottedbox}
  \emph{\textbf{Theorem (Arzela-Ascoli):}} If $\mathcal{F}$ is Uniformly bounded and Equicontinuous on all compact subsets, then it is normal.
\end{dottedbox}

\vskip 0.5cm
\begin{itemize}
  \item Here, \textbf{uniformly bounded} on a (compact) subset $K \subseteq U$ means there exists $B$ such that $|f(z)| < B$ for all $f \in \mathcal{F}$ and $z \in K$. i.e. the same bound $B$ applies to \emph{all} functions in the family.

  \vskip 0.5cm
  \underline{Ex:} Each function of the form $f_n(z) = n$ is bounded, but the family $\{f_n\}$ is not uniformly bounded.
  
  \vskip 0.5cm
  \item A function being \textbf{Equicontinuous} on $K$ means that for all $\epsilon > 0$, there exists $\delta > 0$ such that if $|z_1  -z_2| < \delta$ then $|f(z_1) - f(z_2)| < \epsilon$ for all $z_1, z_2 \in K$.
  
  \vskip 0.5cm
  \underline{Ex:} Suppose $\mathcal{F}$ is a family of functions on $[0, 1]$. If $\{f'(z)\}_{f \in \mathcal{F}}$ is uniformly bounded, then $\mathcal{F}$ is equicontinuous.
  
  \vskip 0.5cm
  \underline{Ex:} $f_n(x) = x^n$ on $[0, 1]$ is \emph{not} equicontinuous. We can see this by lettnig $x_1 = 1, x_2 = 1-\delta$. Then, 
  
  \[ |f_n(x_1) - f_n(x_2)|  \]
  
  [Complete this example later]
\end{itemize}

\vskip 0.5cm
To prove the Arzela-Ascoli Theorem, the key idea we'll use is \emph{diagonalization} (to arrange countably many conditions).

\begin{dottedbox}
  \emph{\textbf{Principle of Diagonalization:}} Given countably many conditions on a sequence $\mathrm{cond}_1, \mathrm{cond}_2, \mathrm{cond}_3, \cdots$ which are inherited on subsequences and sequence $f_1, f_2, f_3, \cdots$ such that for all $j$ any subsetequence has a further subsequence which condition $\mathrm{cond}_j$. Then, there exists a subsequence $f_1^{(\infty)}, f_2^{(\infty)}, \cdots$ satisfying all $\mathrm{cond}_j$.

  \vskip 0.5cm
  \emph{\textbf{Proof-ish:}} Suppose we have
  \[ \begin{matrix}
    f_1^{(1)} & f_2^{(1)} & f_3^{(1)} & \cdots & \text{satisfying } \mathrm{cond_1} \\
    f_1^{(2)} & f_2^{(2)} & f_3^{(2)} & \cdots & \text{satisfying } \mathrm{cond_2} \\
    f_1^{(3)} & f_2^{(3)} & f_3^{(3)} & \cdots & \text{satisfying } \mathrm{cond_3} \\
    \vdots & \vdots & \vdots & \ddots
  \end{matrix} \]
  Then, taking the diagonal gives us a subsequence of $\mathcal{F}$ satisfying all conditions $\mathrm{cond}_j$.
\end{dottedbox}

[Write the rest from recording]



\end{document}
