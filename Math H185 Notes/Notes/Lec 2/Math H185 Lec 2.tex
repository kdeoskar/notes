\documentclass{article}

% Language setting
% Replace `english' with e.g. `spanish' to change the document language
\usepackage[english]{babel}

% Set page size and margins
% Replace `letterpaper' with`a4paper' for UK/EU standard size
\usepackage[letterpaper,top=2cm,bottom=2cm,left=3cm,right=3cm,marginparwidth=1.75cm]{geometry}

% Useful packages
\usepackage{amsmath}
\usepackage{amssymb}
\usepackage{graphicx}
\usepackage[colorlinks=true, allcolors=blue]{hyperref}

\usepackage{hyperref}
\hypersetup{
    colorlinks=true,
    linkcolor=blue,
    filecolor=magenta,      
    urlcolor=cyan,
    pdftitle={Overleaf Example},
    pdfpagemode=FullScreen,
    }

\urlstyle{same}

\usepackage{tikz-cd}

%%%%%%%%%%% Box pacakges and definitions %%%%%%%%%%%%%%
\usepackage[most]{tcolorbox}
\usepackage{xcolor}

% Define the colors
\definecolor{boxheader}{RGB}{0, 51, 102}  % Dark blue
\definecolor{boxfill}{RGB}{173, 216, 230}  % Light blue

% Define the tcolorbox environment
\newtcolorbox{mathdefinitionbox}[2][]{%
    colback=boxfill,   % Background color
    colframe=boxheader, % Border color
    fonttitle=\bfseries, % Bold title
    coltitle=white,     % Title text color
    title={#2},         % Title text
    enhanced,           % Enable advanced features
    attach boxed title to top left={yshift=-\tcboxedtitleheight/2}, % Center title
    boxrule=0.5mm,      % Border width
    sharp corners,      % Sharp corners for the box
    #1                  % Additional options
}
%%%%%%%%%%%%%%%%%%%%%%%%%

\usepackage{biblatex}
\addbibresource{sample.bib}


%%%%%%%%%%% New Commands %%%%%%%%%%%%%%
\newcommand*{\T}{\mathcal T}
\newcommand*{\cl}{\text cl}


\newcommand{\ket}[1]{|#1 \rangle}
\newcommand{\bra}[1]{\langle #1|}
\newcommand{\inner}[2]{\langle #1 | #2 \rangle}
\newcommand{\R}{\mathbb{R}}
\newcommand{\C}{\mathbb{C}}
\newcommand{\V}{\mathbb{V}}
\newcommand{\Hilbert}{\mathcal{H}}
\newcommand{\oper}{\hat{\Omega}}
\newcommand{\lam}{\hat{\Lambda}}
\newcommand{\defn}{\underline{\textbf{Def:}}}


\newcommand{\bigslant}[2]{{\raisebox{.2em}{$#1$}\left/\raisebox{-.2em}{$#2$}\right.}}
\newcommand{\restr}[2]{{% we make the whole thing an ordinary symbol
  \left.\kern-\nulldelimiterspace % automatically resize the bar with \right
  #1 % the function
  \vphantom{\big|} % pretend it's a little taller at normal size
  \right|_{#2} % this is the delimiter
  }}
%%%%%%%%%%%%%%%%%%%%%%%%%%%%%%%%%%%%%%%


\tcbset{theostyle/.style={
    enhanced,
    sharp corners,
    attach boxed title to top left={
      xshift=-1mm,
      yshift=-4mm,
      yshifttext=-1mm
    },
    top=1.5ex,
    colback=white,
    colframe=blue!75!black,
    fonttitle=\bfseries,
    boxed title style={
      sharp corners,
    size=small,
    colback=blue!75!black,
    colframe=blue!75!black,
  } 
}}

\newtcbtheorem[number within=section]{Theorem}{Theorem}{%
  theostyle
}{thm}

\newtcbtheorem[number within=section]{Definition}{Definition}{%
  theostyle
}{def}



\title{Math H185 Lecture 2}
\author{Keshav Balwant Deoskar}

\begin{document}
\maketitle

% \vskip 0.5cm
These are notes taken from lectures on Complex Analysis delivered by Professor Tony Feng for UC Berekley's Math H185 class in the Sprng 2024 semester.
% \pagebreak 

\tableofcontents

\pagebreak

%%%%%%%%%%%%%%%%%%%%%%%%%%%%%%%%%%%%%%%%%%%%%%%%%%%%%%%%%%%%%%%%%%
\section{January 19 - The Complex Plane}
%%%%%%%%%%%%%%%%%%%%%%%%%%%%%%%%%%%%%%%%%%%%%%%%%%%%%%%%%%%%%%%%%%

\vskip 0.5cm

\subsection{What does $\C$ look like geomertrically?}

[Insert Diagram of $\C$ with Real and Imaginary Axes]

\vskip 0.5cm
\defn\; The \textbf{Modulus} or \textbf{absolute value} of $z = a + bi$ is $|z| = a^2 + b^2$. It is the distance to $0$.

\vskip 0.5cm
There is a correspondence between the structure of $\C$ and the vector structure of $\R^2$.
\[ \text{Addition in $\C$} \iff \text{Vector Addition in $\R^2$} \]

\vskip 0.5cm
\underline{\textbf{Exercise:}} Draw the subset of $z \in \C$ defined by 
\begin{enumerate}
  \item $\{ |z| = 1 \}$
  \item $\{ |z - 6| = 1 \}$
  \item $\{ |z - 6i| \leq 1 \}$
  \item $\{ \text{Re}(z) \leq \text{Im}(z) \}$
  \item $\{ |z| = \text{Re}(z) + 1 \}$
\end{enumerate}

\vskip 0.5cm
\underline{\textbf{Answers:}} [Insert figures later.]
\begin{enumerate}
  \item Circle with radius $1$ centered at point $z = 0 + 0i$.
  \item Circle with radius $1$ centered at point $z = 6 + 0i$.
  \item Disk with radius $1$ centered at point $z = 0 + 6i$.
  \item Everything above the line making $45$-degree angle with the real axis.
  \item (Horizontal) Parabola with vertex at $z = 0 + 1i$ since 
  \begin{align*}
    \sqrt{x^2 + y^2} &= x + 1 \\ 
    \implies x^2 + y^2 &= x^2 + 2x + 1 \\
    \implies y^2 = 2x + 1 \\
    \implies x = \frac{y^2 - 1}{2}
  \end{align*}
  So, we get a horizontal parabola.
\end{enumerate}

\vskip 0.5cm
\subsection{Polar Coordinates}
Rather than using the rectangular (cartesian) coordinates to describe $\C$, we can equvalently use \textbf{Polar coordinates} wherein we use the distance from the origin (modulus) and the angle made wtih the real axis (argument).

\begin{align*}
  &(x, y) \rightarrow (r, \theta) \\
  &r = \sqrt{x^2 + y^2} \\
  &\theta = \arctan\left(\frac{y}{x}\right)
\end{align*}

Notice however there is an amibiguity in the angle $\theta$ since $\theta + 2n\pi$ would describe the same point. Thus, we define the \textbf{Principal value branch} as the restriction $\theta \in [0, 2\pi]$.

\vskip 0.5cm
\underline{\textbf{Recall:}} The exponential Function is defined as below for $z \in \C$.
\[ e^z = 1 + \frac{1}{1!} z + \frac{1}{2!} z^2 + \frac{1}{3!} z^3 + \cdots \]

\vskip 0.5cm
Now the usual definitions for the cosine and sine as in $\R$ in terms of the unit circle doesn't quite work for complex numbers. However, we can still define them using the exponential!

\begin{align*}
  \cos(z) &= 1 - \frac{1}{2!}z^2 + \frac{1}{4}z^4 + \cdots\\
  \sin(z) &= z - \frac{1}{3!}z^3 + \frac{1}{5!}z^5 + \cdots
\end{align*}

\vskip 0.5cm
\begin{mathdefinitionbox}{}
  \underline{\textbf{Euler's Theorem:}} $e^{iz} = \cos(z) + i\sin(z)$, $z \in \C$.

\vskip 0.5cm
\underline{\textbf{Proof:}} 
Expanding the exponential, we have
\begin{align*}
  e^{iz} &= 1 + \frac{1}{1!} z + \frac{1}{2!} z^2 + \frac{1}{3!} z^3 + \cdots 
\end{align*}
and note that 
\[ \frac{1}{(2n)!} (iz)^{2n} = \frac{1}{(2n!)} z^{2n} (i^{2n}) = \frac{(-1)^n z^{2n}}{(2n)!} \]

\vskip 0.5cm
Similarly,
\[ \frac{1}{(2n-1)!} (iz)^{2n-1} = \frac{(-1)^{n-1} z^{2n-1}}{(2n-1)!} \]

\vskip 0.5cm
Thus,
\begin{align*}
  e^{iz} &= \underbrace{\sum_{2n} \frac{(-1)^n z^{2n}}{(2n)!}}_{\cos(z)} + \underbrace{\left( \sum_{2n-1} \frac{(-1)^{n-1} z^{2n-1}}{(2n-1)!} \right)}_{\sin(z)} i
\end{align*}
\end{mathdefinitionbox}

\vskip 0.5cm
Additionally, we can convert back from Polar coordinates to Cartesian Coordinates as 

\begin{align*}
  &(r, \theta) \rightarrow (x, y) \\ 
  &x =r\cos(\theta) \\ 
  &y =r\sin(\theta)  
\end{align*}

\vskip 0.5cm
\underline{\textbf{Multiplication:}}
The multiplication of two complex numbers can be thought of in terms of polar coordniates as scaling by $r$ and rotating by $\theta$.

\vskip 1cm
\subsection*{Topology of $\C$}
The set $\C$ is a \textbf{metric space}, and the metric on $\C$ is $\text{dist}(z, w) = |z - w|$ ($\iff$ Eucldiean metric on $\R^2$).

\vskip 0.5cm
\underline{\textbf{Notation:}}
\begin{itemize}
  \item Given a complex number $z_0 \in \C$ and $r > 0, r \in \R$, the set
  \[ B_{r} (z_0) = \{ z \in \C : |z - z_0| < r \} \]
  is called the \textbf{open ball around $z_0$ of radius $r$}. 

  \vskip 0.5cm
  \item Similarly, the set $\overline{B_r (z_0)}$ is the \textbf{closed ball}:
  \[ \overline{B_r (z_0)} = \{ z \in \C : |z - z_0| \leq r \} \]
\end{itemize}

\vskip 0.5cm
\begin{mathdefinitionbox}{Open and Closed sets}
\vskip 0.5cm
  \begin{itemize}
    \item A subset $U \subseteq \C$ is \textbf{open} if for all $z \in U$, there exists $r > 0$ such that $B_r (z) \subset U$.
    
    \item A subset $V \subseteq \C$ is said to be \textbf{closed} if its complement $V^c$ is open in $\C$.
  \end{itemize}
\end{mathdefinitionbox}



\vskip 0.5cm
\underline{\textbf{Ex:}} The set $\C \setminus \R_{\geq 0}$ is closed. This set will be important when studying the complex logarithm.


[Insert figure.]

\vskip 0.5cm
Note that while the closed ball is not open (since the points on the boundary i.e. points with $|z - z_0| = r$ don't satisfy the requirement), a set \emph{can} be both open and closed. For example, the sets $\C$ and $\emptyset$ are both closed and open.

% \printbibliography


\end{document}
