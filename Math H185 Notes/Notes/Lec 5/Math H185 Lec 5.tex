\documentclass{article}

% Language setting
% Replace `english' with e.g. `spanish' to change the document language
\usepackage[english]{babel}

% Set page size and margins
% Replace `letterpaper' with`a4paper' for UK/EU standard size
\usepackage[letterpaper,top=2cm,bottom=2cm,left=3cm,right=3cm,marginparwidth=1.75cm]{geometry}

% Useful packages
\usepackage{amsmath}
\usepackage{amssymb}
\usepackage{graphicx}
\usepackage{enumitem}
\usepackage[colorlinks=true, allcolors=blue]{hyperref}

\usepackage{hyperref}
\hypersetup{
    colorlinks=true,
    linkcolor=blue,
    filecolor=magenta,      
    urlcolor=cyan,
    pdftitle={Overleaf Example},
    pdfpagemode=FullScreen,
    }

\urlstyle{same}

\usepackage{tikz-cd}

%%%%%%%%%%% Box pacakges and definitions %%%%%%%%%%%%%%
\usepackage[most]{tcolorbox}
\usepackage{xcolor}

% Define the colors
\definecolor{boxheader}{RGB}{0, 51, 102}  % Dark blue
\definecolor{boxfill}{RGB}{173, 216, 230}  % Light blue

% Define the tcolorbox environment
\newtcolorbox{mathdefinitionbox}[2][]{%
    colback=boxfill,   % Background color
    colframe=boxheader, % Border color
    fonttitle=\bfseries, % Bold title
    coltitle=white,     % Title text color
    title={#2},         % Title text
    enhanced,           % Enable advanced features
    attach boxed title to top left={yshift=-\tcboxedtitleheight/2}, % Center title
    boxrule=0.5mm,      % Border width
    sharp corners,      % Sharp corners for the box
    #1                  % Additional options
}
%%%%%%%%%%%%%%%%%%%%%%%%%

\newtcolorbox{dottedbox}[1][]{%
    colback=white,    % Background color
    colframe=white,    % Border color (to be overridden by dashrule)
    sharp corners,     % Sharp corners for the box
    boxrule=0pt,       % No actual border, as it will be drawn with dashrule
    boxsep=5pt,        % Padding inside the box
    enhanced,          % Enable advanced features
    overlay={\draw[dashed, thin, black, dash pattern=on \pgflinewidth off \pgflinewidth, line cap=rect] (frame.south west) rectangle (frame.north east);}, % Dotted line
    #1                 % Additional options
}

\usepackage{biblatex}
\addbibresource{sample.bib}


%%%%%%%%%%% New Commands %%%%%%%%%%%%%%
\newcommand*{\T}{\mathcal T}
\newcommand*{\cl}{\text cl}


\newcommand{\ket}[1]{|#1 \rangle}
\newcommand{\bra}[1]{\langle #1|}
\newcommand{\inner}[2]{\langle #1 | #2 \rangle}
\newcommand{\R}{\mathbb{R}}
\newcommand{\C}{\mathbb{C}}
\newcommand{\V}{\mathbb{V}}
\newcommand{\Hilbert}{\mathcal{H}}
\newcommand{\oper}{\hat{\Omega}}
\newcommand{\lam}{\hat{\Lambda}}
\newcommand{\defn}{\underline{\textbf{Def:}}}


\newcommand{\bigslant}[2]{{\raisebox{.2em}{$#1$}\left/\raisebox{-.2em}{$#2$}\right.}}
\newcommand{\restr}[2]{{% we make the whole thing an ordinary symbol
  \left.\kern-\nulldelimiterspace % automatically resize the bar with \right
  #1 % the function
  \vphantom{\big|} % pretend it's a little taller at normal size
  \right|_{#2} % this is the delimiter
  }}
%%%%%%%%%%%%%%%%%%%%%%%%%%%%%%%%%%%%%%%


\tcbset{theostyle/.style={
    enhanced,
    sharp corners,
    attach boxed title to top left={
      xshift=-1mm,
      yshift=-4mm,
      yshifttext=-1mm
    },
    top=1.5ex,
    colback=white,
    colframe=blue!75!black,
    fonttitle=\bfseries,
    boxed title style={
      sharp corners,
    size=small,
    colback=blue!75!black,
    colframe=blue!75!black,
  } 
}}

\newtcbtheorem[number within=section]{Theorem}{Theorem}{%
  theostyle
}{thm}

\newtcbtheorem[number within=section]{Definition}{Definition}{%
  theostyle
}{def}



\title{Math H185 Lecture 4}
\author{Keshav Balwant Deoskar}

\begin{document}
\maketitle

% \vskip 0.5cm
These are notes taken from lectures on Complex Analysis delivered by Professor Tony Feng for UC Berekley's Math H185 class in the Sprng 2024 semester.
% \pagebreak 

\tableofcontents

\pagebreak

%%%%%%%%%%%%%%%%%%%%%%%%%%%%%%%%%%%%%%%%%%%%%%%%%%%%%%%%%%%%%%%%%%
\section{January 26 - Geometry of Holomorphic functions}
%%%%%%%%%%%%%%%%%%%%%%%%%%%%%%%%%%%%%%%%%%%%%%%%%%%%%%%%%%%%%%%%%%

\vskip 0.5cm
Whereas in the reals, we can simply look at the graph of a function and tell whether it's differentiable or not, functions on $\C$ are different.

\vskip 0.25cm
Our intuition that a function $\R^d \rightarrow \R^d$ is differentiable is to see whether it is "smooth-ish".

\vskip 0.25cm
The goal for today is to develop some sort of similar intuition for Holomorphic functions on $\C$. (Note: A characteristic equivalent to Holomorphicity s "conformality". We'll explore this today as well.)

\vskip 1cm
\subsection{Review of Differentiation}
For a function $f : \R^d \rightarrow \R^d$, the derivative is supposed to be the \emph{best linear approximation} to $f$.

\[ f(x) \approx f(x_0) + (x - x_0)f'(x_0) \]
for $x \approx x_0$. (In 1D, $f'(x_0)$ is just a number, but in higher dimensions it's generally a \emph{matrix}).


\vskip 0.5cm
Now, the same idea holds for functions on the complex plane. That is, for $f : \Omega \subset_{open} \C \rightarrow \C$,

\[\boxed{ f(z_0) \approx f(z_0) + \underbrace{f'(z_0)}_{\in \C} (z - z_0), z \approx z_0 }\]

\vskip 0.5cm
\begin{dottedbox}
  Contrast this with $f : \Omega \subset_{open} \R^2 \rightarrow \R^2$ where 

  \[ f(x, y) \approx f(x_0, y_0) + (x - x_0, y - y_0)f'(x_0, y_0) \]

  where 
  \[ f'(x_0, y_0) =  \begin{pmatrix}
    \frac{\partial u}{\partial x} & \frac{\partial v}{\partial x} \\
    \frac{\partial v}{\partial x} & \frac{\partial v}{\partial y}
  \end{pmatrix} \]
\end{dottedbox}

[Listen to lecture recording and write about comparison between derivatives on $\C$ and $\R^2$ later]

\vskip 1cm
\subsection{Cauchy-Riemann Equations}

Consider a complex function $f : \Omega \subset_{open} \rightarrow \C = \R^2$ and denote

\[ f(z) = u(z) + iv(z) = u(x, y) + iv(x, y) = f(x, y)  \]

\vskip 0.5cm
\begin{dottedbox}
  \underline{Lemma:} If $f$ is holomorphic at $z_0 = x_0 + i y_0$, then 
  \begin{align*}
    \frac{\partial u}{\partial x}(z_0) &= \frac{\partial v}{\partial y}(z_0) \\
    \frac{\partial u}{\partial y}(z_0) &= - \frac{\partial v}{\partial x}(z_0) 
  \end{align*}
\end{dottedbox}

\vskip 0.5cm
\underline{\textbf{Proof:}} Let $f$ be holomorphic at $z_0$. Then, 
\[ \lim_{x \in \R \rightarrow 0} \frac{f(z_0 + x) - f(z_0)}{x} = f'(z_0) = \lim_{y \in \R \rightarrow 0} \frac{f(z_0 + iy) - f(z_0)}{y} \]

but the Left Hand Expresson above is, by definition, $\partial f / \partial x (z_0)$, or in other words, 
\[ \frac{\partial u}{\partial x}(z_0) + i \frac{\partial v}{\partial x}(z_0) \]

and similarly the Right Hand Expression is $\frac{1}{i} \cdot \partial f / \partial x (z_0)$, or in other words, 
\[ \frac{1}{i} \left( \frac{\partial u}{\partial xy}(z_0) + i \frac{\partial v}{\partial y}(z_0) \right) \]

Then, since both of them are equal to $f'(z_0)$, we should have 
\[ \boxed{\frac{\partial u}{\partial x}(z_0) + i \frac{\partial v}{\partial x}(z_0) = \frac{1}{i} \left( \frac{\partial u}{\partial xy}(z_0) + i \frac{\partial v}{\partial y}(z_0) \right) }\]

The Real and Imaginary parts of the above must then be equal, so we get 

\begin{align*}
    \frac{\partial u}{\partial x}(z_0) &= \frac{\partial v}{\partial y}(z_0) \\
    \frac{\partial u}{\partial y}(z_0) &= - \frac{\partial v}{\partial x}(z_0) 
\end{align*}

These are the \emph{Cauchy-Riemann Equations}.

\vskip 0.5cm
\underline{\textbf{Remark:}} There is a converse which is harder to prove. 
\begin{dottedbox}
  If $f$ is $C^1$ and the Cauchy-Riemann Equations hold at $z_0$, then $f$ is holomorphic at $z_0$.
\end{dottedbox}

This has important applications in PDEs.

\vskip 1cm
\underline{\textbf{Summary:}} The partial derivative matrix of a holomorphic function has the form 

\[ \begin{bmatrix}
  a & b \\
  -b & a
\end{bmatrix} \]

\vskip 0.25cm
where $a, b \in R$. Then, using polar coordinates $(a, b) \rightarrow (r, \theta)$ wherein $a = r\cos(\theta), b = r \sin(\theta)$ then the matrix is 

\[ r \begin{pmatrix}
  \cos(\theta) & \sin(\theta) \\
  -\sin(\theta) & \cos(\theta) \\
\end{pmatrix} \]

\vskip 0.25cm
So, the derivative $f'(z_0)$ of a complex function $f$ is a linear map of the form "scaling + rotaton". There are \emph{conformal mappings} i.e. they infinitessimally preserve angles or scale to zero.

\vskip 0.5cm
\underline{{Example:}} Consder $f(z) = \lambda z$, $\lambda - r e^{i \theta} \in \C$. Angles are certainly preserved by this map:

[insert figure].

\vskip 0.5cm
\underline{{Example:}} In contrast to the last example, $f(z) = \overline{z}$ does \emph{not} preserve angles so it's not Holomorphic.

[insert figure]

\vskip 0.5cm
\underline{Example:} $f(z) = z^2$: This one's a bit tricky. One may think this map is \emph{not} conformal because the real axis stays fixed while the positive imaginary axis becomes aligned with the negative real axis, changing the angle between them from 90 degrees to 180 degrees.

\vskip 0.5cm
\emph{However,} recall that conformal maps can also \emph{scale to zero}. In fact, that's essentially what happens to numbers in a very small neighborhood around the origin.

[insert image and write some more explanation]

\end{document}
