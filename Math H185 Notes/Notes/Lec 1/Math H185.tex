\documentclass{article}

% Language setting
% Replace `english' with e.g. `spanish' to change the document language
\usepackage[english]{babel}

% Set page size and margins
% Replace `letterpaper' with`a4paper' for UK/EU standard size
\usepackage[letterpaper,top=2cm,bottom=2cm,left=3cm,right=3cm,marginparwidth=1.75cm]{geometry}

% Useful packages
\usepackage{amsmath}
\usepackage{amssymb}
\usepackage{graphicx}
\usepackage[colorlinks=true, allcolors=blue]{hyperref}

\usepackage{hyperref}
\hypersetup{
    colorlinks=true,
    linkcolor=blue,
    filecolor=magenta,      
    urlcolor=cyan,
    pdftitle={Overleaf Example},
    pdfpagemode=FullScreen,
    }

\urlstyle{same}

\usepackage{tikz-cd}

%%%%%%%%%%% Box pacakges and definitions %%%%%%%%%%%%%%
\usepackage[most]{tcolorbox}
\usepackage{xcolor}

% Define the colors
\definecolor{boxheader}{RGB}{0, 51, 102}  % Dark blue
\definecolor{boxfill}{RGB}{173, 216, 230}  % Light blue

% Define the tcolorbox environment
\newtcolorbox{mathdefinitionbox}[2][]{%
    colback=boxfill,   % Background color
    colframe=boxheader, % Border color
    fonttitle=\bfseries, % Bold title
    coltitle=white,     % Title text color
    title={#2},         % Title text
    enhanced,           % Enable advanced features
    attach boxed title to top left={yshift=-\tcboxedtitleheight/2}, % Center title
    boxrule=0.5mm,      % Border width
    sharp corners,      % Sharp corners for the box
    #1                  % Additional options
}
%%%%%%%%%%%%%%%%%%%%%%%%%

\usepackage{biblatex}
\addbibresource{sample.bib}


%%%%%%%%%%% New Commands %%%%%%%%%%%%%%
\newcommand*{\T}{\mathcal T}
\newcommand*{\cl}{\text cl}


\newcommand{\ket}[1]{|#1 \rangle}
\newcommand{\bra}[1]{\langle #1|}
\newcommand{\inner}[2]{\langle #1 | #2 \rangle}
\newcommand{\R}{\mathbb{R}}
\newcommand{\C}{\mathbb{C}}
\newcommand{\V}{\mathbb{V}}
\newcommand{\Hilbert}{\mathcal{H}}
\newcommand{\oper}{\hat{\Omega}}
\newcommand{\lam}{\hat{\Lambda}}

\newcommand{\bigslant}[2]{{\raisebox{.2em}{$#1$}\left/\raisebox{-.2em}{$#2$}\right.}}
\newcommand{\restr}[2]{{% we make the whole thing an ordinary symbol
  \left.\kern-\nulldelimiterspace % automatically resize the bar with \right
  #1 % the function
  \vphantom{\big|} % pretend it's a little taller at normal size
  \right|_{#2} % this is the delimiter
  }}
%%%%%%%%%%%%%%%%%%%%%%%%%%%%%%%%%%%%%%%


\tcbset{theostyle/.style={
    enhanced,
    sharp corners,
    attach boxed title to top left={
      xshift=-1mm,
      yshift=-4mm,
      yshifttext=-1mm
    },
    top=1.5ex,
    colback=white,
    colframe=blue!75!black,
    fonttitle=\bfseries,
    boxed title style={
      sharp corners,
    size=small,
    colback=blue!75!black,
    colframe=blue!75!black,
  } 
}}

\newtcbtheorem[number within=section]{Theorem}{Theorem}{%
  theostyle
}{thm}

\newtcbtheorem[number within=section]{Definition}{Definition}{%
  theostyle
}{def}



\title{Math H185 Notes}
\author{Keshav Balwant Deoskar}

\begin{document}
\maketitle

% \vskip 0.5cm
These are notes taken from lectures on Complex Analysis delivered by Professor Tony Feng for UC Berekley's Math H185 class in the Sprng 2024 semester.
% \pagebreak 

\tableofcontents

\pagebreak

%%%%%%%%%%%%%%%%%%%%%%%%%%%%%%%%%%%%%%%%%%%%%%%%%%%%%%%%%%%%%%%%%%
\section{January 17 - Introduction to Complex Numbers}
%%%%%%%%%%%%%%%%%%%%%%%%%%%%%%%%%%%%%%%%%%%%%%%%%%%%%%%%%%%%%%%%%%

\vskip 0.5cm

\subsection{Real Numbers}
Before jumping into Complex Numbers, let's recall a property of Real Numbers - the set containing which is denoted $\mathbb{R}$.
\vskip 0.5cm

\underline{\textbf{Note:}} If $a \in \mathbb{R}$ then $a^2 \geq 0$.  
So, in this number system  negative real numbers do not have square roots in $\mathbb{R}$. 
\vskip 0.5cm

This is a limitation of $\mathbb{R}$, which we can fix by enlargening our field. (Similar to how the set of rationals was enlargened to the set of reals in Real Analysis).
\vskip 0.5cm


\subsection{Imaginary Numbers}
We can introduce a new kind of object called an "Imaginary number" such that imaginary numbers square to negatve $(\leq 0)$ real numbers.
\vskip 0.5cm

We write $i = \sqrt{-1}$.
\vskip 0.5cm

\underline{\textbf{Proposition:}} Any imaginary number can be expressed as $bi$, $b \in \mathbb{R}$.
\vskip 0.5cm

\underline{\textbf{Proof:}}

\subsection{Complex Numbers}

\begin{mathdefinitionbox}{Complex Numbers}
  \begin{itemize}
  \item A complex number is an expression $z = a + bi$ where $a, b \in \mathbb{R}$
  \item The set of complex numbers is denoted $\mathbb{C}$
  \end{itemize}
\end{mathdefinitionbox}
\vskip 0.5cm

\underline{\textbf{Remark:}} $\mathbb{C}$ is the \underline{algebraic closure} of $\mathbb{R}$. 

\vskip 0.5cm
In a sense, this is saying that there are no more "deficiencies" - Unlike polynomials in the reals, \emph{every} complex polynomials is guaranteed to have some complex roots. We will return to this statement later in the course when studying the Fundamental Theorem of Algebra.

\vskip 0.5cm
Let $z = a + bi$ be a complex number. Then, 
\begin{itemize}
  \item The \emph{real part} of $z$ is $Re(z) = a \in \mathbb{R}$ and the \emph{imaginary part} of $z$ is $Im(z) = b \in \mathbb{R}$.
  \item The \emph{complex conjugate} of $z$ is $\bar{z} = a - bi$
\end{itemize} 

\vskip 0.5cm
\subsection{Operations on Complex Numbers}

"Addition is componentwise"

\begin{align*}
  \text{Addtion: } z &= a + bi \\
                  +w &= c + di \\
              z+w&=(a+c) + (b+d)i
\end{align*}

\vskip 0.5cm
"Multiplication distributes"

For $z = (a + bi)$, $w = (c + di)$ we have 
\begin{align*}
  z \cdot w &= (a + bi) \cdot (c + di) \\
            &= a \cdot (c + di) + bi \cdot (c + di) \\
            &= (ac - bd) + (ad + bc)i
\end{align*}

\vskip 0.5cm
Addition and Multiplication satisfy the usual commutativity, associativity, and distributivity. However, Division is a bit more complicated. 

\vskip 0.5cm
\textbf{\textbf{Division:}} If $z \in \mathbb{C}$, $w \in \mathbb{C} \setminus \{0\}$, then $z/w \in \mathbb{C}$ is the \underline{unique} complex number such that $w \cdot (z/w) = z$.

\vskip 0.5cm
\textbf{Examples:} Write the following complex numbers as $a + bi$ where $a, b \in \mathbb{R}$

\begin{enumerate}
  \item $(9-12i) + (12i - 16) = (9 - 16) + (-12i + 12i) = -7$ 
  \item $(3 + 4i) \cdot (3 - 4i) = 9 -12i + 12i - 16i^2 = 25$
  \item $\frac{50 + 50i}{3 - 4i} = \frac{50 + 50i}{3 - 4i} \cdot \frac{3 + 4i}{3 + 4i} = \frac{150 + 200i + 150i + 200i^2}{25}  = \frac{-50 + 350i}{25} = -2 + 14i$ 
\end{enumerate}

\pagebreak


% \printbibliography


\end{document}
