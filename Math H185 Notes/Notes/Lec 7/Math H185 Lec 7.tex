\documentclass{article}

% Language setting
% Replace `english' with e.g. `spanish' to change the document language
\usepackage[english]{babel}

% Set page size and margins
% Replace `letterpaper' with`a4paper' for UK/EU standard size
\usepackage[letterpaper,top=2cm,bottom=2cm,left=3cm,right=3cm,marginparwidth=1.75cm]{geometry}

% Useful packages
\usepackage{amsmath}
\usepackage{amssymb}
\usepackage{graphicx}
\usepackage{enumitem}
\usepackage[colorlinks=true, allcolors=blue]{hyperref}
\usepackage{mathtools}

\usepackage{hyperref}
\hypersetup{
    colorlinks=true,
    linkcolor=blue,
    filecolor=magenta,      
    urlcolor=cyan,
    pdftitle={Math H185 Lecture 7},
    pdfpagemode=FullScreen,
    }

\urlstyle{same}

\usepackage{tikz-cd}

%%%%%%%%%%% Box pacakges and definitions %%%%%%%%%%%%%%
\usepackage[most]{tcolorbox}
\usepackage{xcolor}

% Define the colors
\definecolor{boxheader}{RGB}{0, 51, 102}  % Dark blue
\definecolor{boxfill}{RGB}{173, 216, 230}  % Light blue

% Define the tcolorbox environment
\newtcolorbox{mathdefinitionbox}[2][]{%
    colback=boxfill,   % Background color
    colframe=boxheader, % Border color
    fonttitle=\bfseries, % Bold title
    coltitle=white,     % Title text color
    title={#2},         % Title text
    enhanced,           % Enable advanced features
    attach boxed title to top left={yshift=-\tcboxedtitleheight/2}, % Center title
    boxrule=0.5mm,      % Border width
    sharp corners,      % Sharp corners for the box
    #1                  % Additional options
}
%%%%%%%%%%%%%%%%%%%%%%%%%

\newtcolorbox{dottedbox}[1][]{%
    colback=white,    % Background color
    colframe=white,    % Border color (to be overridden by dashrule)
    sharp corners,     % Sharp corners for the box
    boxrule=0pt,       % No actual border, as it will be drawn with dashrule
    boxsep=5pt,        % Padding inside the box
    enhanced,          % Enable advanced features
    overlay={\draw[dashed, thin, black, dash pattern=on \pgflinewidth off \pgflinewidth, line cap=rect] (frame.south west) rectangle (frame.north east);}, % Dotted line
    #1                 % Additional options
}

\usepackage{biblatex}
\addbibresource{sample.bib}


%%%%%%%%%%% New Commands %%%%%%%%%%%%%%
\newcommand*{\T}{\mathcal T}
\newcommand*{\cl}{\text cl}


\newcommand{\ket}[1]{|#1 \rangle}
\newcommand{\bra}[1]{\langle #1|}
\newcommand{\inner}[2]{\langle #1 | #2 \rangle}
\newcommand{\R}{\mathbb{R}}
\newcommand{\C}{\mathbb{C}}
\newcommand{\V}{\mathbb{V}}
\newcommand{\Hilbert}{\mathcal{H}}
\newcommand{\oper}{\hat{\Omega}}
\newcommand{\lam}{\hat{\Lambda}}
\newcommand{\defn}{\underline{\textbf{Def: }}}
\newcommand{\defeq}{\vcentcolon=}


\newcommand{\bigslant}[2]{{\raisebox{.2em}{$#1$}\left/\raisebox{-.2em}{$#2$}\right.}}
\newcommand{\restr}[2]{{% we make the whole thing an ordinary symbol
  \left.\kern-\nulldelimiterspace % automatically resize the bar with \right
  #1 % the function
  \vphantom{\big|} % pretend it's a little taller at normal size
  \right|_{#2} % this is the delimiter
  }}
%%%%%%%%%%%%%%%%%%%%%%%%%%%%%%%%%%%%%%%


\tcbset{theostyle/.style={
    enhanced,
    sharp corners,
    attach boxed title to top left={
      xshift=-1mm,
      yshift=-4mm,
      yshifttext=-1mm
    },
    top=1.5ex,
    colback=white,
    colframe=blue!75!black,
    fonttitle=\bfseries,
    boxed title style={
      sharp corners,
    size=small,
    colback=blue!75!black,
    colframe=blue!75!black,
  } 
}}

\newtcbtheorem[number within=section]{Theorem}{Theorem}{%
  theostyle
}{thm}

\newtcbtheorem[number within=section]{Definition}{Definition}{%
  theostyle
}{def}



\title{Math H185 Lecture 7}
\author{Keshav Balwant Deoskar}

\begin{document}
\maketitle

% \vskip 0.5cm
These are notes taken from lectures on Complex Analysis delivered by Professor Tony Feng for UC Berekley's Math H185 class in the Sprng 2024 semester.
% \pagebreak 

\tableofcontents

\pagebreak

%%%%%%%%%%%%%%%%%%%%%%%%%%%%%%%%%%%%%%%%%%%%%%%%%%%%%%%%%%%%%%%%%%
\section{January 31 - Cauchy's Theorem}
%%%%%%%%%%%%%%%%%%%%%%%%%%%%%%%%%%%%%%%%%%%%%%%%%%%%%%%%%%%%%%%%%%

\vskip 0.5cm
\subsection*{Review}

\vskip 0.5cm
\begin{itemize}
  \item Recall that the \underline{boundary} of a set $\Omega \subset \C$ is
  \[ \partial \Omega = \underbrace{\overline{\Omega}}_{\text{closure}} \setminus \underbrace{\Omega^{\circ}}_{\text{interior}} \]

  \item For example, the boundary of the (closed or open) ball of radius $r$ centered at $z_0$ i.e. $\Omega = \partial B_r(z_0)$ is $\partial B_r(z_0) = \partial \overline{B_r(z_0)}$ is the circle of radius $r$ around $z_0$.
  
  \item When integrating over a curve, we can do a sort of $u$-substitution 
  \[ \int_{\gamma} f(z) dz = \int_{u \circ \gamma} f(z) u'(z) dz \] as long as $u$ is holomorphic in an open neighborhood of $\gamma$.
\end{itemize}

\vskip 1cm
\subsection{Last time}
Last time, we calculated the following integral:

\begin{mathdefinitionbox}{}
  \[ \int_{\partial B_r(0)} z^n dz = \begin{cases}
    0\;\;n \neq -1 \\
    2\pi i \; n = 1
  \end{cases}  \]
\end{mathdefinitionbox}

\vskip 0.5cm
Doing the substitution $u(z) = z - z_0$. The integral on the circle around some other point $z_0$ also gives us the same answer i.e. 
\[ \int_{\partial B_r(z_0)} (z - z_0)^n dz = \int_{\partial B_r(0)} z^n dz \]

\vskip 1cm
\subsection{Cauchy's Theorem}

\vskip 0.5cm
\begin{mathdefinitionbox}{}
  \underline{\textbf{Theorem (Cauchy):}} If $U \subset_{open} \C$ has piece-wise smooth boundary and $f(z)$ is holomorphic around a neighborhood of $\overline{U}$ then 
  \[ \int_{\partial U} f(z) = 0 \] 
\end{mathdefinitionbox}

\vskip 0.5cm
As you may have read in many books, this is \emph{\textbf{the most important result in Complex Analysis!}} So many other results fall out as a result of this, but it'll take us some time to really appreciate its importance.

\vskip 0.5cm
\underline{Example:} If we want to calculate the integral
\[ \int_{\partial B_r(0)} z^n dz \]

we know that $z^n$ is holomorphic on the ball for all $n \geq 0$, so Cauchy's Theorem immediately tells us the integral is zero.

\vskip 0.5cm
For $n < 0$ though, the function is not holomorphic on the ball because it is not holomorphic at the origin. So, Cauchy's theorem doesn't immediately apply. We'll need to be bit more clever.

\vskip 0.5cm
Consider the annulus formed the the crcles around $z = 0$ of radius $r$ and $r'$, ($r' < r$). The boundary of this region consists of the two circles, but with the inner circle having the reversed orientation i.e. $\partial U = \partial B_r(0) \cup \partial B_{r'}(0)^{-}$.

\vskip 0.5cm
[Include picture]

\vskip 0.5cm
The origin is excluded, so $z^n$ is holomorphic and Cauchy's theorem tells us the integral is zero, for all $n$.

\vskip 0.5cm
So, 
\begin{align*}
  &\int_{\partial B_r(0)} z^n dz + \int_{\partial B_{r'}(0)^{-}} z^n dz = 0 \\
  \implies&\int_{\partial B_r(0)} z^n dz = - \int_{\partial B_{r'}(0)^{-}} z^n dz \\
  \implies& \boxed{\int_{\partial B_r(0)} z^n dz = \int_{\partial B_{r'}(0)} z^n dz }
\end{align*}

\vskip 0.5cm
This verifies our observation from last lecture that the integral of $z^n$ along a circle centered at the origin is \emph{\textbf{independent of $r$}}.

\vskip 0.5cm
\subsubsection*{Upshots}
\begin{itemize}
  \item We will see that it's useful to apply Cauchy's Theorem to creative choices of $U$.
  \item Cauchy's Theorem allows us to do contour manipulation i.e. we can deform the curve $\gamma$ in the domain where $f$ (the function we're integrating) is holomorphic.
  [Include squiggly wiggly nasty curve diagram and show it's equivalent to just a ball]
\end{itemize}


\vskip 0.5cm
\subsection{Cauchy's Formula}

\begin{mathdefinitionbox}{}
  Suppose $f : \Omega_{open} \rightarrow \C$ be holomorphic on $\Omega$ and let $z_0 \in \Omega, r > 0$ such that $B_r(z_0) \in \Omega$. Then, for all $z \in B_r(z_0)$, 
  \[ f(z) = \frac{1}{2\pi i} \int_{\partial B_r(z_0)}  \frac{f(w)}{w - z} dw  \] 
\end{mathdefinitionbox}

\vskip 0.5cm
This, too, is an incredible statement. Cauchy's formula allows us to calculate the value of the function at \emph{any} point inside $B_r(z_0)$, only using the values of the function on the ball $B_r(z_0)$ or any path homologous to it.

\vskip 0.5cm
\begin{dottedbox}
  \underline{\textbf{Proof:}} Define a $g(w) = \frac{f(w)}{w -  z}$. This function is holomorphic on $B_r(z_0) \setminus \{z\}$. Let $\delta > 0$ such that $B_{\delta}(z_0) \subset B_r(z_0)$ and define $U = B_r(z_0) \setminus B_{\delta}(z)$.

  \vskip 0.5cm
  Then, the boundary of $U$ is $\partial U = \partial B_{r}(z_0) \cup \partial B_{\delta} (z)^{-}$.

  \vskip 0.5cm
  By Cauchy's Theorem, 
  \[ \int_{\partial B_r(z)} g(w) dw = \int_{\partial_{\delta}(z)} g(w) dw \]

  Then, since $ff$ is holomorphic at $z$, we have $f(w) = f(z) + \epsilon(w)$ where $\epsilon$

  [Complete later, when recording is out]
\end{dottedbox}

\vskip 0.5cm
\begin{dottedbox}
  \underline{Summary:} Cauchy's formula says 
  \[ f(z) = \frac{1}{2 \pi i} \int_{\partial B_r(z)} \frac{f(w)}{w-z} dw \]

  \begin{itemize}
    \item [Explain how but] the fact that a holomorphic functions is automtically infinitely differentiable is a result of Cauchy's formula.
  \end{itemize}
\end{dottedbox}

\vskip 0.5cm
\end{document}
