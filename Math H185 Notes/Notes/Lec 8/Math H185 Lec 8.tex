\documentclass{article}

% Language setting
% Replace `english' with e.g. `spanish' to change the document language
\usepackage[english]{babel}

% Set page size and margins
% Replace `letterpaper' with`a4paper' for UK/EU standard size
\usepackage[letterpaper,top=2cm,bottom=2cm,left=3cm,right=3cm,marginparwidth=1.75cm]{geometry}

% Useful packages
\usepackage{amsmath}
\usepackage{amssymb}
\usepackage{graphicx}
\usepackage{enumitem}
\usepackage[colorlinks=true, allcolors=blue]{hyperref}
\usepackage{mathtools}

\usepackage{hyperref}
\hypersetup{
    colorlinks=true,
    linkcolor=blue,
    filecolor=magenta,      
    urlcolor=cyan,
    pdftitle={Math H185 Lecture 8},
    pdfpagemode=FullScreen,
    }

\urlstyle{same}

\usepackage{tikz-cd}

%%%%%%%%%%% Box pacakges and definitions %%%%%%%%%%%%%%
\usepackage[most]{tcolorbox}
\usepackage{xcolor}

% Define the colors
\definecolor{boxheader}{RGB}{0, 51, 102}  % Dark blue
\definecolor{boxfill}{RGB}{173, 216, 230}  % Light blue

% Define the tcolorbox environment
\newtcolorbox{mathdefinitionbox}[2][]{%
    colback=boxfill,   % Background color
    colframe=boxheader, % Border color
    fonttitle=\bfseries, % Bold title
    coltitle=white,     % Title text color
    title={#2},         % Title text
    enhanced,           % Enable advanced features
    attach boxed title to top left={yshift=-\tcboxedtitleheight/2}, % Center title
    boxrule=0.5mm,      % Border width
    sharp corners,      % Sharp corners for the box
    #1                  % Additional options
}
%%%%%%%%%%%%%%%%%%%%%%%%%

\newtcolorbox{dottedbox}[1][]{%
    colback=white,    % Background color
    colframe=white,    % Border color (to be overridden by dashrule)
    sharp corners,     % Sharp corners for the box
    boxrule=0pt,       % No actual border, as it will be drawn with dashrule
    boxsep=5pt,        % Padding inside the box
    enhanced,          % Enable advanced features
    overlay={\draw[dashed, thin, black, dash pattern=on \pgflinewidth off \pgflinewidth, line cap=rect] (frame.south west) rectangle (frame.north east);}, % Dotted line
    #1                 % Additional options
}

\usepackage{biblatex}
\addbibresource{sample.bib}


%%%%%%%%%%% New Commands %%%%%%%%%%%%%%
\newcommand*{\T}{\mathcal T}
\newcommand*{\cl}{\text cl}


\newcommand{\ket}[1]{|#1 \rangle}
\newcommand{\bra}[1]{\langle #1|}
\newcommand{\inner}[2]{\langle #1 | #2 \rangle}
\newcommand{\R}{\mathbb{R}}
\newcommand{\C}{\mathbb{C}}
\newcommand{\V}{\mathbb{V}}
\newcommand{\Hilbert}{\mathcal{H}}
\newcommand{\oper}{\hat{\Omega}}
\newcommand{\lam}{\hat{\Lambda}}
\newcommand{\defn}{\underline{\textbf{Def: }}}
\newcommand{\defeq}{\vcentcolon=}


\newcommand{\bigslant}[2]{{\raisebox{.2em}{$#1$}\left/\raisebox{-.2em}{$#2$}\right.}}
\newcommand{\restr}[2]{{% we make the whole thing an ordinary symbol
  \left.\kern-\nulldelimiterspace % automatically resize the bar with \right
  #1 % the function
  \vphantom{\big|} % pretend it's a little taller at normal size
  \right|_{#2} % this is the delimiter
  }}
%%%%%%%%%%%%%%%%%%%%%%%%%%%%%%%%%%%%%%%


\tcbset{theostyle/.style={
    enhanced,
    sharp corners,
    attach boxed title to top left={
      xshift=-1mm,
      yshift=-4mm,
      yshifttext=-1mm
    },
    top=1.5ex,
    colback=white,
    colframe=blue!75!black,
    fonttitle=\bfseries,
    boxed title style={
      sharp corners,
    size=small,
    colback=blue!75!black,
    colframe=blue!75!black,
  } 
}}

\newtcbtheorem[number within=section]{Theorem}{Theorem}{%
  theostyle
}{thm}

\newtcbtheorem[number within=section]{Definition}{Definition}{%
  theostyle
}{def}



\title{Math H185 Lecture 8}
\author{Keshav Balwant Deoskar}

\begin{document}
\maketitle

% \vskip 0.5cm
These are notes taken from lectures on Complex Analysis delivered by Professor Tony Feng for UC Berekley's Math H185 class in the Sprng 2024 semester.
% \pagebreak 

\tableofcontents

\pagebreak

%%%%%%%%%%%%%%%%%%%%%%%%%%%%%%%%%%%%%%%%%%%%%%%%%%%%%%%%%%%%%%%%%%
\section{February 2 - Cauchy's Formula, Infinite Differentiability}
%%%%%%%%%%%%%%%%%%%%%%%%%%%%%%%%%%%%%%%%%%%%%%%%%%%%%%%%%%%%%%%%%%

\vskip 0.5cm
\subsection*{Recall}

\vskip 0.5cm
\begin{itemize}
  \item Last time, we saw Cauchy's formula, which states that if $f : \C \rightarrow \C$ is holomorphic on an open neighborhood $\supseteq \overline{B_{r}(z_0)}$
  \[ f(w) = \frac{1}{2\pi i} \int_{\partial B_r(z_0)} \frac{f(w)}{w - z} dw \]
  This integral can be thought of as kind of a "weighted average".
  
  \item What we'll see soon is that this integral gives us the infinitely differentiability of a holomorphic function.
  
  \item Something to keep in mind is \emph{when we're allowed to interchange an integral and derivative} i.e. when the following equation is valid 
  \[ \frac{\partial}{\partial z} \int_{\gamma} g(z, w) dw = \int_{\gamma} \frac{\partial g(z, w)}{\partial z} dw \]
  \item We can interchahnge them when $g(z, w)$ and all of its derivatives are both continuous (jointly on both $z, w$).
\end{itemize}

\vskip 0.5cm
\subsection{Infinite differentiability of a holomorphic function}

\vskip 0.5cm
Consider a function $f$ which is holomorphic on an open neighborhood $\supseteq \overline{B_{r}(z_0)}$. Then, by Cauchy's formula, for $z \in B_r(z_0)$ we have 
\[  f(z) = \frac{1}{2\pi i} \int_{\partial B_r(z_0)} \frac{f(w)}{w - z} dw  \]

Then, differentiating both sides of the equation 
\begin{align*}
  f'(z) &= \frac{1}{2\pi i} \int_{\partial B_r(z_0)} \frac{d}{dz} \frac{f(w)}{w - z} dw \\
  &= \frac{1}{2\pi i } \int_{\partial B_r(z_0)} \frac{f(w)}{(w - z^2)} dw 
\end{align*}

But notice that all functions in the above expression are still contniuous! So, we can differentiate again 
\[ f''(z) = \frac{2}{2\pi i} \int_{\partial B_r(z_0)} \frac{f(w)}{(w-z)^3} dz \]
...and again!
\[ f''(z) = \frac{6}{2\pi i} \int_{\partial B_r(z_0)} \frac{f(w)}{(w-z)^4} dz  \]

\vskip 1cm
..and we can keep going! In general, the $n^{\text{th}}$ derivative is given by 

\[\boxed{ f^{(n)} (z) = \frac{n!}{2\pi i} \int_{\partial B_r(z_0)} \frac{f(w)}{(w-z)^{n+1}} dz}  \]

\vskip 0.5cm
So, Given $f$ is holomorphic i.e. it is once complex-differentiable, Cauchy's formula tells us that its derivative also holomorphic, and then that functions derivate is holomorphic, and... so on. 
\begin{dottedbox}
  \underline{Corollary:} If $f$ is holomorphic at $z_0$, then $f$ is infinitely $\C-$ differentiable in at $z_0$.
\end{dottedbox}

\vskip 0.5cm
It is impossible for a once-differentiable complex function to not be infinitely differentiable! This is in stark contrast to real valued functions where we can have once differentiable functions.

\vskip 0.5cm
\subsection{Where Now? Contour Integration}
Next, we use Cauchy's Theorem + Cauchy's formula to do \emph{\textbf{contour integration}}. [Write snazzy section intro]

\vskip 0.5cm
\underline{Example:} Consider the Integral
\[ \int_{\partial B_4(0)} \frac{z}{z^2 + 1} dz \]

The integrand has singularities at $z = \pm i$ as $z^2 + 1 = (z-i)(z+i)$, so the integrand doesn't immediately vanish. How do we actually calculate the integral then? We have three main methods.


\vskip 1cm
\textbf{Method A: By Cauchy's Theorem} 

[Insert image and explain why the following is true]
\[ \int_{\partial B_4(0)} f(z)dz  = \int_{\partial B_{1/2} (i)} f(z)dz + \int_{\partial B_{1/2} (-i)} f(z)dz  \]

Now, the function $f(z)$ can be viewed from two different perspectives:
\[ f(z) = \frac{z}{(z-i)(z+i)} = \frac{g(z)}{(z-i)} = \frac{h(z)}{(z+i)}\]
where $g(z) = $ and $h(z) = $

Viewing $f(z)$ as $g(z) / (z-i)$ on the island where $(z-i)$ has a pole, and as $h(z)/(z+i)$ on the island where $(z+i)$ has a pole, we can then apply Cauchy's formula so that 

\begin{align*}
  \int_{\partial B_4(0)} f(z)dz &= \int_{\partial B_{1/2} (i)} f(z)dz + \int_{\partial B_{1/2} (-i)} f(z)dz  \\
  &= 2\pi i \left( \frac{1}{2i} \right) + 2\pi i \left( \frac{-i}{-2i} \right) \\
  &= 2\pi i
\end{align*}

\vskip 0.5cm
\textbf{Method B: Partial Fraction Decomposition}

Another way to do the integral is to recognize 
\[ \frac{z}{z^2 + 1} = \frac{1}{2 i} \left( \frac{z}{z-i} - \frac{z}{z+1} \right)\]

So, 
\begin{align*}
  \int_{\partial B_4(0)} \frac{z}{z^2 + 1} dz &= \frac{1}{2i} \left( \int \frac{z}{z - i} dz - \int \frac{z}{z + i} dz \right) \\
  &= 2\pi i
\end{align*}


\vskip 0.5cm
\textbf{Method C: U-substitution}

Carry out the substitution $u = z^2 + 1$. Then, $du = 2z dz$. This gives us 
\begin{align*}
  \int_{\partial B_4(0)} \frac{z}{z^2 + 1} dz &=^{?} \int_{B_{16}(i)} \frac{1}{2} \frac{du}{u} = \frac{1}{2} (2\pi i)
\end{align*}

Why are we off by a factor of $2$? 
\begin{dottedbox}
  We changed the contour of integration as needed, but \emph{not} the "rate" at which we traverse the contour.
  [Give more careful explanation with graph]
\end{dottedbox}

So, really, the expression after $u-$ substitution should be 
\begin{align*}
  \int_{\partial B_4(0)} \frac{z}{z^2 + 1} dz &= 2 \times \int_{B_{16}(i)} \frac{1}{2} \frac{du}{u} = 2 \times \frac{1}{2} (2\pi i) \\
  &= 2\pi i
\end{align*}
and this is consistent with our previous methods.

\end{document}
