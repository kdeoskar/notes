\documentclass{article}

% Language setting
% Replace `english' with e.g. `spanish' to change the document language
\usepackage[english]{babel}

% Set page size and margins
% Replace `letterpaper' with`a4paper' for UK/EU standard size
\usepackage[letterpaper,top=2cm,bottom=2cm,left=3cm,right=3cm,marginparwidth=1.75cm]{geometry}

% Useful packages
\usepackage{amsmath}
\usepackage{amssymb}
\usepackage{graphicx}
\usepackage{enumitem}
\usepackage[colorlinks=true, allcolors=blue]{hyperref}

\usepackage{hyperref}
\hypersetup{
    colorlinks=true,
    linkcolor=blue,
    filecolor=magenta,      
    urlcolor=cyan,
    pdftitle={Overleaf Example},
    pdfpagemode=FullScreen,
    }

\urlstyle{same}

\usepackage{tikz-cd}

%%%%%%%%%%% Box pacakges and definitions %%%%%%%%%%%%%%
\usepackage[most]{tcolorbox}
\usepackage{xcolor}

% Define the colors
\definecolor{boxheader}{RGB}{0, 51, 102}  % Dark blue
\definecolor{boxfill}{RGB}{173, 216, 230}  % Light blue

% Define the tcolorbox environment
\newtcolorbox{mathdefinitionbox}[2][]{%
    colback=boxfill,   % Background color
    colframe=boxheader, % Border color
    fonttitle=\bfseries, % Bold title
    coltitle=white,     % Title text color
    title={#2},         % Title text
    enhanced,           % Enable advanced features
    attach boxed title to top left={yshift=-\tcboxedtitleheight/2}, % Center title
    boxrule=0.5mm,      % Border width
    sharp corners,      % Sharp corners for the box
    #1                  % Additional options
}
%%%%%%%%%%%%%%%%%%%%%%%%%

\newtcolorbox{dottedbox}[1][]{%
    colback=white,    % Background color
    colframe=white,    % Border color (to be overridden by dashrule)
    sharp corners,     % Sharp corners for the box
    boxrule=0pt,       % No actual border, as it will be drawn with dashrule
    boxsep=5pt,        % Padding inside the box
    enhanced,          % Enable advanced features
    overlay={\draw[dashed, thin, black, dash pattern=on \pgflinewidth off \pgflinewidth, line cap=rect] (frame.south west) rectangle (frame.north east);}, % Dotted line
    #1                 % Additional options
}

\usepackage{biblatex}
\addbibresource{sample.bib}


%%%%%%%%%%% New Commands %%%%%%%%%%%%%%
\newcommand*{\T}{\mathcal T}
\newcommand*{\cl}{\text cl}


\newcommand{\ket}[1]{|#1 \rangle}
\newcommand{\bra}[1]{\langle #1|}
\newcommand{\inner}[2]{\langle #1 | #2 \rangle}
\newcommand{\R}{\mathbb{R}}
\newcommand{\C}{\mathbb{C}}
\newcommand{\V}{\mathbb{V}}
\newcommand{\Hilbert}{\mathcal{H}}
\newcommand{\oper}{\hat{\Omega}}
\newcommand{\lam}{\hat{\Lambda}}
\newcommand{\defn}{\underline{\textbf{Def:}}}


\newcommand{\bigslant}[2]{{\raisebox{.2em}{$#1$}\left/\raisebox{-.2em}{$#2$}\right.}}
\newcommand{\restr}[2]{{% we make the whole thing an ordinary symbol
  \left.\kern-\nulldelimiterspace % automatically resize the bar with \right
  #1 % the function
  \vphantom{\big|} % pretend it's a little taller at normal size
  \right|_{#2} % this is the delimiter
  }}
%%%%%%%%%%%%%%%%%%%%%%%%%%%%%%%%%%%%%%%


\tcbset{theostyle/.style={
    enhanced,
    sharp corners,
    attach boxed title to top left={
      xshift=-1mm,
      yshift=-4mm,
      yshifttext=-1mm
    },
    top=1.5ex,
    colback=white,
    colframe=blue!75!black,
    fonttitle=\bfseries,
    boxed title style={
      sharp corners,
    size=small,
    colback=blue!75!black,
    colframe=blue!75!black,
  } 
}}

\newtcbtheorem[number within=section]{Theorem}{Theorem}{%
  theostyle
}{thm}

\newtcbtheorem[number within=section]{Definition}{Definition}{%
  theostyle
}{def}



\title{Math H185 Lecture 3}
\author{Keshav Balwant Deoskar}

\begin{document}
\maketitle

% \vskip 0.5cm
These are notes taken from lectures on Complex Analysis delivered by Professor Tony Feng for UC Berekley's Math H185 class in the Sprng 2024 semester.
% \pagebreak 

\tableofcontents

\pagebreak

%%%%%%%%%%%%%%%%%%%%%%%%%%%%%%%%%%%%%%%%%%%%%%%%%%%%%%%%%%%%%%%%%%
\section{January 24 - Power Series}
%%%%%%%%%%%%%%%%%%%%%%%%%%%%%%%%%%%%%%%%%%%%%%%%%%%%%%%%%%%%%%%%%%

\vskip 1cm
\begin{mathdefinitionbox}{Power Series}
  A \textbf{Power Series} is a formal expression
  \[ \sum_{n \geq 0} a_n z^{n}, z_n \in \C \]
  for which operations are defined as:

  \begin{itemize}
    \item \underline{Addition:} 
    \[  \sum_{n \geq 0} a_n z^{n} \]
    \item \underline{multiplication}
  \end{itemize}
\end{mathdefinitionbox}

\begin{itemize}
  \item "Formal" here means we temporarily ignore whether it makes sense to plug in complex numbers into such formulae.
\end{itemize}

\vskip 1cm
\subsection{Convergence}
Defining these formal expressions is cool, but when does a power seres actually define a function? It does so when the seres \textbf{converges}.

\vskip 0.5cm
\underline{Example:} Geometric series:

Let $a \in \C$, then for the geometric series we have $a_n = a^n$. So, 
\[ \sum_{n \geq 0} a^n z^n = 1 + az +  a^2z^2 + \cdots \]

converges if $S_n = \sum_{n \geq 0}^{N-1} a^n z^n$ has a limit.

\vskip 0.5cm
By the same argument as in the reals, we can get a closed form expression for $S_N$:
\[ S_N = \frac{1 - (az)^n}{1-(az)} \]

To deal with convergence, we break into cases and take the limit.
\begin{itemize}
  \item \textbf{$|az| < 1$ case:}
  
  \[ |az| < 1 \implies |az|^{N} \rightarrow^{N \rightarrow \infty} 0 \]
  So, 
  \[ \lim_{N \rightarrow \infty} S_N = \frac{1}{1 - az} \]

  \vskip 0.5cm
  \item \textbf{$|az| > 1$ case:}
  \[ |az| > 1 \implies |az|^{N} \rightarrow^{N \rightarrow \infty} \infty \]
  so 
  \[ \lim_{N \rightarrow \infty} S_N \text{ diverges} \]
  
  \vskip 0.5cm
  \item \textbf{$|az| = 1$ case:}
  
  Diverges if $|az| \neq 1$ and if $|az| = 1$. In this case, both diverge, but in general there may be more complicated behavior.
\end{itemize}

\textbf{Conclusion:} The geometric series converges (absolutely) for when $|z| < |a|$.

\vskip 0.5cm
\begin{dottedbox}
  Recall that 
  \[ \sum_{n \geq 0} z_n, z_n \in \C \] coverges \underline{absolutely} if 
  \[ \sum_{n \geq 0} |z_n| \] converges. 
\end{dottedbox}

\vskip 0.5cm
So, we notice that the series converges for any $z$ such that $|z| < 1/|a|$. This region is just an open disk of radius $|a|$. In general, power series has radii of convergence.

\begin{mathdefinitionbox}{Radius of Convergence}
  \vskip 0.5cm
  \underline{\textbf{Def: }} A complex Power Series
  \[ \sum_{n \geq 0} a_n z^n \]
  has \underline{Radius of Convergence}
  \[ r = \left( \lim_{n \rightarrow \infty} \sup |a_n|^{1/n} \right)^{-1} \in \mathbb{R} \]
\end{mathdefinitionbox}

\vskip 0.5cm
\underline{\textbf{Example:}} For $a_n = a^n$, we have
\[ r = \left( \lim_{n \rightarrow \infty} \sup |a_n|^{1/n} \right)^{-1} = \frac{1}{|a|}  \] which matches with the result obtained earlier.

\begin{dottedbox}
  \underline{\textbf{Theorem:}} 
  \begin{enumerate}
    \item If $|z| < r$, then $f(z)$ converges absolutely.
    \item If $|z| > r$, then it diverges.
    \item At $|z| = r$, more care is needed.
  \end{enumerate}

  \vskip 0.5cm
  \underline{\textbf{Proof Sketch:}} 
  \begin{enumerate}
    \item Consider $z$ such that $|z| < (1 - \epsilon)r$ for some $\epsilon > 0$.
    \begin{align*}
      \implies |a_n z^n| &< |a_n|(1-\epsilon)^n r^n \\
      &\leq |a_n| (1-\epsilon)^n \left(\frac{1}{|a_n|^{1/n}}\right)^n \;\;\text{ Assume $a_n \neq 0$}\\
      &\leq (1-\epsilon)^n \;\;\text{ (If $a_n = 0$, this  inequality is true trivially)}
    \end{align*}
    \[ \implies \text{ Convergence by Geometric series} \]
    Term by term, the series is smaller than the geometric series (which converges), thus it also converges(Dominated Convergence Theorem).

    \vskip 0.5cm
    \item If $|z| > r$, then $|z| > r/(1 - \epsilon)$ for some $\epsilon > 0$ while 
    \[ |a_n|^{1/n} > \left( \lim_{k \rightarrow \infty} \sup |a_k|^{1/k} \right) (1 - \epsilon) \] for infinitely many $n$.

    \begin{align*}
      \implies &|a_n z^n| > \frac{1}{r^n}(1-\epsilon)^n \cdot \frac{r^n}{(1-\epsilon)^n} > 1 \;\;\text{ for inf. many $n$} \\
      \implies &\text{the sum diverges}
    \end{align*}
  \end{enumerate}

\end{dottedbox}

\vskip 0.5cm
\underline{\textbf{Example:}} Consider a polynomial 
\[ f(z) = \sum_{n = 0}^{N} a_n z^n \]

\begin{align*}
  \implies &\lim_{n \rightarrow \infty} \sup |a_n|^{1/n} = 0 \\
  \implies & r = \infty
\end{align*}

\vskip 0.5cm
\underline{\textbf{Example:}} Consider the exponential function
\[ \exp(z) = e^{z} = \sum_{n \geq 0} \frac{z^n}{n!} \]

\begin{align*}
  \implies &r = \lim_{n \rightarrow \infty} (n!)^{1/n} =_{\text{claim}} \infty
\end{align*}

\vskip 0.5cm
\underline{Proof of claim:} We wll show that for any $b$, $\frac{1}{2}b < (n!)^{1/n}$ for all $n >> 0$.

% \begin{align*}
%   n! &= \underbrace{(1 \times 2 \times \cdots \times b)}_{\geq 1} \times \underbrace{(b+1)}_{\geq b} \times \underbrace{(b+2)}_{\geq b} \times \cdots \times \underbrace{(n)}_{\geq b} \\
%   \implies (n!)^{1/n} &> (b^{n-b}^{1/n}) = b^{1-\frac{b}{n}} \\
%   \text{Complete later}
% \end{align*}

Note that we can shift the center of the series from $0$ to some point $z_0$.

So, instead of 
\[ \sum_{n \geq 0} a_n z^n \]

We have
\[ \sum_{n \geq 0} a_n (z - z_0)^n \]

The radius of convergence is given by the same expression.

\vskip 1cm
\subsection{Differentiation of Power Series}
Knowing convergence properties of a series is also useful for the purposes of \emph{differentiation}.

\begin{dottedbox}
  \underline{\textbf{Theorem: }} Let $r$ be the radius of convergence of $f(z) = \sum_{n \geq 0} a_n (z - z_0)^n$. Then, $f$ is (complex) differentiable on $B_r(z_0)$ and 
  \[ f'(z_0) = \sum_{n \geq 0} n a_n (z - z_0)^{n-1} \]

  \vskip 0.5cm
  \underline{\textbf{Proof:}}
  The proof follows the same argument as that in real analysis. Write the proof later. One proof can be found \href{https://proofwiki.org/wiki/Derivative_of_Complex_Power_Series/Proof_1}{here}.
\end{dottedbox}

We will break the proof into two parts:
\begin{enumerate}[label=(\alph*)]
  \item First, we show that if $\sum_{n \geq 0} a_n(z - z_0)^n$ has radius of convergence $r$, then so does $\sum_{n \geq 0} n a_n(z - z_0)^{n-1}$.
  \item Second, we show that if $f(z) = \sum_{n \geq 0} a_n(z - z_0)^n$ then the derivative at $z_0$ is indeed 
  \[ f(z_0) = \sum_{n \geq 0} n a_n(z - z_0)^{n-1} \]
\end{enumerate}

\underline{\textbf{Proof:}} (Check if this is correct in Office Hours)

\begin{enumerate}[label=(\alph*)]
  \item Earlier, we proved that for any complex power series there exists a number $r$ called the radius of convergence such that the series converges for any $z \in \C$ such that $|z| < r$. We also showed that $r$ is given by 
  \[ r = \limsup_{n \rightarrow \infty} |a_n|^{1/n} \]

  So, what we want to do here is show that the radii of convergence for the two series $R, R'$ are equal.

  Now, 
  \begin{align*}
    R' &= \limsup_{n \rightarrow \infty} |n a_n|^{1/n} \\
    &= \limsup_{n \rightarrow \infty} |n|^{1/n} \cdot |a_n|^{1/n} \\
    &= \limsup_{n \rightarrow \infty} |n^{1/n}| \cdot |a_n|^{1/n} \\
    &= \limsup_{n \rightarrow \infty} |1| \cdot |a_n|^{1/n} \\
    &= R
  \end{align*}
  So the two series have equal radii of convergence $(=R)$.

  \item Next, we want to show that if 
  \[ f(z) = \sum_{n \geq 0} a_n(z - z_0)^n \]

  and 
  \[ g(z) = \sum_{n \geq 0} n a_n(z - z_0)^{n-1} \]
  Then, in fact, $f'(z) = g(z)$.
\end{enumerate}

\end{document}
