\documentclass{article}

% Language setting
% Replace `english' with e.g. `spanish' to change the document language
\usepackage[english]{babel}

% Set page size and margins
% Replace `letterpaper' with`a4paper' for UK/EU standard size
\usepackage[letterpaper,top=2cm,bottom=2cm,left=3cm,right=3cm,marginparwidth=1.75cm]{geometry}

% Useful packages
\usepackage{amsmath}
\usepackage{amssymb}
\usepackage{graphicx}
\usepackage{enumitem}
\usepackage[colorlinks=true, allcolors=blue]{hyperref}

\usepackage{hyperref}
\hypersetup{
    colorlinks=true,
    linkcolor=blue,
    filecolor=magenta,      
    urlcolor=cyan,
    pdftitle={Math 185 Homework 7},
    pdfpagemode=FullScreen,
    }

\urlstyle{same}

\usepackage{tikz-cd}

%%%%%%%%%%% Box pacakges and definitions %%%%%%%%%%%%%%
\usepackage[most]{tcolorbox}
\usepackage{xcolor}
\usepackage{csquotes}
% Define the colors
\definecolor{boxheader}{RGB}{0, 51, 102}  % Dark blue
\definecolor{boxfill}{RGB}{173, 216, 230}  % Light blue

% Define the tcolorbox environment
\newtcolorbox{mathdefinitionbox}[2][]{%
    colback=boxfill,   % Background color
    colframe=boxheader, % Border color
    fonttitle=\bfseries, % Bold title
    coltitle=white,     % Title text color
    title={#2},         % Title text
    enhanced,           % Enable advanced features
    attach boxed title to top left={yshift=-\tcboxedtitleheight/2}, % Center title
    boxrule=0.5mm,      % Border width
    sharp corners,      % Sharp corners for the box
    #1                  % Additional options
}
%%%%%%%%%%%%%%%%%%%%%%%%%

\usepackage{biblatex}
\addbibresource{sample.bib}


%%%%%%%%%%% New Commands %%%%%%%%%%%%%%
\newcommand*{\T}{\mathcal T}
\newcommand*{\cl}{\text cl}


\newcommand{\ket}[1]{|#1 \rangle}
\newcommand{\bra}[1]{\langle #1|}
\newcommand{\inner}[2]{\langle #1 | #2 \rangle}
\newcommand{\R}{\mathbb{R}}
\newcommand{\C}{\mathbb{C}}
\newcommand{\V}{\mathbb{V}}
\newcommand{\Hilbert}{\mathcal{H}}
\newcommand{\oper}{\hat{\Omega}}
\newcommand{\lam}{\hat{\Lambda}}

\newcommand{\bigslant}[2]{{\raisebox{.2em}{$#1$}\left/\raisebox{-.2em}{$#2$}\right.}}
\newcommand{\restr}[2]{{% we make the whole thing an ordinary symbol
  \left.\kern-\nulldelimiterspace % automatically resize the bar with \right
  #1 % the function
  \vphantom{\big|} % pretend it's a little taller at normal size
  \right|_{#2} % this is the delimiter
  }}
%%%%%%%%%%%%%%%%%%%%%%%%%%%%%%%%%%%%%%%


\newtcolorbox{dottedbox}[1][]{%
    colback=white,    % Background color
    colframe=white,    % Border color (to be overridden by dashrule)
    sharp corners,     % Sharp corners for the box
    boxrule=0pt,       % No actual border, as it will be drawn with dashrule
    boxsep=5pt,        % Padding inside the box
    enhanced,          % Enable advanced features
    overlay={\draw[dashed, thin, black, dash pattern=on \pgflinewidth off \pgflinewidth, line cap=rect] (frame.south west) rectangle (frame.north east);}, % Dotted line
    #1                 % Additional options
}

\tcbset{theostyle/.style={
    enhanced,
    sharp corners,
    attach boxed title to top left={
      xshift=-1mm,
      yshift=-4mm,
      yshifttext=-1mm
    },
    top=1.5ex,
    colback=white,
    colframe=blue!75!black,
    fonttitle=\bfseries,
    boxed title style={
      sharp corners,
    size=small,
    colback=blue!75!black,
    colframe=blue!75!black,
  } 
}}

\newtcbtheorem[number within=section]{Theorem}{Theorem}{%
  theostyle
}{thm}

\newtcbtheorem[number within=section]{Definition}{Definition}{%
  theostyle
}{def}



\title{Math H185 Homework 7}
\author{Keshav Balwant Deoskar}

\begin{document}
\maketitle


%%%%%%%%%%%%%%%%%%%%%%%%%%%%%%%%%%%%%%%%%%%%%%%%%%%%%%%%%%%%%%%%
\begin{mathdefinitionbox}{Question 1}
\vskip 0.25cm
Let $\{f_n\}$ be a sequence of holomorphic functions on an open subset $U \subseteq \C$ that converges uniformly to a function $f$ on every compact subset of $U$. Show that the sequence $\{f_n'\}$ converges uniformly to $f'$ on every compact subset of $U$. Then argue that the formula
\[ \zeta(s) = \sum_{n = 1}^{\infty} \frac{1}{n^s}  \]
defines a holomorphic function in $s$ for $\text{Re}(s) > 1$.
\end{mathdefinitionbox}
%%%%%%%%%%%%%%%%%%%%%%%%%%%%%%%%%%%%%%%%%%%%%%%%%%%%%%%%%%%%%%%%%

\vskip 0.5cm
\underline{\textbf{Proof:}}

\vskip 0.5cm
We say a sequence of functions $\{f_n\}$ converges to function $f$ on a subset $\Omega \subseteq \C$ if for every $\epsilon > 0$ there is some $N > 0$ so that whenever $z \in \Omega$ and $n > N$, we have 
\[ \left| f(z) - f_n(z) \right| < \epsilon  \]

\vskip 0.25cm
Let's first show that if $\{f_n\}$ is a sequence of holomorphic functions converging to $f$ on every compact subset of $U$, then $f$ is also holomorphic.

\begin{dottedbox}
  Let $D$ be any disc whose closure is contained in $U$ and let $T$ be any triangle contained in $D$. Then, on $D$, $\{f_n\} \rightarrow f$. Since each $f_n$ is holomorphic, Goursat's Theorem tells us:
  \[ \int_{T} f_n(z) dz = 0 \]
  for all $n$. Now, since $\{f_n\} \rightarrow f$ in the closure of $D$, $f$ is continuous and we have 
  \[ \int_{T} f_n(z) dz = \int_{T} f(z) dz \]

  As a result,
  \[ \int_{T} f(z) dz = 0 \]
  Then, Morera's Theorem tells us that $f$ is holomorphic on $D$. Since this holds for any $D$ whose closure is contained in $U$, $f$ is holomorphic on all of $U$.
\end{dottedbox}

\vskip 0.5cm
Now, since the sequence $\{f_n\} \rightarrow f$ uniformly on any disc whose closure is contained in $U$, we can assume WLOG that the sequence converges uniformly on all of $U$. Now, given $\delta > 0$ let 
\[ \Omega_{\delta} = {z \in U\;:\; \overline{D_{\delta}(z)} \subseteq U}  \]

be the set of points which are atleast a distance $\delta$ away from the boundary of $U$. To prove the theorem, it suffices to show that $\{f_n'\}$ converges uniformly to $f'$ on each $\Omega_{\delta}$. We do so using the following inequality (for holomorphic $F$):
\[ \sup_{z \in \Omega_{\delta}} \left| F'(z_) \right| \leq \frac{1}{\delta} \sup_{w \in U} \left| F(w) \right| \]
with $F = f_n - f$.

\begin{dottedbox}
  \emph{Proof of Inequality:}
  \vskip 0.25cm
  For every $z \in \Omega_{\delta}$, the closure of $D_{\delta}(z)$ is contained in $U$ and Cauchy's Integral Formula tells us 
  \[ F'(z) = \frac{1}{2\pi i} \int_{\partial D_{\delta}(z)} \frac{F(w)}{(w-z)^2} dw \]

  Hence,
  \begin{align}
    |F'(z)| &\leq \left| \frac{1}{2\pi i} \right| \int_{\partial D_{\delta}(z)} \frac{\left| F(w) \right|}{\left| w - z \right|^2} |dw| \\
    &\leq \frac{1}{2\pi} \sup_{w \in U} \left| F(w) \right| \frac{1}{\delta^2} 2\pi \delta \\
    &\leq \frac{1}{\delta} \sup_{w \in U} \left| F(w) \right|
  \end{align}
  and this holds for any $z \in \Omega_{\delta}$ so of course, 
  \[ \boxed{\sup_{z \in \Omega_{\delta}} \left| F'(z) \right| \leq \frac{1}{\delta} \sup_{w \in U} \left| F(w) \right|} \]
\end{dottedbox}

\vskip 0.25cm
Applying this with $F = f_n - f$, we have for any $z \in \Omega_{\delta}$ that 
\begin{align*}
  &\left| F'(z) \right| \leq \frac{1}{\delta} \sup_{w \in U} \left| F(w) \right| \\
  \implies&\left| f_n'(z) - f(z) \right| \leq \frac{1}{\delta} \sup_{w \in U} \left| f_n(z) - f(z) \right| \\
\end{align*}
Since $\{f_n\} \rightarrow f$ uniformly on $\Omega_{\delta}$, for any $\epsilon > 0$ there exists $N > 0$ such that 
\[ \left| f_n(z) - f(z) \right| < \epsilon \]
for $z \in \Omega_{\delta}$ and $n > N$. Thus, for any $\epsilon$, the same $N$ guarantees that 
\[ \left| f_n'(z) - f'(z) \right| < \frac{\epsilon}{\delta} = \epsilon' \]
for $z \in \Omega_{\delta}$. Thus, the sequence $\{f_n'\} \rightarrow f'$ uniformly on each $\Omega_{\delta}$. Thus, the same holds on all of $U$.

\vskip 1cm
Now, moving onto the Riemann Zeta function. We define the zeta function by 
\[ \zeta(s) = \sum_{n = 1}^{\infty} \frac{1}{n^s} \]

For $s = x + iy$, we have 
\begin{align*}
  \left| n^{-s} \right| &= \left| n^{-(x+iy)} \right| \\
  &= \left| n^{-x} \cdot n^{-iy} \right| \\
  &= \left| n^{-x} \right| \cdot \left| e^{\ln(n^{-iy})} \right| \\
  &= \left| n^{-x} \right| \cdot \underbrace{\left| e^{-iy\ln(n)} \right|}_{=1} \\ 
  &= \left| n^{-\text{Re}(s)} \right|
\end{align*}

So, 
\[ \sum_{n = 1}^{\infty} \left|\frac{1}{n^s}\right| = \sum_{n = 1}^{\infty} \frac{1}{\left| n^{\text{Re}(s)} \right|} \]

Let's denote Re$(s) = \sigma$. Now, 
\begin{align*}
  &\sum_{n = 1}^{\infty} \frac{1}{\left| n^{\sigma} \right|} \leq \sum_{n = 1}^{\infty} \int_{n}^{n+1} \frac{1}{x^\sigma} dx \\
  \implies &\sum_{n = 1}^{\infty} \frac{1}{\left| n^{\sigma} \right|} \leq \int_{1}^{\infty} \frac{1}{x^{\sigma}} dx
\end{align*}

For $\sigma > 1$, the integral converges:
\begin{align*}
  \int_{1}^{\infty} \frac{1}{x^{\sigma}} dx &= \left[\frac{1}{1-\sigma}x^{1-\sigma}\right]_{x = 1}^{x = \infty} \\
  &= \frac{1}{1-\sigma} \left[ \frac{1}{x^{\sigma - 1}} \right]_{x = 1}^{x = \infty} \\
  &= \frac{1}{1-\sigma} \left[0 - 1\right] \\
  &= \frac{1}{\sigma - 1}
\end{align*}

Therefore, for $\sigma > 1$, the sum converges absolutely. Thus, 
\[ \sum_{n = 1}^{\infty} \frac{1}{n^s} \] defines a holomorphic function on the half place Re$(s) > 1$.

\vskip 0.5cm
\hrule 
\vskip 0.5cm

%%%%%%%%%%%%%%%%%%%%%%%%%%%%%%%%%%%%%%%%%%%%%%%%%%%%%%%%%%%%%%%%
\begin{mathdefinitionbox}{Question }
\vskip 0.5cm
Prove that for Re$(s) > 1$,
\[ \zeta(s) = \frac{s}{s-1} + s\int_{1}^{\infty} \frac{\{x\}}{x^{s+1}} dx \]
where $\{x\}$ denotes the fractional part of $x$. Prove that the right hand side defines a holomorphic function in $s$ for $\{s \in \C : \text{Re}(s) > 0 \} \setminus \{1\}$.
\end{mathdefinitionbox}
%%%%%%%%%%%%%%%%%%%%%%%%%%%%%%%%%%%%%%%%%%%%%%%%%%%%%%%%%%%%%%%%%

\vskip 0.5cm
\underline{\textbf{Proof:}}

Let's look at the integral on the right hand side:
\begin{align*}
  \int_{1}^{\infty} \frac{\{x\}}{x^{s+1}} dx &= \sum_{n = 1}^{\infty} \int_{n}^{n+1} \frac{x-n}{x^{s+1}} dx \\
  &= \sum_{n = 1}^{\infty} \int_{n}^{n+1} \frac{1}{x^{s}} - \frac{n}{x^{s+1}} dx \\
  % &= \sum_{n = 1}^{\infty} \left[ \frac{-s}{x^{s+1}} + \frac{n(s+1)}{x^{s+2}} \right]_{n}^{n+1}\\
\end{align*}

If $|s| > 1$, this can be written as 
\begin{align*}
  \int_{1}^{\infty} \frac{\{x\}}{x^{s+1}} dx &= \int_{1}^{\infty} \frac{1}{x^s} dx - \sum_{n = 1}^{\infty} \int_{n}^{n+1} \frac{n}{x^{s+1}} dx \\
  &= \frac{1}{s-1} - \sum_{n = 1}^{\infty} n \cdot \left[ \frac{x^{-s}}{-s} \right]_{x = n}^{x = n+1} \\
  &= \frac{1}{s-1} + \sum_{n = 1}^{\infty} \frac{n}{s} \left[(n+1)^{-s} - n^{-s}\right] \\
  &= \frac{1}{s-1} - \frac{1}{s} \sum_{n = 1}^{\infty} n \left[n^{-s} - (n+1)^{-s}\right]
\end{align*}
Now, we can re-express the sum by combining terms cleverly:

\begin{align*}
  \sum_{n = 1}^{\infty} n \left[n^{-s} - (n+1)^{-s}\right] &= 1 \cdot \left(1^{-s} - 2^{-s}\right) + 2 \left(2^{-s} - 3^{-s}\right) + 3\cdot \left(3^{-s} + 4^{-s}\right) + \cdots \\
  &= 1^{-s} + \left(2 - 1\right) \cdot 2^{-s} + \left(4-3\right) \cdot 3^{-s} \cdots \\
  &= 1^{-s} + 2^{-s} + 3^{-s} + \cdots \\
  &= \sum_{n = 1}^{\infty} \frac{1}{n^s} \\
  &= \zeta(s)
\end{align*}

This rearranging of terms is only guaranteed to be valid when the sum converges absolutely, and that happens for Re$(s) > 1$.

\vskip 0.5cm
Thus, for Re$(s) > 1$, we have 
\begin{align*}
  &\int_{1}^{\infty} \frac{\{x\}}{x^{s+1}} dx = \frac{1}{s-1} - \frac{1}{s} \zeta(s)  \\
  \implies &\boxed{\zeta(s) = s\int_{1}^{\infty} \frac{\{x\}}{x^{s+1}} dx + \frac{s}{s-1}} \;\;\;\;\;\;\;\;(1)
\end{align*}

\vskip 0.25cm
Earlier, we proved that if we have a sequence of holomorphic functions $\{f_n\}$ which converge uniformly to a function $f$ on an open subset $U \subseteq_{\text{open}} \C$, then $f$ is holomorphic.

\vskip 0.25cm
Consider the sequence of functions $\{f_n\}_{n = 1}^{\infty}$ defined on $U = \{s \in \C \;:\; \text{Re}(s) > 0\} \setminus \{1\}$ where
\[ f_n(z) = \frac{s}{s-1} + s\int_{1}^{n} \frac{\{x\}}{x^{s+1}} dx \]

Each of these functions is holomorphic on $U$ because $s/(s-1)$ is a rational function whose denominator does not vanish in $U$ and the integral evaluates to 
\begin{align*}
  \sum_{k = 1}^{n} \int_{k}^{k+1} \frac{x - k}{x^{s+1}} dx &= \sum_{k = 1}^{n} \left[ \frac{x^{1-s}}{1-s} + \frac{k}{s} x^{-s} \right]_{k}^{k+1} \\
  &= \sum_{k = 1}^{n} \left[ \left(\frac{(k+1)^{1-s}}{1-s} + \frac{k}{s} (k+1)^{-s}\right) - \left(\frac{k^{1-s}}{1-s} + \frac{k}{s} k^{-s}\right) \right] \\
  &= \sum_{k = 1}^{n} \left[ \frac{1}{1-s} \left( (k+1)^{1-s} - k^{1-s} \right) + \frac{k}{s} \left( (k+1)^{-s} - k^{-s} \right) \right] \\
  &= \frac{1}{1-s}\sum_{k = 1}^{n}\left[ \underbrace{(k+1)^{1-s} - k^{1-s} }_{\text{telescoping}} \right] + \frac{k}{s} \sum_{k = 1}^{n} \left[ (k+1)^{-s} - k^{-s} \right] \\
  &= \frac{(n+1)^{1-s} - 1^{1-s}}{1-s} - \frac{1}{s} \left[1 \cdot \left(2^{-s} - 1^{-s}\right) + 2 \cdot \left(3^{-s} - 2^{-s}\right) + \cdots + n\left((n+1)^{-s} - n^{-s}\right)\right] \\
  &= \frac{(n+1)^{1-s} - 1}{1-s} - \frac{1}{s} \left[-1^{-s} - 2^{-s} - 3^{-s} - \cdots -n^{-s} + n(n+1)^{-s}\right] \\
  &= \frac{(n+1)^{1-s} - 1}{1-s} - \frac{1}{s}\sum_{k = 1}^{n} \frac{1}{n^s} + \frac{1}{s} n(n+1)^{-s}
\end{align*}
Each of these terms in holomorphic on $U$, so the entire integral term $s \int_1^n \frac{\{x\}}{x^{s+1}} dx$ is holomorphic on $U$.

\vskip 0.25cm
We also have uniform convergence of $\{f_n\}$ to $f(z)$ where 
\[ f(z) = s\int_{1}^{\infty} \frac{\{x\}}{x^{s+1}} dx + \frac{s}{s-1}  \]

on $U$. Therefore, the right hand side of equation (1) is holomorphic on $U = \{s \in \C \;:\; \text{Re}(s) > 0\} \setminus \{1\}$

\vskip 0.5cm
\hrule 
\vskip 0.5cm


%%%%%%%%%%%%%%%%%%%%%%%%%%%%%%%%%%%%%%%%%%%%%%%%%%%%%%%%%%%%%%%%
\begin{mathdefinitionbox}{Question 3}
\vskip 0.5cm
For $x \in \R$, let $Q_0(x) \equiv \{x\} - 1/2$. Prove by induction that there exist for all $k \geq 0$, bounded functions $Q_k(x)$ satisying all of the following conditions:
\begin{enumerate}[label=(\alph*)]
  \item $\int_{0}^{1} Q_k(x) dx = 0$
  \item $\frac{dQ_{k+1}(x)}{dx} = Q_k(x)$ for all $x \in \R \setminus \mathbb{Z}$
  \item $Q_{k}(x+1) = Q_k(x)$ for all $x \in \R$. 
\end{enumerate}
\end{mathdefinitionbox}
%%%%%%%%%%%%%%%%%%%%%%%%%%%%%%%%%%%%%%%%%%%%%%%%%%%%%%%%%%%%%%%%%

\vskip 0.5cm
\underline{\textbf{Proof:}}

\subsection*{Base Case:}

For $k = 0$, we have $Q_0 = \{x\} - 1/2$. This function is bounded as $\sup |Q_0(x)| = 1/2$. Define $Q_1(x)$ as: 
\begin{align*}
  Q_1(x) \equiv \begin{cases}
    F_1(x) + C_1, \;\;\;\; x\in [0, 1)\\
    Q_1(\{x\}), \;\;\;\; x \not\in [0, 1)
  \end{cases}
\end{align*}
where $F_1(x)$ is the anti-derivative of $Q_0(x)$ and $C_1$ is a constant chosen such that property (a) is satisfied i.e. $C_1 = - \int_{0}^{1} F_1(x) dx$.

\vskip 0.5cm
Let's verify that it satisfies the three properties:
\begin{enumerate}[label=(\alph*)]
  \item The integral over the unit interval is 
  \begin{align*}
    \int_{0}^{1} Q_0(x) dx = &\int_{0}^{1} \{x\} - \frac{1}{2} dx \\
    &= \int_{0}^{1} x - \frac{1}{2} dx \\
    &= \left[ \frac{x^2}{2} - \frac{x}{2} \right]_{0}^{\rightarrow 1} \\
    &= \left(\frac{1}{2} - \frac{1}{2} \right) - \left( 0 - 0 \right) \\
    &= 0 
  \end{align*}

  \vskip 0.5cm
  \item For $x \in \R \setminus \mathbb{Z}$ such that $x \in [0, 1)$ we have $Q_1(x) = F_1(x) + C_1$ so 
  \begin{align*}
    \frac{dQ_1(x)}{dx} &= \frac{dF_1(x)}{dx} + 0 \\
    &= Q_0(x)
  \end{align*}

  and for $x \in \R \setminus \mathbb{Z}$ such that $x \not\in [0, 1)$ we have $Q_1(x) = Q_1(\{x\})$, and so 
  \begin{align*}
    \frac{dQ_1(x)}{dx} &= \frac{dF_1(\{x\})}{dx} + 0 \\
    &= Q_0(\{x\}) \\
    &= \{x\} - \frac{1}{2} \\
    &= Q_0(x)
  \end{align*}

  \vskip 0.5cm
  \item For $x \in \R$, we have 
  \begin{align*}
    Q_1(x+1) = Q_1(\{x + 1\}) = Q_1(\{x\}) = Q_1(x)
  \end{align*}
\end{enumerate}

\subsection*{Inductive Hypothesis and Step}
Okay, now suppose functions $Q_i(x)$ where $0 \leq i \leq n$ are defined such that properties (a), (b), and (c) are satisfied. Let's define $Q_{n+1}(x)$ as 

\[ Q_{n+1}(x) = \begin{cases}
  F_{n+1} + C_{n+1}, \;\;\;\; x \in [0,1) \\
  Q_{n+1}(\{x\}),\;\;\;\; x \not\in [0,1)
\end{cases} \]

where $F_{n+1}$ is the antiderivative of $Q_{n}$ and $C_{n+1}$ is a constant chosen so that property (a) is satisfied i.e. $C_{n+1} = -\int_{0}^{1} F_{n+1}(x) dx$.

\vskip 0.5cm
Let's verify that each of the properties hold.

\begin{enumerate}[label=(\alph*)]
  \item Holds by contruction of $Q_{n+1}(x)$
  
  \vskip 0.5cm
  \item Again, due to the construction,
  For $x \in \R \setminus \mathbb{Z}$ such that $x \in [0, 1)$ we have $Q_1(x) = F_1(x) + C_1$ so 
  \begin{align*}
    \frac{dQ_{n+1}(x)}{dx} &= \frac{dF_{n+1}(x)}{dx} + 0 \\
    &= Q_{n}(x)
  \end{align*}

  and for $x \in \R \setminus \mathbb{Z}$ such that $x \not\in [0, 1)$ we have $Q_1(x) = Q_1(\{x\})$, and so 
  \begin{align*}
    \frac{dQ_1={n+1}(x)}{dx} &= \frac{dF_{n+1}(\{x\})}{dx} + 0 \\
    &= Q_{n}(\{x\}) \\
    &= Q_n(x)
  \end{align*}

  \vskip 0.5cm
  \item For $x \in \R$, we have 
  \begin{align*}
    Q_{n+1}(x+1) = Q_{n+1}(\{x + 1\}) = Q_{n}(\{x\}) = Q_{n}(x)
  \end{align*}
\end{enumerate}

\vskip 0.5cm
\hrule 
\vskip 0.5cm


%%%%%%%%%%%%%%%%%%%%%%%%%%%%%%%%%%%%%%%%%%%%%%%%%%%%%%%%%%%%%%%%
\begin{mathdefinitionbox}{Question 4}
\vskip 0.5cm
With $Q_k(x)$ as in the previous problem, prove the formula 
\[ \zeta(s) = \frac{s}{s-1} - \frac{1}{2} - s \int_{1}^{\infty} \left( \frac{d^k}{dx^k} Q_k(x) \right) x^{-s-1} dx \]
for Re$(s) >> 1$, and deduce that there is an analytic continuation of $\zeta(s)$ to $\{s \in \C \;:\; \text{Re}(s) > -k \} \setminus {1}$
\end{mathdefinitionbox}
%%%%%%%%%%%%%%%%%%%%%%%%%%%%%%%%%%%%%%%%%%%%%%%%%%%%%%%%%%%%%%%%%

\vskip 0.5cm
\underline{\textbf{Proof:}}

We found in Question 2 that for Re$(s) > 1$,
\[ \zeta(s) = \frac{s}{s-1} - s\int_1^{\infty} \frac{\{x\}}{x^{s+1}} dx \]

We can rewrite this in terms of $Q_0(x) = \{x\} - 1/2$ and then simplify to get 
\begin{align*}
  \zeta(s) &= \frac{s}{s-1} - s\int_{1}^{\infty} \frac{Q_0(x) + 1/2}{x^{s+1}} dx \\
  &= \frac{s}{s-1} - s \int_1^{\infty} \frac{1/2}{x^{s+1}} dx - s\int_{1}^{\infty} \frac{Q_0}{x^{s+1}} dx \\
  &= \frac{s}{s-1} - \frac{s}{2} \int_1^{\infty} x^{-s-1} dx - s\int_{1}^{\infty} Q_0(x) \cdot x^{-s-1} dx \\
  &= \frac{s}{s-1} - \frac{s}{2} \left[ \frac{x^{-s}}{-s} \right]_{1}^{\infty} - s \int_{1}^{\infty} Q_0(x) \cdot x^{-s-1} dx \\
  &= \frac{s}{s-1} - \frac{s}{2} \left[0 - \frac{-1}{s} \right] - s \int_{1}^{\infty} Q_0(x) \cdot x^{-s-1} dx \\
  &= \frac{s}{s-1} - \frac{1}{2} - s \int_{1}^{\infty} Q_0(x) \cdot x^{-s-1} dx \\
\end{align*}

Also, recall that in Question 3, that we found for $x \in \R \setminus \mathbb{Z}$
\[ \frac{dQ_{k+1}(x)}{dx} = Q_{k}(x) \]

This relation only fails at the integers, which are a set of measure zero, so they don't contribute to the integral.

\vskip 0.25cm
Therefore, we can write 
\[ \int_{1}^{\infty} Q_0(x) \cdot x^{-s-1} dx = \int_{1}^{\infty} \left( \frac{d^k Q_{k}(x)}{dx^k} \right)\cdot x^{-s-1} dx \]

\vskip 0.5cm
Therefore, 
\[ \boxed{ \zeta(s) = \frac{s}{s-1} - \frac{1}{2} - s\int_{1}^{\infty} \left( \frac{d^k Q_{k}(x)}{dx^k} \right)\cdot x^{-s-1} dx} \]

\vskip 0.5cm
Now, the terms other than the integral are all holomorphic on $\C \setminus \{1\}$, so let's think about where the integral converges absolutely and is holomorphic.

For now, instead of $Q_k$, let's think about $Q_0$. Using Integration by Parts, we have 
\begin{align*}
  \int_1^{\infty} \frac{Q_0(x)}{x^{s+1}} dx &= \left[ \frac{Q_1(x)}{x^{s+1}} \right]_{1}^{\infty} - \int_1^{\infty} \frac{Q_1(x)}{x^{s+2}} dx \\
  &= Q_1(1) - \int_{1}^{\infty} \frac{Q_1(x)}{x^{s+2}} dx
\end{align*}

The first term is just a constant and is holomorphic everywhere. The integral on the other hand is absolutely convergent for Re$(s) + 2 > 1 \implies \text{Re}(s) > -1$.

\vskip 0.5cm
If further apply Integration by Parts on the integral with $Q_1$ in it, we will get another constant term and an integral which converges absolutely for Re$(s) > -2$.

\vskip 0.5cm
We can do this $k$ times until we get the integral with $\frac{Q_k(x)}{x^{s+k}}$, so the zeta function $\zeta(s)$ under this continuation is equal to a bunch of constants + an integral which is absolutely convergent on $\{s \in \C \;:\; \text{Re}(s) > -k \} \setminus {1}$. Therefore, we find that this continuation of $\zeta(s)$ is holomorphic on $\{s \in \C \;:\; \text{Re}(s) > -k \} \setminus {1}$.

\vskip 0.5cm
\hrule 
\vskip 0.5cm






%%%%%%%%%%%%%%%%%%%%%%%%%%%%%%%%%%%%%%%%%%%%%%%%%%%%%%%%%%%%%%%%%
% \begin{mathdefinitionbox}{Question }
% \vskip 0.5cm

% \end{mathdefinitionbox}
% %%%%%%%%%%%%%%%%%%%%%%%%%%%%%%%%%%%%%%%%%%%%%%%%%%%%%%%%%%%%%%%%%

% \vskip 0.5cm
% \underline{\textbf{Proof:}}

% \vskip 0.5cm
% \hrule 
% \vskip 0.5cm



\end{document}
