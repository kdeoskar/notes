\documentclass[11pt]{article}

% basic packages
\usepackage[margin=1in]{geometry}
\usepackage[pdftex]{graphicx}
\usepackage{amsmath,amssymb,amsthm}
\usepackage{custom}
\usepackage{lipsum}

\usepackage{xcolor}
\usepackage{tikz}

\usepackage[most]{tcolorbox}

% page formatting
\usepackage{fancyhdr}
\pagestyle{fancy}

\renewcommand{\sectionmark}[1]{\markright{\textsf{\arabic{section}. #1}}}
\renewcommand{\subsectionmark}[1]{}
\lhead{\textbf{\thepage} \ \ \nouppercase{\rightmark}}
\chead{}
\rhead{}
\lfoot{}
\cfoot{}
\rfoot{}
\setlength{\headheight}{14pt}

\linespread{1.03} % give a little extra room
\setlength{\parindent}{0.2in} % reduce paragraph indent a bit
\setcounter{secnumdepth}{2} % no numbered subsubsections
\setcounter{tocdepth}{2} % no subsubsections in ToC

%%%%%%%%%%%%%%%%%%%%%%%%%%%%%%%%%%%%%%%%%%%%%%%%%%%%%%
%%%%%%%%%%% New Commands %%%%%%%%%%%%%%
\newcommand*{\T}{\mathcal T}
\newcommand*{\cl}{\text cl}


% \newcommand{\ket}[1]{|#1 \rangle}
% \newcommand{\bra}[1]{\langle #1|}
\newcommand{\inner}[2]{\langle #1 | #2 \rangle}
\newcommand{\R}{\mathbb{R}}
\newcommand{\C}{\mathbb{C}}
\newcommand{\bS}{\mathbb{S}}
\newcommand{\V}{\mathbb{V}}
\newcommand{\Hilbert}{\mathcal{H}}
\newcommand{\oper}{\hat{\Omega}}
\newcommand{\lam}{\hat{\Lambda}}

\newcommand{\bigslant}[2]{{\raisebox{.2em}{$#1$}\left/\raisebox{-.2em}{$#2$}\right.}}
\newcommand{\restr}[2]{{% we make the whole thing an ordinary symbol
  \left.\kern-\nulldelimiterspace % automatically resize the bar with \right
  #1 % the function
  \vphantom{\big|} % pretend it's a little taller at normal size
  \right|_{#2} % this is the delimiter
  }}
%%%%%%%%%%%%%%%%%%%%%%%%%%%%%%%%%%%%%%%
%%%%%%%%%%%%%%%%%%%%%%%%%%%%%%%%%%%%%%%%%%%%%%%%%%%%%%


%%%%%%%%%%%%%%%%%%%%%%%%%%%%%%%%%%%%%%%%%%%%%%%%%%%%%%%%%%%%%%%%%
% CUSTOM BOXES AND STUFF
\newtcolorbox{redbox}{colback=red!5!white,colframe=red!75!black}
\newtcolorbox{bluebox}{colback=blue!5!white,colframe=blue!75!black}
%%%%%%%%%%%%%%%%%%%%%%%%%%%%%%%%%%%%%%%%%%%%%%%%%%%%%%%%%%%%%%%%%


\begin{document}

% make title page
\thispagestyle{empty}
\bigskip \
\vspace{0.1cm}

\begin{center}
{\fontsize{22}{22} \selectfont Spring '24}
\vskip 16pt
{\fontsize{36}{36} \selectfont \bf \sffamily Math 185 Final Review}
\vskip 24pt
{\fontsize{18}{18} \selectfont \rmfamily Keshav Balwant Deoskar} 
\vskip 6pt
{\fontsize{14}{14} \selectfont \ttfamily kdeoskar@berkeley.edu} 
\vskip 24pt
\end{center}

% {\parindent0pt \baselineskip=15.5pt \lipsum[1-4]} 

% make table of contents
\newpage
\microtoc
\newpage

% main content
\section{January}

\section{February}

\section{March}

\section{April}

\subsection{April 1: Riemann Sphere}
\vskip 0.5cm
\begin{itemize}
    % \item Helps us study \emph{almost} holomorphic functions. Namely, meromorphic functions.
    \item We want to talk about functions that are holomorphic/meromorphic/have a pole "at" $\infty$. We do this by extending the complex plane and functions on it.
\end{itemize}

\vskip 0.5cm
\subsubsection{One point compactification}

\subsubsection*{Real case}
\begin{redbox}
    Consider the map  $\R \setminus \{0\} \rightarrow \R$ defined as 
    \[ f(x) = \frac{1}{x} \]
    This function is \emph{almost} smooth. The only issue is that in some sense we have $f(0) = \infty \not\in \R$. But what if we just add an "$\infty$" point to $\R$?

    \vskip 0.5cm
    Denote this space $\hat{\R}$ as the one-point compactification of $\R$ is $\R \cup \{\infty\} \cong \mathbb{S}^1$. Now, $f$ can be continued to a map from $\R$ to $\hat{R}$ with the same formula.

    \vskip 0.5cm
    We can further define 
    \[ \frac{1}{\infty} = 0 \] to continue the function to a smooth map from $\hat{\R} \rightarrow \hat{\R}$!

    \vskip 0.5cm
    $\hat{\R}$ is called the \emph{\textbf{extended complex plane}}.
\end{redbox}

\vskip 0.5cm
\subsubsection*{Complex case}

Similarly, we can define the \emph{\textbf{extended complex plane}} $\hat{\C}$ by adding a point at infinity. $\hat{C} = \C \cup \{\infty\}$. Instead of $\bS^1$, this space is topologically equivalent to $\bS^2$ i.e. a sphere.

\vskip 0.5cm
INSERT FIGURE.

\vskip 0.5cm
The function $z \mapsto \frac{1}{z}$, $\C \rightarrow \C$ gets extended continuously to a function with the same formula defined on $\hat{\C} \rightarrow \hat{\C}$ under the convention 
\begin{align*}
    &z \mapsto \frac{1}{z} \\
    &0 \mapsto \infty \\
    &\infty \mapsto 0\\
\end{align*}

\vskip 0.25cm
Under this map, a neighborhood of the $0$ gets mapped into a neighborhood of $\infty$. More precisely, this map takes a chart $\C \subseteq U \ni 0$ and maps it to a chart $0 \in U \subseteq \C$ to a chart $\infty \in V \subseteq \hat{\C}$ and the two charts are equivalent in that we can go back and forth.

\vskip 0.5cm
\begin{bluebox}
    \textbf{Def:} We call a function $f \text{ : } U \subseteq \C \rightarrow V \subseteq$ a \emph{\textbf{biholomorphism}} if 
    \begin{itemize}
        \item bijective
        \item holomorphic
        \item $f^{-1}$ is also holomorphic
    \end{itemize}
\end{bluebox}

This is the natural notion of isomorphism in complex geometry. As such, a number of properties follow:

\begin{redbox}
    \begin{itemize}
        \item Given a chain
        \[ U \xrightarrow[]{f} V \xrightarrow[]{g} \C \] with $f$ holomorphic, we have $g$ holomorphic if and only if $g \circ f$ is holomorphic.
        \item $g$ has a removable singularity/pole/etc. at $z_0 \in V$ if and only if $f$ has the same at $f^{-1}(z_0)$.
        \item $\mathrm{Res}_{z_0}(g) = \mathrm{Res}_{f^{-1}(z_0)}(f)$
    \end{itemize}
\end{redbox}

\vskip 0.5cm
We want to force $\mathrm{inv}(z), z \mapsto 1/z$ to be a biholomorphism. [write more later]

\vskip 0.5cm
\subsubsection{Meromorphic functions}

\vskip 0.5cm
\begin{bluebox}
    \textbf{Def:} If $f(z)$ is holomorphic on $U \setminus \{z_0\}$, then $f(z)$ is said to be \emph{\textbf{meromorphic}} if and only if it extends to holomorphic $\hat{f} \text{ : } U \rightarrow \hat{\C}$. 

    \vskip 0.5cm
    The above is equivalent to saying $f$ meromorphic if and only if 
    \[ \lim_{z \rightarrow z_0} f(z) = \infty \]
\end{bluebox}

% \subsubsection*{Definin}

\begin{bluebox}
    A group homomorphism 
    \[ \rho  G \rightarrow \mathrm{Aut}(V) \]
    is called a \textbf{representation} of $G$ in $V$.
\end{bluebox}

\begin{bluebox}
    given a chain
    \[ U \xrightarrow[]{f} V \xrightarrow[]{g} W \xrightarrow[]{h} X  \xrightarrow[]{i} Y  \]
\end{bluebox}

\pagebreak
\subsection{April 3: }


\end{document}