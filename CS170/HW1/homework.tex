\documentclass[11pt]{article}
\usepackage{cs170}

\def\title{Homework 1}
\def\duedate{Friday 9/6/204, at 10:00 pm (grace period until 11:59pm)}

\begin{document}
\maketitle
Due \textbf{\duedate}


\question{Study Group}
List the names and SIDs of the members in your study group.
If you have no collaborators, explicitly write ``none''.


\question{Course Policies}

\begin{subparts}
    \subpart What dates and times are the exams for CS170 this semester? Are there planned alternate exams?

    \begin{solution}
        The dates and times for the exams are:
        \begin{itemize}
            \item Midterm 1: October 2, 2024 7:00-9:00 pm
            \item Midterm 2: November 5, 2024 7:00-9:00 pm
            \item Final 1: December 18, 2024 3:00-6:00 pm
        \end{itemize}
    \end{solution}

    \subpart Homework is due Fridays at 10:00pm, with a late deadline at 11:59pm. At what time do we recommend you have your homework finished?

    \begin{solution}
        Before 10:00 pm on Fridays.
    \end{solution}

    \subpart We provide 2 homework drops for cases of emergency or technical issues that may arise due to homework submission. 
    If you miss the Gradescope late deadline (even by a few minutes) and need to submit the homework, what should you do?

    \begin{solution}
        Treat it as using one of the two homework drops since absolutely no submissions are taken after 11:59pm.    
    \end{solution}
    
    \subpart What is the primary source of communication for CS170 to reach students? 
    We will send out all important deadlines through this medium, and you are responsible for checking your emails and reading each announcement fully.

    \begin{solution}
        EdStem.
    \end{solution}

    \subpart Please read all of the following: 

    \begin{enumerate}[(i)]
        \item \textbf{Syllabus and Policies:} \url{https://cs170.org/policies/}
        \item \textbf{Homework Guidelines:} \url{https://cs170.org/resources/homework-guidelines/}
        \item \textbf{Regrade Etiquette:} \url{https://cs170.org/resources/regrade-etiquette/}
        \item \textbf{Forum Etiquette:} \url{https://cs170.org/resources/ed-etiquette/}
    \end{enumerate}

    Once you have read them, copy and sign the following sentence on your homework submission.

    ``I have read and understood the course syllabus and policies.''

    \begin{solution}
        I have read and understoo the course syllabus and policies.\\
        - Keshav Balwant Deoskar, 09/01/2024.
    \end{solution}

\end{subparts}


\newpage
\question{Understanding Academic Integrity}

Before you answer any of the following questions, make sure you have read over the syllabus and course policies (\url{https://cs170.org/policies/}) carefully.
For each statement below, write \textit{OK} if it is allowed by the course policies and \textit{Not OK} otherwise.

\begin{subparts}
    \subpart You ask a friend who took CS 170 previously for their homework solutions, 
    some of which overlap with this semester's problem sets. 
    You look at their solutions, then later write them down in your own words.

    \begin{solution}
        Not OK.
    \end{solution}

    \subpart You had 5 midterms on the same day and are behind on your homework. 
    You decide to ask your classmate, who's already done the homework, for help. 
    They tell you how to do the first three problems.

    \begin{solution}
        Not OK.
    \end{solution}

    \subpart You're a serial procrastinator and started working on the homework at 8:00 PM on Monday, and out of desperation searched up a homework problem online and find the exact solution. 
    You then write it in your words and cite the source.

    \begin{solution}
        Not OK.
    \end{solution}

    \subpart You were looking up Dijkstra's on the internet, and inadvertently run into a website with a problem very similar to one on your homework. 
    You read it, including the solution, and then you close the website, 
    write up your solution, and cite the website URL in your homework writeup.

    \begin{solution}
        OK.
    \end{solution}
\end{subparts}


\newpage
\question{Log Identities}

The following subparts will cover several math identities, tricks, and techniques that will be useful throughout the rest of this course.

Simplify the following expressions into a single logarithm (i.e. in the form $\log_a b$):
\begin{enumerate}[(a)]

    \item $\frac{\ln x}{\ln y}$ 

    \item $\ln x + \ln y$

    \item $\ln x - \ln y$

    \item $170 \ln x$ 

\end{enumerate}


\newpage
\question{Asymptotics Practice}
For each pair of functions $f$ and $g$, specify whether $f= O(g)$, $g = O(f)$, or both. 
No justification needed. 
\begin{enumerate}

    \item $f(n) = n^2 + 5n$, $g(n) = 1000(n+1)^2$.

    \item $f(n) = 5 n^3$, $g(n) = n^3 + (\log n)^{10}$.

    \item $f(n) = n^{100}$, $g(n) = (1.01)^n$.

    \item $f(n) = \left(\log n\right)^{10}$, $g(n) = n^{0.1}$.

    \item $f(n) = n \cdot 2^n$, $g(n) = 3^n$ for some constant $a > 1$. 

    \item Consider the factorial function: $n! = 1 \cdot 2 \cdot \hdots \cdot n$.
    $f(n) = n!$, $g(n) = n^n$.

    \item $f(n) = 1 + b + b^2 + \hdots + b^n$, $g(n) = b^n$ for arbitrary constant $b > 0$. 
    
    Does your answer change depending on the value of $b$? If so, specify the range of $b$ for which each statement holds.

\end{enumerate}

\newpage
\question{Recurrence Relations}

For each part, find the asymptotic order of growth of $T$; that is, find a function $g$ such that $T(n) = \Theta(g(n))$. Show your reasoning and \textbf{do not directly apply the Master Theorem; doing so will yield 0 credit}.

In all subparts, you may ignore any issues arising from whether a number is an integer.

\begin{subparts}
    \subpart \(T(n)=2T(n/3)+5n\)

    \subpart An algorithm $\mathcal{A}$ takes $\Theta(n^2)$ time to partition the input into $5$ sub-problems of size $n/5$ each and then recursively runs itself on $3$ of those subproblems. Describe the recurrence relation for the run-time $T(n)$ of $\mathcal{A}$ and find its asymptotic order of growth. 

    \subpart \(T(n) = T(3n/5)+T(4n/5)\) (We have $T(1) = 1$)

    \textit{Hint: first, compute a reasonable upper and lower bound for $T(n)$. Then, try to guess a $T(n)$ of the form $an^b$ and then use induction to argue that it is correct.}

\end{subparts}

\end{document}
