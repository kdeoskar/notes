\documentclass{article}

% Language setting
% Replace `english' with e.g. `spanish' to change the document language
\usepackage[english]{babel}

% Set page size and margins
% Replace `letterpaper' with`a4paper' for UK/EU standard size
\usepackage[letterpaper,top=2cm,bottom=2cm,left=3cm,right=3cm,marginparwidth=1.75cm]{geometry}

% Useful packages
\usepackage{amsmath}
\usepackage{amssymb}
\usepackage{graphicx}
\usepackage[colorlinks=true, allcolors=blue]{hyperref}
\usepackage[most]{tcolorbox}

\usepackage{hyperref}
\hypersetup{
    colorlinks=true,
    linkcolor=blue,
    filecolor=magenta,      
    urlcolor=cyan,
    pdftitle={Overleaf Example},
    pdfpagemode=FullScreen,
    }

\urlstyle{same}

\usepackage{tikz-cd}


%%%%%%%%%%% New Commands %%%%%%%%%%%%%%
\newcommand*{\T}{\mathcal T}
\newcommand*{\cl}{\text cl}
\newcommand{\bigslant}[2]{{\raisebox{.2em}{$#1$}\left/\raisebox{-.2em}{$#2$}\right.}}
\newcommand{\restr}[2]{{% we make the whole thing an ordinary symbol
  \left.\kern-\nulldelimiterspace % automatically resize the bar with \right
  #1 % the function
  \vphantom{\big|} % pretend it's a little taller at normal size
  \right|_{#2} % this is the delimiter
  }}
%%%%%%%%%%%%%%%%%%%%%%%%%%%%%%%%%%%%%%%


\tcbset{theostyle/.style={
    enhanced,
    sharp corners,
    attach boxed title to top left={
      xshift=-1mm,
      yshift=-4mm,
      yshifttext=-1mm
    },
    top=1.5ex,
    colback=white,
    colframe=blue!75!black,
    fonttitle=\bfseries,
    boxed title style={
      sharp corners,
    size=small,
    colback=blue!75!black,
    colframe=blue!75!black,
  } 
}}

\newtcbtheorem[number within=section]{Theorem}{Theorem}{%
  theostyle
}{thm}

\newtcbtheorem[number within=section]{Definition}{Definition}{%
  theostyle
}{def}



\title{Introduction to Topological Manifolds - Some solutions}
\author{Keshav Balwant Deoskar}

\begin{document}
\maketitle

\begin{abstract}
This is a collection of my personal solutions to some exercises from the book 'Introduction to Topological Manifolds' (2nd edition) by John M. Lee.

These solutions are being written up and I work through the book, however I do not write up every exercise/problem and leave many to be updated later.

This has been written solely to deepen my own understanding of the material, but please feel free to contact me with corrections, concerns, or comments at kdeoskar@berkeley.edu.
\end{abstract}

% \pagebreak 

\tableofcontents

\pagebreak

%%%%%%%%%%%%%%%%%%%%%%%%%%%%%%%%%%%%%%%%%%%%%%%%%%%%%%%%%%%%%%%%%%
\section{Chapter 2: Topological Spaces}
%%%%%%%%%%%%%%%%%%%%%%%%%%%%%%%%%%%%%%%%%%%%%%%%%%%%%%%%%%%%%%%%%%
\vskip 0.5cm

%%%%%%%%%%%%%%%%%%%%%%%%%%%%%%%%%%%%%%%%%%%%%%%%%%%%%%%%%%%%%%%%%%
\subsection{Topologies}
%%%%%%%%%%%%%%%%%%%%%%%%%%%%%%%%%%%%%%%%%%%%%%%%%%%%%%%%%%%%%%%%%%

\vskip 0.5cm

\textbf{Exercise 2.5:} Suppose $X$ is a topological space and $Y$ is an open subset of $X$. Show that the collection of all subsets of $X$ contained in $Y$ form a topology on $Y$. 

\vskip 0.5cm

\textbf{Proof:}
Let the topology on $X$ be denoted $\mathcal T_X$ and let's call the collection of open subsets of $X$ contained in $Y$ as:
\[ \mathcal T_Y = \{ U \in \mathcal T_X : U \subseteq Y \} \]

\begin{itemize}
  \item Now, clearly $\emptyset \subseteq Y$ and $\emptyset \in \mathcal T_{Y}$. Also, by definition, $Y \in \mathcal T_Y$.
  \vskip 0.25cm

  \item Let $\cup_{\alpha \in A} U_{\alpha}$ denote an arbitrary union of open sets $U_{\alpha} \in \mathcal T_X$ such that each $U_{\alpha} \subseteq Y$.
  
  Since each $U_{\alpha} \in \mathcal T_X$, the arbitrary union is also an open set in $\mathcal T_X$ i.e. \textbf{it is also an open set on X}. 
  
  To show that it is further a subset of $Y$, we note that for any $x$ in the union, $x$ must be contained by some open set $U_k \subseteq Y$, and so every element on the union is an element of $Y$. Thus, $\cup_{\alpha \in A} U_{\alpha} \subseteq Y$.
  \vskip 0.25cm

  \item Consider some \textbf{finite} intersection of sets $\cap_{\alpha \in A} U_{\alpha}$ where each $U_{\alpha} \in \mathcal T_Y$. Then, since each of these sets is also in the topology on $X$, their finite intersection is \textbf{also an open set on X}.
  
  Now, to show that the intersection is \textbf{a subset of $Y$}, we note that an element $x \in \cap_{\alpha \in A} U_{\alpha}$ must be an element of every $U_{\alpha} \subseteq Y$, so every element of the intersection is also contained in $Y$. Thus, $\cap_{\alpha \in A} U_{\alpha} \subseteq Y$.
\end{itemize}

Since all three properties are satisfied, $\mathcal T_Y$ does indeed form a topology on $Y$.
\vskip 0.5cm
\hrule
\vskip 0.5cm

\textbf{Exercise 2.6:} Let $X$ be a set, and suppose $\{\mathcal T_\alpha\}_{\alpha \in A}$ is a collection of topologies on
$X$. Show that the intersection $\mathcal T = \cap_{\alpha \in A} \mathcal T_{\alpha}$ is a topology on $X$. (The open subsets in this topology are exactly those subsets of $X$ that are open in each of the topologies $\mathcal T_{\alpha}$.)

\vskip 0.5cm

\textbf{Proof:}
\begin{itemize}
  \item Clearly, $\phi \in \mathcal T$ and since $X$ must be a member of any topology we have $X \in \T_{\alpha}$ for every $\alpha \in A$, giving us $X \in \T$.
  
  \item Consider an arbitrary union of open sets $U_i \in \T$. Then, each $U_i$ must be contained within each of the topologies $\T_{\alpha}$, and by definition of a topology, the arbitrary unions of these sets must also be in each $\T_{\alpha}$. So, the arbitrary union must also be in $\T$.
  
  \item Consider a finite union of sets $U_i \in T$. Then, each $U_i$ must be an element of every $\T_{\alpha}$ as well and, again, by the definition of a topology, the finite intersection must also be in each topology i.e. $\cap_{i \in I} U_i \in \T_{\alpha}$. Thus, the finite intersection must be in $\T$ as well. 
\end{itemize}

\vskip 0.5cm
\hrule
\vskip 0.5cm

\subsubsection{Closed subsets}

\textbf{Exercise 2.10:}  Show that a subset of a topological space is closed if and only if it contains all of its limit points.

\vskip 0.5cm

\textbf{Proof:}
Here, we're defining a subset $A \subset X$ to be closed if and only if $X \setminus A$ is open in $X$, and a point $p \in X$ to be a limit point of $A$ if every neighborhood of $p$ contains a point $x \in A$ such that $x \neq p$.

\vskip 0.5cm

\underline{"Forwards Direction":} Let $A$ be closed in $X$ and suppose, for contradiction, that $A$ does not contain one of its limit points $p \in X$. Then, $p \in X \setminus A$ which is open since $A$ is closed. However since it is open, every point in $X \setminus A$ must have a neighborhood which is contained in $X \setminus A$ (by Exercise 2.9 (g)). 

\vskip 0.25cm

But by the definition of a limit point, every neighborhood of $p$ must contain some point $p \neq x \in A$. So, we have a contradiction. Thus, $A$ is closed $\implies$ $A$ contains all of its limit points.

\vskip 1cm

\underbar{"Backwards direction":} Suppose $A \subset X$ contains all of its limit points so $A' \subseteq A$ where $A'$ is the set of limit points.

\vskip 0.25cm

Now, suppose that $A$ is \emph{not} closed. Then, the complement $X \setminus A$ is \emph{not} open. This means that there is at least one point $x \in X \setminus A$ such that for \emph{every} neighborhood $U \ni x$ we have $U \not\subseteq X \setminus A$. In other words, for every neighborhood of $x$, we have $U \cap A \neq \emptyset$. 

These intersection points are then clearly in $A \setminus \{x\}$, meaning they are limit points (since their neighborhoods contain points in both $A$ and $X \setminus A$). But we assumed that $A$ already contained all of its limit points! Thus, we have arrived at a conclusion. 

\vskip 0.5cm

(Credit for the backwards argument: Henno Brandsma at \href{https://math.stackexchange.com/questions/3672998/closed-subset-and-accumulation-points}{this MathStackExchange post}.)

\vskip 0.5cm
\hrule
\vskip 0.5cm

\textbf{Exercise 2.11:} Show that a subset $A \subseteq X$ is dense if and only if every nonempty open subset of $X$ contains a point of $A$.

\vskip 0.5cm

\textbf{Proof:} Write later.

\vskip 0.5cm
\hrule
\vskip 0.5cm

\subsection{Convergence and Continuity}

\textbf{Exercise 2.12:} Show that in a metric space, this topological definition of convergence is equivalent to the metric space definition

(Also 2.13, 2.14 later)

\vskip 0.5cm

\textbf{Proof:} Write later.

\vskip 0.5cm
\hrule
\vskip 0.5cm

\textbf{Exercise 2.13:} Let $X$ be a discrete topological space. Show that the only convergent sequences in $X$ are the ones that are eventually constant, that is, sequences $(x_i)$ such that
$x_i = x$ for all but finitely many $i$.


\vskip 0.5cm

\textbf{Proof:} If we have a sequence such that $x_i = x$ for only finitely many $i$ i.e. we have a finite index set I such that $x_i = x$ whenever $i \in I$, then we can define $U = X \setminus \{x_i : i \in I\}$  to be the set of all points excluding the points which are equal to $x$. Then, $U$ is a neighborhood of $x$ since we have the discrete topology (every subset is an open set) and the sequence doesn't converge to $x$ due to the existence of $U$.

\vskip 0.5cm
\hrule
\vskip 0.5cm

\textbf{Exercise 2.14:} Suppose $X$ is a topological space, $A$ is a subset of $X$, and $(x_i)$ is a
sequence of points in $A$ that converges to a point $x \in X$. Show that $x \in \bar{A}$.

\vskip 0.5cm

\textbf{Proof:} Write later.

\vskip 0.5cm
\hrule
\vskip 0.5cm

\textbf{Exercise 2.15:} A map between topological spaces is continuous if and only if the preimage of every closed subset is closed.

\vskip 0.5cm

\textbf{Proof:} Write later.

% \underline{"Forwards Direction":}

% Suppose a map $f : X \rightarrow Y$ is continuous. That means, for every open set $V \subseteq Y$, the pre-image $f^{-1}(V) \subseteq X$ is open. 

% The complement, $Y \setminus V \subseteq Y$ is closed. What about $f^{-1}(Y \setminus V)$?  

\vskip 0.5cm
\hrule
\vskip 0.5cm

\textbf{Exercise 2.21:} Let $(X_1, \T_1)$ and $(X_1, \T_1)$ be topological spaces and let $f : X_1 \rightarrow X_2$ be a bijective map. Show that $f$ is a homeomorphism if and only if $f(\T_1) = \T_2$ in the sense that $U \in \T_1$ if and only if $f(U) \in \T_2$.

\vskip 0.5cm

\textbf{Proof:} [Re-write this later]
As a reminder, we define a function $f : X \rightarrow Y$ to be a homeomorphism if both $f$ and $f^{-1}$ are continuous maps.

\vskip 0.5cm

\underline{"Forwards Direction":}

Suppose $f$ is a homeomorphism between $X$ and $Y$. Then, since $f : X \rightarrow Y $ is continuous, for any open set $V = f(U) \in \T_2$ we have the pre-image $f^{-1}(V) = U \subseteq X$ being open i.e. $U \in \T_1$. it is surjective  Thus, $f(\T_1) = \T_2$.

\vskip 0.5cm

\underline{"Backwards Direction":}

Suppose $f : X \rightarrow Y$ is a function such that $f(\T_1) = \T_2$ in the sense that $U \in \T_1$ if and only if $f(U) \in \T_2$. So, clearly under for any open set $f(U) \subseteq Y$, the preimage $U \subseteq X$ is also open and similarly with $f^{-1} : Y \rightarrow X$, for any $U \subseteq X$ the preimage $f(U) \subseteq Y$ is also open.

\vskip 0.5cm
\hrule
\vskip 0.5cm

\subsection{Hausdorff Spaces}

\textbf{Exercise 2.35:} Suppose $X$ is a topological space, and for every $p \in X$ there exists a continuous function $f : W \rightarrow \mathbb{R}$ such that $f^{-1}(0) = \{ p \}$. Show that $X$ is Hausdorff.

\vskip 0.5cm

\textbf{Proof:} Consider a point $p \in X$ and the associated map $f : X \rightarrow Y$. Now, consider some other point $q \in X$ with $q \neq p$. Then, $f(p) = 0 \neq f(q)$.

Since $\mathbb{R}$ is a Hausdorff space, there exist open neighborhoods $U$ and $V$ around $f(p)$ and $f(q)$ respectively which are disjoint. 

Then, since $f$ is a continuous map, the pre-images $f^{-1}(U)$ and $f^{-1}(V)$ form disjoint open sets around the points $p$ and $q$ respectively.

Thus, $X$ is a Hausdorff space.

\vskip 0.5cm
\hrule
\vskip 0.5cm

\textbf{Exercise 2.38:} Show that the only Hausdorff topology on a finite set is the discrete topology. 

\vskip 0.5cm

\textbf{Proof:} 

\vskip 0.5cm
\hrule
\vskip 0.5cm

\subsection{Bases and Countability}

\textbf{Exercise 2.51:} Suppose $X$ is a second countable space, then $X$ contains a countable dense subset.

\vskip 0.5cm

\textbf{Proof:} Recall that a subset $A$ is dense in $X$ if $\cl(A) = X$. Further, in Exercise 2.11, we showed that a subset $A \subseteq X$ is dense in $X$ if and only if every nonempty open subset of $X$ contains a point of $A$. 

\vskip 0.5cm

FINISH LATER

\vskip 0.5cm
\hrule
\vskip 0.5cm

\subsection{Manifolds}

\textbf{Exercise 2.54:} Show that a topological space is a 0-manifold if and only if it is a countable discrete space.

\vskip 0.5cm

\textbf{Proof:} Write later.

\vskip 0.5cm
\hrule
\vskip 0.5cm

\textbf{Problem 2-1:} Long question

\vskip 0.5cm

\textbf{Proof:} Do later.

\vskip 0.5cm
\hrule
\vskip 0.5cm

\textbf{Problem 2-7:} Suppose $X$ is a Hausdorff space and $A \subseteq X$.If $p \in X$ is a limit point of $A$, then every neighborhood of $p$ contains infinitely many points of $A$.

\vskip 0.5cm

\textbf{Proof:} Recall that given a topological space $X$ and subset $A \subseteq X$, we call point $p \in X$ a limit point of $A$ if every neighborhood of $p$ contains at least one point of $A$ other than itself. i.e. $U$ intersects $A \setminus \{p\}$.

\vskip 0.5cm

Suppose, for contradiction, that $p \in X$ is a limit point of $A \subseteq X$ having a neighborhood $U$ which has finite intersection with $A$ excluding $p$ i.e. the set $V = (U \cap A) \setminus \{p\}$ has finitely many points. 

\vskip 0.5cm

We know that finite subsets of a Hausdorff space are closed. Then, the complement $V^c$ is an open set containing $p$. But then, $V^{c}$ is an open subset containing $p$ and whose intersection with $A$ is empty i.e. $V \cap A = \emptyset$. The existence of $V^c$ would mean that $p$ is \emph{not a limit point}.

\vskip 0.5cm

We have a contradiction. Thus, if $p$ is a limit point of $A$ then \emph{every} open neighborhood of $p$ must contain infinitely many points of $A$.
\vskip 0.5cm
\hrule
\vskip 0.5cm

\pagebreak
\section{Chapter 3: New Spaces from Old}

\subsection{Subspaces}

\textbf{Exercise 3.1:} Prove that $\T_S$ is a topology on $S$.

\vskip 0.5cm

\textbf{Proof:} We have topological space $X$ with topology $\T_X$ and a generic subset $S \subseteq X$. We then define

\[ \T_S = \{ U \subseteq S \; : \; U = V \cap S,\; V \in \T_X \} \]

and show that $\T_S$ is a topology on $S$ by verifying it satisfies the three requirements. 

\vskip 0.5cm

\begin{itemize}
  \item Trivially, $\emptyset \in \T_S$. And, since $X \in \T_X$ we can express $S$ as being $S = S \cap X$, so certainly we have $S \in \T_S$.
  
  \vskip 0.5cm
  \item Consider an arbitrary union of sets $U = \bigcup_{\alpha \in A} U_{\alpha}$ where $U_{\alpha} \in \T_S$. ssince each $U_{\alpha}$ can be expressed as $V_{\alpha} \cap S$ for some open set $V_{\alpha}$ in $\T_X$, we have 
  
  \begin{align*}
    \bigcup_{\alpha \in A} U_{\alpha} &= \bigcup_{\alpha \in A} (V_{\alpha} \cap S) \\
    &= \left( \bigcup_{\alpha \in A} V_{\alpha} \right) \cap S \\
    &= V \cap S
  \end{align*}
  where $V$ is again an open set of $X$.

  So, arbitrary unions of open sets in $V$ remain open.
  
  \vskip 0.5cm
  \item Now consider some finite union $U = \bigcap_{i \in I} U_i$ for each $U_i \in \T_S, \; |I| < \infty$. Then,
  \begin{align*}
    \bigcap_{i \in I} U_i &= \bigcap_{i \in I} (V_i \cap S) \\
    &= \left(\bigcap_{i \in I} V_i\right) \cap S \\
    &= V \cap S
  \end{align*}
  where again $V$ is open in $X$ because it's a finite union of open sets which are elements of $\T_X$.
\end{itemize}
Thus, $\T_S$ forms a topology on $S$, called the \emph{subspace topology}.

\vskip 0.5cm
\hrule
\vskip 0.5cm

\textbf{Exercise 3.2:} Suppose $S$ is a \textbf{subspace} of $X$. Prove that $B \subseteq S$ is closed in $S$ if and only if it is equal to the intersection of $S$ with some closed set in $X$.

\vskip 0.5cm

\textbf{Proof:} 
\vskip 0.5cm

\underline{"Forwards Direction":} Suppose $B \subseteq S$ is closed in $S$. Then, $S \setminus B$ is open in $S$, and by definition of the subspace topology, there is some set $V$ open in $X$ such that 
\[ S \setminus B = S \cap V \]

Now,
\begin{align*}
  B &= S \setminus (S \setminus B) \\
  &= S \setminus (S \cap V) \\
  &= S \setminus V \\
  &= S \cap (X \setminus V)
\end{align*}
Since $V$ is open in $X$, its complement $X \setminus V$ is closed in $X$. So, we have the forward direction.

\vskip 0.5cm
\underline{"Backwards Direction":} Suppose there exists some set $W \subseteq X$ which is closed in $X$ such that $B = S \cap W$.

Then, we have 
\begin{align*}
  S \setminus B &= S \setminus (S \cap W) \\
  &= S \setminus W \\
  &= S \cap (X \setminus W)
\end{align*}

and $X \setminus W$ is open in $X$. So, by definition, $S \setminus B$ is open in $S$, which means that $B$ is closed in $S$. 

\vskip 0.5cm
\hrule
\vskip 0.5cm

\textbf{Exercise 3.3:} Let $M$ be a metric space and let $ S \subseteq M$ be any subset. Show that the subspace topology on $S$ is the same as the metric topology obtained by restricting the metric of $M$ to pairs of points in $S$.

\vskip 0.5cm

\textbf{Proof:} Write later.

\vskip 0.5cm
\hrule
\vskip 0.5cm

\textbf{Exercise 3.6:} Suppose $S$ is a subspace of the topological space $X$:

\begin{enumerate}
  \item If $U \subseteq S \subseteq X$, $U$ is open in $S$ and $S$ is open in $X$, then $U$ is open in $X$. The same true for "closed" instead of "open".
  
  \item If $U$ is a subset of $S$ that is either relatively open or closed in $X$, then it is also open or closed in $X$, respectively.
\end{enumerate}

\vskip 0.5cm

\textbf{Proof:} Write later.

\vskip 0.5cm
\hrule
\vskip 0.5cm


\textbf{Exercise 3.7:} Suppose $X$ is a topological space and $U 
\subseteq S \subseteq X$.

\begin{enumerate}
  \item Show that the closure of $U$ in $S$ is $\bar{U} \cap S$.
  \item Show that the interior of $U$ in $S$ contains $Int(U) \cap S$; give an example to show they may not be equal.
\end{enumerate}

\vskip 0.5cm

\textbf{Proof:} Write later.

\vskip 0.5cm
\hrule
\vskip 0.5cm

\textbf{Exercise 3.12:} 

\vskip 0.5cm

\textbf{Proof:} Write later.

\vskip 0.5cm
\hrule
\vskip 0.5cm

\subsubsection{Topological Embeddings}

\textbf{Exercise 3.13:} Let $X$ be a topological manifold and $S$ be a subspace of $X$. Show that the inclusion map $S \hookrightarrow X$ is a topological embedding. 

\vskip 0.5cm

\textbf{Proof:} An injective continuous map that is a homeomorphism onto its image (in the subspace topology) is called a topological embedding.

\vskip 0.5cm

From the Characteristic property of Subspaces, we know that the inclusion is a continuous map (due to the diagram below). 

\[\begin{tikzcd}
	& X \\
	S & S
	\arrow["{f = i_S}"', from=2-1, to=2-2]
	\arrow["{i_X \circ f}", from=2-1, to=1-2]
	\arrow["{i_X}"', hook', from=2-2, to=1-2]
\end{tikzcd}\]

The inclusion is certainly injective since every element is mapped only to itself.

\vskip 0.5cm
All we need to do is show that $i_X : S \hookrightarrow X$ is a homeomorphism. Since we've already shown $i_X$ is continuous, we just need to show that $i_X^{-1} : X \rightarrow S$ is continuous.

\vskip 0.5cm
To do so, consider an open set $V \subseteq X$ If $V \cap S = \emptyset$ then $i_X^{-1}$ maps $V$ to the emptyset which is trivially open.

If $V \cap S = B \neq \emptyset$, then it maps $V$ to $B$ -- which we've just written as an intersection of $S$ with an open set of $X$. So, $B$ is (relatively) open in $S$. Thus, $i_X^{-1}$ is continuous.

\vskip 0.5cm
So, the inclusion satisfies all the requirements and is a topological embedding.

\vskip 0.5cm
\hrule
\vskip 0.5cm

\textbf{Exercise 3.17:} Give an example of a topological embedding that is neither an open map nor a closed map.

\vskip 0.5cm

\textbf{Proof:} As pointed out by \href{https://math.stackexchange.com/users/41672/stefan-hamcke}{Stefan Hamcke} in \href{https://math.stackexchange.com/questions/1288421/topological-embedding-which-is-neither-open-nor-closed}{this StackExchange post} any inclusion of a subset is an embedding, so all we need to do is find a subset of topological space $X$ which is neither open nor closed and consider the embedding of the subset into $X$.

\vskip 0.5cm
For instance, if $X = $. FINISH LATER.

\vskip 0.5cm
\hrule
\vskip 0.5cm

\subsection{Product Spaces}

\textbf{Exercise 3.25:} Prove that 
\[ \mathcal{B} = \{ U_1 \times \cdots \times U_n : U_i \text{ is an open subset of }X_i, i = 1,2,3,\dots,n \} \]
is a basis for a topology on the product space $X = \prod_{i = 1}^{n} X_i$.

\vskip 0.5cm

\textbf{Proof:} For $\mathcal{B}$ to be a basis of a topology, it must satisfy the following: 

\begin{itemize}
  \item $\cup_{B \in \mathcal B} B = X$.
  \item For any $B_1, B_2 \in \mathcal B$ and any $x \in B_1 \cap B_2$, there exists some $B_3 \in \mathcal B$ such that $x \in B_3 \subseteq B_1 \cap B_2$.  
\end{itemize}

\emph{To show requirement 1}, we can show set containment in both directions.

\vskip 0.5cm

Suppose $x = (p_1, \dots, p_n) \in X = X_1 \times \cdots \times X_n$. So, $p_i \in X_i$. Then, for each $p_i$ there exists some set $U_i^{p_i}$ open in $X_i$ st $x \in U_i^{p_1}$. So, $x \in U_1^{p_1} \times \cdots \times U_n^{p_n} = B_x$. This holds for any point $x \in X$, so clearly, $X \subseteq \cup_{B \in \mathcal B} B$.

\vskip 0.5cm

On the other hand, suppose $x = (p_1, \dots, p_n) \in \cup_{B \in \mathcal B} B$. Then there is some $B_x = U_1^{p_1} \times \cdots \times U_n^{p_n} $ such that $p_i \in U_i^{p_i}$ and $U_i^{p_i} \subseteq X_i$. So, certainly, we have $x \in X$. This shows $\cup_{B \in \mathcal B} B \subseteq X$. Thus, the two are equal.

\vskip 0.5cm

\emph{To show requirement 2}, suppose we have $B_1, B_2 \in \mathcal B$ and $x \in B_1 \cap B_2$. We can write $x = (p_1, \dots, p_n) $, and express the basis sets as $B_1 = U_1 \times \cdots U_n$, $B_2 = V_1 \times \cdots \times V_n$ where $U_i$'s and $V_i$'s are open sets in $X_i$.

\vskip 0.5cm

Then, $x \in B_1 \cap B_2$ means that $p_i \in U_i \cap V_i$. Since each $U_i, V_i$ is open, the intersection $W_i = U_i \cap V_i \subseteq X_i$ is itself open in $X_i$. Then, we can write $x \in B_3 = (W_1, \dots, W_n) \subseteq B_1 \cap B_2$.  

\vskip 0.5cm

This shows the second requirement is fulfilled, and so $\mathcal B$ generates a topology (the product topology) on $X$.

\vskip 0.5cm
\hrule
\vskip 0.5cm

\textbf{Exercise 3.26:} Show that the product topology on $\mathbb{R}^n = \mathbb{R} \times \cdots \times \mathbb{R}$ is the same as the metric topology induced by the Euclidean distance function.

\vskip 0.5cm

\textbf{Proof:} We can show that the two topologies are equivalent by showing that any open set in the product topology contains an open set in the metric topology, and vice versa.

\vskip 0.5cm
We already know that the product topology is generated by the basis 
\[ \mathcal{B} = \{ \prod_{i = 1}^{n} U_i : U_i \text{ is an open subset of } \mathbb{R}, i = 1,2,3,\dots,n \} \]
so let's right away work with the basis sets. 

\vskip 0.5cm
Let $B \in \mathcal B$ be a basis set and let $\{U_k\}$ be the corresponding open sets in $\mathbb{R}$. Consider a point $x = (p_1, \dots, p_n) \in \mathbb{R}^n$. Since each $U_k$ is open in $\mathbb{R}$, each one will contain some $\epsilon-$ball. That is, we have the points $p_1, \dots, p_n$ (with $p_i \in U_i \subseteq \mathbb{R}$) and positive real numbers $\epsilon_1, \dots, \epsilon_n$ such that $B_{\epsilon_i}(p_i) \subseteq U_i$.

\vskip 0.5cm
So, if we let $\epsilon = min\{ \epsilon_1, \dots, \epsilon_n \}$, there an n-dimensional ball $B_{\epsilon}^{n}(x) \subset \mathbb{R}^{n}$ centered around the point $x = (p_1, \dots, p_n)$. So, any open set in the product topology contains an open set from the metric topology.

\vskip 0.5cm
Next, let $B_{\epsilon}^{n}(x) \subset \mathbb{R}^n$ be the $\epsilon-$ball around the point $x = (p_1, \dots, p_n)$ in $\mathbb{R}^n$. 

By definition, the n-ball is the cartesian product of the $1$-dimensional $\epsilon-$ball around each coordinate $p_i$. i.e.

\[ B_{\epsilon}^{n}(x) = \prod_{i = 1}^{n} B_{\epsilon}^1(p_i) \]

and each of these 1-dimensional balls is open in $\mathbb{R}$, so we have exactly what we need -- $B_{\epsilon}^{n}(x)$ is a member of the product topology since it is the product of open sets in $\mathbb{R}$. 

Thus, the basis of the product topology and the euclidean metric both induce the same topology on $\mathbb{R}^n$.

\underline{\textbf{Characteristic Property of the Product Topology:}}

Suppose $X_1 \times \cdots \times X_n$ is a product space. Then, a map from any topological space $Y$, $f : Y \rightarrow X_1 \times \cdots \times X_n$, is continuous if and only if each of its component functions $f_i = \pi_i \circ f$ is continuous, where $\pi_i : X_1 \times \cdots \times X_n \rightarrow X_i$ is the canonical projection.

\[\begin{tikzcd}
	& {X_1 \times \cdots \times X_n} \\
	Y & {X_i}
	\arrow["f", from=2-1, to=1-2]
	\arrow["{f_i = \pi_i \circ f}"', from=2-1, to=2-2]
	\arrow["{\pi_i}", from=1-2, to=2-2]
\end{tikzcd}\]

\vskip 0.5cm
\underline{\textbf{Proof:}} Suppose each $f_i$ is continuous. To prove that $f$ is continuous, it suffices to show that the pre-image of each basis set $U_1 \times \cdots \times U_n$ is open, since that would guarantee the preimage of any open set is also open.

\vskip 0.5cm
For any $x = (p_1, \dots, p_n) \in U_1 \times \cdots \times U_n$, the pre-image $f^{-1}(x) \in f^{-1}(U_1 \times \cdots \times U_n)$ if and only if $f_i^{-1}(p_i) \in f_i^{-1}(U_i)$. So,
\[ f^{-1}(U_1 \times \cdots \times U_n) = f^{-1}(U_1) \cap \cdots \cap f^{-1}(U_n) \]

finish writing this proof later / delete it

\vskip 0.5cm
\hrule
\vskip 0.5cm

\textbf{Exercise 3.29:} Prove the following only using the Characteristic Property of Product spaces:
\vskip 0.5cm

If $X_1, \dots, X_n$ are topological spaces, each canonical projection $\pi_i : X_1 \times \cdots \times X_n \rightarrow X_i$ is continuous.

\vskip 0.5cm

\textbf{Proof:} The Characteristic Property of Product Spaces tells us that, in the diagram below, $f$ is continuous if and only if every $f_i$ is continuous.

\[\begin{tikzcd}
	& {X_1 \times \cdots \times X_n} \\
	Y & {X_i}
	\arrow["f", from=2-1, to=1-2]
	\arrow["{f_i = \pi_i \circ f}"', from=2-1, to=2-2]
	\arrow["{\pi_i}", from=1-2, to=2-2]
\end{tikzcd}\]

Suppose we have $Y = X$ where $X = X_1 \times \cdots \times X_n$ is the product space and $f = Id_X$ is the identity map on the product space. Then, we have the following commutative diagram

\[\begin{tikzcd}
	& {X} \\
	X & {X_i}
	\arrow["Id_x", from=2-1, to=1-2]
	\arrow["{f_i = \pi_i \circ Id_X}"', from=2-1, to=2-2]
	\arrow["{\pi_i}", from=1-2, to=2-2]
\end{tikzcd}\]

By the Characteristic property, $f_i$ is continuous if and only if $Id_X$ is open.
But $f_i = \pi_i \circ Id_X = \pi_i$, and of course $Id_X$ is continuous since the preimage of any open set in $X$ under the identity map is just itself -- an open set. 

\vskip 0.5cm
Therefore, the projection map is continuous.

\vskip 0.5cm
\hrule
\vskip 0.5cm

\textbf{Exercise 3.32:} Prove proposition 3.31 (long question)

\vskip 0.5cm

\textbf{Proof:} Do later.

\vskip 0.5cm
\hrule
\vskip 0.5cm

\textbf{Exercise 3.34:} Suppose $f_1, f_2 : X \rightarrow \mathbb{R}$ are continuous functions. Their pointwise sum $f_1 + f_2 : X \rightarrow \mathbb{R}$ and pointwise product $f_1 f_2 : X \rightarrow \mathbb{R}$ are real-valued functions defined by 

\[ (f_1 + f_2)(x) = f_1(x) + f_2(x) \;\;\;\;\;\ (f_1 f_2)(x) = f_1(x)f_2(x) \]

Pointwise sums and products of complex-valued functions are defined similarly. Use the characteristic property of the product topology to show that pointwise sums and products of
real-valued or complex-valued continuous functions are continuous.

\vskip 0.5cm

\textbf{Proof:} Do later.

\vskip 0.5cm
\hrule
\vskip 0.5cm

\subsubsection{Infinite Products}

\textbf{Exercise 3.38:} Prove the \textbf{Characteristic Property of Infinite Product Spaces} as stated below:

\vskip 0.5cm
Let $(X_{\alpha})_{\alpha \in A}$ be an indexed family of topological spaces. For any topological space $Y$, a map $f : Y 
\rightarrow \prod_{\alpha \in A} X_{\alpha}$ is continuous if and only if each of its component functions $f_{\alpha} = \pi_{\alpha} \circ f$ is continuous. The product topologyis the uniquetopology on $\prod_{\alpha \in A} X_{\alpha}$ that satisfies this property.

\vskip 0.5cm

\textbf{Proof:} Consider the following (commutative) diagram:

\[\begin{tikzcd}
	& { \prod_{\alpha \in A} X_{\alpha}} \\
	Y & {X_{\alpha}}
	\arrow["f", from=2-1, to=1-2]
	\arrow["{f_{\alpha} = \pi_{\alpha} \circ f}"', from=2-1, to=2-2]
	\arrow["{\pi_{\alpha}}", from=1-2, to=2-2]
\end{tikzcd}\]

\vskip 0.5cm
Suppose $f$ is continuous. That means, for any open set $U \subseteq \prod_{\alpha} X_{\alpha}$, $f^{-1}(U)$ is open in $Y$. We wish to show that $f_{\alpha}^{-1}(u_{\alpha}) = (f_{\alpha}^{-1} ( \pi_{\alpha}^{-1} (U_{\alpha}) ))$ is open in $Y$.

\vskip 0.5cm
Now, $\pi_{\alpha}^{-1}(U_{\alpha}) = U_{\alpha} \times \prod_{\alpha \neq \beta \in A} X_{\beta}$, which is open in the product space by the definition of the product topology. Then, by assumption, $f^{-1}(U_{\alpha} \times \prod_{\alpha \neq \beta \in A} X_{\beta})$ is open and since this holds for any arbitrary $\alpha$ we have that each $f_{\alpha}$ is continuous.

\vskip 0.5cm
Now, suppose each $f_{\alpha}$ is continuous. To prove that $f$ is continuous, it suffices to show that the pre-image of any basis set $B \subseteq \prod_{\alpha in A} X_{\alpha}$ under $f$ is open in $Y$. 

\vskip 0.5cm
Consider some basis set $B$. By definition, $B = \prod_{\alpha \in A} U_{\alpha}$ where $U_{\alpha}$ is open in $X_{\alpha}$ for each $\alpha$ and there are finitely many $\alpha$ such that $U_{\alpha} \neq X_{\alpha}$. Let $I$ denote the (finite) set of indices for which $U_{\alpha} \neq X_{\alpha}$

\vskip 0.5cm
A point $y = (p_1, \dots, p_{n}) \in f^{-1}(B)$ if and only if $f_{\alpha}(y) \in U_{\alpha}$, so $f^{-1}(\prod_{\alpha \in A} U_{\alpha}) = \bigcap_{\alpha \in A} f^{-1}_{\alpha}(U_{\alpha})$.

\vskip 0.5cm
Since each $U_{\alpha}$ is open in $X_{\alpha}$, each $f^{-1}_{\alpha}(U_{\alpha})$ is open in $Y$. But note that for any $\alpha \not\in I$, we have $f^{-1}_{\alpha}(U_{\alpha}) = f^{-1}_{\alpha}(X_{\alpha}) = Y$ since $f_{\alpha}$ is surjective.

So, really, we only have finitely many intersections since 
\begin{align*}
  f^{-1}\left(\prod_{\alpha \in A} U_{\alpha}\right) &= \bigcap_{\alpha \in A} f^{-1}_{\alpha}(U_{\alpha}) \\
  &= Y \cap \left( \bigcap_{\alpha \in I} f^{-1}(U_{\alpha}) \right) \\
  &= \bigcap_{\alpha \in I} f^{-1}(U_{\alpha})
\end{align*}
and this is a finite intersection of sets which are open in $Y$, thus it is open itself. So, each of the $f_{\alpha}$'s being continuous implies the continuity of $f$.

\textbf{Prove the uniqueness later.}

\vskip 0.5cm
Also see \href{https://math.stackexchange.com/questions/2106197/question-on-the-characteristic-property-of-infinite-product-spaces}{this Math.SE post}.

\vskip 0.5cm
\hrule
\vskip 0.5cm

\subsection{Disjoint Union Spaces}

\subsection{Quotient Spaces}

\textbf{Exercise 3.46:} Show that the quotient topology is indeed a topology.

\vskip 0.5cm

\textbf{Proof:} Let $X$ be a topological space and $q : X \rightarrow Y$ be a surjective map. We define the \textbf{quotient topology} on $Y$ by declaring a set $U \subseteq Y$ to be open if and only if $q^{-1}(U)$ is open in $X$.

\[ \T_q = \{ U \subseteq Y : q^{-1}(U) \in \T_X \} \]

\vskip 0.5cm
To show that this is indeed a topology, we note that 
\begin{itemize}
  \item $\emptyset \in \T_q$ and since $q^{-1}(Y) = X \in \T_X$ we have $Y \in \T_q$.
  
  \item Consider an arbitrary union of sets which are open in $Y$: $U = \bigcup_{\alpha \in A} U_{\alpha}$, $U_{\alpha} \in \T_q$. Now, 
  \begin{align*}
    q^{-1}(U) &= q^{-1}\left(\bigcup_{\alpha \in A} U_{\alpha}\right) \\
    &= \bigcup_{\alpha \in A} q^{-1}\left(U_{\alpha}\right)
  \end{align*}
  and since each $q^{-1}(U_{\alpha})$ is open in $X$, the union in the last step is also open in $X$. So, $U$ is open in $Y$.

  \item Consider a finite union of open sets in $Y$: $V = \bigcap_{\substack{i \in I, \\ |I| < \infty}} U_{\alpha}$, $U_{\alpha} \in A$. Now,
  \begin{align*}
    q^{-1}(V) &= q^{-1}\left( \bigcap_{\substack{i \in I, \\ |I| < \infty}} U_{\alpha} \right)\\
    &= \bigcap_{\substack{i \in I, \\ |I| < \infty}} q^{-1}(U_{\alpha})
  \end{align*}
  and since this is a finite intersection of open sets in $X$, it is also open in $X$. Thus, finite intersections of open sets in $Y$ are also open in $Y$. 
\end{itemize}
\vskip 0.5cm
Since the three requirements are satsified, $\T_q$ is indeed a topology.

\vskip 0.5cm
Note that nowhere in this proof did we use the fact that $q$ is \emph{surjective}. It is possible to define this kind of topology even if $q$ isn't surjective, but then it is not called the quotient topolgy. This \href{https://math.stackexchange.com/questions/2608654/proof-that-a-quotient-topology-is-indeed-a-topology}{Math.SE post} may be of interest.

\vskip 0.5cm
\hrule
\vskip 0.5cm

\textbf{Exercise 3.55:} Show that every  wedge sum of Hausdorff spaces is Hausdorff.

\vskip 0.5cm

\textbf{Proof:} Suppose we have an indexed family of \emph{Hausdorff} topological spaces $(X_{\alpha})_{\alpha \in A}$ and a collection of base points $\{p_{\alpha}\}_{\alpha \in A}$. The wedge sum $W = \vee_{\alpha \in A} X_{\alpha}$ is the disjoint union quotiented by identifying the base points together i.e.

\[ W = \vee_{\alpha \in A} X_{\alpha} = \bigslant{\left(\coprod_{\alpha \in A} X_{\alpha} \right)}{ \sim } \]

by the identification $p_{i} \sim p_{j}$ for $i,j \in A$.

(not sure if the above is proper notation for the quotient -- refine later.)

\vskip 0.5cm
Now, consider any two points $x, y \in W$. If they both belong to the same $X_{\alpha}$ then clearly there exist disjoint neighborhoods containing them since each $X_{\alpha}$ is Hausdorff. If they originally belonged to $X_{\alpha}$, then any open neighborhood of $x$ and $y$ not containing the point $p$ to which the base points $\{p_{\alpha}\}_{\alpha \in A}$ have been identified works. If one of the points is $p$

\vskip 0.5cm
(refine this answer later. basic idea is there.)

\vskip 0.5cm
\hrule
\vskip 0.5cm

\subsubsection{Recognizing Quotient Maps Between Known Spaces}

\textbf{Exercise 3.59:} Let $q : X \rightarrow Y$ be a quotient map. For a subset $U \subseteq X$, show that the following are equivalent:
\begin{enumerate}
  \item $U$ is saturated.
  \item $U = q^{-1}(q(U))$
  \item $U$ is a union of fibers.
  \item If $x \in U$, then every point $x' \in X$ such that $q(x) = q(x')$ is also in $U$.
\end{enumerate}

\vskip 0.5cm

\textbf{Proof:} 

Let's recall some required definitions:

\vskip 0.5cm
For map $q : x \rightarrow Y$, any subset of the form $q^{-1}(y) \subseteq X$ for some $y \in Y$ is called a \textbf{fiber of $q$}. A subset $U \subseteq X$ is said to be \textbf{saturated with respect to $q$} if $U = q^{-1}(V)$ for some $V \subseteq Y$.

\vskip 0.5cm
Now, let's tackle the question. To show that any pair of the above statements is equivalent, it suffices to show that $(1) \implies (2)$, $(2) \implies (3)$, $(3) \implies (4)$, and $(4) \implies (1)$.

\begin{enumerate}
  \item Suppose $U$ is saturated with respect to $q$. Then there is some $V \subseteq Y$ such that $U = q^{-1}(V)$. Then, for any $x \in U$ we know $x \in q^{-1}(V)$. So, of course, $q(x) \in V$. Since this holds for every $x \in U$, we have $V = q(U)$. Thus,
  \[ U = q^{-1}(q(U)) \]
  This proves the $(1) \implies (2)$ implication.
  \vskip 0.5cm

  \item Now, suppose the subset $U \subseteq X$ can be expressed as $U = q^{-1}(q(U))$. Then, since every element of $x \in U$ can be written as $x = q^{-1}(y)$ for some $y \in q(U) \subseteq Y$, $U$ is clearly the union of fibers. 
  This proves the $(2) \implies (3)$ implication.
  \vskip 0.5cm

  \item Now, suppose $U$ is a union of fibers. i.e. 
  \[ U = \bigcup_{i \in I} q^{-1}(y_i) \] for $y_i \in Y$.

  Consider some $x \in U$. Then, $x \in q^{-1}(y_j)$ for some specific $j \in I$. Any $x' \in X$ such that $q(x) = q(x')$ also belongs to the same pre-image set i.e. $x' \in q^{-1}(y_j)$. So, certianly, we have $x' \in U$. This proves the $(3) \implies (4)$ implication.
  \vskip 0.5cm

  \item Finally, suppose that if $x \in U$ then every point $x' \in X$ such that $q(x) = q(x')$ is also in $U$. Then, we can write
  \[ U = \bigcup_{i \in I} q^{-1}(y_i) \] for $y_i \in Y$.
  
  In other words,
  \[ U = q^{-1} \underbrace{\left(\bigcup_{i \in I} y_i \right)}_{V} \] for $y_i \in Y$.
  
  So, $U$ is saturated with respect to $q$. This proves the $(4) \implies (1)$ implication. 
\end{enumerate}

Thus, we are done.

\vskip 0.5cm
\hrule
\vskip 0.5cm

\textbf{Exercise 3.61:} A continuous surjective map $q : X \rightarrow Y$ is a quotient map if and only if it takes saturated open subsets to open subsets, or saturated closed subsets to closed subsets.

\vskip 0.5cm

\textbf{Proof:} 

\underline{"Forwards Direction":} Suppose $q : X \rightarrow Y$ is a quotient map between topological spaces. That is, $q$ is surjective map and $Y$ has the quotient topology indeuced by $q$. 

\vskip 0.5cm
Recall that the quotient topology is defined in the following way:

A subset $V \subseteq Y$ is said to be open in $Y$ if and only if $q^{-1}(V)$ is open in $X$.

\vskip 0.5cm
Let $U$ be a saturated open set of $q$. Then, using the results of Exercise 3.59 we can write $U = q^{-1}(q(U))$ and since $U$ itself is open, $q(U)$ is clearly open in $Y$ by the definition of the quotient topology. 

\vskip 0.5cm
If $U$ is a saturated closed set, then we note that $X \setminus U = X \setminus q^{-1}(q(U)) = q^{-1}(Y \setminus q(U))$ where the last equality follows from a basic set theoretic argument.
So, $X \setminus U$ is saturated and open, so our argument from above applies, and we know that $Y \setminus q(U)$ is open in  $Y$. Thus, $q(U)$ is closed in $Y$.

\vskip 0.5cm
\underline{"Backwards Direction":}
Now, suppose $q : X \rightarrow Y$ is a continuous surjective map which takes saturated open/closed subsets in $X$ to open/closed subsets in $Y$. Then, $q$ is immediately a quotient map since for any $V \subset Y$ we are guaranteed to have a pre-image $U$ due to the surjectivity of $q$, meaning $U = q^{-1}(V)$ is saturated -- so for open $V$, $U$ is also open. This means $q$ induces the quotient topology on $Y$ and is thus a quotient map.

\vskip 0.5cm
\hrule
\vskip 0.5cm

\textbf{Exercise 3.63:} Proving various properties of quotient maps.

\vskip 0.5cm

\textbf{Proof:} Make sure to do later.

\vskip 0.5cm
\hrule
\vskip 0.5cm

\subsubsection{The Characteristic Property and Uniqueness}

\textbf{Theorem 3.70: Characteristic Property of the Quotient Topology} 

Suppose $X$ and $Y$ are topological spaces and $q : X \rightarrow Y$ is a quotient map. For any topological space $Z$, a map $f : Y \rightarrow Z$ is continuous if and only if $f \circ q : X \rightarrow Z$ is continuous.

\[\begin{tikzcd}
	X \\
	Y & Z
	\arrow["q"', from=1-1, to=2-1]
	\arrow["f"', from=2-1, to=2-2]
	\arrow["{f \circ q}", from=1-1, to=2-2]
\end{tikzcd}\]

\vskip 0.5cm

\textbf{Proof:} Consider some open set $U \subseteq Z$ and its pre-image $f^{-1}(U) \subseteq Y$. Now, $f^{-1}(U)$ is open if and only if $q^{-1}(f^{-1}(U)) = (f \circ q)^{-1}(U)$ is open in $X$. Thus, if $f \circ q$ is continuous then $f$ is continuous, and vice versa.

\vskip 0.5cm
\hrule
\vskip 0.5cm

\textbf{Exercise 3.71:} Given a topological space $X$, a set $Y$, and a surjective map $q : X \rightarrow Y$, the quotient topology is the only topology on $Y$ for which the characteristic property holds.

\vskip 0.5cm

\textbf{Proof:} Let $Y_q$ be the $Y$ equipped with the quotient topology on $Y$ and $Y_g$ be $Y$ equipped with some other topology on $Y$ such that both $Y_q$ and $Y_g$ satisfy the characteristic property of quotient topologies. 

\vskip 0.5cm
To show that the two topologies are the same, it suffices to show that the identity map on $Y$ is a homeomorphism between $Y_q$ and $Y_g$. So, consider the following two diagrams:

\[\begin{tikzcd}
	X &&& X \\
	{Y_q} & {Y_g} && {Y_g} & {Y_q}
	\arrow["q"', from=1-1, to=2-1]
	\arrow["{f_1 = id_{Y,g}}"', from=2-1, to=2-2]
	\arrow["{f_1 \circ q}", from=1-1, to=2-2]
	\arrow["{q'}"', from=1-4, to=2-4]
	\arrow["{f_2 = id_{Y,q}}"', from=2-4, to=2-5]
	\arrow["{f_2 \circ q}", from=1-4, to=2-5]
\end{tikzcd}\]

Note that $f_1$ and $f_2$ are inverses since they are identity maps on $Y$ in opposite directions. We must show that they are both continuous in order to show $Y_q$ and $Y_d$ are homeomorphic.

\vskip 0.5cm
COMPLETE LATER

See \href{https://math.stackexchange.com/questions/75413/uniqueness-of-the-quotient-topology#:~:text=(1)%20There%20can%20be%20at,is%20unique%20if%20it%20exists.}{this Math.SE post}

\vskip 0.5cm
\hrule
\vskip 0.5cm

\textbf{Also, prove the uniqueness of quotient spaces later. (Theorem 3.75).}

\vskip 0.5cm
\hrule
\vskip 0.5cm

\subsubsection{Adjunction Spaces}

\textbf{Proposition 3.77: Properties of Adjunction Spaces}

\vskip 0.5cm
Let $X \cup_f Y$ be an Adjunction space and let $q : X \coprod Y \rightarrow X \cup_f Y$ be the associated quotient map.

\begin{enumerate}
  \item The restriction of $q$ to $X$ is a topological embedding with whose image $q(X)$ is a closed subspace of $X \cap_f Y$.
  \item The restriction of $q$ to $Y \setminus A$ is a topological embedding with whose image $q(Y \setminus A)$ is an open subspace of $X \cap_f Y$.
  \item $X \cup_f Y$ is the disjoint union of $q(X)$ and $q(Y \setminus A)$.
\end{enumerate}

\vskip 0.5cm

\textbf{Proof:}
Let's recall that the Adjunction space formed by attaching $Y$ to $X$ along $f$ is 
\[ X \cup_f Y = \bigslant{\left(X \coprod Y\right)}{\sim} \]
where the equivalence relation $\sim$ is given by $a \sim f(a), a \in A$ for attaching map $f : A \rightarrow X$ where $A$ is a closed subset of $Y$.

\begin{enumerate}
  \item We approach part (a) by first showing that the image of $\restr{q}{X}$ i.e. $q(X)$ is closed. Then, if we show injectivity, we immediately have that $\restr{q}{X}$ is a topological embedding from Propoisiton 3.69.
  
  \vskip 0.5cm
  Consider a closed subset $B \subseteq X$. To show that $\restr{q}{X}$ is a closed map, we have to show that $q(B)$ is closed in the adjunction space. i.e. we have to show that $q^{-1}(q(B))$ is closed in $X \coprod Y$. In other words, we need to show that the intersections of $q^{-1}(q(B))$ with $X$ and $Y$ are both closed in the respective sets.

  By the form of the equivalence relation, $q^{-1}(q(B)) \cap X = B$ which is closed by assumption and $q^{-1}(q(B)) \cap Y = f^{-1}(Y)$ which is closed in $Y$ by the continuity of $f$.
  
  So, $\restr{q}{X}$ is a closed map.
  
  \vskip 0.5cm
  Next, injectivity. The map $\restr{q}{X}$ is injective since the equivalence relation $\sim$ does not identify any point of $X$ with each other, only points in $A$ with points in $X$.

  Thus, the restriction to $X$ is a topological embedding whose image is a closed subspace of the adjunction space.
  
  \vskip 0.5cm
  \item Now, we show that $\restr{q}{Y \setminus A}$ is a topological embedding whose image in an open subspace of $X \cup_f Y$.
  
  This time, we use Proposition 3.62(d) which states that the restriction of a quotient map to a saturated open/closed subset is also a quotient map.

  Finish writing this later.
  
\end{enumerate}

\vskip 0.5cm
\hrule
\vskip 0.5cm

\textbf{NOTE TO SELF: Re-read and do exercises from Adjunction Spaces section later. Especially proof of Theorem 3.79.}

\vskip 0.5cm
\hrule
\vskip 0.5cm

\subsection{Topological Groups and Group Actions}

\textbf{Exercise 3.83:} Verify that each of the following is a topological group (long question).

\vskip 0.5cm

\textbf{Proof:} Do later.

\vskip 0.5cm
\hrule
\vskip 0.5cm

\textbf{Exercise 3.85:} Prove proposition 3.84 as stated below:

\emph{Any subgroup of a topological group is a topological group with the subspace topology. Any finite product of topologyical groups is a topological group with the direct product group structure and the product topology.}

\vskip 0.5cm

\textbf{Proof:} Do later.

\vskip 0.5cm
\hrule
\vskip 0.5cm

\textbf{NOTE TO SELF: Re-read and do exercises from Adjunction Spaces section later. Especially proof of Theorem 3.79.}

\vskip 0.5cm
\hrule
\vskip 0.5cm

\section{Connectedness and Compactness}

\subsection{Connectedness}

\textbf{Exercise 4.3:} Suppose $X$ is a connected topological space, and $\sim$ is an equivalence relation on $X$ such that every equivalence class is open. Show that there is exactly one
equivalence class, namely $X$ itself.

\vskip 0.5cm

\textbf{Proof:} Do later.

\vskip 0.5cm
\hrule
\vskip 0.5cm

\end{document}

