\documentclass[11pt]{article}

% basic packages
\usepackage[margin=1in]{geometry}
\usepackage[pdftex]{graphicx}
\usepackage{amsmath,amssymb,amsthm}
\usepackage{custom}
\usepackage{lipsum}

\usepackage{xcolor}
\usepackage{tikz}

\usepackage[most]{tcolorbox}
\usepackage{xcolor}
\usepackage{mdframed}

% page formatting
\usepackage{fancyhdr}
\pagestyle{fancy}

\renewcommand{\sectionmark}[1]{\markright{\textsf{\arabic{section}. #1}}}
\renewcommand{\subsectionmark}[1]{}
\lhead{\textbf{\thepage} \ \ \nouppercase{\rightmark}}
\chead{}
\rhead{}
\lfoot{}
\cfoot{}
\rfoot{}
\setlength{\headheight}{14pt}

\linespread{1.03} % give a little extra room
\setlength{\parindent}{0.2in} % reduce paragraph indent a bit
\setcounter{secnumdepth}{2} % no numbered subsubsections
\setcounter{tocdepth}{2} % no subsubsections in ToC


%%%%%%%%%%%%%%%%%%%%%%%%%%%%%%%%%%%%%%%%%%%%%%%%%%%%%%%%%%%%%%%%%
% CUSTOM BOXES AND STUFF
\newtcolorbox{redbox}{colback=red!5!white,colframe=red!75!black, breakable}
\newtcolorbox{bluebox}{colback=blue!5!white,colframe=blue!75!black,breakable}

\newtcolorbox{dottedbox}[1][]{%
    colback=white,    % Background color
    colframe=white,    % Border color (to be overridden by dashrule)
    sharp corners,     % Sharp corners for the box
    boxrule=0pt,       % No actual border, as it will be drawn with dashrule
    boxsep=5pt,        % Padding inside the box
    enhanced,          % Enable advanced features
    breakable,         % Enables it to span multiple pages
    overlay={\draw[dashed, thin, black, dash pattern=on \pgflinewidth off \pgflinewidth, line cap=rect] (frame.south west) rectangle (frame.north east);}, % Dotted line
    #1                 % Additional options
}

% Define the colors
\definecolor{boxheader}{RGB}{0, 51, 102}  % Dark blue
\definecolor{boxfill}{RGB}{173, 216, 230}  % Light blue


% Define the tcolorbox environment
\newtcolorbox{mathdefinitionbox}[2][]{%
    colback=boxfill,   % Background color
    colframe=boxheader, % Border color
    fonttitle=\bfseries, % Bold title
    coltitle=white,     % Title text color
    title={#2},         % Title text
    enhanced,           % Enable advanced features
    breakable,
    attach boxed title to top left={yshift=-\tcboxedtitleheight/2}, % Center title
    boxrule=0.5mm,      % Border width
    sharp corners,      % Sharp corners for the box
    #1                  % Additional options
}
%%%%%%%%%%%%%%%%%%%%%%%%%


\definecolor{lightblue}{RGB}{173,216,230} % Light blue color
\definecolor{darkblue}{RGB}{0,0,139} % Dark blue color

% Define the custom proof environment
\newtcolorbox{ex}[2][Example]{
  colback=red!5!white, % Light blue background
  colframe=red!75!black, % Darker blue border
  coltitle=white, % Title color
  fonttitle=\bfseries, % Title font style
  title={{#2}},
  arc=1mm, % Rounded corners with 4mm radius,
  boxrule=0.5mm,
  left=2mm, right=2mm, top=2mm, bottom=2mm, % Padding inside the box
  breakable, % Allow box to be broken across pages
  before=\vspace{10pt}, % Padding above the box
  after=\vspace{10pt}, % Padding below the box
  before upper={\parindent15pt} % Ensure indentation
}

% Define the custom proof environment
\newtcolorbox{defn}[2][Definition]{
  colback=green!5!white, % Light blue background
  colframe=green!75!black, % Darker blue border
  coltitle=white, % Title color
  fonttitle=\bfseries, % Title font style
  title={{#2}},
  arc=1mm, % Rounded corners with 4mm radius,
  boxrule=0.5mm,
  left=2mm, right=2mm, top=2mm, bottom=2mm, % Padding inside the box
  breakable, % Allow box to be broken across pages
  before=\vspace{10pt}, % Padding above the box
  after=\vspace{10pt}, % Padding below the box
  before upper={\parindent15pt} % Ensure indentation
}


%%%%%%%%%%%%%%%%%%%%%%%%%%%%%%%%%%%%%%%%%%%%%%%%%%%%%%%%%%%%%%%%%


\begin{document}

% make title page
\thispagestyle{empty}
\bigskip \
\vspace{0.1cm}

\begin{center}
{\fontsize{22}{22} \selectfont Professor: James Analitis}
\vskip 16pt
{\fontsize{30}{30} \selectfont \bf \sffamily Physics 141A: Solid State Physics}
\vskip 24pt
{\fontsize{14}{14} \selectfont \rmfamily Homework 4} 
\vskip 6pt
{\fontsize{14}{14} \selectfont \ttfamily kdeoskar@berkeley.edu} 
\vskip 24pt
\end{center}

% {\parindent0pt \baselineskip=15.5pt \lipsum[1-4]} 

% make table of contents
% \newpage


\begin{bluebox}
  \textbf{Question 1:} \\
  \begin{center}
    Include figure
  \end{center}
  The low temperrature specific heat of a metal can be described by the equation $$c_v = \frac{\pi^2}{3} g(E_F) k_B^2 T + \frac{12\pi^4}{5} nk_B \left(\frac{T}{\Theta}\right)^3 $$
  \begin{enumerate}
    \item Explain the physical origin of the two terms.
    \item Using the free electron model, show that the density of states in three dimensions $g(E_F)$ is given by $$ g(E_F) = \frac{3}{2} \frac{n_e}{E_F} $$ where $n_e$ is the carrier density.
    \item In the Drude model we assumed that $n_e$ electrons were solely responsible for the conductivity in a metal. If we follow this model and assume that the carrier density at high temperature (145 K) is denoted $n_{HT}$ and the carrier density at low temperature is denoted $n_{LT}$ (say at 125 K). Extimate the ration $n_{LT} / n_{HT}$ from figure 1. Note that $10^4\text{ Oe} = 1T$ 
    \item The scattering time at high temperature is denoted $\tau_{HT}$ and the scattering rate at low temperature is denoted $\tau_{LT}$. Estimate $\tau_{LT} / \tau_{HT}$. 
    \item Calculate the Fermi energy above and below the phase transition. 
    \item $\mathrm{BaFe}_2 \mathrm{As}_2$ is a metal. Using your knowledge of patterns of properties in the periodic table, argue that most of the mobile electrons likely come from the Fe atoms.
    \item If we want to add more electrons to $\mathrm{BaFe}_2 \mathrm{As}_2$ we can replace its atoms with sp-called "dopants". What transition metal(s) would you choose to replace the Fe site to add electrons?
    \item As more electrons $\delta n$ are added, we expect the Sommerfeld coefficient $\frac{\pi^2}{3} g(E_F) k_B^2$ to increase. Define $c_v^0 = \frac{\pi^2}{3} g(E_{F0}) k_B^2$ where $E_{F0}$ is the Fermi energy when there are no dopants added. Show that $c_v$ will approximately increase as $$ c_v \approx c_v^0 + \frac{\pi^2}{3} \frac{\delta n}{2} \frac{k_B^2 T}{E_{F0}}  $$
    \item Show that if $\mathrm{BaFe}_2 \mathrm{As}_2$ was a two-dimensional, free electron metal, that the Sommerfeld coefficient would not be expected to increase when more electrons are added.  
  \end{enumerate}
\end{bluebox}

\vskip 0.5cm
\textbf{\underline{Solution:}}

\begin{enumerate}
  \item The term proportional to $T^1$ is due to the electrons in the metal, while the term proportional to $T^3$ is due to lattice vibrations (phonons).
  

  \item In the free electron model, with the metal assumed to be a box of volume $V = L^3$ and periodic boundary conditions, we have plane wavefunctions $e^{i\mathbf{k} \cdot \mathbf{r}}$ quantized as $$\mathbf{k} = \frac{2\pi}{L}\mathbf{n} = \frac{2\pi}{L} (n_1, n_2, n_3),~n_i \in \mathbb{N}$$ and energies described by $$ \epsilon(\mathbf{k}) = \frac{\hbar^2 |\mathbf{k}|^2}{2m} $$ with $m$ being the electron mass. Finally, we have the dispersion relation $\omega = v |\mathbf{k}|$.
  \\
  \\
  Then, the total number electrons in the system is $N$ and we have $$ N = \frac{2V}{(2\pi)^3} \int \mathbf{dk}~n_F(\beta(\epsilon(\mathbf{k}) - \mu)) $$ where $\mu$ is the chemical potential (the factor of 2 accounts for the two possible spins).
  \\
  \\
  The total energy of the system is $$ E_{tot} = \frac{2V}{(2\pi)^3} \int \mathbf{dk}~\epsilon(\mathbf{k}) \cdot n_F(\beta(\epsilon(\mathbf{k}) - \mu))  $$
  \\
  \\
  Using the fact that $\epsilon(\mathbf{k})$ only depends on $|\mathbf{k}| =\text{:} k$, we can switch from cartesian $\mathbf{k}$-coordinates to spherical coordinates as write $$ E_{tot} = \frac{2V}{(2\pi)^3} \cdot 4\pi \int_0 ^{\infty} dk~\epsilon(k) \cdot n_F(\beta(\epsilon(k) - \mu))  $$ Switching from $k$ to $\epsilon$ using $$ k = \sqrt{\frac{2m\epsilon}{\hbar^2}} \implies dk = \sqrt{\frac{m}{2\epsilon\hbar^2}} d\epsilon $$ we have 
  \begin{align*}
    E_{tot} &= \frac{2V}{2\pi^2} \int_0^{\infty} \left(\sqrt{\frac{m}{2\epsilon\hbar^2}} d\epsilon\right) \cdot \epsilon \cdot n_F(\beta(\epsilon - \mu)) \\
    &= V \int_{0}^{\infty} d\epsilon~\epsilon \cdot g(\epsilon) \cdot n_F(\beta(\epsilon - \mu))
  \end{align*} where we have density of states 
  \begin{align*}
    g(\epsilon) d\epsilon &= \frac{2}{(2\pi)^3} \cdot 4\pi k^2 dk \\
    &= \frac{2}{(2\pi)^3} \cdot 4\pi \cdot \left( \frac{2m\epsilon}{\hbar^2} \right) \cdot \left( \sqrt{\frac{m}{2\epsilon \hbar^2}} d\epsilon \right) \\
    &= \frac{(2m)^{3/2}}{2\pi^2 \hbar^3} \epsilon^{1/2}
  \end{align*} Now recall that the Fermi Energy was $$ E_F = \frac{\hbar^2 k_F^2}{2m}  $$ and $$ N = \frac{2V}{(2\pi)^3} \int \mathbf{dk}~n_F(\beta(\epsilon(\mathbf{k}) - \mu)) $$ The Fermi-Energy is defined at $T = 0$, and at absolute zero, the Fermi-Dirac Distribution $n_F(\beta(\epsilon - \mu))$ is a step function $\begin{cases}
    1,~k\leq k_F \\
    0,~k>k_F \\
  \end{cases}$. Thus,
  \begin{align*}
    & N = \frac{2V}{(2\pi)^3} \int_{k \leq k_F} \mathbf{dk} \\
    \implies& N = \frac{2V}{(2\pi)^3} \cdot \left(\frac{4}{3}\pi k_F^3\right) \\
    \implies& k_F = (3\pi^2 n)^{1/3},~~~n \text{:}= N/V \\
    \implies& E_F = \frac{\hbar^2 (3\pi^2 n)^{2/3} }{2m} \\
    \implies& \frac{(2m)^{3/2}}{\hbar^3} = \frac{3\pi^2 n}{E_F^{2/3}}
  \end{align*} and substituting this into the expression for $g(\epsilon)$ we obtain $$ \boxed{g(\epsilon) = \frac{3n_e}{2E_F} \left(\frac{\epsilon}{E_F}\right)^{1/2}}$$ where $n = n_e = $ Carrier density 
  \\
  \\
  Therefore, $$ \boxed{g(E_F) = \frac{3n_e}{2E_F} } $$


  \item We have carrier densities $n_{HT}, n_{LT}$ at 145K and 125K respectively. In the Drude model, $R_H = (-1)/ne \implies n = (-1)/(e \cdot R_H)$
  \begin{center}
    \includegraphics*[scale=0.45]{q1_figure.png}
  \end{center}
  So, looking at the Figure above, we estimate 
  \begin{align*}
    \frac{n_{LT}}{n_{HT}} &= \frac{(-1)}{e \cdot (R_H)_{LT}} \cdot \frac{e \cdot (R_H)_{HT}}{(-1)} \\
    &= \frac{(R_H)_{HT}}{(R_H)_{LT}} \\
    &= \frac{-0.012  \cdot 10^{-9} \mathrm{m \Omega cm/Oe} }{-0.038 \cdot 10^{-9} \mathrm{m \Omega cm/Oe}} \\
    &= 0.315789474
  \end{align*}
  \begin{note}
    {We don't need to convert the units from Oe to T, right?}
  \end{note}

  \item In the figure, $\rho_{ab}$ is the same as $\rho_{xx}$. We know $$ \rho_{xx} = \frac{m}{ne^2 \tau} \implies \tau = \frac{m}{ne^2 \rho_{xx}}  $$ So,
  \begin{align*}
    \frac{\tau_{LT}}{\tau_{HT}} &= \frac{n_{HT} \cdot \rho_{xx, HT}}{n_{LT} \cdot \rho_{xx, LT}} \\
    &= \frac{-0.012}{-0.038} \cdot \frac{-0.020}{-0.034}
  \end{align*}


  \item 
  
  \item Write this later.
  
  \item 
\end{enumerate}


\vskip 0.5cm
\hrule
\pagebreak


\begin{bluebox}
  \textbf{Question 2: Classical Model of Thermal Expansion} \\
  In classical statistical mechanics, we write the expectation of $x$ as $$ \mean{x}_{\beta} = \frac{\int dx~xe^{-\beta V(x)}}{\int dx~e^{-\beta V(x)}} $$ Although one cannot generally do such integrals for arbitrary potential $V(x)$ as in Eq. 8.1, one can expand the exponentials as $$ e^{-\beta V(x)} = e^{- \frac{\beta \kappa}{2} (x-x_0)^2 } \left[ 1 + \frac{\beta \kappa_3}{6} (x-x_0)^3 + \cdots \right] $$ and let limits of integration go to $\pm \infty$. \\
  $\triangleright$ Why is this expansion of the exponent and the extension of the limits of integration allowed? \\
  $\triangleright$ Use this expansion to derive $\mean{x}_{\beta}$ to lowest order in $\kappa_3$, and hence show that the coefficient of thermal expansion is $$ \alpha = \frac{1}{L} \frac{dL}{dT} \approx \frac{1}{x_0} \frac{d\mean{x}_{\beta}}{dT} = \frac{1}{x_0} \frac{k_B \kappa_3}{2\kappa^2} $$ with $k_B$ Boltzmann's constant. \\
  $\triangleright$ In what temperature range is the above expansion valid? \\
  $\triangleright$ While this model of thermal expansion in a solid is valid if there are only two atoms, why is it invalid for the case of a many-atom chain? 
\end{bluebox}

\vskip 0.5cm
\textbf{\underline{Solution:}}

$\triangleright$ We know we can do the expansion $$ V(x) = v(x_0) + \frac{\kappa}{2} (x-x_0)^2 + \frac{\kappa^3}{6} (x-x_0)^3 + \cdots $$ when we're very close to $x$. If we further have that $-\beta V(x) \approx \beta \cdot \left[v(x_0) + \frac{\kappa}{2} (x-x_0)^2 + \frac{\kappa^3}{6} (x-x_0)^3 + \cdots\right] $ is small, then we can carry out the expansion of the exponential. 
\\
\\
$\triangleright$ We want to determine the conditions under which calculating $$ \mean{x}_{\beta} = \lim_{b \rightarrow \infty} \frac{\int_{-b}^{+b} dx~xe^{-\beta V(x)}}{\int_{-b}^{+b} dx~e^{-\beta V(x)}} $$ with the approximation $$ e^{-\beta V(x)} = e^{-\frac{\beta \kappa}{2}(x-x_0)^2} \left[ 1 + \frac{\beta \kappa_3}{6}(x-x_0)^3 + \cdots \right] $$ is valid.
\\
\\
Define $y = (x-x_0)$. Then, we have 
\begin{align*}
  \mean{x}_{\beta} &= \frac{\int dy~(y+x_0)e^{e^{-\frac{\beta \kappa}{2}y^2} \left[ 1 + \frac{\beta \kappa_3}{6}y^3 + \cdots \right]  }}{\int dy~e^{e^{-\frac{\beta \kappa}{2}y^2} \left[ 1 + \frac{\beta \kappa_3}{6}y^3 + \cdots \right]  }} \\
  &= \frac{\int dy~y \cdot {e^{-\frac{\beta \kappa}{2}y^2} \left[ 1 + \frac{\beta \kappa_3}{6}y^3 + \cdots \right]  }}{\int dy~{e^{-\frac{\beta \kappa}{2}y^2} \left[ 1 + \frac{\beta \kappa_3}{6}y^3 + \cdots \right]  }} + x_0 ~~~ \text{Since $x_0$ is constant} \\
\end{align*} Now, we're trying to find the region of validity for the approximation where we drop the $\mathcal{O}(y^4)$ terms, so let's drop them and expand to get 
\begin{align*}
  \mean{x}_{\beta} &= \frac{\int dy~y \cdot {e^{-\frac{\beta \kappa}{2}y^2} \left[ 1 + \frac{\beta \kappa_3}{6}y^3 \right]  }}{\int dy~{e^{-\frac{\beta \kappa}{2}y^2} \left[ 1 + \frac{\beta \kappa_3}{6}y^3 \right]  }} + x_0 ~~~ \text{Since $x_0$ is constant}  \\
  &= x_0 + \frac{\beta \kappa_3}{6} \frac{\int dy~y^4{e^{-\frac{\beta \kappa}{2}y^2} }}{\int dy~{e^{-\frac{\beta \kappa}{2}y^2} }} 
\end{align*} Here, we've simplified the expression by using the fact that the integral of an odd function over an interval symmetric about $0$ is zero.
\\
\\
Then, using the fact that $$ \int dy~e^{-ay^2} = \sqrt{\frac{\pi}{a}} $$ and Feynman's trick we have
\begin{align*}
  \int dy~y^4 e^{-ay^2} &= \int dy ~ \frac{\partial^2}{\partial a^2} \left(e^{-ay^2}\right) \\
  &= \frac{d^2}{da^2} \int dy~e^{-ay^2} \\
  &= \frac{d^2}{da^2} \left(\sqrt{\frac{\pi}{a}}\right) \\
  &= \frac{3}{4} \sqrt{\frac{\pi}{a^5}}
\end{align*} Substituting this in, we get 
\begin{align*}
  \mean{x}_{\beta} &= x_0 + \frac{\beta \kappa_3}{6} \cdot \frac{\frac{3}{2}\sqrt{\frac{\pi}{(\beta \kappa/2)^5}}}{\sqrt{\frac{\pi}{(\beta \kappa/2)}}} \\
  &= x_0 + \frac{\kappa_3 (k_B T)}{2\kappa^2}
\end{align*} Thus,
$$ \boxed{\frac{1}{x_0} \cdot \frac{d\mean{x}_{\beta}}{dT}  = \frac{1}{x_0} \frac{k_B \kappa_3}{2\kappa^2} } $$\\
$\triangleright$ In order for the result above to be valid, we need the cubic term to be much less important in contribution than the leading term.
\\
\\
And since we found $\mean{x}_{\beta} \sim k_B T$ we need the leading term $\kappa(x-x_0)^2 \sim k_B T \implies |x-x_0| \sim \sqrt{\frac{k_B T}{\kappa}}$. Then, we need 
\begin{align*}
  &\kappa|x-x_0|^2 >> \kappa_3|x-x_0|^3 \\
  \implies& k_B T >> \kappa_3 \left(\sqrt{\frac{k_B T}{\kappa}}\right)^{3/2} \\
  \implies& k_B T << \frac{\kappa^3}{\kappa_3^2}
\end{align*} $\triangleright$ For a many-atom chain, the quadratic term in the potential would introduce interaction (non-linear) terms, which would change the expression.




\vskip 0.5cm
\hrule
\pagebreak


\begin{bluebox}
  \textbf{Question 3: Normal Modes of a One-Dimensional Monoatomic Chain}\\
  \begin{enumerate}[label=(\alph*)]
    \item Explain what is meant by "normal mode" and by "phonon".
    \item Skip derivation of dispersion relation $\omega(k)$ since it's done in the book and lectures.
    \item Show that the mode with wavevector $k$ has the same pattern of mass displacements aas the node with wavevector $k + 2\pi/a$. Hence show that the dispersion relation is periodic in reciprocal space (k-space).\\
    $\triangleright$ How many \textit{different} normal modes are there? 
    \item Derive the phase and group velocities and sketch them as a function of $k$.\\
    $\triangleright$ What is the sound velocity? \\
    $\triangleright$ Show that the sound velocity iss also given by $v_s = 1/\sqrt{\beta \rho}$ where $\rho$ is the chain density and $\beta$ is the compressibility.
    \item Find the expression for $g(\omega)$, the density of states of modes per angular frequency. \\
    $\triangleright$ Sketch $g(\omega)$.
  \end{enumerate} 
\end{bluebox}

\vskip 0.5cm
\textbf{\underline{Solution:}}

\begin{enumerate}[label=(\alph*)]
  \item A normal mode is a collective oscillation of the atoms/particles in which all particles oscillate with the same frequency, usually denoted $\omega$. A Phonon is a \textit{quantum of vibration} i.e. an excitation of a normal mode up the QHO ladder. This is analogous to how a Photon is an excitation/quantum of an EM Wave.
  

  \item Skip this.
  

  \item Consider modes with wavevectors $k$ and $k + 2\pi/a$.
  \\
  \\
  The mass displacement of lattice point $x_n$ is $\delta x_n = Ae^{i\omega t  - ikna}$. Taking $k \rightarrow k + 2\pi/a$ we get 
  \begin{align*}
    \delta x_n &= Ae^{i\omega t - i(k + 2\pi/a)na} \\
    &= Ae^{i\omega t - ikna } \cdot e^{-i2\pi n} 
  \end{align*}
  and $e^{2\pi(-1)n} = 1$ so the mass displacements are the same.
  \\
  \\
  $\triangleright$ How many normal modes are there? \\
  Recall that if we assume periodic boundary conditions then $$ k = \frac{2nm}{L} $$ but $k$ is identified with $k + 2\pi/a$ so there are exactly $N = L/a$ different normal modes.

  \item The phase and group velocities are 
  \begin{align*}
  v_{phase} &= \frac{\omega(k)}{k} = 2 \sqrt{\frac{\kappa}{m}} \frac{\left| \sin\left(\frac{ka}{2}\right) \right|}{k} \\
  v_{group} &= \frac{d\omega(k)}{dk} = \sqrt{\frac{\kappa}{m}} \cdot a \cos\left(\frac{|k|a}{2}\right) = \frac{a}{2} \omega_0 \sqrt{1 - \frac{\omega^2}{\omega_0^2}} 
  \end{align*} where $\omega_0 = 2 \sqrt{\kappa/m}$
\end{enumerate}

\vskip 0.5cm
\hrule
\pagebreak








% \begin{bluebox}
%   \textbf{Question 1:} 
% \end{bluebox}

% \vskip 0.5cm
% \textbf{\underline{Solution:}}
% \\
% \\
% text
% \vskip 0.5cm
% \hrule
% \pagebreak



% %%%%%%%%%%%%%%%%%%%%%%%%%%%%%%%%%%%%%%%%%%%%%%
% \newpage
% % \section{References}
% %%%%%%%%%%%%%%%%%%%%%%%%%%%%%%%%%%%%%%%%%%%%%%
% \vskip 0.5cm
% \bibliographystyle{plain} % We choose the "plain" reference style
% \bibliography{citation}




\end{document}










