\documentclass[11pt]{article}

% basic packages
\usepackage[margin=1in]{geometry}
\usepackage[pdftex]{graphicx}
\usepackage{amsmath,amssymb,amsthm}
\usepackage{custom}
\usepackage{lipsum}

\usepackage{xcolor}
\usepackage{tikz}

\usepackage[most]{tcolorbox}
\usepackage{xcolor}
\usepackage{mdframed}

% page formatting
\usepackage{fancyhdr}
\pagestyle{fancy}

\renewcommand{\sectionmark}[1]{\markright{\textsf{\arabic{section}. #1}}}
\renewcommand{\subsectionmark}[1]{}
\lhead{\textbf{\thepage} \ \ \nouppercase{\rightmark}}
\chead{}
\rhead{}
\lfoot{}
\cfoot{}
\rfoot{}
\setlength{\headheight}{14pt}

\linespread{1.03} % give a little extra room
\setlength{\parindent}{0.2in} % reduce paragraph indent a bit
\setcounter{secnumdepth}{2} % no numbered subsubsections
\setcounter{tocdepth}{2} % no subsubsections in ToC


%%%%%%%%%%%%%%%%%%%%%%%%%%%%%%%%%%%%%%%%%%%%%%%%%%%%%%%%%%%%%%%%%
% CUSTOM BOXES AND STUFF
\newtcolorbox{redbox}{colback=red!5!white,colframe=red!75!black, breakable}
\newtcolorbox{bluebox}{colback=blue!5!white,colframe=blue!75!black,breakable}

\newtcolorbox{dottedbox}[1][]{%
    colback=white,    % Background color
    colframe=white,    % Border color (to be overridden by dashrule)
    sharp corners,     % Sharp corners for the box
    boxrule=0pt,       % No actual border, as it will be drawn with dashrule
    boxsep=5pt,        % Padding inside the box
    enhanced,          % Enable advanced features
    breakable,         % Enables it to span multiple pages
    overlay={\draw[dashed, thin, black, dash pattern=on \pgflinewidth off \pgflinewidth, line cap=rect] (frame.south west) rectangle (frame.north east);}, % Dotted line
    #1                 % Additional options
}

% Define the colors
\definecolor{boxheader}{RGB}{0, 51, 102}  % Dark blue
\definecolor{boxfill}{RGB}{173, 216, 230}  % Light blue


% Define the tcolorbox environment
\newtcolorbox{mathdefinitionbox}[2][]{%
    colback=boxfill,   % Background color
    colframe=boxheader, % Border color
    fonttitle=\bfseries, % Bold title
    coltitle=white,     % Title text color
    title={#2},         % Title text
    enhanced,           % Enable advanced features
    breakable,
    attach boxed title to top left={yshift=-\tcboxedtitleheight/2}, % Center title
    boxrule=0.5mm,      % Border width
    sharp corners,      % Sharp corners for the box
    #1                  % Additional options
}
%%%%%%%%%%%%%%%%%%%%%%%%%


\definecolor{lightblue}{RGB}{173,216,230} % Light blue color
\definecolor{darkblue}{RGB}{0,0,139} % Dark blue color

% Define the custom proof environment
\newtcolorbox{ex}[2][Example]{
  colback=red!5!white, % Light blue background
  colframe=red!75!black, % Darker blue border
  coltitle=white, % Title color
  fonttitle=\bfseries, % Title font style
  title={{#2}},
  arc=1mm, % Rounded corners with 4mm radius,
  boxrule=0.5mm,
  left=2mm, right=2mm, top=2mm, bottom=2mm, % Padding inside the box
  breakable, % Allow box to be broken across pages
  before=\vspace{10pt}, % Padding above the box
  after=\vspace{10pt}, % Padding below the box
  before upper={\parindent15pt} % Ensure indentation
}

% Define the custom proof environment
\newtcolorbox{defn}[2][Definition]{
  colback=green!5!white, % Light blue background
  colframe=green!75!black, % Darker blue border
  coltitle=white, % Title color
  fonttitle=\bfseries, % Title font style
  title={{#2}},
  arc=1mm, % Rounded corners with 4mm radius,
  boxrule=0.5mm,
  left=2mm, right=2mm, top=2mm, bottom=2mm, % Padding inside the box
  breakable, % Allow box to be broken across pages
  before=\vspace{10pt}, % Padding above the box
  after=\vspace{10pt}, % Padding below the box
  before upper={\parindent15pt} % Ensure indentation
}


%%%%%%%%%%%%%%%%%%%%%%%%%%%%%%%%%%%%%%%%%%%%%%%%%%%%%%%%%%%%%%%%%


\begin{document}

% make title page
\thispagestyle{empty}
\bigskip \
\vspace{0.1cm}

\begin{center}
{\fontsize{22}{22} \selectfont Professor: James Analitis}
\vskip 16pt
{\fontsize{30}{30} \selectfont \bf \sffamily Physics 141A: Solid State Physics}
\vskip 24pt
{\fontsize{14}{14} \selectfont \rmfamily MT2 Review} 
\vskip 6pt
{\fontsize{14}{14} \selectfont \ttfamily kdeoskar@berkeley.edu} 
\vskip 24pt
\end{center}

% {\parindent0pt \baselineskip=15.5pt \lipsum[1-4]} 

% make table of contents
\tableofcontents
\newpage

\section{Chapter 5: The Periodic Table}

\begin{itemize}
  \item To solve for the dynamics of a quantum mechanical system we need to solve Schroedinger's Equation $\hat{H}\psi = E \psi$ but this is intractable for atoms with a dozen-or-so electrons, let alone solids with $~10^{23}$ particles in them.
  \item Instead, we try to draw conclusions from simplified models of atoms.
  \item Recall that an electron in an atom is described by a set of quantum numbers $\ket{n, l, l_z, s_z}$ where
  \begin{align*}
    n &= 1,2,3,\cdots \\
    l &= 0,1,\cdots,n-1 \\
    l_z &= -l,\cdots,0,\cdots,l \\
    s_z &= \pm \frac{1}{2}
  \end{align*}
  \item \textbf{Aufbau Principle:}
  \item \textbf{Madelung's Rule:}
  \item \textbf{Effective Nuclear Charge}
  \item \textbf{Ionization Energy}
  \item \textbf{Electron Affinity}
\end{itemize}

\newpage
\section{Chapter 6: Chemical Bonding}
There are a number of different ways atoms bond together to form molecules and materials.

\subsection{Ionic Bonding}
\begin{itemize}
  \item Essentially, the complete transfer of one (or more) electron(s) from one atom to another. Eg. $Na + Cl \rightarrow Na^{+} + Cl^{-} \rightarrow NaCl$
  \item We define the following: 
  \begin{align*}
    \text{Ionization Energy } &= \text{ Energy required to remove one electron from a neutral atom} \\
    \text{Electron Affinity } &= \text{ Energy gain resulting from adding an electron to a neutral atom} 
  \end{align*}
  \item The energy required to transfer an electron from $A$ to $B$ (when the two atoms are far apart) is $$\Delta_{A + B \rightarrow A^+ + B^{-}} = \text{(Ionization Energy)}_{A} + \text{(Electron Affinity)}_B $$
  \item However that's not the full story. We also have a \textbf{Cohesive Energy} which is a classical result of the coulomb interactions between the two ions as they get closer to each other, which is $$ \text{Cohesive Energy of AB } = \text{ Energy gain from } A+B \rightarrow A^+ + B^- $$
  \item Thus, $$ \Delta E_{A + B \rightarrow AB} = \Delta E_{A + B \rightarrow A^+ + B^-} + \text{ Energy gain from } A^+ + B^- $$ i.e. $$ \boxed{\Delta E_{A+B \rightarrow AB} = \text{(Ionization Energy)}_A - \text{(Electron Affinity)}_{B} - \text{(Cohesive Energy of AB)} }$$
  \item \textbf{An Ionic Bond forms if $\Delta E_{A+B \rightarrow AB} < 0$. }
  \item \underline{Note:} Why $\Delta < 0$? Due to our sign convention.
  \begin{itemize}
    \item Ionization Energy is energy we put into the system, but we denote it as $+$ve.
    \item Electron Affinity is energy that comes out of the system, but we denote it as $-$ve.
    \item Similarly, Cohesive Energy is energy gain so it's energy that comes out of the system. Thus, just like Electron Affinity, we denote it as $-$ve.
    \item So, when electron affinity overpowers ionization energy i.e. more energy comes out than we put in, we have $\Delta E < 0$ and this is Energetically Favorable.
  \end{itemize}
\end{itemize}

\subsection{Covalent Bonding}

\subsection{Wan der Vaal's forces}

\subsection{Metallic Bonding}

\subsection{Hydrogen Bonding}








% \begin{bluebox}
%   \textbf{Question 1:} 
% \end{bluebox}

% \vskip 0.5cm
% \textbf{\underline{Solution:}}
% \\
% \\
% text
% \vskip 0.5cm
% \hrule
% \pagebreak



% %%%%%%%%%%%%%%%%%%%%%%%%%%%%%%%%%%%%%%%%%%%%%%
% \newpage
% % \section{References}
% %%%%%%%%%%%%%%%%%%%%%%%%%%%%%%%%%%%%%%%%%%%%%%
% \vskip 0.5cm
% \bibliographystyle{plain} % We choose the "plain" reference style
% \bibliography{citation}




\end{document}










