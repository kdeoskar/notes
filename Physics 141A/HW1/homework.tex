\documentclass[11pt]{article}

% basic packages
\usepackage[margin=1in]{geometry}
\usepackage[pdftex]{graphicx}
\usepackage{amsmath,amssymb,amsthm}
\usepackage{custom}
\usepackage{lipsum}

\usepackage{xcolor}
\usepackage{tikz}

\usepackage[most]{tcolorbox}
\usepackage{xcolor}
\usepackage{mdframed}

% page formatting
\usepackage{fancyhdr}
\pagestyle{fancy}

\renewcommand{\sectionmark}[1]{\markright{\textsf{\arabic{section}. #1}}}
\renewcommand{\subsectionmark}[1]{}
\lhead{\textbf{\thepage} \ \ \nouppercase{\rightmark}}
\chead{}
\rhead{}
\lfoot{}
\cfoot{}
\rfoot{}
\setlength{\headheight}{14pt}

\linespread{1.03} % give a little extra room
\setlength{\parindent}{0.2in} % reduce paragraph indent a bit
\setcounter{secnumdepth}{2} % no numbered subsubsections
\setcounter{tocdepth}{2} % no subsubsections in ToC


%%%%%%%%%%%%%%%%%%%%%%%%%%%%%%%%%%%%%%%%%%%%%%%%%%%%%%%%%%%%%%%%%
% CUSTOM BOXES AND STUFF
\newtcolorbox{redbox}{colback=red!5!white,colframe=red!75!black, breakable}
\newtcolorbox{bluebox}{colback=blue!5!white,colframe=blue!75!black,breakable}

\newtcolorbox{dottedbox}[1][]{%
    colback=white,    % Background color
    colframe=white,    % Border color (to be overridden by dashrule)
    sharp corners,     % Sharp corners for the box
    boxrule=0pt,       % No actual border, as it will be drawn with dashrule
    boxsep=5pt,        % Padding inside the box
    enhanced,          % Enable advanced features
    breakable,         % Enables it to span multiple pages
    overlay={\draw[dashed, thin, black, dash pattern=on \pgflinewidth off \pgflinewidth, line cap=rect] (frame.south west) rectangle (frame.north east);}, % Dotted line
    #1                 % Additional options
}

% Define the colors
\definecolor{boxheader}{RGB}{0, 51, 102}  % Dark blue
\definecolor{boxfill}{RGB}{173, 216, 230}  % Light blue


% Define the tcolorbox environment
\newtcolorbox{mathdefinitionbox}[2][]{%
    colback=boxfill,   % Background color
    colframe=boxheader, % Border color
    fonttitle=\bfseries, % Bold title
    coltitle=white,     % Title text color
    title={#2},         % Title text
    enhanced,           % Enable advanced features
    breakable,
    attach boxed title to top left={yshift=-\tcboxedtitleheight/2}, % Center title
    boxrule=0.5mm,      % Border width
    sharp corners,      % Sharp corners for the box
    #1                  % Additional options
}
%%%%%%%%%%%%%%%%%%%%%%%%%


\definecolor{lightblue}{RGB}{173,216,230} % Light blue color
\definecolor{darkblue}{RGB}{0,0,139} % Dark blue color

% Define the custom proof environment
\newtcolorbox{ex}[2][Example]{
  colback=red!5!white, % Light blue background
  colframe=red!75!black, % Darker blue border
  coltitle=white, % Title color
  fonttitle=\bfseries, % Title font style
  title={{#2}},
  arc=1mm, % Rounded corners with 4mm radius,
  boxrule=0.5mm,
  left=2mm, right=2mm, top=2mm, bottom=2mm, % Padding inside the box
  breakable, % Allow box to be broken across pages
  before=\vspace{10pt}, % Padding above the box
  after=\vspace{10pt}, % Padding below the box
  before upper={\parindent15pt} % Ensure indentation
}

% Define the custom proof environment
\newtcolorbox{defn}[2][Definition]{
  colback=green!5!white, % Light blue background
  colframe=green!75!black, % Darker blue border
  coltitle=white, % Title color
  fonttitle=\bfseries, % Title font style
  title={{#2}},
  arc=1mm, % Rounded corners with 4mm radius,
  boxrule=0.5mm,
  left=2mm, right=2mm, top=2mm, bottom=2mm, % Padding inside the box
  breakable, % Allow box to be broken across pages
  before=\vspace{10pt}, % Padding above the box
  after=\vspace{10pt}, % Padding below the box
  before upper={\parindent15pt} % Ensure indentation
}


%%%%%%%%%%%%%%%%%%%%%%%%%%%%%%%%%%%%%%%%%%%%%%%%%%%%%%%%%%%%%%%%%


\begin{document}

% make title page
\thispagestyle{empty}
\bigskip \
\vspace{0.1cm}

\begin{center}
{\fontsize{22}{22} \selectfont Professor: James Analitis}
\vskip 16pt
{\fontsize{30}{30} \selectfont \bf \sffamily Physics 141A: Solid State Physics}
\vskip 24pt
{\fontsize{14}{14} \selectfont \rmfamily Homework 1} 
\vskip 6pt
{\fontsize{14}{14} \selectfont \ttfamily kdeoskar@berkeley.edu} 
\vskip 24pt
\end{center}

% {\parindent0pt \baselineskip=15.5pt \lipsum[1-4]} 

% make table of contents
% \newpage

\begin{bluebox}
  \textbf{Question 1:} Show that the Bose occupation factor for the number $\mean{N}$ of excited vibrational "particles" (modes) is 
  $$ \frac{1}{e^{\hbar \omega \beta} - 1} $$
  where $\beta = 1/k_B T$ and $\hbar \omega$ is the energy of this vibrational mode.
\end{bluebox}

\vskip 0.5cm
\textbf{\underline{Solution:}}\\
In the Einstein model of a solid, an excited vibrational "particle" or mode can have energies quantized as $$ E_n = \left( n + \frac{1}{2} \right) \hbar \omega $$ so the probability that a mode $N$ is occupied is proportional to 
$$ e^{-\beta E_n } = e^{-\beta \left(N + 1/2\right)\hbar \omega}$$
The normalization factor would be the partition function 
$$ Z = \sum_{n = 0}^{\infty} e^{-\beta E_n} = \sum_{n = 0}^{\infty} e^{-\beta \hbar \omega \left(n + 1/2\right)} = e^{-\beta \hbar / 2} \sum_{n = 0}^{\infty} e^{- n \beta \hbar \omega}$$
i.e. 
$$ P\left(\text{Mode } N \text{ is occupied} \right) = \frac{e^{-\beta E_n}}{Z} = \frac{e^{- N \beta \hbar \omega} \cdot  e^{-\beta \hbar / 2}}{ e^{-\beta \hbar / 2} \cdot \sum_{n = 0}^{\infty} e^{- n \beta \hbar \omega} } = \frac{e^{-N\beta\hbar\omega}}{\sum_{n = 0}^{\infty} e^{- n \beta \hbar \omega}}$$
\\
\\
and the sum in the denominator is a geometric series so we have 
\begin{align*}
  P\left(\text{Mode } N \text{ is occupied}\right) &= e^{-N\beta\hbar\omega} \cdot \left(1 - e^{-\beta \hbar \omega}\right)
\end{align*} Now, the expected number of occupied vibrational modes can be expressed as 
\begin{align*}
  \mean{N} &= \sum_{N = 0}^{\infty} N \cdot P(\text{Mode } N \text{ is occupied}) \\
  \implies \mean{N} &= \sum_{N = 0}^{\infty} N \cdot \frac{e^{-\beta E_N}}{Z} \\
  &= \frac{1}{Z_1} \sum_{N = 0}^{\infty} Ne^{-\beta \hbar \omega \left(N \right)} \\
  &= \frac{1}{Z_1} \cdot \left( - \frac{1}{\hbar \omega} \frac{\partial Z_1}{\partial \beta} \right) \\
\end{align*} where $$Z_1 = \sum_{N = 0}^{\infty} e^{-N\beta \hbar \omega} = \frac{1}{1-e^{-\beta\hbar\omega}} $$ and we can use this since the $-\beta \hbar \omega (1/2)$ factor is in both the boltzmann factor and the partition function $Z$ so it can be cancelled.
\\
\\
Then,
\begin{align*}
  \frac{\partial Z_1}{\partial \beta} &= \left(-\frac{1}{(1-e^{-\beta\hbar\omega})^2} \right) \cdot \left(\hbar \omega e^{-\beta \hbar \omega}\right) \\
  &=-\hbar \omega \left( \frac{e^{-\beta \hbar \omega}}{\left(1-e^{-\beta \hbar \omega}\right)^2} \right)
\end{align*}
\\
\\
Thus, 
\begin{align*}
  \mean{N} &= -\frac{1}{\hbar \omega} \cdot \frac{1}{Z_1} \cdot \frac{\partial Z_1}{\partial \beta} \\
  &= -\frac{1}{\hbar \omega} \cdot \left(1-e^{-\beta\hbar\omega}\right) \cdot \left(-\hbar \omega\right) \left( \frac{e^{-\beta \hbar \omega}}{\left(1-e^{-\beta \hbar \omega}\right)^2} \right) \\
  &= \frac{e^{-\beta \hbar \omega}}{1-e^{-\beta \hbar \omega}} 
\end{align*}
So, multiplying the numberator and denominator by $e^{+\beta \hbar \omega}$ we get $$ \boxed{\mean{N} = \frac{1}{e^{\beta \hbar \omega - 1}} } $$


\vskip 0.5cm
\hrule
\pagebreak





\begin{bluebox}
  \textbf{Question 2:} Use the Debye approximation to determine the heat capacity of a two dimensional solid as a function of temperature.
  \begin{itemize}
    \item State your assumptions.
    \\
    You will need to leave your answer in terms of an integral that one cannot do analytically.

    \item At high $T$, show that the heat capacity goes to a constant and find that constant.

    \item At low $T$, show that $C_v = KT^n$ and find $n$. Find $K$ in terms of a definite integral. 
    \\
    If you are brace, you can try to evaluate the integral, but you will need to leave your result in terms of the Riemann Zeta function.
  \end{itemize}
\end{bluebox}

\vskip 0.5cm
\textbf{\underline{Solution:}}
\\
\\
\textbf{Assumptions:}
\begin{enumerate}
  \item Energy is carried in waves and the energy of a given mode is described as $$E_n = \hbar \omega(\mathbf{k}) \left(n + \frac{1}{2}\right) $$
  \item We have the dispersion relation $\omega = v|\mathbf{k}|$ where $\mathbf{k} = k_x \mathbf{x} + k_y \mathbf{y}$
  \item We use periodic boundary-conditions with box length $L$, which restricts our $k_x, k_y$ values so that 
  $$ \vec{k} = \frac{2\pi}{L} (n_x, n_y) $$ where $n_x, n_y$ are integers.
  \item We can take the continuum limit and convert the sum into an integral
  $$ \sum_{k} \rightarrow \left(\frac{L}{2\pi}\right)^2 \int \mathrm{d}\mathbf{k} $$
\end{enumerate}
So, since we have two modes for the sound wave now, we have 
\begin{align*}
  \mean{E} &= 2 \cdot \left(\frac{L}{2\pi}\right)^2 \int \mathrm{d}\mathbf{k} \; \hbar \omega(\mathbf{k}) \left( n_B(\omega(\mathbf{k})) + \frac{1}{2} \right)
\end{align*} We can convert to polar coordinates are get 
\begin{align*}
  \mean{E} &= 2 \cdot \left(\frac{L}{2\pi}\right)^2 \int_{0}^{2\pi} \int_{0}^{\infty} k \mathrm{d} k \mathrm{d} \theta \; \hbar \omega(k) \left( n_B(\omega(k)) + \frac{1}{2}  \right) \\
  &= 2 \cdot \left(\frac{L}{2\pi}\right)^2 \int_{0}^{2\pi} \mathrm{d} \theta \int_{0}^{\infty} k \mathrm{d} k  \; \hbar \omega(k) \left( n_B(\omega(k)) + \frac{1}{2}  \right) \\
  &= 4\pi \cdot \left(\frac{L}{2\pi}\right)^2 \int_{0}^{\infty} k \mathrm{d} k  \; \hbar \omega(k) \left( n_B(\omega(k)) + \frac{1}{2}  \right) \\
\end{align*}

We can use the dispersion relation $\omega = vk$ to write everything in terms of $\omega$:

\begin{align*}
  \mean{E} &= \frac{L^2}{\pi} \int_{0}^{\infty} \left(\frac{\omega}{v}\right) \left(\frac{\mathrm{d}\omega}{v}\right) \hbar \omega \left( n_B(\omega) + \frac{1}{2} \right) \\
\end{align*}

and since we're actually interested in the heat capacity, we can drop the temperature independent term we get from the $+\frac{1}{2}$ i.e. we'll abuse notation and write 

\begin{align*}
  \mean{E} &= \frac{L^2}{\pi} \int_{0}^{\infty} \left(\frac{\omega}{v}\right) \left(\frac{\mathrm{d}\omega}{v}\right) \hbar \omega \left( n_B(\omega)\right) \\
\end{align*} or, in other words, we have 

\begin{align*}
  \mean{E} &= \frac{L^2}{\pi} \int_{0}^{\infty}\mathrm{d}\omega \;g(\omega)\; \hbar \omega \left( n_B(\beta \hbar \omega)\right) \\
\end{align*} where $$ g(\omega) = \frac{L^2 \omega}{\pi v^2} $$ is the density of states.
\\
\\
Now, physically, it doesn't make sense for infinitely many modes to be occupied, so we introduce a cutoff-frequency $\omega_c$ defined as $$ 2N = \int_{0}^{\omega_c} g(\omega) \mathrm{d}\omega $$ (there should be $2N$ modes of oscillation)
\\
\\
In any case, using the cut-off frequency and substituting in the Bose-Einstein distribution, we have
$$ \mean{E} = \frac{L^2 \hbar}{\pi v^2} \int_{0}^{\omega_c} \mathrm{d}\omega\; \frac{\omega^2}{e^{\beta \hbar \omega} - 1}$$
\underline{High-Temperature limit:}\\
In the limit $T \rightarrow 0$, we have $\beta \rightarrow 0$ so 
\begin{align*}
  \frac{1}{e^{\beta \hbar \omega} - 1} &\approx \frac{1}{\left(1 + \beta \hbar \omega \right) - 1 } \\
  &= \frac{1}{\beta \hbar \omega} \\
  &= \frac{k_B T}{\hbar \omega}
\end{align*}

So, 
\begin{align*}
  \mean{E} &= \frac{L^2 \hbar}{\pi v^2} \int_{0}^{\omega_c} \omega^2 \cdot \frac{k_B T}{\hbar \omega} \mathrm{d}\omega \\
  &= \int_{0}^{\omega_c} \frac{L^2 \omega}{\pi v^2} \hbar \omega \frac{k_B T}{\hbar \omega} \mathrm{d}\omega \\
  &= k_B T \int_{0}^{\omega_c} g(\omega) \mathrm{d} \omega \\
  &= k_B T \cdot 2N
\end{align*}

Therefore, the Heat Capacity in the high-temperature limit is $$ \boxed{C_v = 2N k_B} $$
\\
\\
\underline{Low-Temperature limit:}
In the low-temperature limit, introducing the cut-off frequency $\omega_c$ doesn't impact the integral much because the bose-factor will cause the integrand to vanish before we even reach $\omega_c$. Thus, we can integrate all the way to infinity instead.
\\
\\
So,
\begin{align*}
  \mean{E} &= \frac{L^2 \hbar}{\pi v^2} \int_{0}^{\infty} \frac{\omega^2}{e^{\beta \hbar \omega} - 1} \mathrm{d}\omega
\end{align*}

Using the substitution $x = \beta \hbar \omega$ we have 
\begin{align*}
  \mean{E} &= \frac{L^2 \hbar}{\pi v^2} \int_{0}^{\infty}  \left( \frac{\mathrm{d}x}{\beta \hbar} \right) \left( \frac{x}{\beta \hbar} \right)^2 \frac{1}{x^x - 1} \\
  \mean{E} &= \frac{L^2 \hbar}{\pi v^2} \left(\frac{1}{\beta \hbar}\right)^3 \int_{0}^{\infty}  \mathrm{d}x \frac{x^2}{x^x - 1} \\
  \mean{E} &= \frac{L^2 \hbar}{\pi v^2} \left(\frac{k_B T}{\hbar}\right)^3 \int_{0}^{\infty}  \mathrm{d}x \frac{x^2}{x^x - 1} \\
\end{align*}

Thus, the heat capacity is given by 
\begin{align*}
  C &= \frac{d\mean{E}}{dT} \\
  &= \frac{L^2 \hbar}{\pi v^2} \left(\frac{k_B}{\hbar}\right)^3 \cdot (3T^2) \cdot \int_{0}^{\infty}  \mathrm{d}x \frac{x^2}{x^x - 1} \\
  &= KT^n
\end{align*} where $$ \boxed{ K = \frac{3 L^2 \hbar}{\pi v^2} \left(\frac{k_B}{\hbar}\right)^3 \cdot \int_{0}^{\infty}  \mathrm{d}x \frac{x^2}{x^x - 1} } $$ and $$ \boxed{n = 3} $$


\vskip 0.5cm
\hrule
\pagebreak





\begin{bluebox}
  \textbf{Question 3:} 
  \begin{itemize}
    \item What is the mean kinetic energy in eV at room temperature of a gaseous (a) He atom (b) Xe atom (c) Ar atom and (d) Kr atom? The gas is classical.
    \item Explain the following values of the molar heat capacity $JK^{-1} mol^{-1}$ all measured at constant pressuer at $298 K$: $$ \mathrm{Al} = 24.35, \mathrm{Pb} = 26.44,  \mathrm{N}_2 = 29.13, \mathrm{Ar} = 20.79, \mathrm{O}_2 = 29.36 $$
  \end{itemize}
\end{bluebox}

\vskip 0.5cm
\textbf{\underline{Solution:}}
\\
\\
\begin{itemize}
  \item The mean kinetic energy is given by the Equipartition Theorem: $$ \mathrm{KE} = \frac{f}{2} k_B T $$\\
  Since He, Xe, Ar, Ke are all mono-atomic and have $3$ degrees of freedom they all have the same mean kinetic energy at $T = 298 K$
  $$ \mathrm{KE} = \frac{3}{2} k_B T = 0.03852 eV $$
  
  \item Since Kinetic Energy (of one atom) is $\frac{f}{2} k_B T$, the heat capacity of an ideal gas atom is $C = \frac{f}{2} k_B$. For an entire mole, we have $C = \frac{f}{2} N k_B = \frac{f}{2} R$ where $R$ is the universal gas constant. 
  \\
  \\
  Since $\mathrm{N}_2, \mathrm{O}_2$ are diatomic they have more degrees of freedom ( it has quadratic degrees of freedom) than $\mathrm{Ar}$ and thus have higher molar heat capacity.
  \\
  \\
  $\mathrm{Al}$ and $\mathrm{Pb}$ are metals so they cannot be treated accurately by the Ideal Gas model because many-particle interactions are important in metals.
\end{itemize}
\vskip 0.5cm
\hrule
\pagebreak






% \begin{bluebox}
%   \textbf{Question 1:} 
% \end{bluebox}

% \vskip 0.5cm
% \textbf{\underline{Solution:}}
% \\
% \\
% text
% \vskip 0.5cm
% \hrule
% \pagebreak



% %%%%%%%%%%%%%%%%%%%%%%%%%%%%%%%%%%%%%%%%%%%%%%
% \newpage
% % \section{References}
% %%%%%%%%%%%%%%%%%%%%%%%%%%%%%%%%%%%%%%%%%%%%%%
% \vskip 0.5cm
% \bibliographystyle{plain} % We choose the "plain" reference style
% \bibliography{citation}




\end{document}










