\documentclass[11pt]{article}

% basic packages
\usepackage[margin=1in]{geometry}
\usepackage[pdftex]{graphicx}
\usepackage{amsmath,amssymb,amsthm}
\usepackage{custom}
\usepackage{lipsum}

\usepackage{xcolor}
\usepackage{tikz}

\usepackage[most]{tcolorbox}
\usepackage{xcolor}
\usepackage{mdframed}

% page formatting
\usepackage{fancyhdr}
\pagestyle{fancy}

\renewcommand{\sectionmark}[1]{\markright{\textsf{\arabic{section}. #1}}}
\renewcommand{\subsectionmark}[1]{}
\lhead{\textbf{\thepage} \ \ \nouppercase{\rightmark}}
\chead{}
\rhead{}
\lfoot{}
\cfoot{}
\rfoot{}
\setlength{\headheight}{14pt}

\linespread{1.03} % give a little extra room
\setlength{\parindent}{0.2in} % reduce paragraph indent a bit
\setcounter{secnumdepth}{2} % no numbered subsubsections
\setcounter{tocdepth}{2} % no subsubsections in ToC


%%%%%%%%%%%%%%%%%%%%%%%%%%%%%%%%%%%%%%%%%%%%%%%%%%%%%%%%%%%%%%%%%
% CUSTOM BOXES AND STUFF
\newtcolorbox{redbox}{colback=red!5!white,colframe=red!75!black, breakable}
\newtcolorbox{bluebox}{colback=blue!5!white,colframe=blue!75!black,breakable}

\newtcolorbox{dottedbox}[1][]{%
    colback=white,    % Background color
    colframe=white,    % Border color (to be overridden by dashrule)
    sharp corners,     % Sharp corners for the box
    boxrule=0pt,       % No actual border, as it will be drawn with dashrule
    boxsep=5pt,        % Padding inside the box
    enhanced,          % Enable advanced features
    breakable,         % Enables it to span multiple pages
    overlay={\draw[dashed, thin, black, dash pattern=on \pgflinewidth off \pgflinewidth, line cap=rect] (frame.south west) rectangle (frame.north east);}, % Dotted line
    #1                 % Additional options
}

% Define the colors
\definecolor{boxheader}{RGB}{0, 51, 102}  % Dark blue
\definecolor{boxfill}{RGB}{173, 216, 230}  % Light blue


% Define the tcolorbox environment
\newtcolorbox{mathdefinitionbox}[2][]{%
    colback=boxfill,   % Background color
    colframe=boxheader, % Border color
    fonttitle=\bfseries, % Bold title
    coltitle=white,     % Title text color
    title={#2},         % Title text
    enhanced,           % Enable advanced features
    breakable,
    attach boxed title to top left={yshift=-\tcboxedtitleheight/2}, % Center title
    boxrule=0.5mm,      % Border width
    sharp corners,      % Sharp corners for the box
    #1                  % Additional options
}
%%%%%%%%%%%%%%%%%%%%%%%%%


\definecolor{lightblue}{RGB}{173,216,230} % Light blue color
\definecolor{darkblue}{RGB}{0,0,139} % Dark blue color

% Define the custom proof environment
\newtcolorbox{ex}[2][Example]{
  colback=red!5!white, % Light blue background
  colframe=red!75!black, % Darker blue border
  coltitle=white, % Title color
  fonttitle=\bfseries, % Title font style
  title={{#2}},
  arc=1mm, % Rounded corners with 4mm radius,
  boxrule=0.5mm,
  left=2mm, right=2mm, top=2mm, bottom=2mm, % Padding inside the box
  breakable, % Allow box to be broken across pages
  before=\vspace{10pt}, % Padding above the box
  after=\vspace{10pt}, % Padding below the box
  before upper={\parindent15pt} % Ensure indentation
}

% Define the custom proof environment
\newtcolorbox{defn}[2][Definition]{
  colback=green!5!white, % Light blue background
  colframe=green!75!black, % Darker blue border
  coltitle=white, % Title color
  fonttitle=\bfseries, % Title font style
  title={{#2}},
  arc=1mm, % Rounded corners with 4mm radius,
  boxrule=0.5mm,
  left=2mm, right=2mm, top=2mm, bottom=2mm, % Padding inside the box
  breakable, % Allow box to be broken across pages
  before=\vspace{10pt}, % Padding above the box
  after=\vspace{10pt}, % Padding below the box
  before upper={\parindent15pt} % Ensure indentation
}


%%%%%%%%%%%%%%%%%%%%%%%%%%%%%%%%%%%%%%%%%%%%%%%%%%%%%%%%%%%%%%%%%


\begin{document}

% make title page
\thispagestyle{empty}
\bigskip \
\vspace{0.1cm}

\begin{center}
{\fontsize{22}{22} \selectfont Professor: James Analitis}
\vskip 16pt
{\fontsize{30}{30} \selectfont \bf \sffamily Physics 141A: Solid State Physics}
\vskip 24pt
{\fontsize{14}{14} \selectfont \rmfamily Homework 2} 
\vskip 6pt
{\fontsize{14}{14} \selectfont \ttfamily kdeoskar@berkeley.edu} 
\vskip 24pt
\end{center}

% {\parindent0pt \baselineskip=15.5pt \lipsum[1-4]} 

% make table of contents
% \newpage



\begin{bluebox}
  \textbf{Question 1:} Consider heat capacity data below for the compound $\mathrm{Ba}\mathrm{Fe}_2\mathrm{As}_2$. THis is measured per mole of the formula unit, $\mathrm{Ba}\mathrm{Fe}_2\mathrm{As}_2$. Note there is a phase transition at about 140$K$, that you can ignore for now.
  \begin{enumerate}
    \item From the heat capacity, estimate how many degrees of freedom each formula unit contributes.
    \item Fit the data using the low-temperature Debye model. Estimate the Debye Temperature.
    \item Argue why the phase transition at 140$K$ must affect the density of the atoms only weakly.
  \end{enumerate}
\end{bluebox}

\vskip 0.5cm
\textbf{\underline{Solution:}}
\\
\\
text
\vskip 0.5cm
\hrule
\pagebreak



\begin{bluebox}
  \textbf{Question 2: Physical Properties of the Free Electron Gas}
  \begin{enumerate}[label=(\alph*)]
    \item Give a sumple but approximate derivaiton of the Fermi gas prediction for heat capacity of the conduction electron in metals.
    \item Give a simple (not approximate) derivations of the Fermi gas prediction for magnetic susceptibility of the conduction electron in metals. Here susceptibility is $\chi = dM/dH = \mu_0 dM/dB$ at small $H$ and is meant to consider the magnetization of the electron spins only.
    \item How are the results of (a) and (b) different from that of a classical gas of electrons? \\
    What other properties of metals may be different from the classical prediction?
    \item The experimental specific heat of potassium metal at low temperatures has the form $$ C = \gamma T + \alpha T^3 $$ where $\gamma = 2.08 \mathrm{mJ mol}^{-1}\mathrm{K}^{-2}$ and $\alpha = 2.6 \mathrm{mJ mol}^{-1}\mathrm{K}^{-4}$. \\
    Explain the origin of each of the two terms in this expression. \\
    Make an estimate of the Fermi energy for potassium metal.
  \end{enumerate} 
\end{bluebox}

\vskip 0.5cm
\textbf{\underline{Solution:}}
\\
\\
text
\vskip 0.5cm
\hrule
\pagebreak



\begin{bluebox}
  \textbf{Question 3:} 
  \begin{enumerate}[label=(\alph*)]
    \item What is the relationship between the carrier density $n$ and the Fermi momentum $k_F$ in two dimensions? 
    \item Show that in two dimensions the free electron density of states $g(E)$ is a constant independent of energy $E$ for $E > 0$ and 0 for $E < 0$. What is the constant? 
    \item Using the fact that the total number of particles $\mean{N}$ is given by $$ \mean{N} = \int_{0}^{\infty} g(E) f(E) dE  $$ where $f(E)$ is the Fermi distribution function, show that in two-dimensions that $$ E_F = \mu + k_B T \ln\left( 1 + e^{-\mu / k_B T} \right) $$ In order to solve this, you can loop up the integral describing $n$ in a table, then note the relationship of $n$ to $E_F$ to get to the final expression.
    \item Estimate the amount that $\mu$ differeny from $E_F$. Explain how this shows that the chemical potential $\mu$ is essentially independent of temperature so long as $k_B T << \mu$. This is an important fundamental point about two-dimensional systems - over a wide range of temperature, the chemical potential can be essentially regarded as the Fermi energy. Indeed, even in higher dimensions, the difference is small.
  \end{enumerate}
\end{bluebox}

\vskip 0.5cm
\textbf{\underline{Solution:}}

\begin{enumerate}
  \item We know that, in the Sommerfeld model, the heat capacity of a metal is $$ C = \tilde{\gamma} \left( \frac{f N k_B}{2} \right) \left( \frac{T}{T_F} \right) $$ where $f$ is the number of degrees of freedom and $\tilde{\gamma} = \pi^2/3$
\end{enumerate}

\vskip 0.5cm
\hrule
\pagebreak













% \begin{bluebox}
%   \textbf{Question 1:} 
% \end{bluebox}

% \vskip 0.5cm
% \textbf{\underline{Solution:}}
% \\
% \\
% text
% \vskip 0.5cm
% \hrule
% \pagebreak



% %%%%%%%%%%%%%%%%%%%%%%%%%%%%%%%%%%%%%%%%%%%%%%
% \newpage
% % \section{References}
% %%%%%%%%%%%%%%%%%%%%%%%%%%%%%%%%%%%%%%%%%%%%%%
% \vskip 0.5cm
% \bibliographystyle{plain} % We choose the "plain" reference style
% \bibliography{citation}




\end{document}










