\documentclass[11pt]{article}

% basic packages
\usepackage[margin=1in]{geometry}
\usepackage[pdftex]{graphicx}
\usepackage{amsmath,amssymb,amsthm}
\usepackage{custom}
\usepackage{lipsum}

\usepackage{xcolor}
\usepackage{tikz}

\usepackage[most]{tcolorbox}
\usepackage{xcolor}
\usepackage{mdframed}

% page formatting
\usepackage{fancyhdr}
\pagestyle{fancy}

\renewcommand{\sectionmark}[1]{\markright{\textsf{\arabic{section}. #1}}}
\renewcommand{\subsectionmark}[1]{}
\lhead{\textbf{\thepage} \ \ \nouppercase{\rightmark}}
\chead{}
\rhead{}
\lfoot{}
\cfoot{}
\rfoot{}
\setlength{\headheight}{14pt}

\linespread{1.03} % give a little extra room
\setlength{\parindent}{0.2in} % reduce paragraph indent a bit
\setcounter{secnumdepth}{2} % no numbered subsubsections
\setcounter{tocdepth}{2} % no subsubsections in ToC


%%%%%%%%%%%%%%%%%%%%%%%%%%%%%%%%%%%%%%%%%%%%%%%%%%%%%%%%%%%%%%%%%
% CUSTOM BOXES AND STUFF
\newtcolorbox{redbox}{colback=red!5!white,colframe=red!75!black, breakable}
\newtcolorbox{bluebox}{colback=blue!5!white,colframe=blue!75!black,breakable}

\newtcolorbox{dottedbox}[1][]{%
    colback=white,    % Background color
    colframe=white,    % Border color (to be overridden by dashrule)
    sharp corners,     % Sharp corners for the box
    boxrule=0pt,       % No actual border, as it will be drawn with dashrule
    boxsep=5pt,        % Padding inside the box
    enhanced,          % Enable advanced features
    breakable,         % Enables it to span multiple pages
    overlay={\draw[dashed, thin, black, dash pattern=on \pgflinewidth off \pgflinewidth, line cap=rect] (frame.south west) rectangle (frame.north east);}, % Dotted line
    #1                 % Additional options
}

% Define the colors
\definecolor{boxheader}{RGB}{0, 51, 102}  % Dark blue
\definecolor{boxfill}{RGB}{173, 216, 230}  % Light blue


% Define the tcolorbox environment
\newtcolorbox{mathdefinitionbox}[2][]{%
    colback=boxfill,   % Background color
    colframe=boxheader, % Border color
    fonttitle=\bfseries, % Bold title
    coltitle=white,     % Title text color
    title={#2},         % Title text
    enhanced,           % Enable advanced features
    breakable,
    attach boxed title to top left={yshift=-\tcboxedtitleheight/2}, % Center title
    boxrule=0.5mm,      % Border width
    sharp corners,      % Sharp corners for the box
    #1                  % Additional options
}
%%%%%%%%%%%%%%%%%%%%%%%%%


\definecolor{lightblue}{RGB}{173,216,230} % Light blue color
\definecolor{darkblue}{RGB}{0,0,139} % Dark blue color

% Define the custom proof environment
\newtcolorbox{ex}[2][Example]{
  colback=red!5!white, % Light blue background
  colframe=red!75!black, % Darker blue border
  coltitle=white, % Title color
  fonttitle=\bfseries, % Title font style
  title={{#2}},
  arc=1mm, % Rounded corners with 4mm radius,
  boxrule=0.5mm,
  left=2mm, right=2mm, top=2mm, bottom=2mm, % Padding inside the box
  breakable, % Allow box to be broken across pages
  before=\vspace{10pt}, % Padding above the box
  after=\vspace{10pt}, % Padding below the box
  before upper={\parindent15pt} % Ensure indentation
}

% Define the custom proof environment
\newtcolorbox{defn}[2][Definition]{
  colback=green!5!white, % Light blue background
  colframe=green!75!black, % Darker blue border
  coltitle=white, % Title color
  fonttitle=\bfseries, % Title font style
  title={{#2}},
  arc=1mm, % Rounded corners with 4mm radius,
  boxrule=0.5mm,
  left=2mm, right=2mm, top=2mm, bottom=2mm, % Padding inside the box
  breakable, % Allow box to be broken across pages
  before=\vspace{10pt}, % Padding above the box
  after=\vspace{10pt}, % Padding below the box
  before upper={\parindent15pt} % Ensure indentation
}


%%%%%%%%%%%%%%%%%%%%%%%%%%%%%%%%%%%%%%%%%%%%%%%%%%%%%%%%%%%%%%%%%


\begin{document}

% make title page
\thispagestyle{empty}
\bigskip \
\vspace{0.1cm}

\begin{center}
{\fontsize{22}{22} \selectfont Professor: James Analitis}
\vskip 16pt
{\fontsize{30}{30} \selectfont \bf \sffamily Physics 141A: Solid State Physics}
\vskip 24pt
{\fontsize{14}{14} \selectfont \rmfamily Homework 5} 
\vskip 6pt
{\fontsize{14}{14} \selectfont \ttfamily kdeoskar@berkeley.edu} 
\vskip 24pt
\end{center}

% {\parindent0pt \baselineskip=15.5pt \lipsum[1-4]} 

% make table of contents
% \newpage


\begin{bluebox}
  \textbf{Question 1:} Consider a one-dimensional spring and mass model of a crystal. Generalize this model to include springs not only between neighbors but also between second nearest neighbors. Let the spring constant between neighbors be called $\kappa_1$ and the spring constant between second neighbors be called $\kappa_2$. Let the mass of each atom be $m$. 
  \begin{enumerate}[label=(\alph*)]
    \item Calculate the dispersion curve $\omega(k)$ for this model.
    \item Determine the sound wave velocity. Show the group velocity vanishes at the Brillouin zone boundary.
  \end{enumerate}
\end{bluebox}

\vskip 0.5cm
\textbf{\underline{Solution:}}
\\
\\
text
\vskip 0.5cm
\hrule
\pagebreak



\begin{bluebox}
  \textbf{Question 2: Normal modes of a One-Dimensional Diatomic Chain} 
  \begin{enumerate}[label=(\alph*)]
    \item What is the difference between an acoustic mode and an optical mode? \\
    $\triangleright$ Describe how particles move in each case.

    \item Derive the dispersion relation for the longitudinal oscillations of a one-dimensional diatomic mass-and-spring crystal where the unit cell is of length $a$ and each unit cell contains one atom of mass $m_1$ and one atom of mass $m_2$ connected together by springs with spring constant $\kappa$.
    
    \item Determine the frequences of the acoustic and optical modes at $k = 0$ as well as at the Brillouin zone boundary.\\
    $\triangleright$ Describe the motion of the masses in each case. \\
    $\triangleright$ Determine the sound velocity and show that the group velocity is zero at the zone boundary.\\
    $\triangleright$ Show that the sound velocity is also given by $v_s = \sqrt{\beta^{-1}/\rho}$ where $\rho$ is the chain density and $\beta$ is the compressibility.
    
    \item Sketch the dispersion in both reduced and extended zone scheme. \\
    $\triangleright$ If there are $N$ unit cells, how many different normal modes are there? \\
    $\triangleright$ How many \emph{branches} of excitations are there? I.e., in reduced zone scheme, how many modes are there at each $k$?

    \item What happens when $m_1 = m_2$?
  \end{enumerate}
\end{bluebox}

\vskip 0.5cm
\textbf{\underline{Solution:}}
\\
\\
text
\vskip 0.5cm
\hrule
\pagebreak


% \begin{bluebox}
%   \textbf{Question 3:} 
% \end{bluebox}

% \vskip 0.5cm
% \textbf{\underline{Solution:}}
% \\
% \\
% text
% \vskip 0.5cm
% \hrule
% \pagebreak
























% \begin{bluebox}
%   \textbf{Question 1:} 
% \end{bluebox}

% \vskip 0.5cm
% \textbf{\underline{Solution:}}
% \\
% \\
% text
% \vskip 0.5cm
% \hrule
% \pagebreak



% %%%%%%%%%%%%%%%%%%%%%%%%%%%%%%%%%%%%%%%%%%%%%%
% \newpage
% % \section{References}
% %%%%%%%%%%%%%%%%%%%%%%%%%%%%%%%%%%%%%%%%%%%%%%
% \vskip 0.5cm
% \bibliographystyle{plain} % We choose the "plain" reference style
% \bibliography{citation}




\end{document}










