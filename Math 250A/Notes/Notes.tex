\documentclass[11pt]{article}

% basic packages
\usepackage[margin=1in]{geometry}
\usepackage[pdftex]{graphicx}
\usepackage{amsmath,amssymb,amsthm}
\usepackage{custom}
\usepackage{lipsum}

\usepackage{xcolor}
\usepackage{tikz}

\usepackage[most]{tcolorbox}
\usepackage{xcolor}
\usepackage{mdframed}

% page formatting
\usepackage{fancyhdr}
\pagestyle{fancy}

\renewcommand{\sectionmark}[1]{\markright{\textsf{\arabic{section}. #1}}}
\renewcommand{\subsectionmark}[1]{}
\lhead{\textbf{\thepage} \ \ \nouppercase{\rightmark}}
\chead{}
\rhead{}
\lfoot{}
\cfoot{}
\rfoot{}
\setlength{\headheight}{14pt}

\linespread{1.03} % give a little extra room
\setlength{\parindent}{0.2in} % reduce paragraph indent a bit
\setcounter{secnumdepth}{2} % no numbered subsubsections
\setcounter{tocdepth}{2} % no subsubsections in ToC


%%%%%%%%%%%%%%%%%%%%%%%%%%%%%%%%%%%%%%%%%%%%%%%%%%%%%%%%%%%%%%%%%
% CUSTOM BOXES AND STUFF
\newtcolorbox{redbox}{colback=red!5!white,colframe=red!75!black, breakable}
\newtcolorbox{bluebox}{colback=blue!5!white,colframe=blue!75!black,breakable}

\newtcolorbox{dottedbox}[1][]{%
    colback=white,    % Background color
    colframe=white,    % Border color (to be overridden by dashrule)
    sharp corners,     % Sharp corners for the box
    boxrule=0pt,       % No actual border, as it will be drawn with dashrule
    boxsep=5pt,        % Padding inside the box
    enhanced,          % Enable advanced features
    breakable,         % Enables it to span multiple pages
    overlay={\draw[dashed, thin, black, dash pattern=on \pgflinewidth off \pgflinewidth, line cap=rect] (frame.south west) rectangle (frame.north east);}, % Dotted line
    #1                 % Additional options
}

% Define the colors
\definecolor{boxheader}{RGB}{0, 51, 102}  % Dark blue
\definecolor{boxfill}{RGB}{173, 216, 230}  % Light blue


% Define the tcolorbox environment
\newtcolorbox{mathdefinitionbox}[2][]{%
    colback=boxfill,   % Background color
    colframe=boxheader, % Border color
    fonttitle=\bfseries, % Bold title
    coltitle=white,     % Title text color
    title={#2},         % Title text
    enhanced,           % Enable advanced features
    breakable,
    attach boxed title to top left={yshift=-\tcboxedtitleheight/2}, % Center title
    boxrule=0.5mm,      % Border width
    sharp corners,      % Sharp corners for the box
    #1                  % Additional options
}
%%%%%%%%%%%%%%%%%%%%%%%%%


\definecolor{lightblue}{RGB}{173,216,230} % Light blue color
\definecolor{darkblue}{RGB}{0,0,139} % Dark blue color

% Define the custom proof environment
\newtcolorbox{ex}[2][Example]{
  colback=red!5!white, % Light blue background
  colframe=red!75!black, % Darker blue border
  coltitle=white, % Title color
  fonttitle=\bfseries, % Title font style
  title={{#2}},
  arc=1mm, % Rounded corners with 4mm radius,
  boxrule=0.5mm,
  left=2mm, right=2mm, top=2mm, bottom=2mm, % Padding inside the box
  breakable, % Allow box to be broken across pages
  before=\vspace{10pt}, % Padding above the box
  after=\vspace{10pt}, % Padding below the box
  before upper={\parindent15pt} % Ensure indentation
}

% Define the custom proof environment
\newtcolorbox{defn}[2][Definition]{
  colback=green!5!white, % Light blue background
  colframe=green!75!black, % Darker blue border
  coltitle=white, % Title color
  fonttitle=\bfseries, % Title font style
  title={{#2}},
  arc=1mm, % Rounded corners with 4mm radius,
  boxrule=0.5mm,
  left=2mm, right=2mm, top=2mm, bottom=2mm, % Padding inside the box
  breakable, % Allow box to be broken across pages
  before=\vspace{10pt}, % Padding above the box
  after=\vspace{10pt}, % Padding below the box
  before upper={\parindent15pt} % Ensure indentation
}


%%%%%%%%%%%%%%%%%%%%%%%%%%%%%%%%%%%%%%%%%%%%%%%%%%%%%%%%%%%%%%%%%


\begin{document}

% make title page
\thispagestyle{empty}
\bigskip \
\vspace{0.1cm}

\begin{center}
{\fontsize{22}{22} \selectfont Lecturer: Richard Borcherds}
\vskip 16pt
{\fontsize{30}{30} \selectfont \bf \sffamily Math 250A: Groups, Rings, and Fields}
\vskip 24pt
{\fontsize{14}{14} \selectfont \rmfamily Notes taken by Keshav Balwant Deoskar} 
\vskip 6pt
{\fontsize{14}{14} \selectfont \ttfamily kdeoskar@berkeley.edu} 
\vskip 24pt
\end{center}

% {\parindent0pt \baselineskip=15.5pt \lipsum[1-4]} 

% make table of contents
% \newpage

These are some lecture notes taken from the Fall 2024 semester offering of Math 250A: Groups, Rings, and Fields taught by Richard Borcherds at UC Berkeley. The primary reference is \cite{Lang02}. This template is based heavily off of the one produced by \href{https://knzhou.github.io/}{Kevin Zhou}.

% \microtoc
\setcounter{tocdepth}{3}
\tableofcontents 

%%%%%%%%%%%%%%%%%%%%%%%%%%%%%%%%%%%%%%%%%%%%%%%%%%%%%%%%%
\newpage
\section{August 29, 2024:}
%%%%%%%%%%%%%%%%%%%%%%%%%%%%%%%%%%%%%%%%%%%%%%%%%%%%%%%%%

\subsection{Logistics:}
First third of the course will be groups, next third will be rings, and we'll conclude with fields.

\subsection{Groups}

\begin{bluebox}
  \begin{definition}
    (Group, Concrete): A Group os a \textbf{symmetry} of some math object.
  \end{definition}
\end{bluebox}
where by "symmetry" we mean a bijection $X \rightarrow X$ "preserving the structure of the object". This is all a bit vague, but we get the idea.
\\
\\
\begin{definition}
  The \textbf{Order} of a group is the number of elements.
\end{definition}

\begin{bluebox}
  \begin{definition}
    (Group, Abstract): A set $X$ along with a binary operation $\cdot \text{ : } X \times X \rightarrow X$, Identity map $\mathbb{Id}$, and inverse map $X \rightarrow X$ such that 
    \begin{enumerate}[label=(\alph*)]
      \item A
      \item $a a^{-1} = a^{-1} a = 1$
      \item $(ab)c = a(bc)$
    \end{enumerate}
    fill this in.
  \end{definition}
\end{bluebox}

How do we show the equivalence of these two definitions? one direction (symmetry $\implies$ axiomatic definition) is obvious. \begin{note}
  {Elaborate.}
\end{note} But can we go the other way around?

\subsection{Cayley Graphs}
\begin{dottedbox}
  \textbf{Problem:} Given an abstract group $G$, find an object $X$ where the symmetries are \textit{isomorphic} to $G$.
\end{dottedbox}

The object $X$ will turns out to be a \textbf{Cayley Graph} - which is a colored, directed graph whose points correspond to the elements of the abstract group $G$. So, the set of points $S$ of the graph is really just the same as $G$.
\\
\\
What about the arrows? These will represent \textbf{left-actions of $G$ on $S$}.

\begin{redbox}
  A \textbf{left-action} of a group $G$ on a set $S$ is a map $G \times S \rightarrow S$ such that 
  \begin{enumerate}[label=(\alph*)]
    \item $(g_1 g_2) \cdot s = g_1 \cdot (g_2 \cdot s)$
    \item $1 \cdot s = s$
  \end{enumerate}
\end{redbox}

In the Cayley Graph, we consider the left-action \[ g(s) = g \underbrace{\cdot}_{\text{mult. in the grp.}} s  \] We notice that the Action is \textbf{Faithful} i.e. if $g(s) = s$ for all $s \in S$ then $g = 1$.
\\
\\
So, \underline{$G \subseteq $ all symmetries of the set $S$}. What we've shown is a weak version of \textbf{Cayley's Theorem.}
\\
\\
To complete our goal we have to tackle the following:
\begin{dottedbox}
  \textbf{Problem:} Add extra structure to $S$ in order to cut down the symmetry group to that of $G$.
\end{dottedbox} The extra structure we'll be adding is a \textbf{right-action} of $G$ on $S$ defined by \[ s \cdot g = sg \]
We check that the left-action preserves structure. From the associative law of $G$, we have
\[ (gs)h = g(sh)  \] i.e. left-actions \textbf{commutes} with right-actions.

\begin{redbox}
  Note that left-actions \textbf{do NOT} commute with left-actions as $g(hs) \neq h(gs)$.
\end{redbox} So, finally, we define the \textbf{Cayley Graph} by the following rule:
\\
For each $g \in G$, for a point $s \in S$, draw a lie from $s$ to $sg$ and demarcate it with the \textit{color} $g$.

\subsection*{Eg. Cayley Graph of the Klein 4-group}
Draw the cayley table and cayleg graph. 
\\
\begin{bluebox}
  What we've shown above is that Abstract Groups are the same as Symmetries.
\end{bluebox} In showing this result we used the notion of group actions. A reasonable question to ask would be "What kind of actions does a group have on \textit{itself}?"

\subsection{Actions of $G$ on itself}
There are 8 ways in which a group $G$ naturally acts on itself. 
\\
\\
Include picture.
\\
\\
\begin{remark}
  {The adjoint action will come up again in a lecture or two.}
\end{remark}

\subsection{Goals}
Okay, so we now have two definitions for groups. The ultimate goal would then be to:
\begin{enumerate}[label=(\alph*)]
  \item Classify all groups.
  \item Find all representations of a group.
\end{enumerate}
It turns out this is far too difficult to be done! So, we restrict to all \textbf{simple groups}, and find that again the problem is too difficult!
\\
But if we decide to classify all \textbf{abelian simple groups}, then we can get somewhere. There are other types of Groups that have been successfully classified, and we'll discuss some of them in this course.

\subsection{What is a representation?}
Write this soon.

\subsection{An attempt at the impossible: Catalog of All Groups}
Let's begin trying to classify groups according to their \textbf{orders}... though of course we'll eventually stop when it becomes too difficult to continue. 
\\
\\
Our initial observations are kind of boring.
\begin{align*}
  &\text{Order 1: } \;\;\;&\text{TRIVIAL} \\
  &\text{Order 2: } \;\;\;&\mathbb{Z}/2\mathbb{Z} \\
  &\text{Order 3: } \;\;\;&\mathbb{Z}/3\mathbb{Z} \\
\end{align*} In fact, for any prime $p$, we can show that there exists precisely \textbf{\textit{one}} group of order $p$ up to isomorphism (consistent with our findings for $p = 2,3$). The way to do this is via \textbf{Lagrange's Theorem}.


% %%%%%%%%%%%%%%%%%%%%%%%%%%%%%%%%%%%%%%%%%%%%%%%%%%%%%%%%%
% \newpage
% \section{}
% %%%%%%%%%%%%%%%%%%%%%%%%%%%%%%%%%%%%%%%%%%%%%%%%%%%%%%%%%


%%%%%%%%%%%%%%%%%%%%%%%%%%%%%%%%%%%%%%%%%%%%%%
\newpage
% \section{References}
%%%%%%%%%%%%%%%%%%%%%%%%%%%%%%%%%%%%%%%%%%%%%%
\vskip 0.5cm
\bibliographystyle{plain} % We choose the "plain" reference style
\bibliography{citation} % Entries are in the refs.bib file




\end{document}










