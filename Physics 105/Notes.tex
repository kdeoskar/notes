\documentclass[11pt]{article}

% basic packages
\usepackage[margin=1in]{geometry}
\usepackage[pdftex]{graphicx}
\usepackage{amsmath,amssymb,amsthm}
\usepackage{custom}
\usepackage{lipsum}

\usepackage{xcolor}
\usepackage{tikz}

\usepackage[most]{tcolorbox}
\usepackage{xcolor}
\usepackage{mdframed}

% page formatting
\usepackage{fancyhdr}
\pagestyle{fancy}

\renewcommand{\sectionmark}[1]{\markright{\textsf{\arabic{section}. #1}}}
\renewcommand{\subsectionmark}[1]{}
\lhead{\textbf{\thepage} \ \ \nouppercase{\rightmark}}
\chead{}
\rhead{}
\lfoot{}
\cfoot{}
\rfoot{}
\setlength{\headheight}{14pt}

\linespread{1.03} % give a little extra room
\setlength{\parindent}{0.2in} % reduce paragraph indent a bit
\setcounter{secnumdepth}{2} % no numbered subsubsections
\setcounter{tocdepth}{2} % no subsubsections in ToC


%%%%%%%%%%%%%%%%%%%%%%%%%%%%%%%%%%%%%%%%%%%%%%%%%%%%%%%%%%%%%%%%%
% CUSTOM BOXES AND STUFF
\newtcolorbox{redbox}{colback=red!5!white,colframe=red!75!black}
\newtcolorbox{bluebox}{colback=blue!5!white,colframe=blue!75!black}

\definecolor{lightblue}{RGB}{173,216,230} % Light blue color
\definecolor{darkblue}{RGB}{0,0,139} % Dark blue color

% Define the custom proof environment
\newtcolorbox{ex}[2][Example]{
  colback=red!5!white, % Light blue background
  colframe=red!75!black, % Darker blue border
  coltitle=white, % Title color
  fonttitle=\bfseries, % Title font style
  title={{#2}},
  arc=1mm, % Rounded corners with 4mm radius,
  boxrule=0.5mm,
  left=2mm, right=2mm, top=2mm, bottom=2mm, % Padding inside the box
  breakable, % Allow box to be broken across pages
  before=\vspace{10pt}, % Padding above the box
  after=\vspace{10pt}, % Padding below the box
  before upper={\parindent15pt} % Ensure indentation
}

% Define the custom proof environment
\newtcolorbox{defn}[2][Definition]{
  colback=blue!5!white, % Light blue background
  colframe=blue!75!black, % Darker blue border
  coltitle=white, % Title color
  fonttitle=\bfseries, % Title font style
  title={{#2}},
  arc=1mm, % Rounded corners with 4mm radius,
  boxrule=0.5mm,
  left=2mm, right=2mm, top=2mm, bottom=2mm, % Padding inside the box
  breakable, % Allow box to be broken across pages
  before=\vspace{10pt}, % Padding above the box
  after=\vspace{10pt}, % Padding below the box
  before upper={\parindent15pt} % Ensure indentation
}


%%%%%%%%%%%%%%%%%%%%%%%%%%%%%%%%%%%%%%%%%%%%%%%%%%%%%%%%%%%%%%%%%


\begin{document}

% make title page
\thispagestyle{empty}
\bigskip \
\vspace{0.1cm}

\begin{center}
{\fontsize{22}{22} \selectfont (Instructor: Chien-I Chiang)}
\vskip 16pt
{\fontsize{36}{36} \selectfont \bf \sffamily Physics 105: Analytical Mechanics notes}
\vskip 24pt
{\fontsize{18}{18} \selectfont \rmfamily Keshav Balwant Deoskar} 
\vskip 6pt
{\fontsize{14}{14} \selectfont \ttfamily kdeoskar@berkeley.edu} 
\vskip 24pt
\end{center}

% {\parindent0pt \baselineskip=15.5pt \lipsum[1-4]} 

% make table of contents
% \newpage

These are some very terse notes taken from UC Berkeley's Physics 105 during the Summer '24 session, taught by Chien-I Chiang.

\vskip 0.5cm
This template is based heavily off of the one produced by \href{https://knzhou.github.io/}{Kevin Zhou}.

% \microtoc
\tableofcontents 

% main 

%%%%%%%%%%%%%%%%%%%%%%%%%%%%%%%%%%%%%%%%%%%%%%
\newpage
\section{First topic}
%%%%%%%%%%%%%%%%%%%%%%%%%%%%%%%%%%%%%%%%%%%%%%

\vskip 0.5cm
text


%%%%%%%%%%%%%%%%%%%%%%%%%%%%%%%%%%%%%%%%%%%%%%
\newpage
\section{July 3, 2024:}
%%%%%%%%%%%%%%%%%%%%%%%%%%%%%%%%%%%%%%%%%%%%%%

\subsection{Finishing up discussion from last lecture}
- Finish this from lecture recording

\vskip 0.5cm
We can parametrize the position of a particle as $\vec{r} = \vec{r}(q_k, t)$

We have 
\[ \frac{\partial K}{\partial \dot{q_{k}}} = \frac{1}{2}m \left[ 2 \frac{\partial \vec{r}}{\partial q_k} \cdot \frac{\partial \vec{r}}{\partial q_{m}} \dot{q_{m}} + \cdots \right] \]

Then, 
\[ \begin{cases}
  \frac{\partial K}{\partial \dot{q_{k}}} \dot{q_{k}} = m \left[ \left( \frac{\partial \vec{r}}{\partial q_k} \cdot \frac{\partial \vec{r}}{\partial q_{m}} \dot{q_{k}} \dot{q_{m}}\right) + \frac{\partial \vec{r}}{\partial q_{k}} \cdot \frac{\partial \vec{r}}{\partial t} \dot{q_{k}} \right] \\
  2K = m \left[ \left( \frac{\partial \vec{r}}{\partial q_k} \cdot \frac{\partial \vec{r}}{\partial q_{m}} \dot{q_{k}} \dot{q_{m}}\right) + 2 \frac{\partial \vec{r}}{\partial q_{k}} \cdot \frac{\partial \vec{r}}{\partial t} \dot{q_{k}} + \frac{\partial \vec{r}}{\partial t} \cdot \frac{\partial \vec{r}}{\partial t}\right]
\end{cases} \]

\vskip 0.5cm
\begin{thought}
{
  The expression for $2K$ is obtained by expandingo out 
\[ K = \frac{1}{2}m \frac{d\vec{r}}{dt} \cdot \frac{d\vec{r}}{dt} \] 
in terms of indices -- write this out explicitly later
}
\end{thought}


\vskip 0.5cm
Which gives us the relation

\begin{align*}
  \frac{\partial K}{\partial \dot{q}_k} \dot{q}_k &= 2K - m \frac{\partial \vec{r}}{\partial t} \cdot \underbrace{\left( \frac{\partial \vec{r}}{\partial q_k} \dot{q}_k + \frac{\partial \vec{r}}{\partial t} \right) }_{ = \frac{d \vec{r}}{dt}} \\
  &= 2K - \vec{p} \frac{\partial \vec{r}}{\partial t}
\end{align*}

\vskip 0.5cm
The question we were originally considering is 
\color{blue} \textbf{When is $H = E$?} \color{black}

\vskip 0.5cm
Now,
\begin{align*}
  H &= \frac{\partial L}{\partial \dot{q}_k} \dot{q} - L \\
  &= \frac{\partial K}{\partial \dot{q}_k} \dot{q}_k = \left( K - V\right) \\
  &= 2K - \vec{p} \cdot \frac{\partial \vec{r}}{\partial t} - K + V \\
  &= K + V - \vec{p} \cdot \frac{\partial \vec{r}}{\partial t}
\end{align*}

\begin{redbox} 
So we see that $H = E = K + V$ only when \[ \frac{\partial \vec{r}}{\partial t} = 0 \] i.e. when $\vec{r} = \vec{r}(q_k, t)$ has no time dependence i.e. $\vec{r} = \vec{r}(q_k)$
\end{redbox}

\vskip 0.5cm
Earlier, we considered the following setup:
% \begin{center}
%   \includegraphics*[scale=0.5]{July 3/pic 1.jpeg}
% \end{center}

\vskip 0.5cm
and we showed that 
\[ H = E - m\omega^2 \rho^2 \]

So, let's check that 
\[ m\omega^2 \rho^2 = \vec{p} \cdot \frac{\partial \vec{r}}{\partial t} \]

\begin{bluebox}
\begin{align*}
  \vec{p} \cdot \frac{\partial \vec{r}}{\partial t} &= \vec{p} \cdot \left( -\rho \omega \sin(\omega t) \hat{x} + \rho \omega \sin(\omega t) \hat{y} \right) \\
  &= \vec{p} \cdot \left[ \rho  \omega \hat{\phi} \right] \\
  &= m v_{\phi} \rho \omega \\
  &= m \rho^2 \omega^2 
\end{align*}
where $v_{\phi} = \rho \omega$
\end{bluebox}

Since the hamiltonian itself has no time dependence, \textbf{$H$ is conserved}. However, \textbf{$E$ is not}. We can check that 
\[ dH = dE = d(m \omega^2 \rho^2) \] is indeed zero. 

\vskip 0.5cm
% \begin{center}
%   \includegraphics*[scale=0.5]{July 3/pic 2.jpeg}
% \end{center}
[Include figure]

\vskip 0.5cm
\begin{bluebox}
  If we break the force on the bead into a normal force (denoted $N$) and a centripetal(?) force, then

\begin{align*}
  dW &= \overbrace{N \rho}_{\text{torque about z-axis}} d\phi \\
  &= \frac{d {l}_z}{dt} d\phi \\
  &= d \left(\rho m \rho \omega\right) \omega \\
  &= d\left(m \rho^2 \omega^2\right)
\end{align*}

This is the energy that goes into the system.

\vskip 0.5cm
By energy conservation, $dW = dE$.
\begin{align*}
  \implies 0 = dE - dW = dE - d(m\rho^2 \omega^2)
\end{align*}

i.e. $E - m\rho^2 \omega^2 = H$ is a conserved quantity.
\end{bluebox}

\vskip 0.25cm
So, the \textbf{Hamiltonian being conserved} and the \textbf{Hamiltonian being equal to Energy} are two different scenarios with two different conditions.

\begin{redbox}
  \begin{itemize}
    \item $\frac{\partial L}{\partial t} = 0 \implies H$ is conserved.
    \item For $\vec{r} = \vec{r}(q_k, t)$, $\frac{\partial \vec{r}}{\partial t} = 0 \implies H = E$
  \end{itemize}
\end{redbox}

\vskip 0.5cm
Now we move on to a powerful technique.

\vskip 1cm
\subsection{The Method of Lagrange Multipliers}

\vskip 0.5cm
% \begin{center}
%   \includegraphics*[scale=0.5]{July 3/pic 2.jpeg}
% \end{center}

\vskip 0.5cm
We have a block constrained to move on the $xy$-plane, and we have gravity. Previously, we would say 
\[ L = \frac{1}{2} m \left( \dot{x}^2 + \dot{y}^2 + \dot{z}^2\right) - mgz \] i.e we would start with an unconstrained lagrangian, and then plug in the constraints $z = 0, \dot{z} = 0$
\begin{align*}
  &L = \frac{1}{2}m(\dot{x}^2 + \dot{y}^2) \\
  \implies & \begin{cases}
    m \ddot{x} = 0 \\
    m \ddot{y} = 0
  \end{cases} 
\end{align*}

\vskip 0.5cm
Alternatively, we can implement the constraint $\ddot{z} = 0$ in the following way: We have the original lagrangian
\begin{align*}
  L' = \frac{1}{2} m \left( \dot{x}^2 + \dot{y}^2 + \dot{z}^2\right) - mgz + \lambda z
\end{align*}

where $\lambda$ is the Lagrange multiplier and we can think of $z$ as being the constraint function $f(z)$ and our constraint is $f(z) = 0$.

\vskip 0.5cm
If we treat $\lambda$ as an independent degree of freedom, we can write the Euler-Lagrange equation for $\lambda$ as 
\[ \frac{\partial L}{\partial \lambda} - \frac{d}{dt} \left(\frac{\partial L}{\partial \dot{\lambda}}\right) = 0 \implies z = 0 \text{ (constraint)} \]

On the other hand, if we look at the equation of motion for $z$, we get 
\[  \frac{\partial L}{\partial z} - \frac{d}{dt} \left(\frac{\partial L}{\partial \dot{z}}\right) = 0 \implies \lambda - mg - m\ddot{z} = 0 \implies m\ddot{z} = \lambda - mg \]

and using the constraint $z = 0 \implies \ddot{z} = 0$ we get $-mg + \lambda = 0 \implies \lambda = mg$. Okay, but what physical meaning does $\lambda$ have? It has to do with the \textbf{Normal force}. i.e. $\lambda$ is encoding the \textbf{constraint} that the block can only move on the $xy$-plane due to the Normal force. 

\begin{redbox}
  So, in general, for $N$ constraints we have Lagrange Multipliers $\lambda_1, \cdots, \lambda_N$.
\end{redbox}

\begin{bluebox}
  \textbf{Why do we call $\lambda$ a Lagrange Multiplier?}
  \vskip 0.5cm
  Recall from Calc 3 that if we have contours of a function $f(x,y)$ on the $xy$-plane and we are constrained to move along some other curve $g(x,y) = c$ on the plane, if we ask 
  \color{blue}
  "What is the extremum of $f(x,y)$ as we move along the curve $g(x,y) = c$?"
  \color{black} then visually we can tell that the extremum corresponds to the point where $g(x,y)$ intersects the contour of $f(x,y)$ only once. This is because at such a point, the gradients of the two functions are parallel:

  \[ \nabla g = \lambda \nabla f \]
  This constant multiplier is the \textbf{Lagrange Multiplier}
\end{bluebox}

So, in general, if we have a Lagrangian 
\[ L = \frac{1}{2}m \left(\dot{x}^2 + \dot{y}^2 + \dot{z}^2\right) - V(x,y,z) \]
we know that $\delta L = 0$ gives the Equations of Motion. But if we want to do this variation $\delta L$ under some constraint $C(x,y,z) = 0$ then we need to consider 
\[ \delta L = \lambda \delta C \implies L' = L - \lambda C \]


\vskip 0.5cm
\begin{redbox}
  Generally, if we have $P$ constraints, $C_{l}(q_1, \cdots, t) = 0$, $l = 1, \cdots, P$ on the lagrangian $L$, we can write a new lagrangian 
  \[ L' = L + \sum_{l = 1}^{P} \lambda_l C_l \]
  The Euler-Lagrange equation for $\lambda_l$ leads to $C_l = 0$ and the Euler-Lagrange equation for the generalized coordinate $q_k$ is 
  \[ \left( \frac{\partial L }{\partial q_{k}} - \frac{d}{dt} \left( \frac{\partial L}{\partial \dot{q}_{k}} \right) - \sum_{l = 1}^{P} \lambda_l \frac{C_l}{q_k} \right) = 0 \] 

  \begin{align*}
    \implies& \frac{d}{dt} \left( \frac{\partial L}{\partial \dot{q}_{k}} \right) = \frac{\partial L }{\partial q_{k}} + \underbrace{\sum_{l = 1}^{P} \lambda_l \frac{C_l}{q_k} }_{\text{generalized force}}
  \end{align*}
\end{redbox}

On the physical point of view, consider the following system:

\vskip 0.5cm
[include picture of block and sledge which can both move]
\vskip 0.5cm

If we consider the system as a whole, the normal forces due to the block and the sledge are equal and opposite, so they cancel each other out - and so does the work that they do(?).

\vskip 0.5cm
Howeverm if we consider the block only - we do have a normal force. The block is constrained the only move on the surface of the slope, so we can write 

\begin{align*}
  L' = K - V + \int^{\vec{r}} \vec{F}_C(\vec{r}) \cdot d\vec{r}'
\end{align*}

\begin{thought}
  {This is a bit handwavy - watch the lecture recording and think about this}
\end{thought}

Then, if we conpare this with 
\[ L' = L - V + \sum_{l} \lambda_l C_l \]

we have 
\begin{align*}
  &\sum_{l} \lambda_l C_l = \int^{\vec{r}} \vec{F}_C \cdot d\vec{r}' = \int^{\vec{r}} \vec{F} \cdot \left( \frac{\partial \vec{r}'}{\partial q_k} \cdot \mathrm{d}q_k \right) \\
  \implies& \frac{\partial}{\partial q_k} \left( \sum_{l} \lambda_l c_l \right) = \vec{F} \cdot \left( \frac{\partial \vec{r}}{\partial q_{k}} \right) \equiv \mathcal{F}_k \text{  (generalized force)}
\end{align*}

% \begin{ex}{Fecko, Exercise 12.1.5}
    
% \end{ex}

% \begin{ex}{Fecko, Exercise 12.2.2}
    
% \end{ex}

% \begin{note}{Note}
% \end{note}

% \begin{thought}{A thought}
% \end{thought}

% \begin{highlight}{A hightlight}
% \end{highlight}



\end{document}










\end{document}