\documentclass[11pt]{article}

% basic packages
\usepackage[margin=1in]{geometry}
\usepackage[pdftex]{graphicx}
\usepackage{amsmath,amssymb,amsthm}
\usepackage{custom}
\usepackage{lipsum}

\usepackage{xcolor}
\usepackage{tikz}

\usepackage[most]{tcolorbox}
\usepackage{xcolor}
\usepackage{mdframed}

% page formatting
\usepackage{fancyhdr}
\pagestyle{fancy}

\renewcommand{\sectionmark}[1]{\markright{\textsf{\arabic{section}. #1}}}
\renewcommand{\subsectionmark}[1]{}
\lhead{\textbf{\thepage} \ \ \nouppercase{\rightmark}}
\chead{}
\rhead{}
\lfoot{}
\cfoot{}
\rfoot{}
\setlength{\headheight}{14pt}

\linespread{1.03} % give a little extra room
\setlength{\parindent}{0.2in} % reduce paragraph indent a bit
\setcounter{secnumdepth}{2} % no numbered subsubsections
\setcounter{tocdepth}{2} % no subsubsections in ToC


%%%%%%%%%%%%%%%%%%%%%%%%%%%%%%%%%%%%%%%%%%%%%%%%%%%%%%%%%%%%%%%%%
% CUSTOM BOXES AND STUFF
\newtcolorbox{redbox}{colback=red!5!white,colframe=red!75!black}
\newtcolorbox{bluebox}{colback=blue!5!white,colframe=blue!75!black}

\definecolor{lightblue}{RGB}{173,216,230} % Light blue color
\definecolor{darkblue}{RGB}{0,0,139} % Dark blue color

% Define the custom proof environment
\newtcolorbox{ex}[2][Example]{
  colback=red!5!white, % Light blue background
  colframe=red!75!black, % Darker blue border
  coltitle=white, % Title color
  fonttitle=\bfseries, % Title font style
  title={{#2}},
  arc=1mm, % Rounded corners with 4mm radius,
  boxrule=0.5mm,
  left=2mm, right=2mm, top=2mm, bottom=2mm, % Padding inside the box
  breakable, % Allow box to be broken across pages
  before=\vspace{10pt}, % Padding above the box
  after=\vspace{10pt}, % Padding below the box
  before upper={\parindent15pt} % Ensure indentation
}

% Define the custom proof environment
\newtcolorbox{defn}[2][Definition]{
  colback=blue!5!white, % Light blue background
  colframe=blue!75!black, % Darker blue border
  coltitle=white, % Title color
  fonttitle=\bfseries, % Title font style
  title={{#2}},
  arc=1mm, % Rounded corners with 4mm radius,
  boxrule=0.5mm,
  left=2mm, right=2mm, top=2mm, bottom=2mm, % Padding inside the box
  breakable, % Allow box to be broken across pages
  before=\vspace{10pt}, % Padding above the box
  after=\vspace{10pt}, % Padding below the box
  before upper={\parindent15pt} % Ensure indentation
}


%%%%%%%%%%%%%%%%%%%%%%%%%%%%%%%%%%%%%%%%%%%%%%%%%%%%%%%%%%%%%%%%%


\begin{document}

% make title page
\thispagestyle{empty}
\bigskip \
\vspace{0.1cm}

\begin{center}
{\fontsize{22}{22} \selectfont (Instructor: Chien-I Chiang)}
\vskip 16pt
{\fontsize{36}{36} \selectfont \bf \sffamily Physics 105: Analytical Mechanics notes}
\vskip 24pt
{\fontsize{18}{18} \selectfont \rmfamily Keshav Balwant Deoskar} 
\vskip 6pt
{\fontsize{14}{14} \selectfont \ttfamily kdeoskar@berkeley.edu} 
\vskip 24pt
\end{center}

% {\parindent0pt \baselineskip=15.5pt \lipsum[1-4]} 

% make table of contents
% \newpage

These are some very terse notes taken from UC Berkeley's Physics 105 during the Summer '24 session, taught by Chien-I Chiang.

\vskip 0.5cm
This template is based heavily off of the one produced by \href{https://knzhou.github.io/}{Kevin Zhou}.

% \microtoc
\tableofcontents 

% main 

%%%%%%%%%%%%%%%%%%%%%%%%%%%%%%%%%%%%%%%%%%%%%%
\newpage
\section{First topic}
%%%%%%%%%%%%%%%%%%%%%%%%%%%%%%%%%%%%%%%%%%%%%%

\vskip 0.5cm
text




% \begin{ex}{Fecko, Exercise 12.1.5}
    
% \end{ex}

% \begin{ex}{Fecko, Exercise 12.2.2}
    
% \end{ex}

% \begin{note}{Note}
% \end{note}

% \begin{thought}{A thought}
% \end{thought}

% \begin{highlight}{A hightlight}
% \end{highlight}



\end{document}










\end{document}