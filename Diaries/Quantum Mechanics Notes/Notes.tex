\documentclass[11pt]{article}

% basic packages
\usepackage[margin=1in]{geometry}
\usepackage[pdftex]{graphicx}
\usepackage{amsmath,amssymb,amsthm}
\usepackage{custom}
\usepackage{lipsum}

\usepackage{xcolor}
\usepackage{tikz-cd}

\usepackage[most]{tcolorbox}
\usepackage{xcolor}
\usepackage{mdframed}

% page formatting
\usepackage{fancyhdr}
\pagestyle{fancy}

\renewcommand{\sectionmark}[1]{\markright{\textsf{\arabic{section}. #1}}}
\renewcommand{\subsectionmark}[1]{}
\lhead{\textbf{\thepage} \ \ \nouppercase{\rightmark}}
\chead{}
\rhead{}
\lfoot{}
\cfoot{}
\rfoot{}
\setlength{\headheight}{14pt}

\linespread{1.03} % give a little extra room
\setlength{\parindent}{0.2in} % reduce paragraph indent a bit
\setcounter{secnumdepth}{2} % no numbered subsubsections
\setcounter{tocdepth}{2} % no subsubsections in ToC


%%%%%%%%%%%%%%%%%%%%%%%%%%%%%%%%%%%%%%%%%%%%%%%%%%%%%%%%%%%%%%%%%
% CUSTOM BOXES AND STUFF
\newtcolorbox{redbox}{colback=red!5!white,colframe=red!75!black, breakable}
\newtcolorbox{bluebox}{colback=blue!5!white,colframe=blue!75!black,breakable}

\newtcolorbox{dottedbox}[1][]{%
    colback=white,    % Background color
    colframe=white,    % Border color (to be overridden by dashrule)
    sharp corners,     % Sharp corners for the box
    boxrule=0pt,       % No actual border, as it will be drawn with dashrule
    boxsep=5pt,        % Padding inside the box
    enhanced,          % Enable advanced features
    breakable,         % Enables it to span multiple pages
    overlay={\draw[dashed, thin, black, dash pattern=on \pgflinewidth off \pgflinewidth, line cap=rect] (frame.south west) rectangle (frame.north east);}, % Dotted line
    #1                 % Additional options
}

% Define the colors
\definecolor{boxheader}{RGB}{0, 51, 102}  % Dark blue
\definecolor{boxfill}{RGB}{173, 216, 230}  % Light blue


% Define the tcolorbox environment
\newtcolorbox{mathdefinitionbox}[2][]{%
    colback=boxfill,   % Background color
    colframe=boxheader, % Border color
    fonttitle=\bfseries, % Bold title
    coltitle=white,     % Title text color
    title={#2},         % Title text
    enhanced,           % Enable advanced features
    breakable,
    attach boxed title to top left={yshift=-\tcboxedtitleheight/2}, % Center title
    boxrule=0.5mm,      % Border width
    sharp corners,      % Sharp corners for the box
    #1                  % Additional options
}
%%%%%%%%%%%%%%%%%%%%%%%%%


\definecolor{lightblue}{RGB}{173,216,230} % Light blue color
\definecolor{darkblue}{RGB}{0,0,139} % Dark blue color

% Define the custom proof environment
\newtcolorbox{ex}[2][Example]{
  colback=red!5!white, % Light blue background
  colframe=red!75!black, % Darker blue border
  coltitle=white, % Title color
  fonttitle=\bfseries, % Title font style
  title={{#2}},
  arc=1mm, % Rounded corners with 4mm radius,
  boxrule=0.5mm,
  left=2mm, right=2mm, top=2mm, bottom=2mm, % Padding inside the box
  breakable, % Allow box to be broken across pages
  before=\vspace{10pt}, % Padding above the box
  after=\vspace{10pt}, % Padding below the box
  before upper={\parindent15pt} % Ensure indentation
}

% Define the custom proof environment
\newtcolorbox{defn}[2][Definition]{
  colback=green!5!white, % Light blue background
  colframe=green!75!black, % Darker blue border
  coltitle=white, % Title color
  fonttitle=\bfseries, % Title font style
  title={{#2}},
  arc=1mm, % Rounded corners with 4mm radius,
  boxrule=0.5mm,
  left=2mm, right=2mm, top=2mm, bottom=2mm, % Padding inside the box
  breakable, % Allow box to be broken across pages
  before=\vspace{10pt}, % Padding above the box
  after=\vspace{10pt}, % Padding below the box
  before upper={\parindent15pt} % Ensure indentation
}


%%%%%%%%%%%%%%%%%%%%%%%%%%%%%%%%%%%%%%%%%%%%%%%%%%%%%%%%%%%%%%%%%


\begin{document}

% make title page
\thispagestyle{empty}
\bigskip \
\vspace{0.1cm}

\begin{center}
% {\fontsize{22}{22} \selectfont Lecturer: Alexander Givental}
% \vskip 16pt
{\fontsize{30}{30} \selectfont \bf \sffamily Miscellaneous Notes on Quantum Mechanics}
% \vskip 24pt
% {\fontsize{14}{14} \selectfont \rmfamily Random tidbits} 
\vskip 6pt
{\fontsize{14}{14} \selectfont \ttfamily kdeoskar@berkeley.edu} 
\vskip 24pt
\end{center}

% {\parindent0pt \baselineskip=15.5pt \lipsum[1-4]} 

% make table of contents
% \newpage

This is a collection of random facts, theorems, exercises, and illustrations during my journey of learning Quantum Mechanics. Topics may be organized in a slightly nonsensical manner (sorry). Any errors are due to my own ignorance - please feel free to reach out and correct me!
\\
\\
This template is based heavily off of the one produced by \href{https://knzhou.github.io/}{Kevin Zhou}.

% \microtoc
\setcounter{tocdepth}{3}
\tableofcontents 



% %%%%%%%%%%%%%%%%%%%%%%%%%%%%%%%%%%%%%%%%%%%%%%%%%%%%%%%%%
% \newpage
% \section{Conservation of Observables}
% %%%%%%%%%%%%%%%%%%%%%%%%%%%%%%%%%%%%%%%%%%%%%%%%%%%%%%%%%

% \subsection{When is an observable conserved?}



%%%%%%%%%%%%%%%%%%%%%%%%%%%%%%%%%%%%%%%%%%%%%%%%%%%%%%%%%
\newpage
\section{Adiabatic Theorem}
%%%%%%%%%%%%%%%%%%%%%%%%%%%%%%%%%%%%%%%%%%%%%%%%%%%%%%%%%




% %%%%%%%%%%%%%%%%%%%%%%%%%%%%%%%%%%%%%%%%%%%%%%%%%%%%%%%%%
% \newpage
% \section{}
% %%%%%%%%%%%%%%%%%%%%%%%%%%%%%%%%%%%%%%%%%%%%%%%%%%%%%%%%%


%%%%%%%%%%%%%%%%%%%%%%%%%%%%%%%%%%%%%%%%%%%%%%
\newpage
% \section{References}
%%%%%%%%%%%%%%%%%%%%%%%%%%%%%%%%%%%%%%%%%%%%%%
\vskip 0.5cm
\bibliographystyle{plain} % We choose the "plain" reference style
\bibliography{citation} % Entries are in the refs.bib file




\end{document}










