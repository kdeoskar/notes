\documentclass[11pt]{article}

% basic packages
\usepackage[margin=1in]{geometry}
\usepackage[pdftex]{graphicx}
\usepackage{amsmath,amssymb,amsthm}
\usepackage{custom}
\usepackage{lipsum}

\usepackage{xcolor}
\usepackage{tikz-cd}

\usepackage[most]{tcolorbox}
\usepackage{xcolor}
\usepackage{mdframed}

% page formatting
\usepackage{fancyhdr}
\pagestyle{fancy}

\renewcommand{\sectionmark}[1]{\markright{\textsf{\arabic{section}. #1}}}
\renewcommand{\subsectionmark}[1]{}
\lhead{\textbf{\thepage} \ \ \nouppercase{\rightmark}}
\chead{}
\rhead{}
\lfoot{}
\cfoot{}
\rfoot{}
\setlength{\headheight}{14pt}

\linespread{1.03} % give a little extra room
\setlength{\parindent}{0.2in} % reduce paragraph indent a bit
\setcounter{secnumdepth}{2} % no numbered subsubsections
\setcounter{tocdepth}{2} % no subsubsections in ToC


%%%%%%%%%%%%%%%%%%%%%%%%%%%%%%%%%%%%%%%%%%%%%%%%%%%%%%%%%%%%%%%%%
% CUSTOM BOXES AND STUFF
\newtcolorbox{redbox}{colback=red!5!white,colframe=red!75!black, breakable}
\newtcolorbox{bluebox}{colback=blue!5!white,colframe=blue!75!black,breakable}

\newtcolorbox{dottedbox}[1][]{%
    colback=white,    % Background color
    colframe=white,    % Border color (to be overridden by dashrule)
    sharp corners,     % Sharp corners for the box
    boxrule=0pt,       % No actual border, as it will be drawn with dashrule
    boxsep=5pt,        % Padding inside the box
    enhanced,          % Enable advanced features
    breakable,         % Enables it to span multiple pages
    overlay={\draw[dashed, thin, black, dash pattern=on \pgflinewidth off \pgflinewidth, line cap=rect] (frame.south west) rectangle (frame.north east);}, % Dotted line
    #1                 % Additional options
}

% Define the colors
\definecolor{boxheader}{RGB}{0, 51, 102}  % Dark blue
\definecolor{boxfill}{RGB}{173, 216, 230}  % Light blue


% Define the tcolorbox environment
\newtcolorbox{mathdefinitionbox}[2][]{%
    colback=boxfill,   % Background color
    colframe=boxheader, % Border color
    fonttitle=\bfseries, % Bold title
    coltitle=white,     % Title text color
    title={#2},         % Title text
    enhanced,           % Enable advanced features
    breakable,
    attach boxed title to top left={yshift=-\tcboxedtitleheight/2}, % Center title
    boxrule=0.5mm,      % Border width
    sharp corners,      % Sharp corners for the box
    #1                  % Additional options
}
%%%%%%%%%%%%%%%%%%%%%%%%%


\definecolor{lightblue}{RGB}{173,216,230} % Light blue color
\definecolor{darkblue}{RGB}{0,0,139} % Dark blue color

% Define the custom proof environment
\newtcolorbox{ex}[2][Example]{
  colback=red!5!white, % Light blue background
  colframe=red!75!black, % Darker blue border
  coltitle=white, % Title color
  fonttitle=\bfseries, % Title font style
  title={{#2}},
  arc=1mm, % Rounded corners with 4mm radius,
  boxrule=0.5mm,
  left=2mm, right=2mm, top=2mm, bottom=2mm, % Padding inside the box
  breakable, % Allow box to be broken across pages
  before=\vspace{10pt}, % Padding above the box
  after=\vspace{10pt}, % Padding below the box
  before upper={\parindent15pt} % Ensure indentation
}

% Define the custom proof environment
\newtcolorbox{defn}[2][Definition]{
  colback=green!5!white, % Light blue background
  colframe=green!75!black, % Darker blue border
  coltitle=white, % Title color
  fonttitle=\bfseries, % Title font style
  title={{#2}},
  arc=1mm, % Rounded corners with 4mm radius,
  boxrule=0.5mm,
  left=2mm, right=2mm, top=2mm, bottom=2mm, % Padding inside the box
  breakable, % Allow box to be broken across pages
  before=\vspace{10pt}, % Padding above the box
  after=\vspace{10pt}, % Padding below the box
  before upper={\parindent15pt} % Ensure indentation
}


%%%%%%%%%%%%%%%%%%%%%%%%%%%%%%%%%%%%%%%%%%%%%%%%%%%%%%%%%%%%%%%%%


\begin{document}

% make title page
\thispagestyle{empty}
\bigskip \
\vspace{0.1cm}

\begin{center}
% {\fontsize{22}{22} \selectfont Lecturer: Alexander Givental}
% \vskip 16pt
{\fontsize{30}{30} \selectfont \bf \sffamily Miscellaneous Notes on Solid State Physics}
% \vskip 24pt
% {\fontsize{14}{14} \selectfont \rmfamily Random tidbits} 
\vskip 6pt
{\fontsize{14}{14} \selectfont \ttfamily kdeoskar@berkeley.edu} 
\vskip 24pt
\end{center}

% {\parindent0pt \baselineskip=15.5pt \lipsum[1-4]} 

% make table of contents
% \newpage

This is a collection of random facts, theorems, exercises, and illustrations during my journey of learning Solid State Physics. Topics may be organized in a slightly nonsensical manner (sorry). Any errors are due to my own ignorance - please feel free to reach out and correct me!
\\
\\
This template is based heavily off of the one produced by \href{https://knzhou.github.io/}{Kevin Zhou}.

% \microtoc
\setcounter{tocdepth}{3}
\tableofcontents 



%%%%%%%%%%%%%%%%%%%%%%%%%%%%%%%%%%%%%%%%%%%%%%%%%%%%%%%%%
\newpage
\section{Why are there exactly 14 possible Bravais Lattices?}
%%%%%%%%%%%%%%%%%%%%%%%%%%%%%%%%%%%%%%%%%%%%%%%%%%%%%%%%%

The different types of lattice structures possible for a (ideal) crystalline material are called the \textbf{Bravais Lattices}. It's often mentioned that there are 14 such distinct 3-Dimensional Bravais Lattices, but why is this the case? 
\\
\\
The main characteristic of the Bravais Lattices is that they are the "building blocks" in our repeating crystalline structure. Thus, they should have certain properties like translational symmetry \begin{note}{Add more detail}\end{note}.
\\
\\ 
When it comes to actual materials, we live in 3D i.e. $\R^n$ so any Bravais Lattice is (equivalent to) some subset of $\zee^3$. As mentioned above, we want the lattices to have spatial symmetries i.e. invariance under certain elements of $GL(3, \zee)$.
\\
\\
It turns out that Bravais lattices are identified with conjugacy classes of certain finite subgroups of $GL(3,\zee)$ called \textbf{Bravais Subgroups}, and it follows from a theorem by Jordan that there are 14 such conjugacy classes. In general, Jordan's theorem tells us $GL(n, \zee)$ has finitely many conjugacy classes of finite subgroups \cite{Burde13}.
\\
\\
\begin{thought}
{Need to add much more; good discussion of crystallographic groups and possibly proof of special case $n=3$.}
\end{thought} See \cite{Whitman11}.



%%%%%%%%%%%%%%%%%%%%%%%%%%%%%%%%%%%%%%%%%%%%%%%%%%%%%%%%%
\newpage
\section{What is the Berry Phase? Is it topological or geometrical?}
%%%%%%%%%%%%%%%%%%%%%%%%%%%%%%%%%%%%%%%%%%%%%%%%%%%%%%%%%

Given a system with hilbert space $\mathcal{H}$, starting off in some eigenstate $\ket{n}$ whose hamiltonian remains constant, we can solve for the time-evolution of the state by solving the Time-Dependent Schr\"{o}dinger Equation. We know that $t$ moments after the start, the system will be in the state $e^{-\frac{i}{\hbar}\hat{H} t}\ket{n}$.
\\
\\
But what if the Hamiltonian also changes with respect to some set of parameters $\mathbf{\lambda} = (\lambda_1, \cdots, \lambda_N)$ as $\hat{H}(\mathbf{\lambda})$? If the hamiltonian changes \emph{slowly} from $\hat{H}(0)$ to $\hat{H}(T)$, the Adiabatic Theorem tells us that the $n^{th}$ eigenstate $\ket{n}$ of $\hat{H}(0)$ evolves such that it is the $n^{th}$ eigenstate of $\hat{H}(t)$ for each $t \in [0, T]$.
\\
\\
Now, rather than just $e^{-\frac{i}{\hbar}\hat{H}(0)t}$, it picks up the phase: $$ e^{-\frac{i}{\hbar}\theta_n(t) } \text{ where } \theta_n(t) \equiv \int_{0}^{t} E_n(t') \mathrm{d}t'$$ This $\theta_n$ is called the \textbf{Dynamical Phase}. But this may not be the entire story. Although $\ket{n}$ remains an eigenstate, it may not be exactly the same i.e. it may pick up an additional phase $e^{i\gamma_n}$ where $\gamma_n$ is called the \textbf{Geometrical Phase}, since such a state would be physically equivalent.
\\
\\
Berry purportedly described the Dynamical and Geometrical Phases, respectively, as answering the two questions "How long have you been moving?" and "Where have you gone?" \cite{GriffithsQM}.

\subsection{Motivation/Expression for the Geometric Phase $\gamma$}
Consider the Hamiltonian above $H(\mathbf{\lambda}(t))$ where the parameter $\mathbf{\lambda}$ is varying with time, causing the hamiltonian to vary as well. Let $\ket{\psi(t)}$ denote an eigenstate of $$H(\mathbf{\lambda}(t)) \ket{\psi(t)} = E(t)\ket{\psi(t)} $$
Schr\"odinger's Equation states that $$ i\hbar\frac{d\ket{\psi}}{dt} = H(t)\ket{\psi(t)} $$
\\
\\
At first, we might try the usual ansatz of $\ket{\psi} = e^{-\frac{i}{\hbar}\theta_n(t)} \ket{n}$ but plugging this into Schro\"odinger's Equation, we see that it fails to be a solution because of the differential on the LHS.
\\
\\
Instead let's the ansatz $\ket{\psi} = e^{i\gamma_n(t)} e^{-\frac{i}{\hbar}\theta_n(t)} \ket{n}$. Plugging this in,
\begin{align*}
  &i\hbar \left\{ e^{-\frac{i}{\hbar}\theta_n(t)} \frac{d\ket{n}}{dt} + \left( i \dot{\gamma}_n(t) - \frac{i}{\hbar} E(t) \right) e^{i\gamma_n(t)} e^{-\frac{i}{\hbar}\theta_n(t)} \ket{n} \right\} = E(t)e^{-\frac{i}{\hbar}\theta_n(t)} \ket{n} \\
  \implies& i\hbar e^{-\frac{i}{\hbar}\theta_n(t)}  \frac{d\ket{n}}{dt} -\hbar \dot{\gamma}_n(t) e^{-\frac{i}{\hbar}\theta_n(t)} \ket{n}  + E(t)e^{-\frac{i}{\hbar}\theta_n(t)} \ket{n} = E(t)e^{-\frac{i}{\hbar}\theta_n(t)} \ket{n} \\
\end{align*}
Dotting both sides with $\bra{n}$ then gives us 
\begin{align*}
  &i\hbar e^{-\frac{i}{\hbar}\theta_n(t)} \braket{n}{\frac{dn}{dt}} = \hbar \dot{\gamma}_n(t)e^{-\frac{i}{\hbar}\theta_n(t)} \ket{n}  \\
  \implies& \dot{\gamma}_n(t) = i\braket{n}{\frac{dn}{dt}}
\end{align*}
giving us 
\begin{align*}
  \gamma_n(t) &= \int_0^{t} A(t') dt' \\
  A(t) &= i \braket{n}{\frac{dn}{dt}}
\end{align*}
\begin{note}
  {Write about the relation with gauge transformations and how this geometrical factor cannot be ignored when we have a closed loop in parameter space.}
\end{note}



%%%%%%%%%%%%%%%%%%%%%%%%%%%%%%%%%%%%%%%%%%%%%%%%%%%%%%%%%
\newpage
\section{Quantum Geometrical Tensor and Quantum Metric}
%%%%%%%%%%%%%%%%%%%%%%%%%%%%%%%%%%%%%%%%%%%%%%%%%%%%%%%%%






% %%%%%%%%%%%%%%%%%%%%%%%%%%%%%%%%%%%%%%%%%%%%%%%%%%%%%%%%%
% \newpage
% \section{}
% %%%%%%%%%%%%%%%%%%%%%%%%%%%%%%%%%%%%%%%%%%%%%%%%%%%%%%%%%


%%%%%%%%%%%%%%%%%%%%%%%%%%%%%%%%%%%%%%%%%%%%%%
\newpage
% \section{References}
%%%%%%%%%%%%%%%%%%%%%%%%%%%%%%%%%%%%%%%%%%%%%%
\vskip 0.5cm
\bibliographystyle{plain} % We choose the "plain" reference style
\bibliography{citation} % Entries are in the refs.bib file




\end{document}










